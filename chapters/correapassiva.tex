\documentclass[output=paper]{LSP/langsci} 
\author{Letícia M. Sicuro Corrêa\affiliation{Pontifícia Universidade Católica do Rio de Janeiro, Laboratório de Psicolinguística e Aquisição da Linguagem (LAPAL)}\and
Marina R. A. Augusto\affiliation{Universidade do Estado do Rio de Janeiro, Laboratório de Psicolinguística e Aquisição da Linguagem (LAPAL)}\lastand 
João C. de Lima-Júnior\affiliation{Pontifícia Universidade Católica do Rio de Janeiro, Laboratório de Psicolinguística e Aquisição da Linguagem (LAPAL)}
}
\title{Passivas} 
\ChapterDOI{10.5281/zenodo.889433}
\abstract{}
\maketitle
\begin{document}
\section{Introdução} 
\label{sec:correapassiva_correapassiva_intro}

No estudo da aquisição da sintaxe da língua materna, estruturas passivas\is{passiva} (cf. (\ref{ex:correapassiva_1}) e (\ref{ex:correapassiva_2})) têm recebido considerável atenção. Essa atenção foi inicialmente motivada pelo destaque dado a essas estruturas no início da proposta gerativista, em meados do século XX \citep{chomsky1957,chomsky1965}. Posteriormente, os resultados da pesquisa acerca da produção e da compreensão de passivas\is{passiva} por crianças, aliados aos desenvolvimentos da pesquisa linguística em direção a um maior entendimento da natureza das línguas humanas, têm trazido renovado interesse na aquisição dessas estruturas.

\ea\label{ex:correapassiva_1} A maçã foi mordida pela menina. 
\z
\ea\label{ex:correapassiva_2} A música foi ouvida pela menina. 
\z

Sentenças na \textit{voz passiva}\is{passiva} têm baixa frequência na fala de crianças, são de difícil compreensão, em comparação a outras estruturas, mostram-se particularmente afetadas quando há problemas de linguagem, como no caso de PEL ou DEL,\footnote{PEL (\isi{Perturbação Específica da Linguagem}), termo utilizado em Portugal, ou DEL (Distúrbio/Déficit Específico da Linguagem), termo utilizado no Brasil.} e evidenciam sérios comprometimentos na sintaxe de adultos, decorrentes de um tipo de \isi{afasia} adquirida (\isi{agramatismo}). Sua produção e compreensão são custosas e sua aquisição foi tradicionalmente tida como tardia. Diferentes hipóteses têm sido formuladas para explicar o modo como a aquisição de passivas\is{passiva} transcorre e as possíveis causas da dificuldade que apresentam. Recentemente, contudo, constata-se que a produção de sentenças passivas\is{passiva} pode ser elicitada (ou induzida) em crianças de relativa tenra idade. Fatores de ordem pragmática têm sido apontados como possível fonte das dificuldades de compreensão por parte de crianças em tarefas experimentais, ainda que de forma não conclusiva. Em que consiste a dificuldade das crianças? Como conciliar resultados aparentemente contraditórios entre compreensão e produção? O que, afinal, a criança deve adquirir no que diz respeito a passivas?\is{passiva} Como esse processo transcorre? Essas são algumas das questões a que o estudo da aquisição de passivas\is{passiva} na língua materna busca responder.

Passivas\is{passiva} estão presentes na maior parte das línguas conhecidas \citep{keenandryer2007}. Trata-se de uma forma como as relações temáticas entre os argumentos de um predicador/verbo são expressas na sintaxe. Caracterizam-se por apresentarem o sujeito com papel temático diferente de \isi{agente} ou de \textit{experienciador},\is{experienciador} como nas formas ativas\is{ativa} canônicas (\ref{ex:correapassiva_3}) e (\ref{ex:correapassiva_4}).

\ea\label{ex:correapassiva_3} A menina mordeu a maçã.
\z
\ea\label{ex:correapassiva_4} A menina ouviu a música.
\z

Línguas diferem, contudo, quanto à forma como essas estruturas se realizam ou mesmo quanto aos tipos de verbo que admitem sua formação \citep{keenandryer2007}. Em línguas como o português e o inglês\il{inglês}, por exemplo, passivas\is{passiva} podem ser formadas com um verbo transitivo em forma participial. Nessas estruturas, o \textit{agente}/ \is{agente}\textit{experienciador}\is{experienciador} é expresso como complemento de uma preposição (\textit{por}, em português) em um sintagma preposicional (referido na literatura como \textit{by-phrase}),\is{by-phrase} que pode ser omitido (\ref{ex:correapassiva_5}). 

\ea\label{ex:correapassiva_5} A maçã foi mordida (pela menina).
\z

De um ponto de vista funcional, o fato de um elemento diferente do agente\is{agente}\slash experienciador\is{experienciador} ocupar a posição proeminente de sujeito vem satisfazer certas demandas discursivas, como, por exemplo, na manutenção de um tópico, quando este não coincide com o \isi{agente}/\isi{experienciador} na relação temática expressa na sentença em questão.  A frequência no uso dessas estruturas varia, não obstante, entre línguas, o que pode ter impacto em sua aquisição. 

Neste capítulo, introduzimos a pesquisa sobre aquisição de passivas,\is{passiva} caracterizando brevemente seu curso e o estado-da-arte, levando em conta resultados recentes obtidos em português europeu (PE) e brasileiro (PB). Para isso, começamos por estabelecer uma importante distinção entre \textit{sujeito lógico}\is{sujeito!lógico} e \textit{sujeito gramatical}\is{sujeito!gramatical} e sua relação com papéis temáticos.

Sabemos que sentenças ou expressões linguísticas são formadas pela combinação de elementos do léxico (palavras, morfemas) em constituintes de uma estrutura hierárquica. A estrutura hierárquica subjacente às sentenças ou expressões linguísticas com verbos transitivos de qualquer língua pode ser representada em (\ref{ex:correapassiva_6}), em que VP é um sintagma verbal, A é o sujeito lógico\is{sujeito!lógico} (ou argumento externo de um predicador/verbo) e B é o objeto lógico (argumento interno de um predicador). Na interface entre a sintaxe e a semântica, atribui-se caracteristicamente ao sujeito lógico\is{sujeito!lógico} o papel temático de \isi{agente} da ação ou de \isi{experienciador} do estado ou processo apresentado pelo verbo. Ao objeto lógico, é atribuído o papel de \isi{tema} ou de \isi{paciente} (se animado, com verbo de ação). 

\ea\label{ex:correapassiva_6}
\begin{forest} baseline
  [VP
    [A]
    [V'
      [V]
      [B]]
]
\end{forest}
\z

Em estruturas na voz ativa\is{ativa} com verbo transitivo, o sujeito gramatical\is{sujeito!gramatical} coincide com o sujeito lógico.\is{sujeito!lógico} Em estruturas passivas,\is{passiva} contudo, o sujeito gramatical\is{sujeito!gramatical} corresponde ao objeto lógico ou argumento interno do verbo.

A possibilidade de o argumento interno do verbo ocupar a posição de sujeito gramatical\is{sujeito!gramatical} em uma sentença é explicada, na teoria linguística gerativista, como decorrente da disponibilidade de uma operação sintática (universal) que desloca um constituinte de sua posição de origem (na estrutura em (\ref{ex:correapassiva_7})) para uma dada posição na estrutura que se constrói quando sentenças são geradas em uma determinada língua. Diz-se, no caso das passivas,\is{passiva} que o argumento interno do verbo é movido para a posição argumental de sujeito gramatical\is{sujeito!gramatical} na língua em questão (o que é denominado movimento-A,\is{movimento!movimento-A}\is{cadeia-A|see {movimento!movimento-A}} de argumental), em função de propriedades do verbo ou da possibilidade de haver um elemento no léxico responsável por voz gramatical a que o verbo se associe.\footnote{A estrutura em (\ref{ex:correapassiva_7}) é uma ilustração do movimento-A,\is{movimento!movimento-A} ou seja, o movimento de um elemento para uma posição argumental, especificamente, no caso da passiva,\is{passiva} do objeto lógico para a posição de sujeito sintático. Há diferentes análises em relação à estrutura passiva:\is{passiva} uma das mais influentes tem sido a de \citet{collins2005} que adota uma projeção específica VoiceP para caracterizar a passiva.\is{passiva} \citet{limajunioraugusto2015,limajunioraugustoemprep} propõem uma releitura de \citet{collins2005}, a partir da adoção de um nó PassiveP.}

\ea\label{ex:correapassiva_7}
\begin{forest} baseline
  [TP
    [B$_t$,name=Bt]
    [T'
      [VP
      [A]
      [V'
      [V]
      [B,name=B]
      ] 
]]]
      \draw[->] (B) to[out=south west, in=south] (Bt); 
\end{forest}
\z

Independentemente da explicação fornecida para o \textit{movimento de argumento} em uma gramática gerativa, são as propriedades formais, que podem ser percebidas pela criança nos dados da fala, que informam que, na língua em questão, o objeto lógico de um verbo transitivo pode se realizar como sujeito gramatical.\is{sujeito!gramatical} Em português, por exemplo, a estrutura passiva\is{passiva} mais característica apresenta o verbo auxiliar – \textit{SER} – com o qual se combina a forma participial do verbo principal (\textit{-ado}). Essas propriedades, assim como a presença de um PP\footnote{Prepositional Phrase (PP); em português, Sintagma ou Grupo Preposicional. Usaremos aqui estes termos em inglês\il{inglês}, por ser o mais usual.} (\textit{by-phrase}),\is{by-phrase} tornam visível o tipo de relação entre função/posição sintática e papel temático que essas estruturas apresentam, ou seja, a relação entre sujeito-gramatical\is{sujeito!gramatical} e \isi{tema}.

O estudo da aquisição de passivas\is{passiva} começa por verificar se e quando a criança interpreta o sujeito dessas sentenças como \isi{paciente}/\isi{tema}; se e quando ela produz essa estrutura e que fatores afetam o desempenho linguístico de crianças. 

Neste capítulo, trazemos resultados da pesquisa em aquisição de passivas\is{passiva} e hipóteses relativas à natureza das dificuldades que a criança apresenta. Veremos que o modo como a criança identifica as propriedades que possibilitam a geração de estruturas desse tipo por sua gramática ainda não foi suficientemente clarificado, o que nos motiva a dar continuidade ao estudo da aquisição de passivas\is{passiva} na língua materna. Antes, porém, de apresentarmos esses resultados, vamos caracterizar os tipos de estruturas passivas\is{passiva} que vêm sendo considerados no estudo da aquisição da linguagem.  

\section{Tipos de estruturas passivas}
\label{sec:correapassiva_tipos_passivas}
\is{passiva}
Até então, vimos nos referindo à estrutura mais característica da \textit{voz passiva}, a \textit{passiva sintática}\is{passiva!sintática} ou \textit{eventiva}. Esta pode ser subdividida em \textit{agentiva} (com verbos de ação) (\ref{ex:correapassiva_1}) e \textit{não-agentiva} (\ref{ex:correapassiva_2}) (com verbos de percepção ou de estado psicológico), que, em lugar de um \textit{agente},\is{agente} têm um \textit{experienciador}\is{experienciador} no complemento da preposição. 

Esta é a estrutura que a gramática tradicional apresenta como \textit{voz passiva} e a que atraiu o interesse de linguistas, no início da teoria gerativista. Observa-se, nos pares em (\ref{ex:correapassiva_1}-\ref{ex:correapassiva_3}) e (\ref{ex:correapassiva_2}-\ref{ex:correapassiva_4}), que ativas\is{ativa} e passivas compartilham o mesmo verbo (\textit{morder}/\textit{ouvir}) com formas distintas. Em ambas as estruturas, o verbo atua como um predicador que estabelece o mesmo tipo de relação temática com os elementos nominais.

A suposta equivalência ou semelhança de sentido entre sentenças ativas\is{ativa} e passivas\is{passiva} foi enfatizada nos primeiros modelos da teoria gerativista. Nestes, sentenças passivas\is{passiva} eram geradas por meio de uma regra do tipo transformacional, a partir de uma estrutura equivalente à de ativas\is{ativa} \citep{chomsky1957,chomsky1965}. Assumindo-se que uma gramática gerativa representa a competência linguística do falante da língua, considerava-se que a aquisição de passivas\is{passiva} iria, então, requerer a incorporação de \textit{uma regra do tipo transformacional} na gramática (\textit{língua interna}) da criança. Com o desenvolvimento da pesquisa linguística, no entanto, não só a aquisição da linguagem deixou de ser concebida como aquisição de regras específicas de uma língua \citep{chomsky1981}, como ficou evidente que passivas\is{passiva} não podem ser derivadas de sentenças ativas,\is{ativa} sem que se altere seu sentido, como demonstram as sentenças (\ref{ex:correapassiva_8}) e (\ref{ex:correapassiva_9}), dada a presença de \textit{todo} (um quantificador). 

\ea\label{ex:correapassiva_8} Toda criança comeu sua maçã.
\z
\ea\label{ex:correapassiva_9} A sua maçã foi comida por toda criança.
\z

Estudos linguísticos recentes não só consideram ativas\is{ativa} e passivas\is{passiva} como estruturas independentes na gramática da língua, como agregam, como \textit{passivas},\is{passiva} diferentes tipos de estruturas em função do predicado que apresentam. 

Considerando-se, particularmente, o português, as passivas sintáticas ou eventivas,\is{passiva!sintática} formadas com o auxiliar SER, se distinguem das \textit{passivas adjetivais},\is{passiva!adjetival} subdivididas em \textit{estativas}, com ESTAR (\ref{ex:correapassiva_10}) e \textit{resultativas}, com FICAR (\ref{ex:correapassiva_11}).

\ea\label{ex:correapassiva_10} A maçã estava mordida.
\z
\ea\label{ex:correapassiva_11} A criança ficou assustada.
\z

Em línguas como o inglês\il{inglês}, a distinção entre passivas sintáticas\is{passiva!sintática} e adjetivais,\is{passiva!adjetival} do tipo estativo se mantém exclusivamente em função do significado do verbo no particípio, visto que ambas são formadas com o auxiliar BE (\ref{ex:correapassiva_12}) e (\ref{ex:correapassiva_13}). Nessa língua, o verbo GET é usado na formação das passivas resultativas (\ref{ex:correapassiva_14}).

\ea\label{ex:correapassiva_12} The child was kissed by her mother.
\z
\ea\label{ex:correapassiva_13} The child was scared.
\z
\ea\label{ex:correapassiva_14} The child got scared.
\z

No estudo da aquisição de passivas,\is{passiva} as subdivisões aqui apresentadas vêm sendo exploradas. Comparam-se, por exemplo, passivas agentivas (com verbos de ação) e não-agentivas (com verbos psicológicos e de percepção) e verifica-se a ordem de aquisição entre estas e as diferentes passivas adjetivais.\is{passiva!adjetival} Considera-se, ainda, a possibilidade de omissão do sintagma-preposicionado nas passivas eventivas, contrastando-se passivas longas\is{passiva!longa} (com PP, \textit{by-phrase})\is{by-phrase} e curtas\is{passiva!curta} (sem PP, \textit{by-phrase}).\is{by-phrase} Numa situação de fala, há omissão do PP (\textit{by-phrase})\is{by-phrase} quando o \isi{agente} não é conhecido pelo falante ou não é do seu interesse dá-lo a conhecer, ou mesmo em função do que é do conhecimento do ouvinte. A interação entre fatores de ordem sintática e pragmática pode, assim, ser explorada no estudo da aquisição dessas estruturas.

O português apresenta, ainda, o que se denomina passiva pronominal\is{passiva!pronominal} ou passiva de –\textit{se} (\ref{ex:correapassiva_15}). Em PB, seu uso é restrito à língua culta, na modalidade escrita. Na língua oral, a forma singular é preferida, o que sugere que a estrutura é analisada como sentença com sujeito indeterminado (\ref{ex:correapassiva_16}). Em PE, o uso de passiva pronominal\is{passiva!pronominal} é mais amplo. Essa estrutura é produtiva tanto na língua oral quando na escrita \cite{correia2003}.

\ea\label{ex:correapassiva_15} Ouviram-se gritos no corredor.
\z

\ea\label{ex:correapassiva_16} Ouviu-se gritos no corredor.
\z

Vimos que o modo como estruturas passivas\is{passiva} são caracterizadas tem implicações para o estudo da sua aquisição. A seguir, traçamos um breve histórico dessa pesquisa, trazendo seus principais resultados.

\section{A aquisição de passivas:\is{passiva} breve histórico e principais resultados}
\label{sec:correapassiva_breve_historia}
\subsection{Primeiros estudos}
\label{subsec:correapassiva_primeiros_estudos}

Os primeiros estudos sobre a aquisição de passivas,\is{passiva} orientados pelas propostas iniciais da linguística gerativista, buscaram verificar em que medida haveria evidência da presença da regra de geração de passivas\is{passiva} na gramática da criança. Em um trabalho seminal, \citet{menyuk1969} apontou, por exemplo, que havia evidências esparsas para a presença da regra transformacional de passivas\is{passiva} na fala espontânea de crianças de três a sete anos falantes de inglês\il{inglês}. A baixa frequência dessas estruturas na fala espontânea também foi constatada no clássico estudo longitudinal de \citet{brown1973}. 

No que diz respeito à compreensão, a aquisição dessa regra seria crucial para a condução da análise sintática da sentença, o que permitiria à criança interpretar o sujeito gramatical\is{sujeito!gramatical} como \isi{tema}/\isi{paciente}, independentemente do tipo de relação semântica que a sentença apresenta. Com o objetivo de verificar em que medida a criança teria o conhecimento linguístico necessário para interpretar as relações semânticas da sentença com base exclusivamente em informação sintática, uma série de estudos experimentais foi conduzida, com vistas a identificar o efeito de fatores tais como \textit{\isi{reversibilidade} de papéis temáticos} e \textit{plausibilidade} da relação \isi{agente}-ação-\isi{paciente} no evento descrito pelo verbo da sentença no desempenho linguístico da criança \citep{bever1970,strohnernelson1974}. Esses estudos foram conduzidos originalmente em inglês\il{inglês} e o padrão de desempenho tem sido amplamente replicado. Reversibilidade\is{reversibilidade} de papéis temáticos pode ser ilustrada em (\ref{ex:correapassiva_17}) e (\ref{ex:correapassiva_18}).

\ea\label{ex:correapassiva_17} A boneca foi beijada pela menina.
\z
\ea\label{ex:correapassiva_18} A menina foi beijada pela mãe.
\z

Em (\ref{ex:correapassiva_17}), os papéis temáticos \textit{tema}\is{tema} e \textit{agente}\is{agente} não são intercambiáveis entre os elementos nominais \textit{a boneca} e \textit{a menina}, dado que (em função de \textit{animacidade}, um traço semântico do nome) apenas \textit{a menina} pode ser tomada como \isi{agente} de \textit{beijar}. Considera-se, então, que a sentença apresenta uma relação temática \textit{irreversível}. Já em (\ref{ex:correapassiva_18}), tanto \textit{a menina} como \textit{a mãe} podem ser o \isi{agente} ou o \isi{tema} do verbo \textit{beijar}. Sentenças em que os papéis temáticos são intercambiáveis entre os elementos nominais são então chamadas \textit{passivas reversíveis}.\is{passiva} A \isi{reversibilidade} de papéis temáticos é um fator que pode afetar mesmo a compreensão de sentenças ativas\is{ativa} por crianças em determinadas tarefas e seu efeito na compreensão de passivas\is{passiva} pode perdurar em idade escolar \citep{augustocorrea2012}.

Crianças também se mostram afetadas pela \textit{plausibilidade do evento} descrito pela sentença. Assim, uma sentença que apresenta uma relação semântica implausível como (\ref{ex:correapassiva_19}) mostra-se particularmente difícil para crianças de três anos e mesmo na faixa dos cinco anos de idade a taxa de acertos na compreensão de sentenças implausíveis é inferior à obtida com eventos plausíveis \citep{strohnernelson1974}.

\ea\label{ex:correapassiva_19} O gato foi perseguido pelo rato.
\z

A compreensão de passivas\is{passiva} foi inicialmente investigada por meio de técnicas experimentais clássicas como \textit{acting out} (tarefa de manipulação de brinquedos/objetos) e identificação de imagens. A primeira consiste na apresentação de uma sentença a partir da qual a criança monta uma cena com bonecos/animais de brinquedo e/ou objetos em termos de relações do tipo \isi{agente}-ação-\isi{paciente}/\isi{tema}. Na tarefa de identificação de imagens, as crianças devem escolher, dentre duas ou mais figuras, aquela que melhor combina com o enunciado apresentado pelo experimentador e as imagens apresentam a \isi{reversibilidade} dos papéis de \isi{agente} e \isi{paciente}. Por exemplo, diante de uma instrução como \textit{Mostra a figura que combina com o que eu vou dizer: O macaco foi lavado pelo elefante}, apresentam-se para as crianças uma figura com o elefante \isi{agente} e o macaco \isi{paciente}, e outra em que o macaco é \isi{agente} e o elefante é \isi{paciente}. 

O desempenho de crianças em função de fatores como \isi{reversibilidade} e plausibilidade é sugestivo do uso de estratégias cognitivas - procedimentos heurísticos de busca de solução para a situação-problema que a tarefa apresenta. Uma estratégia que seria utilizada em línguas do tipo SVO, como o inglês\il{inglês} ou o português, é a chamada estratégia NVN \citep{bever1970}: interprete uma sequência Nome-Verbo-Nome como \isi{agente}/ação/objeto, a não ser que sinalizado o contrário. A criança teria dificuldade na identificação das sinalizações que bloqueariam o uso da estratégia.

\subsection{Possíveis relações entre sintaxe e semântica}
\label{subsec:correapassiva_possiveis_relacoes}

Numa linha de investigação voltada para possíveis relações entre sintaxe e semântica na aquisição da linguagem, os tipos de verbos formadores de estruturas passivas\is{passiva} foram explorados. \citet{maratsos_etal1985} compararam o desempenho de crianças na compreensão de passivas\is{passiva} com verbos de ação (como \textit{segurar}, \textit{sacudir}, \textit{lavar}) e de estado mental (de percepção, como \textit{ouvir}; de cognição, como \textit{saber}; e de estado emocional, como \textit{gostar}), contrastando, desse modo, o que denominamos de passivas\is{passiva} \textit{agentivas} e \textit{não-agentivas} na seção 2. O procedimento inicialmente considerado mais adequado para o estudo de verbos de estado mental foi o uso de perguntas de compreensão relativas aos participantes do evento, apresentado em (\ref{ex:correapassiva_20}).

\ea\label{ex:correapassiva_20}
O menino foi chamado pela mãe.\\
Pergunta: Quem chamou? Quem é que chamou?
\z

Os resultados evidenciaram que passivas\is{passiva} formadas por verbos de estado mental acarretam mais dificuldade de compreensão do que as formadas por verbos de ação para crianças de 4 e 5 anos de idade. O método de investigação apresentou, contudo, dificuldade para crianças. Os pesquisadores observaram que a taxa de acertos nos verbos de ação foi consideravelmente inferior à reportada em outros estudos com metodologia tradicional. A técnica da seleção de imagens foi então utilizada, criando-se convenções pictóricas para a expressão de estado mental e de percepção, com as quais as crianças foram familiarizadas. Crianças de 4 a 11 anos foram testadas. Os resultados revelaram que o número de acertos ultrapassa 85\% nas passivas\is{passiva} de ação no grupo de 4 e 90\% no grupo de 5 anos, em consonância com estudos prévios. Nas passivas\is{passiva} de estado mental, o grupo de 9 anos se aproxima dessa faixa e apenas aos 11 anos o desempenho é semelhante em todos os tipos de sentença testados. Os pesquisadores sugerem que as crianças podem buscar uma generalização quanto ao tipo de relações semânticas que podem ser expressas por passivas,\is{passiva} cuja interpretação não seria, portanto, determinada pela sintaxe, mas construída em função de informação sintática e semântica. 

Os estudos conduzidos a partir da década de 90 mostram-se mais linguisticamente orientados e exploram outras metodologias, como o Julgamento de Valor de Verdade, com técnica introduzida em estudo de \citet{crain_mckee1985}. Histórias breves são contadas à criança, acompanhadas de manipulação de brinquedos correspondentes aos personagens, por parte do experimentador. Ao final do relato, uma afirmação sobre o que ocorreu na história é apresentada e a criança deve decidir se esta está de acordo com o que ouviu. Em (\ref{ex:correapassiva_21}), tem-se um exemplo desse tipo de material \citep{limajunior2012}. 

\ea\label{ex:correapassiva_21}
O Sapinho Popó, o gatinho Mimi e o cachorrinho Zecão foram passear no bosque. Quando avistou um lindo jardim, Zecão saiu correndo na frente, ele começou a arrancar as flores e a pisar toda a grama. O sapinho Popó e o gatinho Mimi disseram para ele não fazer aquilo, porque é errado destruir a natureza. O cachorrinho Zecão nem ligou. Então, o gatinho Mimi amarrou o cachorrinho Zecão com uma corda. O Sapo Popó disse que ele só sairia dali quando pedisse desculpas pelo que fez.\\
Um Fantoche afirma: Eu sei o que aconteceu nessa história: O cachorrinho Zecão foi amarrado pelo gatinho Mimi. (V) / O gatinho Mimi foi amarrado pelo cachorrinho Zecão (F) 
\z

Em estudo realizado com crianças entre 3;6 e 5;5 anos de idade, por meio dessa técnica, a distinção entre verbos de ação e de estado mental foi retomada e foi também manipulado o tipo de passiva,\is{passiva} em função da presença/omissão do PP – passivas curtas\is{passiva!curta} (sem o \isi{agente} explícito) e longas\is{passiva!longa} (com o \isi{agente} explícito) \citep{foxgrodzinsky1998}. Os resultados obtidos revelaram que a dificuldade de crianças com verbos de estado mental restringe-se à compreensão de passivas longas.\is{passiva!longa} Esses resultados são interpretados como indicativos de dificuldades na atribuição do papel temático para o elemento nominal (NP) contido no PP, quanto este não pode ser entendido como causador, como no caso dos verbos de ação.

\subsection{Fatores de ordem pragmática e discursiva}
\label{subsec:correapassiva_fatores_prag_disc}

Mais recentemente, questões de ordem pragmática associadas à metodologia para a avaliação do desempenho de crianças têm sido levantadas. \citet{obrien_etal2006} sugerem, por exemplo, que a dificuldade atestada na compreensão de passivas longas\is{passiva!longa} pode ser atribuída a uma metodologia que não atende a condições de felicidade para seu uso. Consideram ser necessário introduzir um terceiro personagem na história apresentada à criança, de modo a tornar a informação trazida pelo PP de uma passiva longa\is{passiva!longa} informativa. A história acima apresentada (cf. (\ref{ex:correapassiva_21})) ilustra um material que atende ao critério de felicidade ou de sucesso estipulado, por esses autores, para o uso de passivas longas.\is{passiva!longa} No estudo de \citeauthor{obrien_etal2006}, crianças de 3 anos tiveram um melhor desempenho nas condições com passivas longas\is{passiva!longa} em que a condição de felicidade (ou de sucesso) foi atendida, mesmo com verbos de estado mental. Um estudo com crianças que adquirem o PB revelou, contudo, que a presença de um terceiro personagem não é, por si só, um fator decisivo para tornar a compreensão de passivas mais acessível para crianças dessa faixa etária \citep{limajunior2016}, embora a manutenção do tópico do discurso por meio do sujeito da passiva crie uma expectativa para o uso dessa estrutura, visto que a não manutenção do tópico do discurso pelo sujeito da passiva dificulta a tarefa particularmente para as crianças mais jovens\citep{limajunior2016}. 
Os textos em (\ref{ex:correapassiva_22a}) e (\ref{ex:correapassiva_22b}) ilustram o material utilizado em \citet{limajunior2016}: 

\noindent\parbox{\textwidth}{\ea\label{ex:correapassiva_22}
\ea\label{ex:correapassiva_22a}
Tópico Mantido pela sentença-alvo final.\\
Essa é a história de um \textbf{leão} bonzinho. Ele vive com o seu grande amigo, o porco. Um dia, \textbf{esse leão} colocou um filme assustador no DVD. A cada cena do filme, um deles gritava: -AHHHH! Passado um tempo, o \textbf{leão} ficou com fome e resolveu ir à cozinha comer e pegar água pro porco. Nessa hora, uma cena horrível apareceu no vídeo e, \textbf{o leão} deu um rugido bem alto: URRAU! No susto, \textbf{O LEÃO} foi agarrado pelo porco.
\ex\label{ex:correapassiva_22b}
Tópico Não-Mantido pela sentença-alvo final.\\
Essa é a história de \textbf{um leão} bonzinho. \textbf{Ele} vive com o seu grande amigo, o porco. Um dia, \textbf{esse leão} colocou um filme assustador no DVD. A cada cena do filme, um deles gritava: -AHHHH! Passado um tempo, \textbf{o leão} ficou com fome e resolveu ir à cozinha comer e pegar água pro porco. Nessa hora, uma cena horrível apareceu no vídeo e, \textbf{o leão} deu um rugido bem alto: URRAU! No susto, \textbf{O PORCO} foi agarrado pelo leão.
\zl}

A satisfação de critérios de felicidade ou sucesso não parece ser, contudo, suficiente para eliminar a dificuldade das crianças com verbos de estado mental. Em experimentos conduzidos em PB e em PE, apenas com contextos felizes, os resultados vão na mesma direção dos obtidos por \citet{foxgrodzinsky1998}, ou seja, passivas longas\is{passiva!longa} e verbos de estado mental acarretaram mais dificuldade para crianças de cinco anos de idade (\citealt{limajunior2012}, para o PB; \citealt{estrela2013}, para o PE). Verificou-se, ainda, com base nos dados do PB, que a direcionalidade das relações temáticas implicadas pelo verbo de estado mental também é um fator a ser considerado. Os verbos psicológicos se subdividem em verbos do tipo \isi{experienciador}-\isi{tema} (\textit{A mãe admira o menino}) e \isi{tema}-\isi{experienciador} (\textit{A mãe magoa o menino}). O fato de o segundo tipo de verbos psicológicos indicar claramente que o objeto lógico é afetado ou, ainda, que o sujeito semântico poderia agir intencionalmente poderia vir a facilitar a interpretação das formas passivas desse tipo de verbos. Efetivamente, constatou-se que passivas como em (\ref{ex:correapassiva_23}), em que o \isi{experienciador} é o complemento da preposição, são mais difíceis para crianças de cinco anos do que passivas como (\ref{ex:correapassiva_24}), em que este ocupa a posição de sujeito, de forma atípica \citep{limajunioraugusto2014}. 

\ea\label{ex:correapassiva_23}
O menino foi admirado pela mãe.
\z
\ea\label{ex:correapassiva_24}
O menino foi magoado pela mãe.
\z

\subsection{Comparando tipos de passivas}
\label{subsec:correapassiva_comparando}

Outro fator que recebeu atenção em estudos recentes conduzidos em PE e em PB foi o tipo de auxiliar formador de passivas,\is{passiva} na acepção ampla do termo introduzida na Secção \ref{sec:tipos_passivas}. Passivas \textit{eventivas}, formadas com o auxiliar SER foram contrastadas com \textit{passivas adjetivais} (estativas, com ESTAR, e resultativas, com FICAR) em tarefa de julgamento de valor de verdade, como já exemplificado, ou de julgamento de gramaticalidade. Em tarefas de julgamento de gramaticalidade, o participante deve avaliar se um dado enunciado corresponde a uma sentença bem formada da língua. No caso de crianças, faz-se necessário criar-se um contexto no qual seja possível colocar a forma como enunciados se apresentam como objeto de avaliação. No estudo de \citet{estrela2013}, conduzido em PE, com essa técnica, dois fantoches de animais disputam o melhor uso do português e a criança deve avaliar qual deles produziu o enunciado na melhor forma. Pares de sentenças com combinações de eventivas, resultativas e estativas foram contrastados. Os resultados demonstraram que, aos cinco anos, as crianças não parecem distinguir esses tipos de sentenças em seus julgamentos. As respostas ficam em torno de 50\% para cada tipo de estrutura contrastada. Somente aos seis anos, houve diferença significativa entre os pares de contrastes, exceto entre resultativas (\ref{ex:correapassiva_25}) e eventivas (\ref{ex:correapassiva_26}).

\ea[*]{\label{ex:correapassiva_25}O espelho ficou partido com o martelo.}
\z
\ea[]{\label{ex:correapassiva_26} O espelho foi partido com o martelo.}
\z

O estudo conduzido em PB, com tarefa de julgamento de valor de verdade, como já ilustrado, contrastou passivas estativas\is{passiva} (\ref{ex:correapassiva_27}) e eventivas curtas (\ref{ex:correapassiva_28}). O contexto apresentava um vídeo em que uma personagem sofria uma ação, como ser \textit{amarrada}, sendo que o resultado dessa ação poderia manter-se até o final do vídeo ou não (ela poderia se libertar ou não). A última imagem era objeto de avaliação pela criança a partir do enunciado, contendo uma passiva estativa ou uma eventiva, proferido por um fantoche. A análise dos dados demonstrou que a presença do auxiliar não levou a respostas distintas, o que parece indicar que, para as crianças de cinco anos, falantes de PB, a forma participial é mais saliente, levando à atribuição da leitura adjetival mesmo a passivas eventivas, embora o efeito de distinções aspectuais tenha de ser verificado \citep{limajunior2012}. 

\ea\label{ex:correapassiva_27}
O menino estava/está amarrado.
\z
\ea\label{ex:correapassiva_28}
O menino \textit{foi amarrado}.
\z

Resultados obtidos em outras línguas, como o espanhol\il{espanhol} \citep{pierce1992}, o grego\il{grego} \citep{terziexler2002} e o catalão \citep{chocarro2009} têm padrão semelhante ao que foi reportado até então. Seria a dificuldade na compreensão de passivas indicativa de que a gramática (língua interna) da criança não é capaz de gerá-las?

Ainda que o comportamento médio obtido nos diferentes grupos etários testados indique um progressivo aumento de respostas corretas em função de idade, diferenças individuais indicam que a aquisição de passivas\is{passiva} pode se dar mais precocemente do que vinha sendo considerado. A análise de dados individuais revela que há crianças que já aos três anos não apresentam dificuldades de compreensão mesmo nas condições de maior demanda, como ressaltam \citet{rubin2009} para o PB e \citet{estrela2013} para o PE.

\subsection{Novas evidências de produção}
\label{subsec:correapassiva_novas_evidencias}

Quanto à produção, dados da fala espontânea obtidos em PB sugerem que passivas\is{passiva} estativas são as primeiras a serem produzidas por crianças, por volta do segundo ano de vida \citep{minellolopes2013}. No que concerne às eventivas, passivas curtas\is{passiva!curta} são mais frequentes do que longas em PE \citep{estrela2013} e em PB \citep{perotino1995}. No entanto, em estudos com base em fala espontânea de crianças falantes de sesotho\il{sesotho}\footnote{\il{sesotho}Língua falada no Lesoto e na África do Sul; pertence à família das línguas bantas.} \citep{demuth1990,demuth_etal2010} e de inuktitut\il{inuktitut}\footnote{\il{inuktitut}Língua falada por cerca de 23000 habitantes dos territórios do nordeste do Canadá, pertencente à família esquimo-aleutiana.} \citep{allencrago1996}, constatou-se que passivas longas\is{passiva!longa} são produzidas já aos dois anos de idade. \citet{demuth1990} observa que a aquisição de passivas\is{passiva} em sesotho\il{sesotho} parece estar relacionada com o fato de que, nessa língua, o sujeito dessas estruturas tem de ser o tópico do discurso. Inuktitut,\il{inuktitut} por sua vez, é uma língua polissintética, ou seja, em que palavras são compostas por dois ou mais morfemas, com regras morfológicas complexas. Ainda que do tipo SVO, a informação sintática proveniente da morfologia é preponderante relativamente à ordem dos constituintes. Nesta língua, a estrutura passiva\is{passiva} apresenta marcas morfológicas tanto no \isi{agente} quanto no verbo, por meio de infixo. Dados da produção espontânea de três crianças revelam que as estruturas são usadas de forma produtiva e que incluem passivas\is{passiva} não agentivas (com verbos de percepção e estado mental). Tanto a estrutura da língua quanto a frequência de uso, consideravelmente superior à reportada para o inglês\il{inglês}, são apresentadas como possíveis explicações para a produção precoce constatada. Em suma, a visibilidade da informação sintaticamente relevante, restrições discursivas e frequência de uso seriam fatores que afetam o curso do desenvolvimento na aquisição dessas estruturas. 

A produtividade de passivas\is{passiva} na fala de crianças falantes de \ili{inglês} pode, não obstante, ter sido subestimada. Uma série de estudos recentes tem explorado o efeito de \textit{priming}\is{priming@\textit{priming}} sintático na elicitação da produção de passivas\is{passiva} por crianças \citep{bencinivalian2008,manetti2012,messenger_etal2012}. A noção de \textit{priming}\is{priming@\textit{priming}} está associada a pré-ativação; trata-se de um efeito de memória implícita devido à influência que a exposição a um determinado estímulo exerce sobre a resposta a outro estímulo subsequente. O \textit{prime} (estímulo apresentado previamente à produção induzida) apresenta à criança um tipo de estrutura, que poderá ser por ela também utilizada em situação semelhante. Tal como na técnica tradicional de repetição, a criança tem de formular o novo enunciado com seus próprios recursos. Diferentemente daquela, a proposição a ser enunciada é nova nessa tarefa, o que aumenta sua demanda. Nos testes para elicitação de passivas,\is{passiva} é comum que figuras sejam descritas, ora pelo experimentador, ora pelo participante, em um tipo de jogo de cartas. O uso de passivas\is{passiva} pela criança tende a aumentar, ao longo da atividade, em função do uso dessa estrutura por parte do experimentador. Constatou-se, por meio dessa técnica, que crianças de três e quatro anos de idade falantes de inglês\il{inglês} são capazes de produzir passivas,\is{passiva} independentemente de \isi{reversibilidade} e da agentividade dos verbos \citep{bencinivalian2008,messenger_etal2012}. O estudo de \citet{messenger_etal2012}, em particular, contrasta os resultados de produção elicitada por \textit{priming}\is{priming@\textit{priming}} e os de compreensão em uma tarefa clássica de identificação de imagens. Os autores argumentam que a dificuldade encontrada nos dados de compreensão pode estar relacionada ao tipo de demanda imposta pela tarefa. Adaptando-se a tarefa de \citet{messenger_etal2012}, de modo a tornar o \textit{prime} coincidente com o tópico do discurso, em experimento conduzido com crianças falantes de PB, o efeito de \textit{priming}\is{priming@\textit{priming}} foi particularmente expressivo, comparado com o de estudos anteriores, sugerindo que, uma vez que a produção de passivas\is{passiva} venha a atender a demandas discursivas, o peso do custo da produção pode ser relativizado \citep{limajunior2016}.

\subsection{Quadro síntese da pesquisa em aquisição de passivas}
\label{subsec:correapassiva_quadro}

Apresentam-se, nesta secção, as Tabelas \ref{tab:correapassiva_correa1}, \ref{tab:correapassiva_correa2} e \ref{tab:correapassiva_correa3}, que condensam os principais resultados reportados.
\begin{table}
\begin{tabular}{p{2cm}p{2cm}p{1.5cm}p{1cm}p{5cm}}
\lsptoprule
Fonte                  & Metodologia                                                  & Língua               & Idade         & Principais resultados                                                                                        \\
\midrule
\citet{bever1970}             & Manipulação de brinquedos                                    & inglês\il{inglês}               & 2 a 5 anos    & 2 anos: pouco uso de estratégias (desempenho no nível de chance)                                             \\
                       &                                                              &                      &               & 3 anos: uso de estratégias (eventos prováveis e N-V-N)                                                       \\
                       &                                                              &                      &               & Aumento de acertos com a idade                                                                               \\
\citet{strohnernelson1974} & Manipulação de brinquedos                                    & inglês\il{inglês}               & 2 a 5 anos    & 2 e 3 anos: influência de estratégias (eventos prováveis)                                                    \\
                       &                                                              &                      &               & 5 anos: boas taxas de acerto (70\% eventos pouco prováveis; 90--100\% eventos prováveis)                     \\
\citet{maratsos_etal1985}  & a) Pergunta para identificação do agente                     & inglês\il{inglês}               & 4--5 anos     & a) Verbos de ação: 67\%                                                                                      \\
                       & b) Seleção de imagem                                         &                      &               & b) Verbos de ação: percentuais acima de 85\%                                                                 \\
                       &                                                              &                      &               & Verbos de estado mental: 4 (34\%); 5 (65\%); 7 (62\%); 9 (87\%); 11 anos (99\%)                              \\
\citet{foxgrodzinsky1998}   & Julgamento de Valor Verdade                                  & inglês\il{inglês}               & 3;6--5;5 anos & Passivas com verbos de ação curtas e longas: 100\%; com verbos de estado mental curtas: 86,5\%; longas: 46\% \\
\citet{pierce1992}            & Identifcação da figura                                       & espanhol\il{espanhol}             & 3;7--5;9 anos & 5 anos; percentuais acima de 66,7\% tanto para ordem S-V ou V-S (possível no espanhol\il{espanhol})                       \\
\citet{perotino1995}         & Produção espontânea                                          & Português Brasileiro & 3--5 anos     & Número reduzido de passivas perifrásticas                                                                    \\
\citet{terziexler2002}    & Seleção de imagens                                           & grego\il{grego}                & 3;8--5;10     & 3 anos: passivas com verbos de ação (3\%); adjetivais (83\%); verbos de estado mental (20\%)                 \\
                       &                                                              &                      &               & 4 anos: (33\%); (77\%); (13\%)                                                                               \\
                       &                                                              &                      &               & 5 anos: (44\%), (89\%); (20\%)  \\
\lspbottomrule
\end{tabular}
\caption{Principais resultados reportados na literatura}
\label{tab:correapassiva_correa1}
\end{table}

\begin{table}
\begin{tabular}{p{2cm}p{2cm}p{1.5cm}p{1cm}p{5cm}}
\lsptoprule
Fonte                   & Metodologia                                          & Língua               & Idade     & Principais resultados                                               \\
\midrule
\citet{obrien_etal2006}      & Julgamento de Valor Verdade                          & inglês\il{inglês}               & 3;0 anos  & Verbos de ação: sem ($>$50\%); com ($>$80\%)                        \\
                        &                                                      &                      &           & Verbos de estado mental: sem ($>$60\%); com ($>$90\%)               \\
\citet{bencinivalian2008} & \textit{Priming} sintático em descrição de imagens   & inglês\il{inglês}               & 3;0 anos  & Mais passivas produzidas após \textit{priming} (14\%)               \\
\citet{rubin2009}              & Manipulação de brinquedos e Identificação de imagens & Português Brasileiro & 3;0--4;11 & Passivas de ação longas: resultados na chance                       \\
                        &                                                      &                      &           & Passivas curtas: acima da chance                                    \\
\citet{chocarro2009}           & Identificação de imagens                             & Catalão              & 3;1--5;11 & Passivas curtas: 3;0 anos (55\%) 4;0 (70\%); 5;0 (87\%)             \\
                        &                                                      &                      &           & Passivas longas: 3;0 anos (12\%); 4;0 (36\%); 5;0 (31\%)            \\
\citet{demuth_etal2010}       & Identificação de imagens                             & Sesotho\il{sesotho}              & 3;0 anos  & Passivas de ação: acima da chance                                   \\
                        &                                                      &                      &           & Passivas de estado mental: acima da chance                          \\
\citet{manetti2012}           & Descrição de figuras e \textit{Priming} sintático    & italiano\il{italiano}             & 3;6--4;6  & Não houve produção de passivas na tarefa de descrição de figuras.   \\
                        &                                                      &                      &           & Percentual de 17\% de passivas longas na tarefa com \textit{priming}\\
\lspbottomrule
\end{tabular}
\caption{Principais resultados reportados na literatura (Cont.)}
\label{tab:correapassiva_correa2}
\end{table}


\begin{table}
\begin{tabular}{p{2cm}p{2cm}p{1.5cm}p{1cm}p{5cm}}
\lsptoprule
Fonte                      & Metodologia                                & Língua               & Idade          & Principais resultados                                                                              \\
\midrule
\citet{messenger_etal2012}       & \textit{Priming} sintático                 & inglês\il{inglês}               & 3;4--4;11 anos & Mais passivas produzidas após \textit{priming} para verbos de ação e de estado mental              \\
\citet{minellolopes2013}       & Dados de produção espontânea de 4 crianças & Português Brasileiro & 1;6--5;6       & 49 ocorrências de passivas adjetivais (1ª ocor. -- 1;10,21)                                        \\
                           &                                            &                      &                & 01 ocorrência de passiva agentiva (3;6,28)                                                         \\
\citet{limajunior2012}         & Julgamento de Valor de Verdade             & Português Brasileiro & 5;5 anos       & Passivas curtas: de ação (86\%); de estado mental (66,7\%)                                         \\
                           &                                            &                      &                & Passivas longas: de ação (66,7\%); de estado mental (37,5\%)                                       \\
\citet{estrela2013}               & Julgamento de Valor de Verdade             & Português Europeu    & 3;1--5;11      & Passivas curtas de ação: 3 anos (43\%); 4 anos (86\%), 5 anos (100\%)                              \\
                           &                                            &                      &                & Passivas curtas de estado mental: 3 anos (57\%); 4 anos (57\%); 5 anos (71\%)                      \\
                           &                                            &                      &                & Passivas longas de ação: 4 anos (43\%); 4 anos (86\%); 5 anos (100\%)                              \\
                           &                                            &                      &                & Passivas longas de estado mental: 3 anos (43\%); 4 anos (43\%); 5 anos (57\%)                      \\
\citet{limajunioraugusto2014} & Julgamento de Valor de Verdade             & Português Brasileiro & 5;4 anos       & Passivas longas de estado mental do tipo tema-experienciador com traço de afetado: (79,2\%)       \\
\citet{limajunior2016}           & Identificação de imagem                    & Português Brasileiro & 4--5 anos      & A não manutenção do tópico pelo sujeito da passiva dificulta a compreensão para crianças de 4 anos \\
\citet{limajunior2016}          & \textit{Priming} sintático                 & Português Brasileiro & 3--6 anos      &           Mais passivas produzidas após \textit{priming} para verbos de ação em relação a estudos anteriores \\                                                       \lspbottomrule                                
\end{tabular}
\caption{Principais resultados reportados na literatura (Cont.)}
\label{tab:correapassiva_correa3}
\end{table}

Em suma, passivas\is{passiva} causam dificuldade para crianças em tarefas de compreensão e sua produção é usualmente pouco frequente na fala de crianças, exceto em algumas línguas. No entanto, essa dificuldade não parece decorrer da impossibilidade de essa estrutura ser gerada pela gramática de crianças de tenra idade. 

\section{Algumas propostas teóricas sobre a aquisição de passivas}
\label{sec:correapassiva_algumas_propostas}

Teorias da aquisição da linguagem tendem a ser de dois tipos: procedimental, i.e. em que se busca explicar o modo como a aquisição transcorre, considerando as propriedades dos dados linguísticos (\textit{input linguístico}) e como a criança os processa; estrutural, i.e. em que se busca explicar o desempenho da criança com base no estado de sua gramática (língua interna). Uma explicação de natureza procedimental, de grande influência na década de 70 do último século, foi considerar que o processo de aquisição dar-se-ia de forma estratégica, ou seja, que a criança lidaria com a tarefa de aquisição como uma situação-problema a qual tentaria resolver fazendo uso dos recursos cognitivos a seu dispor. Logo foi observado, contudo, que as estratégias identificadas no comportamento de crianças em tarefas de compreensão se apresentam como um meio para chegarem a uma resposta diante de uma dificuldade, e não como um procedimento de aquisição \citep{cromer1976}. No caso das passivas, por exemplo, não é claro de que modo um procedimento em que o sujeito é imediatamente tomado como \isi{agente} (estratégia NVN) ou uma análise em que a relação semântica mais plausível entre os constituintes é assumida poderiam levar à identificação da informação gramaticalmente relevante acerca da estrutura em aquisição. 

Explicações focadas no estado da gramática da criança ao longo do desenvolvimento vêm sendo apresentadas, particularmente, a partir da década de 80. Para \citet{borerwexler1987}, por exemplo, a dificuldade das crianças com verbos de estado mental, reportada por \citet{maratsos_etal1985}, indicaria que o estado da gramática da criança só lhes permitiria interpretar passivas eventivas\is{passiva} (com verbos de ação) como adjetivais, o que não funcionaria com verbos de estado mental. Uma sentença como (\ref{ex:correapassiva_29}) pode ser interpretada como (\ref{ex:correapassiva_30}). No entanto, a interpretação de (\ref{ex:correapassiva_31}) como (\ref{ex:correapassiva_32}) resulta em anomalia.

\ea\label{ex:correapassiva_29} A porta foi aberta.\z
\ea\label{ex:correapassiva_30} A porta está aberta.\z
\ea\label{ex:correapassiva_31} A conversa foi ouvida.\z
\ea\label{ex:correapassiva_32} A conversa está ouvida. (?!)\z

Segundo esses autores, a gramática da criança não gera sentenças que envolvem o movimento-A,\is{movimento!movimento-A} de modo que a formação de \textit{cadeias argumentais} (relação entre a posição original na estrutura em (\ref{ex:correapassiva_7}) e a posição para a qual o argumento interno é movido) não é possível, sendo o seu desenvolvimento atribuído a maturação.

A \textit{Hipótese do Déficit de Formação de Cadeias-A} de \citeauthor{borerwexler1987}\footnote{Essa hipótese recebeu algumas reformulações posteriores em \citet{babyonyshev_etal2001}, em que se propõe a \textit{External Argument Requirement Hypothesis}. Em \citet{wexler2004} é apresentada a hipótese da \textit{Universal Phase Requirement}, e, ainda, em \citet{hyamssnyder2005}, uma alternativa com a \textit{Universal Freezing Hypothesis}.} não explica, contudo, como argumentam \citet{foxgrodzinsky1998}, a relativa facilidade de crianças na compreensão de passivas curtas\is{passiva!curta} não agentivas e o fato de não terem dificuldade com a formação de cadeias argumentais com verbos inacusativos \citep{friedmanncosta2010}. Sentenças como \textit{O copo quebrou}, por exemplo, em que o argumento interno de \textit{quebrar} ocupa a posição de sujeito sintático, não apresentam problemas para crianças. 

Como alternativa, \citet{foxgrodzinsky1998} formulam a chamada \textit{Hipótese do Déficit da Transmissão do Papel Temático}. Esta hipótese prevê dificuldades na atribuição do papel temático para o complemento da preposição, quanto este não pode ser entendido como \textit{causador} (\isi{agente}). Essa hipótese, contudo, também não dá conta de todos os resultados. Como visto anteriormente, crianças de três anos de idade podem produzir sentenças passivas longas,\is{passiva!longa} mesmo com verbos de estado mental em línguas como inglês\il{inglês}, uma vez elicitadas por \textit{priming}.\is{priming@\textit{priming}}

Dadas essas considerações, uma teoria de aquisição de passivas\is{passiva} terá de caracterizar o estado da língua interna da criança que permite a geração dessa estrutura em tenra idade; explicar como se dá a identificação da informação de ordem morfológica que determina a intepretação do sujeito gramatical\is{sujeito!gramatical} como objeto lógico; prever o custo relativo da compreensão dessas estruturas em variadas condições.

Na versão atual da teoria gerativa (cf. \citealt{chomsky1995,chomsky2005,chomsky2007}), a faculdade de linguagem consiste fundamentalmente de operações universais que possibilitam a combinação de elementos do léxico em estruturas hierárquicas com base em suas propriedades formais (ou gramaticais). Essas propriedades mostram-se legíveis na forma com que enunciados linguísticos se apresentam (na ordenação dos constituintes, na morfologia, na prosódia). Crianças de tenra idade são sensíveis ao que se apresenta de forma sistemática nos enunciados linguísticos, o que facilita o reconhecimento de informação gramatical (cf. \citealt{correa2009,correa2014} e referências ali contidas). Resultados obtidos com bebês de 18 meses adquirindo o português brasileiro sugerem sensibilidade à relação entre o verbo auxiliar (no caso SER) e o particípio do verbo principal, constituindo uma unidade. O tempo de atenção das crianças à audição de histórias em que passivas \is{passiva}são apresentadas é maior do que para histórias em que o morfema de particípio foi substituído por um morfema de aspecto imperfeito (Exemplo: O chão foi molhado vs O chão foi molhava) \citep{limajunior2016}.  Logo, a identificação precoce do padrão com que passivas\is{passiva} se apresentam na língua deve constituir o primeiro passo do processo de aquisição dessas estruturas (ver \citealt{correa_etal2016}).

Prever o custo relativo da compreensão dessas estruturas em diferentes condições também requer uma articulação do estudo da aquisição da linguagem com o estudo do processamento (produção e compreensão) de enunciados linguísticos por adultos (cf. \citealt{correaaugusto2007,correa2014,limajuniorcorrea2015}. Em que medida condições de menor custo, como passivas\is{passiva} irreversíveis (com sujeito inanimado), podem contribuir para a estabilização das propriedades de passivas\is{passiva} na língua em aquisição, tornando efetiva a interpretação semântica do sujeito de estruturas passivas?\is{passiva} Em que medida o efeito de verbos de estado mental é pós-sintático, ou seja, decorrente da interpretação semântica do enunciado e não de dificuldades na análise sintática da sentença? Qual a natureza da dificuldade imposta pela semântica de verbos de estado mental em interação com as propriedades sintáticas de estruturas passivas?\is{passiva} Essas são questões que ainda estão por investigar. 


\section{Considerações finais}
\label{sec:correapassiva_conclusao}

Neste capítulo, alguns dos principais estudos acerca da aquisição de passivas\is{passiva} conduzidos ao longo das últimas cinco décadas foram considerados, juntando-se a estes, resultados recentemente obtidos com dados do PB e do PE. Vimos que vários fatores podem influenciar o desempenho da criança e que uma plena habilidade da compreensão dessas estruturas é tardia. Os resultados obtidos até então apontam para um gradual domínio de habilidades de produção e compreensão em função do custo associado a fatores como tipo de passiva (adjetival,\is{passiva!adjetival} eventiva); tipo de verbo (de ação, de estado menta), presença explícita do sintagma preposicionado (\textit{by-phrase}),\is{by-phrase} assim como \isi{reversibilidade} de papéis temáticos. No entanto, isso não significa que a capacidade de geração dessas estruturas não esteja disponível em tenra idade.

Várias hipóteses vêm sendo apresentadas para dar conta dos resultados experimentais obtidos. Tanto explicações estritamente procedimentais, em forma de estratégias, quanto explicações focadas no estado da gramática da criança em um particular momento mostram-se, contudo, insatisfatórias. É importante explicitar como a criança extrai dos dados da fala a informação gramaticalmente relevante que lhe permite desencadear as operações gramaticais necessárias para a análise e a produção de estruturas passivas.\is{passiva} Em línguas como o português, essa informação parece concentrar-se no complexo \textit{Aux}-\textit{ser}+\textit{Particípio}. A abordagem procedimental para a aquisição de passivas\is{passiva} em \citet{correa_etal2016} explora o modo como a identificação desse complexo pode vir a desencadear a operação sintática que possibilita ao objeto lógico assumir a posição de sujeito da sentença, no caso das passivas.\is{passiva} Diante das questões que estão por ser respondidas, uma teoria de natureza ao mesmo tempo procedimental e estrutural, como a que essa abordagem ilustra, parece ser necessária para irmos adiante. 

\section*{Nota}
Este capítulo foi elaborado durante a vigência do projeto \textit{Processamento e aquisição da linguagem sob ótica minimalista: extensão e comparação de modelos }da primeira autora (Conselho Nacional de Desenvolvimento Científico e Tecnológico - CNPq) e inclui resultados obtidos na dissertação de Mestrado e na tese de Doutorado de João Claudio de Lima Junior (Bolsista FAPERJ-Nota 10), orientadas pela segunda e pela primeira autora, respectivamente.João C.de Lima Junior é atualmente bolsista PDJ (pós-doutorado júnior) CNPq, no LAPAL (Laboratório de Psicolinguística e Aquisição da Linguagem), PUC-Rio.

{\sloppy
\printbibliography[heading=subbibliography,notkeyword=this]
}
\end{document}