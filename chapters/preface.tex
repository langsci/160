\addchap{Prefácio}
% \begin{refsection}

As questões que se relacionam com a aquisição de uma língua materna ou de uma língua não materna são relevantes em diversas áreas, incluindo a Linguística, a Psicologia, a Educação ou a Terapia da Fala. Por isso mesmo, as licenciaturas nestas diferentes áreas incluem frequentemente disciplinas, ou módulos de disciplinas, cujo objeto de estudo é a aquisição e o desenvolvimento linguísticos. Esse mesmo facto tornou evidente, entre quem leciona estas disciplinas, a necessidade de um livro de carácter introdutório sobre estas questões, escrito em português.

Por outro lado, os estudos sobre a aquisição do português, como língua materna ou como língua não materna, têm conhecido um desenvolvimento acelerado nas últimas décadas. Quem leciona unidades curriculares na área da aquisição do desenvolvimento linguístico conta já, portanto, com um conjunto relevante de dados e estudos. Fazia-se, assim, sentir a necessidade de coligir parte relevante deste trabalho num formato de texto de divulgação, acessível a estudantes universitários de várias áreas. 

O livro que agora apresentamos pretende fazer isto mesmo: servir de porta de entrada ao problema da aquisição de uma língua materna ou não materna, tomando para isso o português como exemplo. Sendo crescente o interesse no trabalho (muitas vezes numa perspetiva comparatista) sobre a aquisição de diferentes variedades do português, o volume é constituído por capítulos da autoria de vários especialistas portugueses e brasileiros. Não quer isto dizer que se olhe para o fenómeno da aquisição com um foco redutor, considerando apenas o caso de uma língua. Pelo contrário, assume-se que só compreenderemos o fenómeno da aquisição se considerarmos um conjunto vasto de dados que permitam a comparação dos percursos de aquisição de diversas línguas. Na verdade, a maioria dos autores que colaboram no volume partilha a ideia de que a capacidade da linguagem é inata no ser humano, consistindo a aquisição de uma ou mais línguas em particular na atualização dessa capacidade.\footnote{É essa a perspetiva assumida por Noam Chomsky e apresentada de forma particularmente clara em \citet{chomsky1986}.} Isto é, está subjacente à maioria dos capítulos incorporados no presente volume uma visão inatista do problema da aquisição,\largerpage facto que se reflete de várias formas no texto. Nomeadamente, nos capítulos dedicados ao desenvolvimento fonológico e sintático, será claro que a  perspetiva sobre a linguagem aqui assumida é, de um modo geral, generativista. Aliás, é nessa medida que as organizadoras do volume rejeitam o termo ``aquisição da linguagem'', geralmente usado para designar as disciplinas que tratam estas questões: numa perspetiva inatista, a linguagem, i.e. a capacidade da linguagem, é inata; o que se adquire são línguas particulares.

Apesar da perspetiva teórica adotada pela maioria dos autores, o que se pretende com este volume é, antes de mais, levantar questões de uma forma acessível a um público ainda pouco especializado e apresentar dados que possam alimentar a reflexão e criar interesse no trabalho aprofundado numa ou outra área específica. Não deixámos, pois, de criar espaço para a apresentação, numa perspetiva histórica, de hipóteses que se tornaram clássicas sobre a natureza do fenómeno da aquisição das línguas, nomeadamente, a hipótese behaviorista de Skinner, a hipótese inatista de Chomsky e a hipótese cognitivista de Piaget, sendo ainda referidas as abordagens de Vygotsky e de Bruner. É esse o tema central do capítulo 1, de Inês Sim-Sim. Embora esse facto não seja objeto do capítulo, sabemos que o debate sobre a existência de conhecimento linguístico inato, que opôs Chomsky a Piaget \citep{piattelli-palmarini1980}, continua a alimentar a discussão entre defensores de uma perspetiva generativista e defensores de uma perspetiva \textit{usage-based} \citep{tomasello2003}.

Os capítulos seguintes exploram as áreas mais estudadas no domínio da aquisição e desenvolvimento linguísticos, a fonologia e a sintaxe, fazendo, sempre que possível, pontes com a fonética, a morfologia e a semântica. A aquisição do português como língua materna é explorada em detalhe, sendo também apresentados capítulos sobre aquisição em contexto bilingue ou de língua não materna. Por fim, os cinco capítulos que fecham o volume debruçam-se sobre: (i) a avaliação linguística de crianças com desenvolvimento típico e com desenvolvimento atípico, capítulos produzidos na perspetiva da Linguística Clínica; (ii) a interação entre o conhecimento implícito, o conhecimento metalinguístico e a escrita nos primeiros anos de ensino formal, capítulos desenvolvidos na perspetiva da Linguística Educacional. Passamos a apresentar sumariamente cada um dos capítulos do presente volume que apresentam investigação sobre estruturas específicas da aquisição do português.

O capítulo 2 é da responsabilidade de Cristina Name e de Sónia Frota e retoma estudos recentes na área da perceção em bebés, revendo as questões centrais de investigação nesta área e apresentando os resultados disponíveis até ao momento. Trata-se de uma área de investigação\largerpage recente no caso do português, que permite a exploração de aspetos ligados às interfaces gramaticais, dada a relação estreita entre aspetos fonológicos, em particular prosódicos, e aspetos sintáticos, nos momentos iniciais do percurso de desenvolvimento linguístico infantil. As autoras enquadram a investigação disponível sobre o português do Brasil (PB) e o português europeu (PE) na produção científica internacional, apresentando o estado da arte neste domínio e dando conta da investigação em curso nos dois países.

Os capítulos seguintes centram-se na aquisição da fonologia de língua materna em contexto de desenvolvimento típico. O capítulo 3, de Carmen Matzenauer e de Teresa Costa, dá conta da aquisição das unidades fonológicas mínimas, os segmentos, no PE e no PB, confrontando os resultados com os obtidos para outras línguas do mundo descritas para o efeito. Retomam mais de três décadas de investigação no domínio da aquisição fonológica, centrando-se em aspetos da aquisição fonológica como o ponto de articulação, o modo de articulação e o vozeamento, discutindo os dados com base na perspetiva não-linear da fonologia, assumida também nos capítulos seguintes. As autoras exploram a relevância do conceito de classe natural na descrição dos padrões de desenvolvimento segmental. Observam, ainda, a aquisição de segmentos que são alvo de processos fonológicos do sistema gramatical dos adultos. O capítulo 4 é da responsabilidade de Maria João Freitas, sendo dedicado à unidade prosódica sílaba. É descrita a ordem de aquisição dos vários constituintes internos à sílaba no PE e no PB, sendo estes percursos comparados com os descritos para a aquisição de outras línguas. É dado relevo à interface entre desenvolvimento silábico e aquisição das unidades segmentais, crucial para a avaliação e a intervenção terapêuticas. Alguns padrões que violam princípios de boa formação silábica são discutidos tendo em conta os dados de produção das crianças, no sentido de mostrar que dados da aquisição podem ser usados como forma de refletir sobre a análise das estruturas-alvo. Finalmente, o capítulo 5, da autoria de Raquel Santana Santos, dá continuidade à descrição da aquisição de estruturas prosódicas, centrando-se no acento e na palavra prosódica. Estas categorias, menos estudadas do que o segmento e a sílaba no domínio da aquisição de língua materna e não materna, são cruciais para o desenvolvimento fonológico infantil, sendo de aquisição precoce e estabelecendo interface com outras unidades linguísticas. Uma vez mais, é dada ênfase ao PB e ao PE, embora a discussão dos dados apresentados seja feita numa perspetiva comparada. 

Os capítulos seguintes são dedicados ao desenvolvimento\largerpage sintático. Em primeiro lugar, apresentam-se os principais marcos do desenvolvimento linguístico no período em que emergem as primeiras combinações de palavras. Assim, o capítulo 6, de Letícia Corrêa e Marina Augusto, centra-se na aquisição da estrutura do sintagma nominal, destacando-se questões como a omissão de determinantes em estádios iniciais de aquisição ou a concordância de género e número interna ao sintagma nominal. Esta secção explora ainda a questão da concordância de número e pessoa entre o verbo e o sintagma nominal com função sintática de sujeito. No capítulo 7, de Ana Lúcia Santos e Ruth Lopes, exploram-se as principais características das primeiras combinações de palavras produzidas pelas crianças, sendo dada uma particular atenção à convergência precoce com a gramática alvo no que diz respeito à ordem de palavras, particularmente no que decorre da posição do verbo na frase. Assim, contrasta-se a ordem de palavras das primeiras produções de crianças portuguesas com a ordem de palavras encontrada, por exemplo, nas primeiras combinações de palavras de crianças falantes de alemão.\il{alemão} É ainda apresentada informação sobre dois fenómenos relacionados, característicos de estádios iniciais da produção: infinitivos raiz e frases com sujeito nulo.

O capítulo 8, da autoria de João Costa e Elaine Grolla, debruça-se sobre a aquisição de pronomes. Assim, explora-se a produção de pronomes, destacando-se a questão dos pronomes clíticos, e a compreensão de pronomes (clíticos e fortes). Discute-se ainda o fenómeno de omissão de pronomes nas produções das crianças, em articulação com dados de compreensão da construção de objeto nulo em português.

Os capítulos seguintes centram-se em estruturas que se sabe serem de desenvolvimento menos precoce, nomeadamente, passivas, interrogativas Qu- e estruturas de subordinação. O capítulo 9, de Letícia Corrêa e Marina Augusto, apresenta dados da aquisição de frases passivas, sendo considerados dados do PE e do PB. O capítulo 10, de Maria Lobo e Carla Soares-Jesel, explora a aquisição de interrogativas, relativas e clivadas, sendo apresentados quer dados relativos à produção (espontânea ou em situação experimental), quer dados de compreensão. Finalmente, o capítulo 11, de Ana Lúcia Santos, apresenta alguns dados sobre a aquisição de estruturas completivas, baseados quer na análise de discurso espontâneo quer em recolhas experimentais. É tratada a aquisição de completivas infinitivas (incluindo as de infinitivo flexionado, disponíveis em português) e de completivas finitas (tratando-se a questão da aquisição de contrastes de modo). 

O capítulo 12, da autoria de Letícia Almeida e Cristina Flores, explora a aquisição em situações de bilinguismo, área de estudos cada vez mais relevante, dados os movimentos migratórios na sociedade contemporânea. São discutidas diferentes situações de bilinguismo (simultâneo, sucessivo), sendo ainda considerada a relação entre os diferentes sistemas linguísticos no que diz respeito à sua representação mental. No final do capítulo, o caso dos falantes de herança é tratado como um caso particular de bilinguismo. Já o capítulo 13, de Ana Madeira, discute as questões\largerpage específicas que se levantam à aquisição de uma língua como língua não materna, quer em idade adulta quer na infância. 

Os capítulos seguintes centram-se nos instrumentos disponíveis para avaliação do desenvolvimento linguístico e que permitem identificar casos de desenvolvimento atípico. Assim, o capítulo 14, de Fernanda L. Viana, Carla Silva, Iolanda Ribeiro e Irene Cadime, faz um levantamento de diferentes métodos de avaliação do desenvolvimento linguístico e, particularmente, dos instrumentos de avaliação linguística estandardizados para o português europeu. Nos capítulos seguintes, é salientada a importância do trabalho interdisciplinar entre terapeutas da fala e linguistas, no sentido de tornar a avaliação cada vez mais rigorosa e a intervenção cada vez mais eficaz. O capítulo 15, da autoria de Marisa Lousada, Dina Alves e Maria João Freitas, trata da avaliação dos aspetos fonéticos e fonológicos em contexto clínico, nem sempre adequadamente identificados como sendo de naturezas distintas nos materiais disponíveis e recrutados na prática clínica. O foco central da secção é a fonologia, sendo feita uma reflexão sobre os contributos da perspetiva da fonologia não-linear para o aperfeiçoamento da prática clínica e refletindo-se sobre variáveis linguísticas a ter em consideração na construção de instrumentos de avaliação fonológica e na planificação da intervenção. O capítulo 16, de Alexandrina Martins e Sónia Vieira, desenvolve ainda a questão da avaliação linguística, centrando-se agora em aspetos sintáticos do desenvolvimento. São considerados resultados obtidos em estudos sobre Perturbações Específicas da Linguagem, Síndrome de Down, Síndrome de Williams e, ainda, Perturbações do Espetro do Autismo. 

Por fim, os dois últimos capítulos exploram potenciais correlações entre conhecimento implícito, consciência linguística e escrita. O capítulo 17, de Ana Luísa Costa, Armanda Costa e Anabela Gonçalves, centra-se no desenvolvimento da consciência sintática. Explora-se particularmente a relação entre conhecimento sintático explícito e escrita, sendo apresentados exemplos de estudos que apontam para relações de interdependência entre conhecimento sintático específico e sucesso na escrita de diferentes tipos de texto. No capítulo 18, Ana Ruth Miranda e João Veloso mostram de que modo os dados da escrita e da consciência fonológica podem ser usados como forma de aceder ao conhecimento fonológico implícito e de refletir sobre a natureza das representações fonológicas. Tratam, ainda, a questão das relações entre conhecimento metafonológico e literacia. Em ambos os casos, a discussão, embora focada nos resultados disponíveis para o PB e para o PE, retoma questões de investigação clássicas nestes domínios, mostrando de que forma estudos que fomentam cruzamentos entre dados da aquisição e desenvolvimento linguísticos, da consciência fonológica e da escrita podem contribuir para o progresso no conhecimento sobre o processamento linguístico nos primeiros anos de percurso académico infantil.\\

\hfill Maria João Freitas

\hfill Ana Lúcia Santos




% 
% \printbibliography[heading=subbibliography]
% \end{refsection}