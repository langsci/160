\documentclass[output=paper]{LSP/langsci} 
\author{Ana Madeira\affiliation{Centro de Linguística da Universidade Nova de Lisboa \& Faculdade de Ciências Sociais e Humanas da Universidade Nova de Lisboa}
}
\title{Aquisição de língua não materna}  
\abstract{}
\ChapterDOI{10.5281/zenodo.889441}
\maketitle
\begin{document}
\section{Introduction} 
\label{sec:madeira_intro}
Este capítulo pretende descrever alguns dos modelos teóricos, conceitos e questões de investigação mais relevantes na área de Aquisição de Segunda Língua (ASL).
Começamos por definir os conceitos de \textit{língua segunda},\is{língua segunda} \textit{língua estrangeira}\is{língua estrangeira} e \textit{língua não materna}\is{língua não materna} (ver Secção \ref{sec:madeira_conceitos}) e descrever as principais características que aproximam e distinguem a aquisição de \isi{língua não materna} da aquisição de língua materna (ver Secção \ref{sec:madeira_aquisicao}). Discutimos depois algumas destas características mais detalhadamente: o efeito da idade de início de exposição à língua (ver Secção \ref{sec:madeira_periodo_critico}), o papel dos fatores individuais (ver Secção \ref{sec:madeira_fatoreS_individuais}) e a influência do conhecimento linguístico prévio (ver Secção \ref{sec:madeira_papel_conhecimento}). Na Secção \ref{sec:madeira_desenvol_conhecimento}, descrevem-se alguns aspetos do desenvolvimento do conhecimento linguístico na \isi{língua não materna} e, finalmente, na Secção \ref{sec:madeira_conclusao}, apresenta-se uma breve síntese.


\section{Os conceitos de \textit{língua segunda}, \textit{língua estrangeira} e \textit{língua não materna}}
\label{sec:madeira_conceitos}

No domínio da didática das línguas, estabelece-se frequentemente uma distinção entre os conceitos de \textit{língua segunda}\is{língua segunda|seealso{L2}} e \textit{língua estrangeira}\is{língua estrangeira}. Esta distinção assenta, sobretudo, nas diferenças entre os contextos que estão tipicamente associados a cada uma das situações de aprendizagem. Considera-se, geralmente, que ``o termo \textit{LS} [língua segunda]\is{língua segunda|seealso{L2}} deve ser aplicado para classificar a \textit{aprendizagem e o uso} de uma língua não-nativa dentro de fronteiras territoriais em que ela tem uma função reconhecida; enquanto que o termo \textit{LE} [\isi{língua estrangeira}] deve ser usado para classificar a aprendizagem e o uso em espaços onde essa língua não tem qualquer estatuto sociopolítico'' \citep[1]{leiria2004}. Por outras palavras, quando falamos em \textit{língua segunda},\is{língua segunda|seealso{L2}} estamos a referir-nos a um contexto de aprendizagem em que o falante não-nativo se encontra no seio de uma comunidade em que a língua é utilizada num grande número de situações de comunicação, tendo o falante, assim, oportunidade para participar em interações comunicativas quer com falantes nativos da língua quer com outros falantes não-nativos. No caso da \textit{língua estrangeira},\is{língua estrangeira} por seu lado, o aprendente encontra-se num contexto em que a exposição à língua ocorre sobretudo em situações de aprendizagem formal, nas quais os conteúdos linguísticos lhe são apresentados sequencialmente e de forma estruturada. Deste modo, os dois contextos caracterizam-se por diferenças significativas quer na quantidade e qualidade de estímulos linguísticos quer nas oportunidades de participação em interações comunicativas de que o aprendente dispõe.

Não é evidente, contudo, que o contexto influencie o processo de aquisição da língua de modo significativo. De facto, muitos estudos têm procurado demonstrar que não existem evidências convincentes de efeitos do contexto de aquisição/aprendizagem quer nas sequências de desenvolvimento (e.g. \citealt{pica1983}) quer no nível de competência final (e.g. \citealt{long1983}). Por esta razão, é prática comum utilizar o termo \textit{língua não materna}\is{língua não materna} (\isi{L2}) para designar qualquer língua que é adquirida/aprendida depois da língua materna (\isi{L1}), independentemente do contexto.

\section{Aquisição de L2 e aquisição de L1: algumas diferenças e semelhanças}
\label{sec:madeira_aquisicao}

A aquisição de \isi{L2} apresenta diversas características que a distinguem da aquisição de \isi{L1}, entre as quais se podem destacar as seguintes: 

\begin{enumerate}
\item Enquanto o processo de aquisição da \isi{L1} começa nos primeiros meses de vida, a \isi{L2} é adquirida mais tarde. A primeira exposição à língua, para que se considere um caso de aquisição de uma \isi{L2}, nunca ocorre antes dos 4 anos de idade \citep{schwartz2004} e, frequentemente, ocorre apenas na adolescência ou mesmo em idade adulta, fora do chamado \textit{período crítico}\is{período crítico} para a aquisição da linguagem \citep{lenneberg1967} (ver Secção \ref{sec:madeira_periodo_critico}).

\item  O processo de aquisição da \isi{L2} é influenciado por \textit{fatores} ou \textit{diferenças individuais} \citep{domyeiskehan2003}, entre os quais se incluem a aptidão linguística, a motivação, os estilos cognitivos, as estratégias de aprendizagem, a personalidade, as atitudes, etc. (ver Secção \ref{sec:madeira_fatoreS_individuais}).

\item A aquisição de uma \isi{L2} é caracterizada por efeitos de influência tanto da \isi{L1} dos aprendentes \citep{odlin2005} como de outras \isi{L2} que estes tenham adquirido anteriormente \citep{rothman_etal2013} (ver Secção \ref{sec:madeira_papel_conhecimento}).

\item A aquisição da \isi{L2} caracteriza-se por muita variabilidade, que é visível quer quando se comparam aprendentes (os quais, expostos a idênticas condições de aquisição/aprendizagem, podem diferir no desenvolvimento do seu conhecimento gramatical e atingem frequentemente níveis finais de proficiência distintos), quer quando se comparam as produções linguísticas de um único aprendente num determinado estádio de desenvolvimento.

\item Muitos aprendentes de \isi{L2} nunca atingem um nível de competência (quase-) nativo (e.g. \citealt{hylternstamabrahamsson2003}), mesmo após uma exposição prolongada à língua-alvo, ocorrendo frequentemente fenómenos de \isi{fossilização} \citep{selinker1972}, isto é, de estabilização do conhecimento linguístico em fases precoces de desenvolvimento.

\item   Ao contrário do que se verifica com a \isi{L1}, muitos autores defendem que o ensino formal e as correções desempenham um papel fundamental na aquisição da \isi{L2}, embora o efeito de diferentes tipos de estímulos no desenvolvimento do conhecimento linguístico (por exemplo, a exposição passiva à língua, a participação em interações comunicativas, a exposição a dados linguísticos estruturados e a explicações gramaticais, a correção de erros, as reformulações, etc.) continue a ser objeto de debate (veja-se, por exemplo, \citealt{long1996}). 

\end{enumerate}

Estas diferenças entre a \isi{L1} e a \isi{L2} levaram muitos investigadores a concluir que se trata de processos de natureza diferente, que culminam na construção de tipos distintos de conhecimento linguístico: no caso da \isi{L1}, estamos perante um processo natural, através do qual as crianças constroem, a partir dos estímulos linguísticos a que estão expostas, um sistema de conhecimento implícito das propriedades abstratas da gramática; no caso da \isi{L2}, estamos na presença de um processo ativo de aprendizagem, que resulta na construção de representações gramaticais explícitas e conscientes. Em síntese, alguns autores defendem que os dois processos são fundamentalmente diferentes (veja-se, por exemplo, a \textit{Hipótese da Diferença Fundamental}, de \citealt{bley-vroman1989}).

No entanto, os processos de aquisição de uma \isi{L1} e a de uma \isi{L2} apresentam também algumas características em comum. Assim, características como as que abaixo se enunciam indicam que quer a aprendizagem quer a aquisição desempenham um papel na construção do conhecimento da \isi{L2}:

\begin{enumerate}
\item Os falantes não-nativos exibem um comportamento linguístico criativo, na medida em que, tal como os falantes nativos, têm a capacidade de produzir e compreender formas e estruturas que nunca ouviram antes.

\item Ainda que alguns dos erros que se verificam nas produções de falantes não-nativos possam ser atribuídos à influência de conhecimento linguístico prévio, em particular, da sua \isi{L1}, um número significativo de erros é comum a diferentes falantes não-nativos (que se distinguem não só pela sua \isi{L1}, mas também pela idade, contexto de aquisição/aprendizagem, etc.), exibindo a sistematicidade que caracteriza os erros produzidos pelas crianças ao longo do processo de aquisição e desenvolvimento da sua \isi{L1}.

\item Observa-se um desenvolvimento sequencial na \isi{L2}, à semelhança do que acontece com a \isi{L1}, pelo menos no que diz respeito a certas propriedades gramaticais, verificando-se percursos de desenvolvimento comuns a todos os aprendentes, independentemente da sua \isi{L1}, idade, contexto de aquisição/aprendizagem, etc.

\end{enumerate}

Estas características comuns têm levado muitos investigadores a propor que a \isi{L2} também envolve pelo menos alguns aspetos de aquisição. 
\section{Idade e efeitos de período crítico}
\label{sec:madeira_periodo_critico}
\is{período crítico}
Vimos na Secção \ref{sec:madeira_aquisicao} que, ao contrário do processo de aquisição da \isi{L1}, que ocorre naturalmente, por mera exposição à língua da família e/ou da comunidade em que a criança está inserida, resultando num nível de competência nativo, a aquisição da \isi{L2} é um processo mais lento, que poderá requerer maior esforço por parte do aprendente, sendo facilitada pela aprendizagem explícita, e que raramente resulta no desenvolvimento de um nível de competência (quase-)nativo. Um dos fatores que contribuem para estas diferenças entre os dois processos é a idade de início de exposição regular à língua. Estas diferenças indicam que as crianças possuem uma capacidade natural para a aquisição da linguagem, que os adultos já não possuem – por outras palavras, à semelhança do que acontece com outras capacidades biológicas, como a visão, por exemplo, existe um \isi{período crítico} para a aquisição de línguas, ou seja, um período durante o qual os mecanismos naturais que são usados na aquisição estão ativos, permitindo que esta ocorra por mera exposição à língua \citep{lenneberg1967}.\footnote{Este fenómeno, que afeta especificamente a faculdade cognitiva da linguagem, não deve ser confundido com o declínio das capacidades cognitivas gerais, que ocorre naturalmente com a idade e que se reflete no declínio progressivo da capacidade de aprender línguas. É importante salientar também que a hipótese do \isi{período crítico} é relevante para a aquisição, ou seja, para o desenvolvimento linguístico que ocorre quando o indivíduo está exposto naturalmente a uma língua, e não para a aprendizagem formal da língua \citep{krashen1981}.}

Como foi referido na Secção \ref{sec:madeira_aquisicao}, de acordo com \citet{schwartz2004}, podemos falar em \textit{aquisição de L2}\is{L2} quando a primeira exposição à língua ocorre após os 4 anos de idade. No entanto, no caso de crianças que iniciam a sua exposição à língua antes dos 8 anos de idade \citep[3]{haznedargavruseva2008}, tem sido defendido que, pelo menos em certos domínios gramaticais, o processo de desenvolvimento apresenta características em comum com a aquisição de \isi{L1}, por um lado, e com a aquisição de \isi{L2} por adultos, por outro \citep{schwartz2004}. Muitos estudos têm demonstrado que existe um limite de idade para o desenvolvimento de competência nativa na \isi{L2} e que as probabilidades de se atingir um nível de proficiência nativo vão diminuindo com a idade \citep{hylternstamabrahamsson2003}. Contudo, poderão existir limites diferentes para diferentes aspetos da competência linguística. Sabe-se, por exemplo, que a capacidade de desenvolver um nível de competência nativo (ou melhor, quase-nativo) se perde muito mais cedo no domínio da \isi{fonologia} – alguns autores referem os 6 anos, embora o número de estudos sobre \isi{L2} na infância seja ainda reduzido – do que no domínio da sintaxe – por volta dos 15 anos, de acordo com \citet{patkowski1980}.

Assim, a idade determina a existência de diferenças importantes não só entre a aquisição de \isi{L2} por adultos e a aquisição de \isi{L1}, mas também na própria aquisição de \isi{L2}, entre adultos e crianças. 

\section{Fatores individuais na aquisição de L2}
\label{sec:madeira_fatoreS_individuais}
\is{L2}
Para além da idade, outros fatores extralinguísticos têm sido identificados como sendo relevantes na aquisição de \isi{L2} \citep{domyeiskehan2003}. Entre os fatores individuais mais relevantes incluem-se os seguintes:

\begin{enumerate}
\item a aptidão para a aprendizagem de línguas estrangeiras,\is{língua estrangeira} que é determinada por um conjunto de características cognitivas, como capacidades de memória e estilos de aprendizagem, que tornam o indivíduo um bom aprendente de línguas; de acordo com \citet[38]{skehan1989}, ``a aptidão constitui o melhor indicador de sucesso na aprendizagem de línguas”;\footnote{No original: ``aptitude is consistently the best predictor of language learning success.''}

\item a motivação, que está estreitamente relacionada com as razões que levam um indivíduo a aprender uma \isi{L2} e é considerada um dos fatores mais determinantes para o sucesso na aquisição/aprendizagem da língua, por influenciar a quantidade de tempo e de esforço que um aprendente está disposto a investir no processo de aprendizagem;

\item os estilos cognitivos (global/analítico; visual/auditivo; etc.), que estão relacionados com o tipo de perspetiva que os indivíduos adotam na resolução de problemas e determinam as suas preferências face ao processo de aquisição/aprendizagem da língua, definindo o modo como recolhem, processam e memorizam a informação;

\item as estratégias de aprendizagem de línguas, ou seja, as estratégias metacognitivas, cognitivas, sociais e afetivas que cada aprendente desenvolve para obter, processar e memorizar informação linguística de modo mais eficaz; 

\item os estilos de personalidade (e.g. introvertido/extrovertido); 

\item as atitudes mais ou menos positivas que o falante não-nativo apresenta em relação à língua-alvo, à cultura que lhe está associada e aos seus falantes.
\end{enumerate}

Estes fatores interagem, contribuindo para as diferenças qualitativas que se observam entre falantes não-nativos quanto ao modo como se desenvolve o conhecimento de certas propriedades linguísticas, e influenciando quer o ritmo de desenvolvimento quer as probabilidades de sucesso na aquisição.

\section{O papel do conhecimento linguístico prévio}
\label{sec:madeira_papel_conhecimento}

Observámos que as gramáticas dos falantes não-nativos tendem a divergir das gramáticas dos falantes nativos da língua-alvo, mesmo em estádios muito avançados de aquisição. Estas divergências são frequentemente atribuídas, total ou parcialmente, à influência do conhecimento linguístico prévio. Nesta secção, consideramos diferentes posições que têm sido defendidas, desde os anos 50 do séc. XX até ao presente, quanto ao papel que o conhecimento de outras línguas – e, em particular, da \isi{L1} – desempenha na aquisição de uma \isi{L2}.

\subsection{A Análise Contrastiva}
\label{subsec:madeira_analise_cont}
\is{análise contrastiva}
De acordo com o modelo da Análise Contrastiva,\is{análise contrastiva} que dominou os estudos de ASL durante as décadas de 50 e 60 do séc. XX, a \isi{L1} dos falantes não-nativos influencia significativamente a aprendizagem de uma \isi{L2}. Este modelo está associado ao comportamentalismo\is{comportamentalismo|see {aquisição da linguagem!perspetiva behaviorista}}, uma teoria do domínio da psicologia que defende que os comportamentos se desenvolvem através de um processo de aprendizagem, que corresponde a um processo de formação de hábitos \citep{skinner57}. Tal como acontece com qualquer outro comportamento, também a aprendizagem de uma língua envolve a formação de hábitos linguísticos. No caso das línguas estrangeiras,\is{língua estrangeira} considera-se que a aprendizagem ocorre por um processo de imitação e prática repetida das estruturas linguísticas \citep{skinner57}. Uma vez que o principal obstáculo à aprendizagem provém da interferência do conhecimento prévio, na aprendizagem da \isi{L2}, a interferência provém sobretudo da \isi{L1}. Assim, o grau de dificuldade da aprendizagem é determinado pelo esforço requerido para aprender uma forma da língua-alvo e depende da semelhança ou diferença que existe entre as formas da \isi{L1} e as da língua-alvo: as formas que são idênticas nas duas línguas são fáceis de aprender (ocorrendo, neste caso, a \textit{transferência}\is{transferência}, ou \isi{influência positiva}, da \isi{L1}), enquanto as formas diferentes são difíceis de aprender (observando-se, então, efeitos de \textit{interferência}, ou \isi{influência negativa}, da \isi{L1}). As dificuldades dos aprendentes manifestam-se através dos erros que eles produzem, que refletem os seus ``maus'' hábitos linguísticos. Esta ideia corresponde à chamada \textit{Hipótese da Análise Contrastiva},\is{análise contrastiva} segundo a qual ``o aluno que entra em contacto com uma \isi{língua estrangeira} achará alguns aspetos dessa língua bastante fáceis e outros extremamente difíceis. Os elementos que são semelhantes aos da sua língua materna serão simples para ele, e aqueles que são diferentes serão difíceis'' \citep[23]{lado1957}.\footnote{No original: ``[\ldots] the student who comes into contact with a foreign language will find some features of it quite easy and others extremely difficult. These elements that are similar to his native language will be simple for him, and those that are different will be difficult.''} É, pois, possível prever todos os erros na \isi{L2} a partir da identificação das diferenças entre a \isi{L1} dos aprendentes e a língua-alvo.

Contudo, verifica-se, por um lado, que muitos dos erros preditos pela \textit{Hipótese da Análise Contrastiva}\is{análise contrastiva} não ocorrem, de facto, e, por outro lado, que se observam erros nas produções dos falantes não-nativos que não são preditos por esta hipótese. Estas duas situações são claramente ilustradas pelas tendências que se observam na aquisição dos padrões de colocação dos pronomes átonos em português \isi{L2} por falantes nativos de línguas como o espanhol\il{espanhol} \citep{madeiraxavier2009}.

Em português, os pronomes pessoais átonos (também chamados \textit{pronomes clíticos}\is{pronome!clítico}) ocorrem obrigatoriamente associados a uma forma verbal. Aparecem em posição enclítica (ou seja, pós-verbal) (cf. \ref{ex:madeira_1a}), exceto na presença de certos constituintes, como, por exemplo, a negação – nestes casos, o pronome é proclítico (isto é, pré-verbal) (cf. \ref{ex:madeira_1b}). Em espanhol,\il{espanhol} pelo contrário, o pronome clítico\is{pronome!clítico} precede sempre o verbo finito (cf. \ref{ex:madeira_2}).

\ea\label{ex:madeira_1}
\ea\label{ex:madeira_1a} O João comeu-\textbf{o}.
\ex\label{ex:madeira_1b} O João não \textbf{o} comeu.
\zl

\ea\label{ex:madeira_2}
\ea\label{ex:madeira_2a} Juan \textbf{lo} comió.
\ex\label{ex:madeira_2b} Juan no \textbf{lo} comió.
\zl

Assumindo a \textit{Hipótese da Análise Contrastiva},\is{análise contrastiva} podemos fazer duas predições relativamente à aquisição dos padrões de colocação dos clíticos\is{pronome!clítico} em português \isi{L2} por falantes nativos de espanhol:\il{espanhol} por um lado, prediz-se que, por interferência do espanhol,\il{espanhol} estes aprendentes produzam próclise, mesmo quando não ocorre na frase qualquer elemento proclisador, como em (\ref{ex:madeira_3a}); por outro lado, não se espera que produzam frases agramaticais como (\ref{ex:madeira_3b}), uma vez que elas também não são possíveis na sua \isi{L1} – espera-se, sim, que, por transferência do espanhol,\il{espanhol} produzam estruturas-alvo como a ilustrada em (\ref{ex:madeira_1b}) acima.

\ea\label{ex:madeira_3}
\ea[]{\label{ex:madeira_3a} O João \textbf{o} comeu.}
\ex[*]{\label{ex:madeira_3b} O João não comeu-\textbf{o}.}
\zl

Na verdade, nenhuma destas predições é cumprida: o que se verifica é uma generalização da ordem verbo-clítico a todos os contextos, idêntica à que se observa na aquisição de português \isi{L1} \citep{costa_etal2015}. Assim, nos estádios iniciais de aquisição da língua, os falantes nativos de espanhol\il{espanhol} tendem a produzir frases-alvo como \textit{O João comeu-o} (cf. (\ref{ex:madeira_1b}) acima) – ou seja, a \textit{Hipótese da Análise Contrastiva}\is{análise contrastiva} prediz que os falantes não-nativos vão produzir um erro que, afinal, não produzem; por outro lado, aquilo que se observa, nos níveis iniciais, é uma tendência para produzir estruturas como a ilustrada em (\ref{ex:madeira_3b} (\textit{O João não comeu-o}) – ou seja, verifica-se um erro que, de acordo com a \textit{Hipótese da Análise Contrastiva},\is{análise contrastiva} não deveria ocorrer.

Na verdade, estes padrões não são exclusivos de falantes nativos de espanhol,\il{espanhol} observando-se, de modo generalizado, em falantes de outras \isi{L1}. Este facto indicia que há, pelo menos, alguns aspetos da aquisição da \isi{L2} que não são determinados pela \isi{L1} dos aprendentes. 

\subsection{O conceito de \textit{interlíngua}}
\label{subsec:madeira_interlingua}
\is{interlíngua}
A ideia de que nem todos os desvios que se observam nas produções de falantes nativos podem ser atribuídos à influência da \isi{L1} é desenvolvida no âmbito da \textit{Análise de Erros}, um modelo introduzido por \citet{corder1967}. Considera-se que existem três causas principais para os erros \citep{richards1971}: (i) a \isi{L1} (erros de interferência ou interlinguísticos); (ii) a \isi{L2} (erros intralinguísticos, que resultam das próprias características gramaticais da língua-alvo – por exemplo, erros de colocação de pronomes clíticos\is{pronome!clítico} como os que observámos acima, que não ocorrem, geralmente, na aquisição de outras línguas românicas e parece deverem-se aos padrões particulares de colocação dos pronomes clíticos\is{pronome!clítico} em português); e (iii) o percurso natural de desenvolvimento linguístico (erros de desenvolvimento, como é o caso das \isi{regularizações} que os falantes não-nativos fazem em determinadas etapas da aquisição – por exemplo, a produção de formas como \textit{cãos}, em vez de \textit{cães}, e \textit{ouvo}, em vez de \textit{ouço}).

Uma das conclusões principais da Análise de Erros\footnote{Outra contribuição importante da \textit{Análise de Erros} é de natureza metodológica. Ao contrário da \textit{Análise Contrastiva},\is{análise contrastiva} a \textit{Análise de Erros} desenvolve procedimentos metodológicos cuidadosos para a análise de erros nas produções dos falantes não-nativos \citep{corder1974}. Tipicamente, uma análise de erros apresenta os seguintes passos: recolha de dados (de produção espontânea ou induzida); identificação dos erros; descrição dos erros (classificação e quantificação); explicação dos erros.} é a de que a maioria dos erros produzidos pelos falantes não-nativos não são erros de interferência, mas sim erros intralinguísticos e erros de desenvolvimento. Em oposição à \textit{Análise Contrastiva},\is{análise contrastiva} este modelo assume uma perspetiva positiva dos erros, considerando que estes são importantes porque refletem o conhecimento linguístico dos falantes não-nativos, dando pistas sobre os modos sistemáticos como esse conhecimento se desenvolve. Partindo da hipótese de que, tal como a aquisição de \isi{L1}, a aquisição de \isi{L2} é um processo determinado por princípios regulares, \citet{selinker1972} introduz o conceito de \textit{interlíngua}\is{interlíngua} para designar os sistemas gramaticais de transição que os falantes não-nativos constroem no decurso do processo de desenvolvimento da competência linguística na \isi{L2}.

A hipótese de que muitas das características das produções de falantes não-nativos resultam de percursos naturais de aquisição (\textit{Hipótese da Ordem Natural}, \citealt{krashen1981}) foi confirmada através de um conjunto de estudos realizados na década de 70 do séc. XX (conhecidos como \textit{Estudos de Ordens de Morfemas}), que investigaram ordens naturais na aquisição de morfemas\is{morfema} gramaticais. Estes estudos são importantes por serem os primeiros trabalhos de investigação que, assentando numa base empírica forte, mostram muito claramente que existe uma ordem de aquisição mais ou menos fixa e que, portanto, pelo menos certos aspetos do processo de aquisição de \isi{L2} são regulares e sistemáticos. Por exemplo, com base em dados de produção induzida (obtidos através de uma técnica de conversa estruturada, baseada em imagens, conhecida como \textit{Bilingual Syntax Measure}), \citet{dulayburt1973} comparam três grupos de crianças (6--8 anos), falantes nativas de espanhol,\il{espanhol} que tinham iniciado a aquisição/aprendizagem do inglês\il{inglês} em idades diferentes e que tinham diferentes graus de exposição à língua na altura do estudo. Os resultados revelam ordens de aquisição idênticas nos três grupos de crianças, que coincidem parcialmente com as observadas na aquisição de \isi{L1}. Estes resultados são corroborados por diversos outros estudos que se realizaram ao longo dos anos 70, quer com crianças com outras \isi{L1} quer com adultos, confirmando-se que, independentemente da idade de início de aquisição/aprendizagem, da \isi{L1}, do contexto e do grau de exposição à \isi{L2}, as ordens de aquisição dos morfemas\is{morfema} gramaticais são idênticas para todos os falantes não-nativos. 

Para além dos estudos sobre ordens de morfemas,\is{morfema} diversos trabalhos sobre construções sintáticas\is{sintaxe} vieram evidenciar a sistematicidade do processo de aquisição de \isi{L2}. Estes estudos incidiram sobre diferentes tipos de fenómenos sintáticos,\is{sintaxe} como é o caso de frases negativas (e.g. \citealt{wode1978}), mostrando que existem sequências de desenvolvimento fixas na aquisição destas estruturas.

Em suma, os resultados dos estudos sobre ordens e sequências de aquisição indicam que, tal como na aquisição de \isi{L1}, o desenvolvimento gramatical na aquisição de \isi{L2} se caracteriza pela existência de sequências de desenvolvimento sistemáticas, confirmando a hipótese de que se trata de um processo criativo de construção de uma gramática mental, que recorre a mecanismos cognitivos universais, os quais determinam o modo como os falantes organizam e interpretam os dados linguísticos. Assiste-se, pois, a uma desvalorização, ou mesmo negação, do papel da \isi{L1} na aquisição da \isi{L2}. 

Porém, esta posição extrema, que não toma em conta as diferenças evidentes entre \isi{L1} e \isi{L2}, levanta muitas questões. Existe demasiada evidência empírica de efeitos de influência da \isi{L1} na \isi{L2} para que uma posição extremada seja neste caso sustentável. Assim, durante os anos 80, reaparece o consenso de que a \isi{L1} desempenha um papel na aquisição, reconhecendo-se que a influência do conhecimento linguístico prévio se pode manifestar de diferentes formas – não apenas através da \isi{transferência}, mas também, por exemplo, através de estratégias de ``evitamento'' de determinadas estruturas ou formas, ou a nível de efeitos no ritmo ou no percurso de aquisição. A investigação passa, portanto, a incidir sobretudo sobre a natureza do processo de influência da \isi{L1}, sobre os modos como os falantes não-nativos usam a sua \isi{L1} na aquisição de uma \isi{L2}, sobre as condições em que tal acontece e sobre os fatores que permitem explicar este fenómeno.

\subsection{O modelo generativo}
\label{subsec:madeira_modelo_generativo}

Um dos modelos teóricos que está na base de muita da investigação realizada em ASL nas últimas décadas resulta da aplicação de princípios da \isi{linguística generativa} à aquisição de \isi{L2}. Este modelo procura integrar os papéis desempenhados pelos mecanismos cognitivos, pelos estímulos linguísticos e pela \isi{L1} para explicar o processo de aquisição da \isi{L2} e as propriedades das gramáticas de \isi{interlíngua}. Segundo a perspetiva generativa, a aquisição da \isi{L1} é determinada por mecanismos mentais específicos para a linguagem (a que se dá o nome de \textit{faculdade da linguagem})\is{faculdade da linguagem}, assumindo-se que as crianças constroem uma gramática mental da sua \isi{L1} sem necessidade de instrução e de correções, por mera exposição aos chamados \textit{dados linguísticos primários}, que correspondem aos estímulos linguísticos em contexto. Uma das questões que se colocam, quando investigamos a aquisição do conhecimento gramatical, é se a aquisição de \isi{L2} é ou não orientada pelos mesmos princípios que a aquisição da \isi{L1}. 

De acordo com um conjunto de hipóteses, os mecanismos que orientam a aquisição da \isi{L1} permanecem ativos na aquisição de \isi{L2}, desempenhando um papel idêntico nos dois processos. A diferença entre eles prende-se com a influência do conhecimento linguístico prévio, que, de acordo com alguns autores, é particularmente evidente nos estádios iniciais de aquisição (porém, de acordo com outros autores que defendem que a \isi{L1} não desempenha nenhum papel na aquisição da \isi{L2}, não existe qualquer diferença entre os dois processos – veja-se, em particular, \citealt{epstein_etal1996}). Esta hipótese é conhecida na literatura como a \textit{Hipótese da Transferência Plena/Acesso Pleno}\is{acesso pleno} \citep{schwartzsprouse1996}. A Hipótese da Transferência Plena\is{transferência}/Acesso Pleno\is{acesso pleno} defende que a gramática da \isi{L1} é transferida na sua totalidade, correspondendo ao estádio inicial da aquisição de \isi{L2}, e é reestruturada gradualmente, à medida que o aprendente é exposto a dados da \isi{L2} que são incompatíveis com as regras da gramática da sua \isi{interlíngua}. Uma vez que a reestruturação da gramática assenta nos mecanismos que estão disponíveis na aquisição de \isi{L1}, prediz-se que os falantes não-nativos possam adquirir plenamente todas as propriedades gramaticais da língua-alvo. Porém, o sucesso total na aquisição não é inevitável, podendo não ocorrer se o aprendente não tiver acesso a dados linguísticos suficientes para reestruturar determinados aspetos da sua gramática de \isi{interlíngua}.

Por outro lado, de acordo com um outro conjunto de hipóteses, a aquisição da \isi{L1} e da \isi{L2} são processos fundamentalmente diferentes (veja-se, por exemplo, a \textit{Hipótese da Diferença Fundamental}, de \citealt{bley-vroman1989}, já referida na Secção \ref{sec:madeira_aquisicao}), como consequência da existência de um \isi{período crítico} para a aquisição da linguagem. Nesta perspetiva, enquanto a aquisição de \isi{L1} é um processo natural, que depende de mecanismos cognitivos específicos da linguagem, que são ativados pela mera exposição a dados linguísticos, a aquisição/aprendizagem da \isi{L2} assenta quer na \isi{transferência} de propriedades da \isi{L1} do aprendente quer em mecanismos gerais de aquisição/aprendizagem. Assim, prediz-se que o conhecimento que um falante não-nativo desenvolve da gramática da \isi{L2} seja sempre incompleto e diferente do conhecimento gramatical do falante nativo.

O principal argumento que tem sido utilizado na literatura de aquisição de \isi{L2} em favor do primeiro conjunto de hipóteses e contra o segundo corresponde ao chamado \textit{argumento da pobreza de estímulo em L2}\is{L2}\is{pobreza do estímulo} \citep{schwartzsprouse2013}. No caso da \isi{L1}, existe uma distância entre os dados linguísticos primários e o sistema de conhecimento que a criança constrói, já que esta desenvolve conhecimento de propriedades gramaticais muito subtis e complexas, especificamente linguísticas, para as quais não existe evidência direta nos dados linguísticos e às quais seria impossível ou, pelo menos, muito difícil chegar recorrendo apenas a mecanismos e princípios cognitivos gerais. É este argumento que leva à hipótese de que os seres humanos possuem uma \isi{faculdade da linguagem}, que permite às crianças filtrar os dados linguísticos e determinar quais as gramáticas que podem gerar aqueles dados. No caso da \isi{L2}, o que é que constituiria evidência para a existência de problemas de \isi{pobreza do estímulo}? De acordo com \citet{white2003}, por exemplo, seria necessário encontrar evidência de que os falantes não-nativos têm conhecimento de propriedades que são especificamente linguísticas (ou seja, propriedades que não ocorrem noutros domínios de cognição) e que não estão presentes na \isi{L1} nem estão diretamente presentes nos dados linguísticos. Na Secção \ref{sec:madeira_desenvol_conhecimento}, veremos um exemplo desta evidência.


\subsection{Efeitos de influência da L1 em diferentes componentes da gramática}
\label{subsec:madeira_efeitos_influencia}
\is{L1}
Certos domínios da gramática são, aparentemente, mais suscetíveis a influência da \isi{L1} do que outros. Nesta secção, consideramos brevemente duas áreas em que se considera habitualmente que é mais provável que ocorra \isi{transferência}, nomeadamente, a \isi{fonologia} e o \isi{léxico}.

Sabemos que a \isi{fonologia} é umas das áreas em que se observam efeitos mais evidentes de influência da \isi{L1} (e.g. \citealt{broselow1988}). Os falantes não-nativos são, regra geral, facilmente identificados pelo seu ``sotaque'' estrangeiro, sendo fácil identificar a sua proveniência a partir das características da sua pronúncia. Embora se reconheça o papel importante da \isi{transferência} na aquisição da \isi{fonologia} da \isi{L2}, reconhece-se que, tal como em outras áreas da gramática, a \isi{transferência} interage com outros fatores. Por exemplo, diversos estudos demonstram que existe uma correlação entre a regularidade e o grau de exposição à língua e o desenvolvimento da competência fonológica: num estudo realizado com falantes não-nativos de inglês,\il{inglês} \citet{bongaerts_etal1995} verificaram que, aplicando métodos de ensino adequados e assegurando uma exposição prolongada e intensa à língua, é possível desenvolver um nível muito avançado de competência fonológica na \isi{L2}, ultrapassando os efeitos de \isi{fossilização} que seriam esperados em falantes que iniciam a sua exposição à língua na adolescência ou já em idade adulta. Assim, o que muitos dos trabalhos de investigação no domínio da aquisição da \isi{fonologia} na \isi{L2} mostram é que, apesar do papel incontestável da influência da \isi{L1} neste domínio, é necessário considerar também outros fatores – tais como o tipo e a quantidade de dados linguísticos a que o aprendente tem acesso – se queremos compreender plenamente o modo como os falantes não-nativos desenvolvem competência fonológica.

Na aquisição do \isi{léxico} (veja-se, entre outros, \citealt{leiria2001}), observa-se frequentemente evidência de \isi{transferência} de padrões lexicais da \isi{L1} para a \isi{L2}. Esta \isi{transferência} manifesta-se, por exemplo, no uso de empréstimos, sobretudo, mas não só, quando a \isi{L1} e a \isi{L2} são tipologicamente muito próximas. Este uso pode ser observado em enunciados como os que se mostram em (\ref{ex:madeira_4a}) e (\ref{ex:madeira_4b}), produzidos, respetivamente, por um falante nativo de espanhol\il{espanhol} e um nativo de inglês. \il{inglês}

\ea\label{ex:madeira_4}
\ea\label{ex:madeira_4a} A noite anterior ela esteve com o seu chefe tomando umas \textbf{copas} ($=$uns copos).
\ex\label{ex:madeira_4b} Às 18 horas, ela \textbf{chamou} o namorado ($=$telefonou ao namorado).  [CAL2]\footnote{CAL2 - Corpus de aquisição de L2 (Centro de Linguística da Universidade Nova de Lisboa). Disponível em \url{http://cal2.clunl.edu.pt/}.}
\zl

Ocorre também \isi{transferência} das propriedades de seleção das palavras em enunciados como em (\ref{ex:madeira_5a}), produzido por um falante nativo de espanhol\il{espanhol} (língua em que um verbo como \textit{ver} seleciona um complemento direto preposicionado quando este denota uma entidade humana específica), e (\ref{ex:madeira_5b}), que corresponde a um decalque da estrutura do inglês\il{inglês} \textit{wants her to go}.

\ea\label{ex:madeira_5}
\ea\label{ex:madeira_5a} Desceu as escadas e \textbf{viu ao guarda de segurança}.
\ex\label{ex:madeira_5b} O namorado diz-lhe que ele vai ir para Roma [\ldots] e ele \textbf{quer-a ir} com ele. [CA\isi{L2}]
\zl

Em alguns casos, a existência de uma forma ou estrutura semelhante na \isi{L1} tem um efeito facilitador, que é visível na maior rapidez com que se desenvolve o conhecimento dessa forma ou estrutura. Por exemplo, \citet{ardhomburg1983} compararam a aprendizagem de vocabulário em inglês\il{inglês} \isi{L2} por falantes nativos de espanhol\il{espanhol} e de árabe\il{árabe} \isi{L1}. Enquanto parte do vocabulário do espanhol\il{espanhol} e do inglês\il{inglês} é muito semelhante (sobretudo vocabulário de base latina), tal não acontece no caso do árabe.\il{árabe} Este facto influencia a rapidez com que os falantes nativos de espanhol\il{espanhol} aprendem vocabulário em inglês,\il{inglês} em contraste com os falantes de árabe,\il{árabe} e é particularmente notório quanto a palavras do inglês\il{inglês} que não apresentam quaisquer semelhanças com palavras do espanhol\il{espanhol} (ou seja, vocabulário de base germânica). Estes resultados sugerem, pois, que a existência de um corpo de vocabulário semelhante no espanhol\il{espanhol} \isi{L1} liberta os aprendentes para a aprendizagem de vocabulário menos familiar. 

De modo geral, o papel da \isi{L1}, que é indubitavelmente um dos fatores mais determinantes na aquisição de uma \isi{L2}, e a forma seletiva como os efeitos de influência da \isi{L1} se fazem sentir na aquisição de diferentes domínios da gramática continuam a ser pouco compreendidos e permanecem entre as questões de maior relevo na investigação sobre a aquisição de \isi{L2}.

\subsection{Aquisição de L3}
\label{subsec:madeira_aquisicao_l3}
\is{L3}
Considera-se atualmente que a aquisição de \isi{L2} poderá apresentar características diferentes, consoante se trate de uma segunda língua\is{língua segunda|seealso{L2}} ou de uma terceira ou quarta, cronologicamente. A aquisição de uma \isi{L3} poderá ser influenciada pelo conhecimento que o falante tem, não apenas da sua \isi{L1}, mas também de outras \isi{L2}, podendo a \isi{transferência} ser feita a partir de qualquer uma das línguas adquiridas previamente.

Uma das questões que constituem objeto de debate prende-se com o próprio conceito de \textit{L3}\is{L3} \citep{rothman_etal2013}. Assumindo um critério meramente cronológico, pode considerar-se que a \isi{L3} é a terceira língua que o falante adquire, ou seja, a sua segunda língua\is{língua segunda|seealso{L2}} não materna;\is{língua não materna} distingue-se, assim, não apenas da \isi{L2}, mas também da L4, L5, etc. Em alternativa, pode assumir-se que a \isi{L3} é qualquer \isi{língua não materna} adquirida por indivíduos que já adquiriram ou estão no processo de adquirir pelo menos uma outra \isi{língua não materna} – a \isi{L3} pode ser, neste caso, a terceira língua, a quarta ou a quinta, por exemplo. Por outro lado, assumindo que o critério determinante é o nível de proficiência atingido nas línguas previamente adquiridas, podemos considerar que a \isi{L3} não é necessariamente a terceira língua do falante, por ordem de aquisição – pode ser a segunda língua\is{língua segunda|seealso{L2}} que ele adquiriu se, entretanto, o nível de proficiência que ele possui nesta língua foi ultrapassado pelo nível que atingiu noutra \isi{língua não materna} que adquiriu posteriormente. Em conclusão, não existe um consenso sobre o que constitui uma \isi{L3}. 

Porém, independentemente da definição que se adote, a maioria dos investigadores concorda que é importante distinguir a \isi{L3} da \isi{L2}, por várias razões: o multilinguismo dos aprendentes tem um efeito aditivo na aquisição de \isi{L3} \citep{cenoz2003}, na medida em que o facto de que já ocorreu a aquisição de uma \isi{língua não materna} confere, aos aprendentes, vantagens cognitivas na aquisição posterior de outras línguas não maternas;\is{língua não materna} os aprendentes de \isi{L3} apresentam consciência e competências metalinguísticas mais desenvolvidas; e dispõem de uma maior variedade de fontes de conhecimento linguístico, o que lhes permite o acesso a um maior conjunto de propriedades gramaticais (não apenas as da \isi{L1}, mas também as da \isi{L2}).
Assim, no caso da \isi{L3}, a questão prende-se não apenas com o modo como a \isi{L1} poderá influenciar a aquisição, mas também com o papel que outra ou outras \isi{L2} poderão assumir na aquisição da \isi{L3}. Duas das hipóteses que têm sido defendidas relativamente ao estádio inicial na aquisição de \isi{L3} são as seguintes:

\begin{enumerate}
\item Os falantes transferem apenas a partir da \isi{L2}. Por exemplo, \citet{bardelfalk2007} defendem que a \isi{L2} funciona como um filtro entre a \isi{L1} e a \isi{L3}, bloqueando acesso direto à \isi{L1}; toda a \isi{transferência} é, portanto, feita a partir da \isi{L2} ou é, pelo menos, mediada pela \isi{L2}.

\item  A \isi{transferência} pode ocorrer a partir quer da \isi{L1} quer da \isi{L2}. Neste caso, diversos fatores podem determinar de qual das duas línguas deverá ocorrer a \isi{transferência}: por exemplo, a distância tipológica entre as línguas – é mais provável que ocorra \isi{transferência} entre línguas que sejam tipologicamente próximas, pelo menos de acordo com a perceção dos falantes, do que entre aquelas que são tipologicamente distantes (veja-se o \textit{Modelo da Primazia Tipológica} de \citealt{rothman2011}); outros fatores como o prestígio da \isi{L1} e da \isi{L2}, o nível de proficiência na \isi{L2} e o uso mais ou menos recente de cada uma das línguas poderão contribuir para determinar qual é a língua que constitui a principal fonte de influência. 
\end{enumerate}

De acordo com a investigação realizada até ao momento, porém, nem sempre os falantes selecionam uma única língua como a sua base para \isi{transferência}. A língua de base pode mudar ao longo do tempo e, numa determinada etapa, os aprendentes podem selecionar diferentes tipos de informação de cada uma das suas línguas. Contudo, parece claro que a existência de semelhanças formais entre as línguas e, de modo geral, a proximidade tipológica favorecem a \isi{transferência} de uma determinada língua na aquisição de \isi{L3}. 

Em suma, apesar de haver muitos aspetos do processo de aquisição de \isi{L3} que não estão ainda bem compreendidos, parece claro que este apresenta algumas características que o distinguem da aquisição de uma \isi{L2}. Por exemplo, no caso da \isi{L3}, esta poderá ser influenciada pelo conhecimento que o falante tem, não apenas da sua \isi{L1}, mas também de outras línguas não maternas\is{língua não materna} previamente adquiridas, podendo a \isi{transferência}, de acordo com algumas perspetivas, ser realizada a partir de qualquer uma destas línguas. 

\section{O desenvolvimento do conhecimento linguístico}
\label{sec:madeira_desenvol_conhecimento}

Uma questão importante na aquisição de \isi{L2} refere-se ao modo como os falantes não-nativos desenvolvem conhecimento de propriedades pertencentes a diferentes domínios gramaticais. Nesta secção, procuramos descrever algumas propriedades da aquisição de \isi{L2}, concentrando-nos em três domínios: a \isi{morfologia} flexional, a \isi{sintaxe} e o discurso.  


\subsection{Aquisição da morfologia flexional}
\label{subsec:madeira_aquisicao_morfologia}
\is{morfologia}
Embora a influência de conhecimento linguístico prévio não seja particularmente evidente no caso da \isi{morfologia} flexional, esta é uma área que apresenta dificuldades visíveis na aquisição de uma \isi{L2}, em especial quando a \isi{L2} é uma língua com paradigmas morfológicos ricos, como o português, e a \isi{L1} dos aprendentes se caracteriza pela ausência de \isi{morfologia} flexional \citep{white2003}. As dificuldades manifestam-se sobretudo ao nível da produção, persistem em estádios de desenvolvimento avançados e não são, geralmente, acompanhadas por um atraso das propriedades sintáticas\is{sintaxe} associadas \citep{lardiere2000,prevostwhite2000}. A produção destas formas morfológicas caracteriza-se pela variabilidade, que se traduz na alternância entre formas-alvo e formas desviantes. Veja-se os exemplos em (\ref{ex:madeira_6}), retirados de produções escritas de falantes nativos de chinês\il{chinês} de nível elementar, aprendentes de português europeu \isi{L2}:

\ea\label{ex:madeira_6}
\ea\label{ex:madeira_6a} XinNa não \textbf{bebeu} muito [\ldots]. Depois ela \textbf{dormir} [\ldots].
\ex\label{ex:madeira_6b} As férias \textbf{ficaram} muito long. [\ldots] As férias \textbf{passou} rápido.
\ex\label{ex:madeira_6c} Eu \textbf{estudei} lingua português na Universidade [\ldots] eu \textbf{compreendei} muito.
\ex\label{ex:madeira_6d} E nós sempre \textbf{foram} ao cinema [\ldots]. Nós também \textbf{visitamos} castelos [\ldots]. [CA\isi{L2}]
\zl

Segundo alguns autores, esta variabilidade deve-se a défices nas representações gramaticais subjacentes, ou seja, a um conhecimento defetivo de propriedades morfossintáticas abstratas (e.g. \citealt{hawkinschan1997}). De acordo com esta hipótese, designada como a \textit{Hipótese do Défice Representacional}, a aquisição da \isi{morfologia} flexional deverá estar relacionada com o desenvolvimento das propriedades sintáticas\is{sintaxe} correspondentes. Assim, no caso da flexão verbal de pessoa e número, que está estreitamente ligada a fenómenos sintáticos\is{sintaxe} como a possibilidade de sujeitos nulos e certos aspetos de ordens de palavras na frase, a hipótese prediz que a variabilidade na realização das formas morfológicas deverá estar associada a défices no conhecimento destes fenómenos sintáticos. \is{sintaxe}

Contudo, vários estudos têm encontrado evidência contra esta predição. Por exemplo, \citet{lardiere2000} e \citet{prevostwhite2000} mostram que existe uma dissociação entre a \isi{morfologia} e o conhecimento sintático – ou seja, os falantes não-nativos demostram ter adquirido as propriedades sintáticas, apesar de continuarem a manifestar dificuldades na produção da \isi{morfologia} flexional. Uma explicação alternativa para a variabilidade da \isi{morfologia}, a \textit{Hipótese da Ausência Superficial da Flexão} \citep{prevostwhite2000}, defende que o défice não se situa ao nível do conhecimento gramatical, mas apenas ao nível do uso, designadamente na realização morfológica das formas. Os défices devem-se, de acordo com esta hipótese, a dificuldades no acesso às formas morfológicas durante a produção. 


\subsection{Aquisição de propriedades sintáticas}
\label{subsec:madeira_aquisicao_sint}
\is{sintaxe}
Observámos na Secção \ref{subsec:madeira_modelo_generativo} que existem diferentes hipóteses relativamente ao desenvolvimento do conhecimento gramatical, que fazem predições distintas quanto à existência de efeitos de influência da \isi{L1} (em particular, nos estádios iniciais) e às probabilidades de uma aquisição completa das propriedades gramaticais da \isi{L2}. 

Muitos estudos mostram que a gramática da \isi{L1} pode, de facto, constituir o ponto de partida na aquisição de \isi{L2}. Vamos tomar como exemplo uma propriedade sintática que caracteriza a gramática de línguas como o português, nomeadamente, a existência de sujeitos nulos. As línguas naturais variam quanto à possibilidade de permitir a omissão do sujeito gramatical em orações finitas: a omissão é possível em línguas como o português, o espanhol\il{espanhol} e o italiano\il{italiano} (línguas de \isi{sujeito nulo}), mas não em línguas como o inglês,\il{inglês} o francês\il{francês} e o alemão\il{alemão} (línguas de sujeito obrigatório). Existe evidência de que os falantes de línguas de \isi{sujeito nulo} tendem a omitir e a aceitar a omissão de sujeitos quando aprendem línguas que não permitem sujeitos nulos. Por exemplo, num estudo sobre a aquisição de sujeitos em inglês\il{inglês} \isi{L2}, \citet{white1985pro} observou que, numa tarefa de juízos de gramaticalidade, falantes nativos de espanhol\il{espanhol} de nível elementar e intermédio apresentavam taxas elevadas de aceitação de frases (agramaticais) com sujeitos nulos. Uma comparação dos resultados deste grupo com os de um grupo de falantes nativos de francês,\il{francês} que apresentava taxas de aceitação destas frases consideravelmente mais baixas, revela um claro efeito de influência da \isi{L1}, observando-se, no entanto, que as taxas de aceitação do grupo de espanhol\il{espanhol} \isi{L1} diminuíam à medida que o nível de proficiência aumentava, indicando que este grupo estaria a adquirir as propriedades do inglês.\il{inglês} Por outro lado, em línguas de \isi{sujeito nulo} como o espanhol\il{espanhol} \citep{montrulrodriguez2006} ou o português \citep{madeira_etal2009}, os sujeitos nulos são adquiridos cedo, mesmo quando a \isi{L1} dos aprendentes não possui sujeitos nulos.

Um outro exemplo, referente à aquisição de \isi{interrogativas} em inglês\il{inglês} \isi{L2}, confirma que é possível adquirir propriedades sintáticas\is{sintaxe} da \isi{L2}, mesmo quando essas propriedades não estão representadas na gramática da \isi{L1} e não existe evidência direta da sua existência nos dados linguísticos. Sabemos que, em português, é possível formar \isi{interrogativas} de dois modos diferentes: movendo o constituinte interrogativo para o início da frase (\ref{ex:madeira_7a}) ou mantendo-o i\textit{n situ}, na sua posição canónica (\ref{ex:madeira_7b}) (a este propósito, veja-se o Capítulo 10). 

\ea\label{ex:madeira_7}
\ea\label{ex:madeira_7a} \textbf{Para onde} foi o João?
\ex\label{ex:madeira_7b} O Joao foi \textbf{para onde}?
\zl

Quando a interrogativa\is{interrogativas} envolve movimento do constituinte interrogativo, o movimento é bloqueado em domínios ilha, de que são exemplos as orações relativas e adverbiais (cf. \ref{ex:madeira_8}--\ref{ex:madeira_9}). Neste caso, só é possível fazer a pergunta com o constituinte interrogativo \textit{in situ}.

\ea\label{ex:madeira_8}
\ea[]{\label{ex:madeira_8a} O João conheceu o jornalista [que escreveu \textbf{o quê}]?}
\ex[*]{\label{ex:madeira_8b} \textbf{O que} conheceu o João o jornalista [que escreveu -]?}
\zl

\ea\label{ex:madeira_9}
\ea[]{\label{ex:madeira_9a} O João conheceu o jornalista [quando esteve \textbf{onde}]?}
\ex[*]{\label{ex:madeira_9b} \textbf{Onde} conheceu o João o jornalista [quando esteve -]?}
\zl

Muitas línguas dispõem de apenas uma destas estratégias de formação de \isi{interrogativas}. Por exemplo, em inglês,\il{inglês} só se pode formar \isi{interrogativas} com movimento, enquanto em chinês\il{chinês} e em indonésio\il{indonésio} as interrogativas-Qu não envolvem movimento. Na aquisição de inglês\il{inglês} \isi{L2}, prediz-se que os falantes nativos de chinês\il{chinês} e de indonésio\il{indonésio} desconheçam as restrições a que o movimento de constituintes interrogativos está sujeito, uma vez que estas restrições não estão visivelmente presentes nos dados linguísticos nem são ensinadas explicitamente em contexto de sala de aula. No entanto, vários estudos mostram que falantes não-nativos de inglês\il{inglês} que têm uma \isi{L1} sem movimento-Qu demonstram conhecimento destas restrições.\footnote{Embora línguas como o chinês\il{chinês} e o indonésio\il{indonésio} possuam movimento em algumas construções – por exemplo, \citet{huang1982} mostra que o chinês\il{chinês} tem movimento em orações relativas e em construções de topicalização, o qual está sujeito às mesmas restrições que o movimento-Qu em inglês\il{inglês} – e embora existam restrições também quanto à ocorrência de constituintes interrogativos\is{interrogativas} em domínios ilha nestas línguas (veja-se o exemplo do chinês\il{chinês} em (\ref{ex:madeira_i}), que mostra um constituinte-Qu dentro de uma oração relativa), estas restrições parecem diferir daquelas que se observam nas \isi{interrogativas} em línguas como o inglês\il{inglês} (veja-se (\ref{ex:madeira_ii}), que exibe também um constituinte-Qu dentro de uma relativa e cujo equivalente em inglês\il{inglês} seria agramatical).

\ea[*]{\label{ex:madeira_i}
\gll tou-le sheme de neige ren bei dai-le?\\
roubar quê \textsc{DE} aquela pessoa por apanhado\\ 
\glt `O homem que roubou o quê foi apanhado?'\jambox{\citep[380]{huang1982move}}
}
\z
\ea[]{\label{ex:madeira_ii}
\gll shei yao mai de shu zui gui?\\
quem querer comprar \textsc{DE} livro mais caro\\ 
\glt `Os livros que quem quer comprar são mais caros?'\jambox{\citep[381]{huang1982move}}
}
\z
Assim, pode considerar-se que os falantes não-nativos de inglês\il{inglês} que têm como \isi{L1} uma língua com estas características precisam de adquirir não só o movimento em \isi{interrogativas}, mas também as condições que restringem esse movimento.
}

Um estudo de \citet{martohardjonoflynn1995} realizado com falantes nativos de chinês\il{chinês} e indonésio,\il{indonésio} por exemplo, revela uma percentagem elevada de respostas-alvo numa tarefa de juízos de gramaticalidade sobre diferentes tipos de estruturas \isi{interrogativas}. Observamos que as taxas de rejeição de \isi{interrogativas} com movimento a partir de domínios ilha são muito próximas dos valores apresentados por um grupo de falantes nativos de inglês\il{inglês} (cf. Tabela \ref{tab:madeira_rejeicao}).

\begin{table}
  \begin{tabular}{ll}
    \lsptoprule
    Língua  & Taxas de rejeição \\
    \midrule
Chinês\il{chinês} L1 & 65\%\\
Indonésio\il{indonésio} L1 & 74\%\\
Inglês\il{inglês} L1 & 92\%\\
\lspbottomrule
  \end{tabular}
  \caption{Taxas de rejeição de interrogativas com movimento a partir de domínios ilha}
  \label{tab:madeira_rejeicao}
\end{table}

Estes resultados mostram que o argumento da \isi{pobreza do estímulo} é relevante na aquisição de \isi{L2} – isto é, os falantes não-nativos podem exibir conhecimento de propriedades especificamente linguísticas (como, neste caso, restrições sobre o movimento de constituintes interrogativos)\is{interrogativas} que não estão presentes na \isi{L1}, não são evidentes nos dados linguísticos e não são ensinadas explicitamente. Em suma, pode afirmar-se que existe evidência clara de que a tarefa de aquisição de \isi{L2} por adultos é orientada por mecanismos e princípios cognitivos especificamente linguísticos, semelhantes àqueles que orientam a aquisição de \isi{L1}.

Além disso, embora os resultados dos inúmeros estudos existentes sobre a aquisição de propriedades sintáticas\is{sintaxe} não sejam categóricos, existem fortes indícios de que, pelo menos na aquisição de certas propriedades sintáticas,\is{sintaxe} os efeitos de influência da \isi{L1} não são significativos, observando-se semelhantes percursos de desenvolvimento em falantes de diferentes línguas maternas. Assim, há pelo menos certos aspetos da aquisição da sintaxe da \isi{L2} que não são determinados pela \isi{L1} dos aprendentes, mas que poderão ser consequência de sequências naturais de desenvolvimento linguístico.

\subsection{Aquisição de propriedades discursivas}
\label{subsec:madeira_aquisicao_discurs}
\is{propriedades discursivas}
Muita investigação recente na área de ASL tem demonstrado que propriedades estritamente gramaticais, em particular as propriedades sintáticas,\is{sintaxe} são mais fáceis de adquirir do que propriedades que implicam a integração de conhecimentos de diferentes domínios da gramática ou de conhecimentos gramaticais e de outros sistemas cognitivos. Estas últimas propriedades, chamadas \textit{propriedades de interface},\is{interface} apresentam, frequentemente, atrasos no desenvolvimento e efeitos de \isi{fossilização}, e caracterizam-se por um elevado grau de variabilidade e pela presença de efeitos residuais de influência da \isi{L1} em estádios mais avançados de aquisição. A hipótese de que as propriedades gramaticais, sobretudo as sintáticas, são mais fáceis de adquirir do que as propriedades que estão na \isi{interface} entre a sintaxe e outros domínios cognitivos é conhecida na literatura como a \textit{Hipótese da Interface}\is{interface} (e.g. \citealt{soracefiliaci2006}).

Vamos ilustrar este fenómeno com o exemplo da aquisição de sujeitos nulos e expressos, que já referimos na Secção \ref{subsec:madeira_aquisicao_sint}. Em línguas de \isi{sujeito nulo}, o sujeito nem sempre é opcional, ou seja, os sujeitos nulos e os sujeitos expressos não ocorrem em variação livre. Assim, existem contextos em que a realização do sujeito é impossível (é o caso das orações coordenadas com sujeitos correferentes, como em \ref{ex:madeira_11a}) e contextos em que o sujeito é obrigatoriamente realizado (por exemplo, em coordenadas com sujeitos referencialmente disjuntos, como em \ref{ex:madeira_11b}).

\ea\label{ex:madeira_11}
\ea\label{ex:madeira_11a} O João$_j$ encontrou o Pedro$_i$ no cinema, mas (*ele$_j$) não lhe falou.
\ex\label{ex:madeira_11b} O João$_j$ encontrou o Pedro$_i$ no cinema, mas *(ele$_i$) não lhe falou.
\zl

Por outro lado, existem contextos em que pode haver alternância. No entanto, mesmo nestes contextos, a escolha entre um \isi{sujeito nulo} e um sujeito expresso não parece ser verdadeiramente livre. Neste sentido, a distribuição de sujeitos nulos e expressos parece obedecer a condições discursivas distintas. Por exemplo, em orações subordinadas,\footnote{Falamos de orações subordinadas completivas de indicativo. No caso das orações completivas de conjuntivo, a interpretação é diferente (veja-se, sobre esta questão, 11.4.3.).} como as ilustradas em (\ref{ex:madeira_12}) abaixo, os sujeitos nulos tendem a ser interpretados como idênticos ao sujeito da oração matriz: na frase em (\ref{ex:madeira_12a}) abaixo, a interpretação mais natural é aquela em que foi o João que reprovou no exame. Pelo contrário, nestes contextos, os sujeitos expressos tendem a ser interpretados como distintos do sujeito da oração matriz: assim, na frase em (\ref{ex:madeira_12b}), a interpretação mais natural será aquela em que o Pedro ou outra pessoa (mas não o João) reprovou no exame.

\ea\label{ex:madeira_12}
\ea\label{ex:madeira_12a} O João$_j$ disse ao Pedro$_i$ que [-]$_j$ reprovou no exame.
\ex\label{ex:madeira_12b} O João$_j$ disse ao Pedro$_i$ que ele$_{i/k}$ reprovou no exame.
\zl

Existem muitos estudos sobre a aquisição de sujeitos nulos em \isi{L2} (e.g. \citealt{montrulrodriguez2006,soracefiliaci2006,rothman2009,madeira_etal2012}), que combinam diferentes pares de línguas e que, de maneira geral, mostram que os falantes não-nativos tendem a sobregeneralizar a produção de sujeitos expressos e a interpretá-los como sendo correferentes com um antecedente em posição de sujeito, mesmo em contextos que não favorecem esta interpretação. O domínio das condições discursivas que determinam a  distribuição de sujeitos nulos e sujeitos expressos aparece tarde e desenvolve-se gradualmente. Estas dificuldades não podem dever-se exclusivamente à influência da \isi{L1}, uma vez que têm sido observadas quer em falantes de línguas de sujeito obrigatório quer em falantes de línguas de \isi{sujeito nulo}. De igual modo, têm sido observadas em outros domínios como, por exemplo, na erosão da \isi{L1}, na aquisição de \isi{L1}, na aquisição bilingue e na aquisição de língua de herança.

Não é muito claro por que razão as \isi{propriedades discursivas}, ou a integração destas propriedades com outros tipos de conhecimento gramatical, levantam tantas dificuldades aos falantes não-nativos. No caso dos sujeitos, a realização de sujeitos redundantes poderá corresponder a uma estratégia de compensação por défices na \isi{morfologia} verbal \citep{margazabel2006}; poderá ser um efeito de exposição a dados linguísticos variáveis, uma vez que os próprios falantes nativos frequentemente produzem sujeitos redundantes \citep{rothman2009}; ou poderá dever-se a problemas de processamento que resultam do diferente estatuto das formas pronominais envolvidas (nulas vs. lexicais), na linha do que é proposto por \citet{costaambulate2010} para a aquisição de sujeitos pronominais em português \isi{L1}. De modo mais geral, os atrasos e a opcionalidade persistente que se observam no desenvolvimento destas propriedades poderão dever-se a dificuldades na integração de informação gramatical e discursiva. Assim, o que se desenvolve tardiamente não será tanto o conhecimento gramatical e discursivo, mas sim as estratégias de processamento necessárias para integrar estes diferentes tipos de conhecimento.  

\section{Conclusão}
\label{sec:madeira_conclusao}

Apesar das diferenças evidentes que existem entre a aquisição de \isi{L2} e a aquisição de \isi{L1} (variabilidade, efeitos de \isi{fossilização}, etc.), podemos afirmar que a grande questão de investigação é idêntica nos dois casos: como é que um indivíduo constrói uma gramática a partir dos dados linguísticos a que está exposto? No caso da \isi{L2}, para explicar o processo de aquisição e as propriedades particulares das gramáticas de \isi{interlíngua}, é necessário identificar e explicar, não apenas o papel desempenhado pelos mecanismos cognitivos e pelos diferentes tipos de dados linguísticos (bem como os modos como estes interagem), mas também o papel do conhecimento linguístico prévio. Além disso, importa determinar qual é o efeito de outros fatores, tais como a idade de início de exposição à língua, o contexto de aquisição/aprendizagem e as diferenças individuais no processo de aquisição da \isi{L2}. Outras questões que têm assumido particular relevo na investigação mais recente em ASL prendem-se, por exemplo, com as diferenças que existem entre a aquisição de \isi{L2} e de \isi{L3}, e com o diferente estatuto dos vários domínios gramaticais na aquisição e, em particular, de propriedades linguísticas que exigem a integração de diferentes tipos de conhecimento, as chamadas \textit{propriedades de interface}.\is{interface}







{\sloppy
\printbibliography[heading=subbibliography,notkeyword=this]
}
\end{document}