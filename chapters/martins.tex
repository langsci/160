\documentclass[output=paper]{LSP/langsci} 
\author{Alexandrina Martins\affiliation{Universidade de Lisboa, Centro de Linguística e Universidade de Aveiro}\lastand 
Sónia Vieira\affiliation{Universidade de Lisboa, Centro de Linguística}
}
\title{Avaliação linguística em contextos de desenvolvimento típico e atípico: aspetos sintáticos}  
\rohead{\thechapter\hspace{0.5em}Avaliação linguística em contextos de desenvolvimento típico e atípico}
\abstract{\noabstract}
\ChapterDOI{10.5281/zenodo.889463}
\maketitle
\begin{document}
\section{Introdução}
\label{sec:martins_intro}

Um dos maiores objetivos dos pais das crianças que procuram uma opinião especializada é obterem informação acerca da necessidade de intervenção, sendo indispensável que os dados retirados da avaliação permitam, para além da identificação de áreas de intervenção, a confirmação da necessidade da mesma. É, por isso, de extrema relevância a possibilidade de separar performances que seguem um percurso típico daquelas que se apresentam como atípicas\is{desenvolvimento atípico} em termos de aquisição de linguagem, tentando perceber qual o grau de variabilidade que poderá ser considerado como expectável para os parâmetros normativos.

Importa sempre considerar que o desenvolvimento linguístico é caracterizado por uma considerável variação, conforme comprovam Fenson et al. (1994 \textit{apud} \citealt{bishop1997}) através da utilização da \textit{MacArthur Scale of Communicative Development Inventory}. O relato descreve 80\% das crianças com 16 meses de idade compreendendo entre 78 e 303 palavras, bem como uma variação na \isi{produção} que vai desde as 154 palavras produzidas no caso de 10\% das crianças com melhores resultados até nenhuma palavra produzida, no caso de 10\% das crianças, as que obtiveram resultados mais baixos. 

\subsection{Importância da avaliação linguística}
\label{subsec:martins_importancia}

No processo de avaliação, vários são os instrumentos que podem ser utilizados, de acordo com o caso em questão ou o objetivo de tal momento. A sua utilização deve permitir a recolha e registo de informação que facilite a tomada de decisão e, tanto quanto possível, a compreensão da problemática \citep{coltoncovert2007}. 

Assim sendo, poderemos dispor de \textit{checklists}, escalas de avaliação formais, observação direta, testes \textit{standardizados} ou provas informais, oferecendo cada uma destas, ainda que de forma diferente e dentro das limitações que apresentam, uma abordagem importante quando se procura caracterizar o perfil linguístico de um indivíduo (veja-se o levantamento de metodologias de avaliação no capítulo 14). Em alguns destes casos, uma primeira abordagem poderá levar à perceção da necessidade de uma avaliação mais detalhada de áreas linguísticas específicas, sendo para isso necessário recorrer a métodos mais sensíveis ou mais dirigidos às mesmas. A necessidade da avaliação de áreas linguísticas específicas advém do facto de, cada vez mais, se reconhecer que uma criança com uma Perturbação da Linguagem poderá revelar dificuldades que atingem de forma modular e heterogénea as diferentes componentes da linguagem \citep{bishop1997}. 

As perturbações da linguagem podem ser categorizadas como primárias\is{perturbação primária} ou secundárias.\is{perturbação secundária} Nas perturbações primárias\is{perturbação primária} da linguagem, e tal como o nome indica, as dificuldades linguísticas ocorrem sem que nenhuma causa subjacente o justifique \citep{schuele2004}.

A nomenclatura mais conhecida para este tipo de grupo é \textit{Perturbação Específica da Linguagem} (PEL),\is{Perturbação Específica da Linguagem} nos casos em que os valores de QI não-verbal estão acima dos 85, isto é, são valores normais. Estas crianças apresentam capacidades auditivas normais, resultados em termos de testes de inteligência não-verbal dentro dos expectáveis para a sua idade e não evidenciam qualquer défice a nível neurológico, representando por isso um desafio para o profissional responsável pela sua avaliação \citep{leonard2014}. A questão da avaliação do perfil linguístico das PEL torna-se particularmente delicado, quando considerada a existência de diferentes subtipos, distinguindo-se os mesmos entre si pela afeção divergente das diferentes componentes. Assim, uma criança poderá revelar défices puramente sintáticos (PEL-Sintática), sem que apresente défices fonológicos, lexicais ou pragmáticos, podendo também ocorrer o percurso inverso. Ou seja, é possível a existência de défices seletivos num determinado módulo da linguagem e não nos outros, levando à identificação de subtipos dentro das PEL, afetando um ou vários módulos de linguagem \citep{friedmannnovogrodsky2008}.

No caso das perturbações secundárias\is{perturbação secundária} da linguagem, os défices são consequência de uma etiologia conhecida, tal como alterações cromossómicas (e.g. \isi{Síndrome de Down}, \isi{Síndrome de Williams}), alterações sensoriais (e.g. surdez), perturbações desenvolvimentais (\isi{Perturbações do Espectro Autista}) ou alterações neurológicas (e.g. \isi{afasia}). 

Sendo a sintaxe uma das áreas da linguagem mais afetadas em crianças com PEL e em crianças com \isi{Síndrome de Williams} (SW) e \isi{Síndrome de Down} (SD), é essencial avaliar os parâmetros subjacentes a este subdomínio linguístico, de modo a que, para além de ampliar o conhecimento acerca das manifestações linguísticas das patologias, seja possível programar a intervenção a implementar. Este fator torna-se ainda mais importante quando se considera que não estamos perante grupos homogéneos visto que todos eles diferem quanto às suas capacidades sintáticas. De acordo com \cite{leonard1998}, é enganador quando na literatura inicial sobre perturbações da linguagem se referia que as crianças exibiam um atraso ou um desvio face ao normal desenvolvimento da linguagem. Para Leonard, esta dicotomia não corresponde à realidade, visto que o que distingue crianças com perturbação das que seguem o desenvolvimento típico é o facto de as primeiras exibirem perfis linguísticos irregulares que não se assemelham a nenhuma das fases de desenvolvimento linguístico típico. Ou seja, o desempenho sintático de uma criança com perturbação pode igualar-se ao de uma criança de desenvolvimento típico dois anos mais nova relativamente à \isi{produção} de uma determinada estrutura, mas ser semelhante ao de uma criança três anos mais nova relativamente à \isi{produção} de outra estrutura sintática. 

Os testes utilizados para o Português Europeu permitem uma importante análise das capacidades linguísticas das crianças, no entanto são concebidos para uma abordagem global de várias áreas da linguagem, não permitindo a obtenção de dados muito específicos sobre áreas particulares, nomeadamente a sintaxe \citep{afonso_2011}. Por esta razão, a avaliação da performance perante estruturas sintáticas específicas, consideradas como marcadores\is{marcadores clínicos} de determinadas perturbações, como sendo o caso da Perturbação Específica da Linguagem-Sintática (PEL-S),\is{Perturbação Específica da Linguagem} apresenta-se como crucial. Testes que abrangem aspetos específicos da linguagem são muitas vezes utilizados para complementar baterias completas e normalmente incluem um maior número de itens dedicados a esses mesmos aspetos do conhecimento linguístico, bem como mais níveis de dificuldade, permitindo colher informações precisas e detalhadas sobre as capacidades específicas de linguagem da criança a avaliar. 

Quantificar o desempenho linguístico de uma criança, através do recurso a testes de linguagem, poderá ser uma tarefa relativamente fácil. No entanto, traduzir essa quantificação num padrão de performance e tirar conclusões a respeito dos mecanismos que estão na base desse desempenho linguístico poderá ser um desafio para o avaliador. Várias podem ser as razões que justificam os resultados obtidos, sendo que uma fraca performance em testes de linguagem pode ter como origem diversos fatores, como o conhecimento desviante ou ausente, dificuldade na análise, a sobrecarga de memória, entre outras \citep{crainthornton1998}. 

Assim, é importante ter em conta que, se uma criança não produzir uma determinada estrutura sintática, isso poderá ser devido a duas situações: um problema de competência que se irá traduzir na performance; ou um problema de performance apesar de a criança ser competente. Neste último caso, poderá significar que certos fatores extra-linguísticos poderão estar a impossibilitar o acesso ao conhecimento, pelo que é necessária alguma cautela na interpretação dos resultados, visto que mesmo que a criança não produza uma determinada estrutura sintática, não significa que não a tenha adquirido \citep{valianaubry2005}. 

Os testes deverão ser criados com vista a uma avaliação da capacidade das crianças em compreender ou utilizar (produzir) um aspeto particular da língua, sendo concebidas tarefas relevantes a realizar \citep{peccei2006}.

\subsection{Avaliação da \isi{compreensão} sintática}
\label{subsec:martins_avaliacao_comp}

No que diz respeito à \isi{compreensão} de estruturas sintáticas específicas, pretende-se avaliar a interpretação que a criança atribui a determinada frase, podendo para isso ser criadas tarefas perante as quais é requerida uma resposta a um estímulo falado ou escrito, através do olhar, do apontar ou de uma ação. A avaliação da \isi{produção} pode ser realizada recorrendo a diferentes tipos de tarefas, desde as mais naturalistas, como a análise de \isi{discurso espontâneo}, a tarefas mais estruturadas, como a \isi{produção} provocada de frases.

Quanto à avaliação da \isi{compreensão}, várias são as tarefas que podem ser utilizadas, podendo para tal o avaliador recorrer a diferentes tipos de materiais, adaptando-os à idade e características da criança. As tarefas mais comummente utilizadas são tarefas de \textit{act-out},\is{act@\emph{act-out}} tarefas com imagens (através das quais se pode recorrer à seleção ou juízo das mesmas) e tarefas de \isi{juízo de valor de verdade}. 

A tarefa de act-out\is{act@\emph{act-out}} foi já utilizada por Chomsky, quando em 1969 (\textit{apud} \citealt{mcdaniel_etal1998}) utilizou esta metodologia de representação para estudar o conhecimento sintático das crianças. Desde então, tem sido tradicionalmente utilizada no estudo da aquisição da linguagem, consistindo numa situação de jogo no qual a criança manipula os brinquedos à sua disposição de acordo com as frases apresentadas, seguindo a instrução ``Faz o que eu digo''. As maiores vantagens desta são, por um lado, não limitar a resposta, como numa tarefa de identificação das imagens e, por outro lado, facilitar a análise de erro \citep{correa1995}. No entanto, são necessários alguns cuidados na montagem desta tarefa, pois corre-se o risco de se tornar pouco funcional, sendo importante a criação de um cenário que legitime a utilização das estruturas sintáticas que se pretende testar. É importante ter também em mente a limitação de não ser possível saber se a criança aceita outras leituras para a frase ouvida \citep{crainthornton1998}.

Nas tarefas que envolvem identificação de imagens, é pedido à criança que identifique a imagem correspondente à frase ouvida. Neste caso, podem ser utilizadas imagens desenhadas (coloridas ou a preto e branco) ou mesmo fotografias de brinquedos ou objetos. No caso das tarefas de \isi{avaliação de imagem}, é pedido que a criança julgue a frase proferida pelo avaliador como verdadeira ou falsa, segundo a sua interpretação da imagem. As principais vantagens das tarefas com imagens são a rapidez de aplicação e o facto de possibilitar avaliar crianças muito novas. Para além disso, permite avaliar estruturas com diferentes níveis de complexidade. No entanto, como qualquer tarefa, também possui as suas limitações, pois, para além de ser difícil representar todo o tipo de ações, a análise dos resultados deve ser cuidada, pois apenas indica uma preferência e não se a criança aceita outras leituras \citep{schmittmiller2010}. 

A tarefa de \isi{juízo de valor de verdade} é uma ferramenta que envolve a apresentação de uma frase num contexto de cada vez. A tarefa da criança é julgar as frases proferidas por um boneco/fantoche como verdadeiras ou falsas, em diferentes conjuntos de contextos (histórias representadas com pequenos bonecos) \citep{crainthornton1998}.  

\subsection{Avaliação da \isi{produção} sintática}
\label{subsec:martins_avaliacao_prod}

Relativamente à avaliação da \isi{produção} sintática, é possível usar diferentes tipos de métodos, dos mais naturalistas (\isi{discurso espontâneo}) aos mais estruturados.

A análise do \isi{discurso espontâneo} foi um dos métodos mais frequentemente utilizado nos últimos 50 anos por investigadores e clínicos. A principal vantagem deste tipo de avaliação é o facto de não ser requerido um conhecimento prévio e aprofundado da respetiva língua para a criação de materiais de estímulo. Além disso, pode ser aplicada a qualquer criança, independentemente das suas capacidades linguísticas, cognitivas ou mesmo da sua idade.

Apesar de a análise do \isi{discurso espontâneo} providenciar dados ricos quanto ao conhecimento linguístico de uma criança, esta escolha enfrenta várias limitações metodológicas.  Isto porque, num ambiente espontâneo, não há forma de prever que tipo de estruturas sintáticas serão produzidas pela criança \citep{erlam2006}. Torna-se, por isso, fácil subestimar as suas capacidades linguísticas, uma vez que, da amostra em análise poderão não constar certas estruturas, sem que a razão para tal resida no facto de a criança não ser capaz de as produzir, mas simplesmente porque a amostra é reduzida ou porque perante a situação em causa não houve necessidade para tal \citep{mcdaniel_etal1998}.

No entanto, outras técnicas mais estruturadas e menos dispendiosas em termos de tempo são valorizadas, quer por clínicos quer por investigadores, como é o caso das técnicas de \isi{elicitação}. O uso de técnicas de \isi{elicitação} é preferível em diversas situações, nomeadamente, quando o objetivo é estudar uma estrutura sintática em particular; quando a sua frequência de ocorrência no \isi{discurso espontâneo} é limitada (por exemplo, se quisermos analisar a \isi{produção} de passivas);\is{passiva} ou quando o tempo é um fator a considerar. A frequência de ocorrência, especialmente de estruturas complexas, é um ponto particularmente importante quando se comparam as técnicas de \isi{elicitação} com o método de análise de \isi{discurso espontâneo}, visto que neste último a criança poderá de forma deliberada  evitar a \isi{produção} de estruturas complexas escolhendo, ao invés, formas mais simples de comunicar.

Nas técnicas de \isi{elicitação}, que incluem a \isi{produção} provocada e a imitação provocada, o investigador/clínico providencia um contexto que motive a criança a produzir uma determinada estrutura sintática. Uma vez que as produções verbais podem ser provocadas através do uso de contextos cuidadosamente controlados, muitas das dificuldades experienciadas aquando a análise do \isi{discurso espontâneo} de uma criança são minimizadas. 

No caso da \isi{produção} provocada, cria-se um contexto capaz de ``provocar'' a \isi{produção} de uma determinada estrutura sintática através de um contexto cuidadosamente desenhado para o efeito \citep{thornton1998}.

A técnica de imitação provocada é descrita como a \isi{repetição}, por parte do sujeito, de uma frase que foi momentos antes proferida pelo examinador/investiga\-dor. Assume-se, assim, que, se a criança reproduz corretamente uma frase que lhe é dada a imitar, então esse desempenho é representativo do seu conhecimento gramatical. Vários autores, entre eles \cite{contiramsden_etal2001}, mostraram que a imitação de frases pode ser um marcador\is{marcadores clínicos} sensível na identificação de crianças com perturbações específicas da linguagem. 

Os estudos que utilizam técnicas de \isi{elicitação} apresentam três vantagens relativamente à \isi{produção} espontânea: permitem explorar estruturas que ainda não foram produzidas pela criança; oferecem uma janela de oportunidade para estudar o processo de aquisição de uma determinada estrutura antes que esta esteja totalmente adquirida; e permitem um determinado nível de controlo\largerpage metodológico que não está disponível no \isi{discurso espontâneo} \citep{hirshpasekgolinkoff1996}.

A função e as características de determinadas estruturas sintáticas alvo, assim como os contextos em que estas ocorrem, devem ser tidos em conta na seleção do método mais apropriado na avaliação do conhecimento dessas mesmas estruturas \citep{zukowski2004}.

O estudo de crianças com perturbação da linguagem revela-se essencial para que, em contexto clínico, seja possível diagnosticar um caso com precisão a partir de um determinado perfil sintático. Pretende-se, assim, distinguir os casos em que o distúrbio poderá ser transitório, acabando por se resolver com o tempo, daqueles em que criança apresenta alterações de linguagem e que podem corresponder ao início de uma longa história de défices linguísticos, havendo a possibilidade de se traduzirem em problemas académicos e, muitas vezes, comportamentais \citep{bishopedmunson1987}. 

As medidas de competência sintática, tanto a nível de \isi{compreensão} como de \isi{produção}, estão fortemente relacionadas com a posterior aprendizagem da \isi{leitura} e capacidade de ortografia, como comprovam os resultados obtidos em testes de linguagem pré-escolar, sendo o valor de \isi{extensão média do enunciado} (EME) um forte preditor da proficiência da \isi{leitura} \citep{bishopadams1990}. A \isi{leitura} requer competências ao nível da descodificação e \isi{compreensão} que se baseiam em conhecimentos gramaticais e que vão muito além das tão estudadas capacidades fonológicas, sendo a competência sintática preponderante na aquisição de \isi{leitura} \citep{schuele2004}. 
\section{Alguns dados sobre desenvolvimento sintático em contexto atípico\is{desenvolvimento atípico}}
\label{sec:martins_dados}

\largerpage A identificação de crianças com perturbações da linguagem poderá ser facilitada se identificarmos fenómenos no comportamento linguístico que ocorrem regularmente nessas crianças, os chamados \isi{marcadores clínicos}. A identificação de um marcador\is{marcadores clínicos} tem implicações teóricas e clínicas, visto poder providenciar pistas sobre os mecanismos subjacentes à perturbação. Para ser considerado um bom teste ou um bom marcador clínico,\is{marcadores clínicos} o comportamento em questão deve estar presente em indivíduos que têm essa perturbação e ausentes em quem não tem. Assim, para avaliar o conhecimento gramatical é essencial ter em conta que as técnicas utilizadas tenham sido testadas em crianças com desenvolvimento típico e que estas tenham obtido sucesso quando confrontadas com as mesmas \citep{zukowski2004}. Este fator assume extrema importância quando se considera, tal como previamente mencionado, que o desenvolvimento linguístico das crianças com PEL é por definição desfasado em relação ao desenvolvimento típico nos aspetos da língua atingidos, sendo que estes défices não são acompanhados por limitações noutras áreas do desenvolvimento, afastando assim um diagnóstico alternativo. 

\subsection{Sintaxe e perturbação específica da linguagem}
\label{subsec:martins_sintaxe_ple}

Vários são os estudos que descrevem as dificuldades morfossintácticas associadas a PEL. \citet{zebib_etal2012} sugerem que a \isi{complexidade sintática} é uma área prejudicada em crianças com PEL. Ou seja, embora o aumento da complexidade seja uma marca do desenvolvimento sintático, a tentativa de evitar a complexidade mesmo durante a adolescência parece caracterizar indivíduos com PEL, que tendem a utilizar estruturas sintáticas mais frequentes tais como coordenadas ao invés de subordinadas. As estratégias utilizadas para contornar estruturas com maior nível de complexidade dependem da língua, da idade ou mesmo da gravidade das dificuldades linguísticas \citep{jakubowicztuller2008}.

Uma das áreas descrita como sendo de particular dificuldade para as crianças com PEL refere-se aos morfemas\is{morfema} gramaticais, elementos que expressam informações puramente gramaticais, como \isi{género}, número, pessoa ou tempo. 

Nos últimos anos, tem-se verificado um aumento de estudos para o português de crianças com PEL, talvez devido a um crescimento do interesse pela investigação e uma maior disponibilidade de instrumentos de avaliação da linguagem. Vários trabalhos têm mostrado que crianças com PEL falantes do português exibem, comparativamente aos seus pares com desenvolvimento típico, défices morfológicos e sintáticos que constituem \isi{marcadores clínicos} importantes para o seu diagnóstico. 

A pesquisa para o português do Brasil realizada por \cite{puglisi_etal2005} corrobora o facto de as questões gramaticais ligadas à sintaxe serem um desafio para crianças com PEL. Os autores concluíram que as crianças com PEL possuem um conhecimento restrito de quais as estruturas sintáticas a selecionar durante a \isi{produção} de uma frase, mais especificamente dos seus elementos gramaticais, como é o caso das \isi{preposições}, que têm a função de relacionar frases ou palavras numa oração. Os mesmos resultados foram encontrados em crianças falantes do italiano\il{italiano} com PEL (Sabbadini et al. 1987 \textit{apud} \citealt{leonard2000}).

\largerpage Outro estudo para o português \citep{araujo2007} centrou-se na caracterização do desempenho gramatical em 70 crianças de idade escolar com e sem PEL. Diferentes características gramaticais foram estudadas através da análise do \isi{discurso espontâneo} das crianças. Os resultados foram semelhantes aos já referidos por \cite{puglisi_etal2005}, reforçando a ideia de que um traço marcante nas crianças com PEL é a reduzida \isi{produção} de palavras funcionais tais como \isi{preposições}, pronomes e conjunções. Uma das consequências imediatas desta situação é a \isi{produção} de estruturas sintaticamente menos complexas e, consequentemente, uma menor extensão média de enunciado.\is{extensão média do enunciado} Estes estudos vieram também reforçar os dados já existentes para outras línguas ao verificar-se que as crianças com PEL apresentam mais problemas com a morfologia verbal do que nominal. 

\cite{befilopes_etal2008} verificaram que crianças com PEL falantes do português obtiveram piores desempenhos durante a \isi{produção} de \isi{narrativas}, quando comparadas com os seus pares com desenvolvimento típico, sendo que as crianças com PEL produziram frases sintaticamente menos complexas e com maior número de erros gramaticais.

Num trabalho desenvolvido por \cite{silveira2011}, foram comparadas crianças com PEL e crianças com desenvolvimento típico no sentido de explorar a \isi{concordância} em \isi{género} no interior do sintagma determinante (DP). Os resultados mostraram que a \isi{concordância} em \isi{género} é problemática em crianças com PEL, em particular quando há necessidade de atribuir \isi{género} a nomes não existentes para o português (pseudonomes). 

Dentro do grupo de crianças com PEL, a heterogeneidade tem sido alvo de grande debate ao longo dos tempos, sendo que cada indivíduo apresenta um perfil linguístico diferente, podendo apresentar disparidades não só entre as várias componentes linguísticas, como também em termos de \isi{compreensão} e \isi{produção}. Dentro do grupo, podem ser encontrados indivíduos com dificuldades mais marcadas em termos de \isi{produção} ou em termos de \isi{compreensão}, no entanto são vários os casos em que não são observáveis discrepâncias entre as capacidades linguísticas compreensivas\is{compreensão} e expressivas \citep{gillamkamhi2010}. 

Diversos estudos apontam para dificuldades na \isi{compreensão} de estruturas sintáticas, mais especificamente em frases relativas \citep{friedmannnovogrodsky2004,costa_etal2009}, passivas\is{passiva} \citep{vanderlely1996}, \isi{interrogativas} de objeto \citep{friedmannnovogrodsky2011}, bem como focalizações e frases com \isi{movimento} dativo \citep{vanderlelyharris1990,friedmannnovogrodsky2007}.

\subsection{Sintaxe e outras perturbações da linguagem}
\label{subsec:martins_sintaxe_outras}

A \isi{Síndrome de Williams} e a \isi{Síndrome de Down} caracterizam-se por alterações cognitivas significativas em termos gerais mas, no entanto, apresentam-se como altamente específicas ao nível das capacidades linguísticas, sendo que em cada síndrome existem défices linguísticos em diferentes níveis.

Clahsen \& Almazan (1998 \textit{apud} \citealt{guasti2002}) levaram a cabo um estudo sobre dois fenómenos sintáticos (e.g. passivas\is{passiva} e ligação anafórica) e ainda um fenómeno morfológico, mais precisamente sobre a marcação do passado em \isi{verbos} regulares e irregulares, em crianças com SW com idades compreendidas entre os 11;2 e os 15;4. Os resultados indicaram que, no que diz respeito aos fenómenos sintáticos, os sujeitos com SW produziram respostas 100\% corretas, obtendo um desempenho superior ao grupo de controlo. No entanto, ao nível morfológico, o desempenho destas crianças, relativamente à marcação do passado em \isi{verbos} irregulares foi inferior ao do grupo controlo, verificando-se um uso excessivo do morfema\is{morfema} –ed, apenas aplicável em \isi{verbos} regulares. Num outro estudo, \cite{clahsenalmazan2001} centraram-se no desempenho dos plurais regulares e irregulares em crianças com SW, verificando dificuldades acrescidas ao nível das construções irregulares. \cite{volterra_etal1996} num estudo com crianças com SW italianas reportaram a existência de défices morfossintáticos tais como \isi{concordância} sujeito-verbo, uso incorreto de infinitivos em frases de \isi{verbos} finitos e substituições incorretas de \isi{preposições}.

Estes autores defendem que o perfil das competências morfossintáticas em crianças com SW aparenta ser bastante diferente do apresentado por crianças com PEL. Enquanto estes últimos exibem maiores défices em aspetos sintáticos centrais, como questões de \isi{movimento} e de complexidade,\is{complexidade sintática} as crianças com SW apresentam dificuldades em mecanismos que envolvem exceções. Estes resultados sugerem que as alterações morfossintáticas em crianças com SW advêm de um défice na recuperação da informação lexical enquanto o sistema computacional da linguagem parece permanecer intacto.

Relativamente à \isi{Síndrome de Down} (SD), vários estudos referem que, apesar da heterogeneidade interindividual, quando considerados diferentes subsistemas da linguagem, o domínio morfossintático é aquele que apresenta maior grau de afeção. Crianças com SD\is{Síndrome de Down} revelam menor \isi{extensão média do enunciado} quando comparadas com crianças com desenvolvimento típico emparelhadas em termos de QI não-verbal, bem como quando comparadas com crianças com défices cognitivos de causa desconhecida. Alguns estudos longitudinais referem um declínio pela altura da adolescência ou da idade adulta em indivíduos com SD,\is{Síndrome de Down} no que diz respeito à sua performance sintática \citep{roberts_etal2007}. 

De acordo com \cite{ringclahsen2005} foram encontrados padrões distintos de perturbação em crianças com SD\is{Síndrome de Down} e SW. Num estudo levado a cabo por estes autores, estruturas como passivas\is{passiva} e ligações anafóricas foram estudadas em adolescentes. Os participantes com SD\is{Síndrome de Down} revelaram mais dificuldades em interpretar passivas\is{passiva} e frases com pronomes \isi{reflexos}, enquanto apresentaram um melhor desempenho em frases ativas e em frases com pronomes não-reflexos. Frequentemente forneceram respostas invertidas, em particular nas frases passivas\is{passiva} perante as quais interpretaram o primeiro sintagma nominal como o agente da frase. Quanto aos participantes com SW, verificou-se que nas tarefas que envolviam ligações anafóricas e frases ativas/passivas,\is{passiva} e tal como já tinha sido estudado por Clahsen \& Almazan (1998 \textit{apud} \citealt{guasti2002}), não foram reveladas dificuldades significativas. Este padrão distinto, em termos de desempenho, entre sujeitos com SD e SW que possuem idades mentais e QIs semelhantes sugere que os problemas experienciados pelos adolescentes com SD\is{Síndrome de Down} relativamente às passivas\is{passiva} e ligação anafórica não poderão ser devidos aos baixos níveis de inteligência. Estes resultados servem também para indagar acerca da natureza das dificuldades apresentadas, sendo que não se trata de um simples atraso em relação à gramática das crianças com desenvolvimento típico, uma vez que são revelados resultados opostos nos dois grupos. Ou seja, no desenvolvimento típico é esperado que aos 3 anos de idade as crianças compreendam frases com pronomes \isi{reflexos}, mas que aos 4 anos as crianças ainda cometam erros de interpretação relativamente a pronomes não-reflexos, o oposto do que se observa em indivíduos com SD.\is{Síndrome de Down}

Também o estudo da aquisição da linguagem em crianças com Perturbações do Espectro do Autismo (PEA) se apresenta como importante, uma vez que as dificuldades na aquisição linguística continuam a ser uma questão de grande interesse para os pais, sendo este um dos sinais mais comuns que alerta para a existência de alguma não-conformidade em relação ao desenvolvimento considerado normal \citep{lord_etal2004}. 

Apenas muito recentemente tem surgido interesse no estudo do desenvolvimento sintático em crianças com PEA, embora vários tenham sido já os estudos que têm oferecido importantes contributos para a temática. \cite{perovicjanke2013}, dedicando-se ao estudo da aquisição de estruturas sintáticas complexas neste grupo, descrevem um incompleto ou imperfeito domínio de uma série de estruturas sintáticas relatado em crianças e adultos com autismo abarcando vários níveis do espectro, como sendo frases relativas, \isi{interrogativas} QU, interpretação de \isi{reflexos} ou passivas.\is{passiva} 

Apesar de vários autores considerarem a possibilidade de uma etiologia de base comum entre PEL e PEA \citep{tagerflusbergjoseph2003,ruser_etal2007,leyfer_etal2008}, em estudos que visam comparar a performance de crianças com as duas perturbações, foram encontradas diferenças significativas de desempenho em determinadas tarefas, tais como tarefas de \isi{repetição} de frases, sendo que as crianças com PEL apresentaram desempenhos inferiores \citep{whitehouse_etal2008}. 

No que diz respeito à construção de \isi{interrogativas}, os dois grupos de crianças utilizam as estratégias mais simples com maior frequência do que as crianças com desenvolvimento típico, sendo que, tal como as crianças com PEL, também as crianças com PEA tendem a evitar a \isi{complexidade sintática}. No entanto, crianças com PEA diferem das crianças com PEL, produzindo uma maior número de perguntas inapropriadas para a situação específica da tarefa \citep{tuller_etal2012}. 

Como descrito, vários poderão ser os défices revelados, pelo que durante o processo de avaliação é importante identificar quais as alterações sintáticas que caracterizam a criança de forma a diagnosticar e tratar casos de perturbação sintática o mais precocemente possível. 

\section{A avaliação do conhecimento sintático em alguns instrumentos de avaliação}
\label{sec:martins_avaliacao_conhecimento}

A escassez de testes de avaliação normalizados para a população portuguesa até há, aproximadamente, 20 anos atrás fez com que o \isi{discurso espontâneo} fosse a técnica mais utilizada pelos terapeutas da fala na recolha de determinadas estruturas sintáticas. Mas devido a todas as condicionantes atrás referidas houve necessidade de serem desenvolvidas técnicas de \isi{elicitação} mais estruturadas e utilizadas quer em contexto experimental quer clínico. Atualmente, em Portugal estão disponíveis alguns instrumentos que avaliam as diversas componentes da linguagem na criança, incluindo a sintaxe. Apesar da sua variedade, neste subcapítulo iremos apenas descrever com maior detalhe os instrumentos que apresentam dados normativos para a população portuguesa e/ou que são frequentemente utilizados pelos terapeutas da fala em contexto clínico. São eles, por ordem de data de publicação/criação, o Teste de Avaliação da Linguagem Oral - ALO \citep{simsim1997}; a Grelha de Observação da Linguagem – Nível escolar (GOL-E) \citep{suakaysantos2003}; o Teste de Identificação de Competências Linguísticas – TICL \citep{viana2004}; o Teste de Avaliação da Linguagem na Criança – TALC \citep{suakaytavares2006}; o teste Schlichting: Teste de Avaliação da Competência Sintática - Sin:TACS \citep{vieir2011}; e o Teste de Linguagem ALPE \citep{mendes_etal2014}. É importante referir que a GOL-E não apresenta dados normativos para o português europeu. No entanto, a sua menção neste subcapítulo deve-se ao facto de, não só ser um teste frequentemente utilizado na prática clínica, como também ser o único exclusivamente direcionado para idades escolares (a ALO abrange idades pré-escolares e escolares). A existência de testes sintáticos para idades escolares é de especial relevância, tendo em conta que algumas estruturas sintáticas apenas são consideradas adquiridas a partir desta fase, como é o caso das passivas\is{passiva} \citep{simsim1997}.

O teste Sin:TACS está ainda em fase de publicação e, como tal, não pertence ao grupo de instrumentos frequentemente utilizados na prática clínica. No entanto, justifica-se a sua menção neste subcapítulo por ser o único instrumento aferido para a população pré-escolar portuguesa focado exclusivamente na sintaxe, em oposição aos restantes testes, que têm como objetivo a obtenção de um perfil linguístico relativamente às várias componentes da Gramática.\largerpage[2]

As tabelas \ref{tab:martins_alo}--\ref{tab:martins_tl-alpe} descrevem de forma resumida os subtestes, tipos de tarefas e exemplos de aspetos (morfo)sintáticos que compõem cada um destes testes.\newpage

\begin{table}[p]
\caption{Descrição sumária das características sintáticas do teste ALO}
\resizebox{\textwidth}{!}{
\begin{tabular}{p{4cm}p{4cm}p{4cm}}
\lsptoprule
Subtestes e tipo de tarefas                                                                                                                                                               & Aspetos (morfo)sintáticos avaliados                                                           & Exemplos                                                                                                                                                                \\
\midrule
\textbf{Compreensão de estruturas complexas}. Avaliada através da resposta a questões (tipo-QU) colocadas pelo avaliador acerca das frases previamente proferidas pelo mesmo.                      & Frases simples na voz ativa e passiva; Frases complexas envolvendo coordenação e subordinação & a) Hoje ou vamos à feira ou vamos ao jardim. Onde vamos hoje?; b) O cão do meu vizinho ladra sempre que me vê chegar da escola. Quando é que o cão do meu vizinho ladra? \\
\textbf{Completamento de frases}. Tarefa de Produção Elicitada (avalia a produção morfossintática). A criança terá de produzir uma ou mais palavras em falta numa frase proferida pelo examinador. & Produção de nomes; artigos; pronomes e \isi{verbos}                                                 & a) O Bruno estava a ver na televisão um \underline{\hspace{3em}} de terror (\textit{filme}); b) O macaco subiu à \underline{\hspace{3em}} e pôs-se a comer a banana (\textit{árvore})                                          \\
\textbf{Reflexão morfossintática}. Tarefa de Juízo de Gramaticalidade. A criança terá de identificar se a frase é correta gramaticalmente e corrigi-la em caso negativo.                            & Concordância verbal; ordem de palavras; conjunções, preposições; etc.                         & a) *Os óculos da Maria era cinzento; b) *Muro cavalo o saltou o; c) *O bebé fez barulho antes que adormecer\\
\lspbottomrule
\end{tabular}}
\label{tab:martins_alo}
\end{table}
\begin{table}[p]
\caption{Descrição sumária das características sintáticas do teste \mbox{GOL-E}}
\resizebox{\textwidth}{!}{
\begin{tabular}{p{4cm}p{4cm}p{4cm}}
\lsptoprule
Subtestes e tipo de tarefas                                                                                                                                                                                                                         & Aspetos (morfo)sintáticos avaliados                                             & Exemplos                                                                                 \\
\midrule
\textbf{Reconhecimento de frases agramaticais}. Tarefa de Juízo de Gramaticalidade. A criança ouve as frases proferidas pelo examinador e é-lhe perguntado se a frase é ou não gramatical. Em caso negativo, é pedido à criança que produza a frase correta. & Pronomes clíticos; pronomes relativos; concordância nominal; concordância, etc. & a) *Ele se penteia-se sozinho; b) *O livro está na mesa é meu; c) *Ele comeu duas banana \\
\textbf{Coordenação e subordinação de frases}. Tarefa de Produção Elicitada. A criança ouve duas frases isoladas e terá de produzir apenas uma através de um processo de coordenação ou subordinação.                                                        & Frases complexas envolvendo coordenação e subordinação                          & a) O João caiu. Fez uma ferida; b) A chávena caiu. A chávena não se partiu               \\
\textbf{Ordem de palavras na frases}. Tarefa de Produção Elicitada. A criança é convidada a ordenar corretamente palavras por forma a construir uma frase.                                                                                                   & Frases simples; Interrogativas tipo QU                                          & a) chora bebé o; b) casa onde a é; c) anos tens quantos                                  \\
\textbf{Derivação de palavras}. Tarefa morfológica de completamento de palavras.                                                                                                                                                                             & Processos de formação de palavras através de nominalização e adjetivalização    & a) O homem que pinta é um pin\underline{\hspace{3em}} (\textit{tor}); b) Um rapaz que gosta de comer muito é um co\underline{\hspace{3em}} (\textit{milão})         \\ 
\lspbottomrule
\end{tabular}}
\label{tab:martins_gol-e}
\end{table}

\begin{table}
\caption{Descrição sumária das características sintáticas do teste TICL}
\resizebox{\textwidth}{!}{
\begin{tabular}{p{4cm}p{4cm}p{4cm}}
\lsptoprule
Subtestes e tipo de tarefas                                                                                                                                                              & Aspetos (morfo)sintáticos avaliados                                                           & Exemplos                                                                                                                                                                                                                                                                                                            \\
\midrule
\textbf{Conhecimento morfossintático}. Tarefas de produção elicitada. Completamento de palavras numa frase.                                                                                       & Concordância género-número; Pretérito perfeito; Plurais irregulares; Grau dos adjetivos; etc. & a) O balão é redondo. A bola é \underline{\hspace{3em}} (\textit{redonda}). A bola e o balão são \underline{\hspace{3em}} (\textit{redondos}); b) Neste desenho ele está a pintar mas neste desenho ele já \underline{\hspace{3em}} (\textit{pintou}); c) Este bolo não é bom. Este bolo é bom. Este bolo é ainda \underline{\hspace{3em}} (\textit{melhor}); d) Aqui está um cão. Aqui estão dois \underline{\hspace{3em}} (\textit{cães}) \\
\textbf{Conhecimento morfossintático}. Compreensão de estruturas complexas avaliada através da resposta a questões (sintagma-Q) colocadas pelo avaliador acerca das frases proferidas pelo mesmo. & Frases SVO e frases complexas                                                                 & a) O carro azul da tia teve um furo. De que cor é que é o carro da tia?; b) O leão que o tigre mordeu saltou por cima da cobra. Quem é que saltou por cima da cobra?                                                                                                                                                \\
\textbf{Reflexão sobre a língua}. Tarefa de Juízo de Gramaticalidade. A criança terá de repetir a frase proferida pelo examinador, aferir a sua gramaticalidade e corrigi-la.                     & Concordância verbal; Pronome                                                                  & a) *Os meninos joga à bola; b) *Mim pendura isto  \\
\lspbottomrule
\end{tabular}}
\label{tab:martins_TICL}
\end{table}

\begin{table}
\caption{Descrição sumária das características sintáticas do teste TALC}
\resizebox{\textwidth}{!}{
\label{tab:martins_talc}
\begin{tabular}{p{4cm}p{4cm}p{4cm}}
\lsptoprule
Subtestes e tipo de tarefas                                                                                                                                    & Aspetos (morfo)sintáticos avaliados                                                                                                               & Exemplos                                                                                                                                                                                                                                                                                                                                                        \\
\midrule
\textbf{Compreensão de frases complexas}. Tarefa de seleção de imagem. A criança terá de apontar para a imagem que corresponde à frase proferida pelo examinador.       & Compreensão de frases relativas, passivas\is{passiva} e expressões correlativas                                                                               & a) O homem que está a escovar o cão é magro; b) O elefante está a ser empurrado pelo touro; c) Nem o livro nem o copo estão em cima da mesa                                                                                                                                                                                                                     \\
\textbf{Produção de constituintes morfossintáticos}. Tarefa de Produção Elicitada. A criança terá de completar frases ou responder a questões colocadas pelo examinador & Plurais regulares e irregulares; preposições e conjunções; flexão verbal de pessoa e tempo; argumentos do verbo\is{verbos} (objeto direto e objeto indireto) & a) Olha tantos brinquedos que o menino tem aqui. Aqui estão dois \underline{\hspace{3em}} (\textit{leões}) e aqui dois \underline{\hspace{3em}} (\textit{carros}) e aqui dois \underline{\hspace{3em}} (\textit{pincéis}) e aqui duas \underline{\hspace{3em}} (\textit{bolas}).; b) O menino está sentado à mesa e a mãe não está contente porque ele tem as mãos sujas. Eu acho que ele tem \underline{\hspace{3em}} (resposta esperada: \textit{que lavar as mãos}/\textit{de lavar as mãos})\\
\lspbottomrule
\end{tabular}}
\end{table}

\begin{table}
\caption{Descrição sumária das características sintáticas do teste Sin:TACS}
\resizebox{\textwidth}{!}{
\label{tab:martins_sintacs}
\begin{tabular}{p{4cm}p{4cm}p{4cm}}
\lsptoprule
Subtestes e tipo de tarefas                                                                                        & Aspetos (morfo)sintáticos avaliados                                                                                                 & Exemplos                                                                                                    \\
\midrule
\textbf{Imitação Exata}. É pedido à criança que repita a mesma estrutura sintática proferida momentos antes pelo examinador & Predicação verbal (intransitivos, transitivos); Frases complexas coordenadas, etc.                                                  & a) Este dorme; b) Este vai comer; c) Este voa e este não                                                    \\
\textbf{Imitação com variação}. A criança produz a mesma estrutura sintática que o examinador mas com variação lexical      & Frases subordinadas completivas; Frases subordinadas adverbiais; frases subordinadas relativas; Frases complexas adversativas, etc. & a) Eu acho que levo o carro; b) Este tem um pincel para pintar; c) A menina que partiu a perna tem um balão \\
\textbf{Produção elicitada}. A criança terá de completar uma frase e responder a uma questão colocada pelo examinador       & Frase passiva; Frase subordinada adverbial causal                                                                                   & a) O leão morde este palhaço. E este palhaço \underline{\hspace{3em}} (resposta esperada: \textit{É mordido pelo cão})         \\   
\lspbottomrule
\end{tabular}}
\end{table}

\begin{table}
\caption{Descrição sumária das características sintáticas do teste TL-ALPE}
\label{tab:martins_tl-alpe}
\resizebox{\textwidth}{!}{
\begin{tabular}{p{4cm}p{4cm}p{4cm}}
\lsptoprule
Subtestes e tipo de tarefas                                                                                                                                                                                            & Aspetos (morfo)sintáticos avaliados                                                                    & Exemplos                                                                                                                                                                                                                                                                                                     \\
\midrule
\textbf{Produção de frases simples e complexas}. Tarefa de produção elicitada. Através do uso de imagens, é esperado que a criança produza frases simples e complexas através da colocação de questões por parte do examinador. & Frases simples e frases com coordenação                                                                & a) ``O que é que a menina está a fazer?'' (resposta esperada: {[}A menina{]} lava/está a lavar os dentes); b) ``O que é que o menino tem?'' (resposta esperada: O menino tem um chocolate e um sumo)                                                                                                         \\
\textbf{Produção de constituintes morfossintáticos}. A criança deverá completar frases proferidas pelo examinador ou responder a questões.                                                                                      & Concordância de número e género; possessivos; conjugação verbal                                        & a) Aqui está uma \underline{\hspace{3em}} (\textit{bola}), aqui estão três \underline{\hspace{3em}} (\textit{bolas}); b) O par da rapariga é o \underline{\hspace{3em}} (\textit{rapaz}); c) De quem é este olho? (resposta esperada: \textit{É teu}); d) O que vai acontecer aos copos? (resposta esperada: \textit{Vão cair})                                                                        \\
\textbf{Compreensão de frases simples e complexas}. A criança deverá responder a questões proferidas pelo examinador.                                                                                                           & Frases simples na voz ativa e passiva; frases complexas coordenadas; frases complexas subordinadas     & a) A Carolina mostrou a sua mochila vermelha à Ana. A quem é que a menina mostrou a mochila?; b) A boneca da Joana foi comprada pela Rita. Quem comprou a boneca?; c) O João pegou no lápis e fez um desenho. O que é que o João fez primeiro?; d) A mãe pediu ao João que pusesse a mesa. Quem pôs a mesa? \\
\textbf{Agramaticalidade morfossintática}. Tarefa de Juízo de Gramaticalidade. A criança deverá aferir se a frase é agramatical e em caso negativo justificar.                                                                  & Concordância de número entre o sujeito e o verbo; Omissão de objeto direto; Concordância dentro do SN. & a) *Os meninos brinca no parque; b) *A mãe comprou ao Pedro; c) *A menino foi ao circo; d) *O Manuel leu dois livro   \\
\lspbottomrule
\end{tabular}}
\end{table}
\clearpage
Fazendo uma breve análise ao quadro apresentado, verifica-se que os testes aqui descritos utilizam tarefas de \isi{compreensão} sintática (ex: apontar para a imagem correspondente a uma frase proferida pelo examinador) e/ou tarefas de \isi{elicitação} verbal como completar frases, ou responder a questões. Incluem ainda testes que visam avaliar a capacidade para emitir um \isi{juízo de gramaticalidade}. 

Apesar de inúmeras vantagens na utilização de testes de avaliação mais formais, estes não são isentos de limitações no que diz respeito às técnicas utilizadas bem como ao tipo de estruturas sintáticas avaliadas.

Em relação às técnicas usadas na avaliação sintática, há a considerar que a idade da criança pode ser um fator que condiciona os resultados obtidos. Isto porque crianças mais novas podem ter dificuldade em perceber o que lhes é pedido, como é o caso das tarefas de \isi{juízo de gramaticalidade}. Este tipo de tarefa, além de assumir um elevado grau de complexidade, não permite avaliar qual a interpretação da criança para a estrutura testada, uma vez que apenas é questionada sobre se tal frase é gramaticalmente correta \citep{white2003}. Para além disso, algumas tarefas (tais como imitação de frases ou resposta a questões) não apresentam um contexto funcional, podendo pôr em causa a motivação intrínseca da criança para participar no teste. De acordo com \cite{vinther2002}, criar um objetivo comunicativo usando imagens ou objetos favorece a \isi{produção} de determinadas estruturas sintáticas, no sentido de existir motivação por parte da criança de expressar algo a alguém. No que diz respeito à resposta a questões, adiciona-se ainda a possibilidade de a criança não compreender a estrutura frásica utilizada, que corresponde na maioria dos caso a \isi{interrogativas} QU. Estas estruturas são complexas do ponto de vista gramatical, vindo a ser destacadas como problemáticas nos grupos com perturbações de linguagem. 

Relativamente às estruturas sintáticas que caracterizam estes testes, verifica-se que nem todas permitem um diagnóstico preciso de perturbações da linguagem sintáticas. Afonso (2011) analisou os itens morfossintáticos em três testes (GOL-E; ALO; e TALC) no que diz respeito à sua precisão no diagnóstico de casos de PEL-S. Para esse efeito, a autora comparou as estruturas sintáticas alvo presentes nestes testes com resultados em artigos nacionais e internacionais chegando à conclusão de que determinados \isi{marcadores clínicos} na detecção de PEL-S, tal como a \isi{compreensão} e \isi{produção} de frases relativas, passivas\is{passiva} ou \isi{interrogativas} são inexistentes ou pouco representados. 

Há também a considerar que vários testes incluem itens que avaliam morfologia e não exclusivamente sintaxe. No entanto, a morfologia verbal é ainda uma área cujo impacto em perturbações da linguagem necessita de um estudo mais aprofundado, quer para o português (europeu e do Brasil)\largerpage quer para outras línguas. 

Apesar de algumas limitações, os testes que avaliam a competência sintática apresentam inúmeras vantagens. São de rápida utilização e cotação e permitem a recolha de enunciados que poderiam não ser produzidos pela criança num contexto mais natural, como é o caso do \isi{discurso espontâneo}. Permitem, ainda, que seja possível comparar o desempenho de uma criança face a um grupo normativo, chegando dessa forma a um diagnóstico mais fiável e válido.

\section{Conclusão}
\label{sec:martins_conclusao}

A experiência profissional leva à conclusão de que o tempo e os recursos despendidos na avaliação inicial de qualquer caso se transformam numa mais-valia, uma vez que permitem definir a linha de base de um indivíduo, caracterizando deste modo as suas áreas fortes e fracas em termos de comunicação e linguagem, o que irá permitir traçar o caminho mais adequado para cada caso. 

Medir a competência sintática na criança é, provavelmente, uma das áreas mais difíceis na avaliação da linguagem da criança. Talvez por esse motivo, e corroborando as afirmações de \cite{schlichtinglutjespelberg2003}, a sintaxe tenha sido uma área frequentemente negligenciada pelos terapeutas da fala no processo de avaliação e intervenção ao longo dos anos. Um aumento de estudos nacionais e internacionais sobre o comportamento sintático de determinadas perturbações, tais como PEL, \isi{Síndrome de Williams}, \isi{Síndrome de Down} e \isi{Perturbações do Espectro Autista}, tem sido preponderante na identificação de estruturas sintáticas específicas consideradas como \isi{marcadores clínicos}. Um maior conhecimento do comportamento sintático em crianças com desenvolvimento típico versus crianças com patologia assim como uma maior variedade e disponibilidade de testes de avaliação da linguagem (e mais especificamente da sintaxe) para uso clínico tem permitido nos últimos tempos uma identificação mais precoce e eficaz de casos de perturbação sintática. Uma das consequências imediatas é o evitamento de uma intervenção tardia de impacto nefasto ao nível emocional e académico nas fases de adolescência ou mesmo adulta (especialmente em casos de PEL-S). 

Considerando que o desenvolvimento típico da linguagem é um processo complexo e multifactorial, uma maior precisão na avaliação de um determinado perfil sintático torna-se crucial na distinção de uma criança que se situa num extremo baixo da normalidade e outra que se atrasa ou desvia do normal desenvolvimento linguístico.

\section*{Abreviaturas}
Adicionar: PEA - Perturbação do Espectro Autista; PEL – Perturbação específica da linguagem; PEL-S – Perturbação específica da linguagem sintática; SW – Síndrome de Williams; SD – Síndrome de Down.




































{\sloppy
\printbibliography[heading=subbibliography,notkeyword=this]
}
\end{document}