\documentclass[output=paper]{LSP/langsci} 
\author{Carmen Matzenauer\affiliation{Universidade
Católica de Pelotas}\lastand 
Teresa Costa\affiliation{Universidade de Lisboa, Centro de Linguística}
}
\title{Aquisição da fonologia em língua materna: os segmentos}  
\abstract{}
\ChapterDOI{10.5281/zenodo.889421}
\maketitle
\begin{document}
\section{Introdução}
\label{sec:matzenauer_intro}

A aquisição da componente segmental de uma língua envolve o domínio de um intrincado sistema de especificidades fonológicas. A complexidade desse processo é intensificada não só pela constituição interna dos segmentos, uma vez que os traços distintivos\is{traço distintivo} apresentam diferentes estádios de aquisição, mas também pela existência de padrões combinatórios entre os sons e as unidades suprassegmentais, tais como a sílaba e a palavra. Desta forma, iniciaremos este capítulo com uma breve descrição das principais caraterísticas fonológicas inerentes aos sistemas consonântico e vocálico no português (variedades europeia -- PE e brasileira -- PB) assim como dos respetivos padrões fonotáticos. Essa descrição visa essencialmente definir as ferramentas teóricas necessárias para a compreensão dos padrões de aquisição segmental, apresentados nas secções \ref{sec:matzenauer_padroes_mundo} e \ref{sec:matzenauer_padroes_portugues}.

\paragraph*{O sistema segmental do português}

No plano fonológico, o Português apresenta dezanove segmentos consonânticos com valor distintivo: \textipa{/p b t d k g f v s z S Z l L R \;R m n \textltailn/}. A constituência interna destes sons determina a sua organização em classes naturais, em função dos traços de ponto e do modo de articulação (PA e MA, respetivamente), assim como de \isi{vozeamento}.

\begin{table}
  \begin{tabular}{lllll}
    \lsptoprule
\multicolumn{2}{l}{Modo de Articulação - MA} & \multicolumn{3}{l}{Ponto de Articulação - PA} \\
\midrule
Oclusivas \textipa{/p b t d k g/}    & {[$-$contínuo, $-$soante]}      & \multicolumn{2}{l}{labial\is{ponto de articulação!labial}}       & \textipa{/p b f v m/}        \\
Fricativas\is{modo de articulação!fricativa} \textipa{/f v s z S Z/}   & {[$+$contínuo, $-$soante]}      & Coronal\is{ponto de articulação!coronal}       & {[$+$ant]}       & \textipa{/t d s z n l R/}       \\
Nasais\is{modo de articulação!nasal} \textipa{/m n \textltailn/}      & {[$+$soante, $+$nasal]}         &               & {[$-$ant]}       & \textipa{/S Z \textltailn} \textipa{L/}      \\
Líquidas\is{modo de articulação!líquida} \textipa{/l L R \;R/}    & {[$+$soante, $\pm$lateral]}  & \multicolumn{2}{l}{Dorsal}       & \textipa{/k g \;R/}       \\
\lsptoprule
\multicolumn{2}{l}{Consoantes vozeadas}      & \multicolumn{3}{l}{Consoantes não vozeadas}   \\
\midrule
\multicolumn{2}{l}{\textipa{/b d g v z Z m n \textltailn} \textipa{l L R \;R/}}                    & \multicolumn{3}{l}{\textipa{/p t k f s S/}}\\
\lspbottomrule
  \end{tabular}
  \caption{Classificação dos sons consonânticos - MA, PA e vozeamento}
  \label{tab:matzenauer_consoantes}
\end{table}

\paragraph*{O sistema vocálico do português}

O sistema fonológico do Português integra um conjunto de sete vogais: \textipa{/i e E a u o O/}. No plano fonético, o leque de segmentos vocálicos é mais alargado, na decorrência da atuação de processos fonológicos como o vocalismo átono e a nasalização.\is{modo de articulação!nasal} Assume-se também que as semivogais/glides \textipa{[j]} e \textipa{[w]} constituem realizações fonéticas de vogais fonológicas subjacentes (\textipa{/i/} e \textipa{/u/}). Os principais traços de PA,\is{traço distintivo} que permitem distinguir os diferentes segmentos vocálicos, são listados na Tabela \ref{tab:matzenauer_vogais}.

\begin{table}
  \begin{tabular}{lll}
    \lsptoprule
                        & {[}$-$recuado{]} & {[}$+$recuado{]} \\
    \midrule
{[$+$alta]}           & \textipa{/i/}                & \textipa{/u/}                \\
{[$-$alta, $-$baixa]} & \textipa{/e}                & \textipa{/o/}                \\
{[$+$baixa]}          & \textipa{/E/}                & \textipa{/a/ /O/}             \\
\lspbottomrule
  \end{tabular}
  \caption{Classificação das vogais - PA (Ponto de Articulação)}
  \label{tab:matzenauer_vogais}
\end{table}

\paragraph*{Padrões de distribuição segmental na palavra e na sílaba}

As consoantes do português são contrastivas em Ataque silábico,\footnote{Onset, em PB.} em posição inicial e medial de palavra, com exceção dos sons \textipa{/R/}, \textipa{/L/} e \textipa{/\textltailn/}, que não ocorrem em início de palavra. Em Ataque ramificado, as sequências fonológicas são compostas por segmentos oclusivos\is{modo de articulação!oclusiva} ou fricativos\is{modo de articulação!oclusiva} labiais seguidos das líquidas\is{modo de articulação!líquida} anteriores (como em \textit{pr}ato ou \textit{fl}oresta). Em posição de Coda, assume-se comummente na literatura teórica que três segmentos fonológicos estão subjacentes às produções, quer no PE quer no PB: \textipa{/l R s/} \citep{mateusdandrade2000}. A realização fonética varia em função do contexto segmental adjacente à direita e da variedade da língua dos falantes (por exemplo, o alvo sal é produzido como sa\textipa{[\textltilde]} no PE e como sa\textipa{[w]} no PB). No PB, tem sido defendida a existência do arquifonema nasal\is{modo de articulação!nasal} /N/ que surge em posição de Coda em palavras como ca\textit{n}ta e fi\textit{m} \citep{camarajunior1970,lamprecht_etal2004}.\footnote{Já no PE, a nasalidade das vogais é representada através de um autossegmento nasal\is{modo de articulação!nasal} associado ao nó Núcleo \citep{mateusdandrade2000}.}

Quanto aos segmentos vocálicos, estes ocorrem em núcleo silábico, que poderá ser ramificado, na presença de ditongos, em vocábulos como p\textit{ai} ou m\textit{au}.

Em suma, os segmentos consonânticos e vocálicos do Português apresentam diversas caraterísticas fonológicas e articulatórias e diferentes padrões distribucionais na sílaba e na palavra (para mais informação, consultar \citet{freitas2017} e \citet{santos2017}, neste volume). Nas secções que se seguem, será analisada a forma como essas especificidades segmentais e fonotáticas são adquiridas pelos falantes.

\section{A investigação em desenvolvimento segmental}
\label{sec:matzenauer_investigacao}

A aquisição da linguagem pela criança tem constituído um foco privilegiado de estudos nas últimas décadas. Neste âmbito, os fenómenos característicos do processo de aquisição fonológica e a definição de estádios de desenvolvimento têm sido descritos e explicados à luz de diferentes modelos teóricos, dos quais se mencionam, pela frequência dos estudos sobre a formação de inventários segmentais, a \textit{Fonologia Linear}\is{fonologia linear} \citep{chomskyhalle1968}; a \textit{Fonologia Natural}\is{fonologia natural} \citep{stampe1973}; a \textit{Geometria de Traços} \citep{clements1985,clementshume1995} e a \textit{Teoria da Otimidade} \citep{princesmolensky1993,mccarthyprince93}. Cada uma das abordagens teóricas atribui ao fenómeno da aquisição fonológica uma interpretação diferenciada.

\subsection{Fonologia Linear\is{fonologia linear} \citep{chomskyhalle1968}}
\label{subsec:matzenauer_linear}

Na Fonologia Linear\is{fonologia linear} ou Generativa Clássica, proposta por \citet{chomskyhalle1968}, são
fundamentais as noções de \textit{regras} e de \textit{traços fonológicos}: os segmentos são conjuntos de traços distintivos\is{traço distintivo} binários (matrizes de traços sem ordenamento), e o mapeamento entre a representação fonológica, abstrata, e a representação fonética dá-se por meio de regras, num processamento linguístico que prevê derivação serial. A aquisição dos segmentos implica a incorporação, no sistema fonológico da criança, da coocorrência de traços que os caracteriza e do valor contrastivo dos mesmos.

Sob esses pressupostos, o segmento \textipa{/s/}, por exemplo, é o resultado da coocorrência dos traços [$-$soante, $+$contínuo, $+$coronal,\is{ponto de articulação!coronal} $+$anterior, $-$vozeado],\is{vozeamento!vozeado} enquanto o segmento \textipa{/S/} é caracterizado por uma matriz de traços semelhante, diferenciando-se de \textipa{/s/} apenas porque contém o traço [$-$anterior]: a integração desses segmentos no inventário fonológico da criança implica a aquisição dessas coocorrências de traços, bem como do valor distintivo do traço [$\pm$anterior].\is{traço distintivo}

Neste quadro teórico, o processo de substituição de uma consoante por outra, no decurso da aquisição fonológica, é interpretado como a aplicação de uma regra de alteração de traço(s), no mapeamento entre o \textit{input} fonológico e o \textit{output} fonético. Nesta perspetiva, a aquisição fonológica surge como um processo de \textit{aquisição de regras}. 

\subsection{Fonologia Natural\is{fonologia natural} \citep{stampe1973}}
\label{subsec:matzenauer_natural}

Na Fonologia Natural,\is{fonologia natural} proposta por \citet{stampe1973}, é central a noção de \textit{processos fonológicos}, que são considerados naturais, universais e inatos. Estes constituem-se por operações mentais de simplificação, através dos quais segmentos ou sequências que se mostram difíceis para a criança são substituídos por outros sem a propriedade complexa. Nos processos fonológicos incluídos nesta proposta teórica, divididos em `processos de estrutura silábica' e `processos de substituição', encontram-se, por exemplo: a redução de encontro consonantal, o apagamento de líquida\is{modo de articulação!líquida} em coda, a substituição por oclusiva,\is{modo de articulação!oclusiva} a anteriorização, a posteriorização e a substituição de líquida.\is{modo de articulação!líquida}

Neste modelo, sendo inatos os processos fonológicos, a aquisição fonológica implica a eliminação, ordenação ou limitação daqueles que não integram a gramática do sistema alvo da criança -- a unidade de análise neste contexto não é os traços, mas os processos.

\subsection{Geometria de Traços\is{geometria de traços} \citep{clements1985,clementshume1995}}
\label{subsec:matzenauer_geometria}

Enquanto propostas teóricas não lineares, a Fonologia Autossegmental e a Geometria de Traços\is{geometria de traços} assumem o pressuposto de que os segmentos são constituídos por traços organizados numa hierarquia, de modo a representar a possibilidade de cada traço funcionar isoladamente (como autossegmento) ou em conjuntos solidários com outros traços, vinculados ao mesmo nó de classe. Os segmentos passam a ter uma estrutura interna, formalizada por meio de uma Geometria de Traços,\is{geometria de traços} com configuração arbórea. As regras são representadas pela ligação ou desconexão de linhas de associação nessa estrutura.

Neste modelo teórico, o desenvolvimento fonológico é entendido como a construção gradual da estrutura que caracteriza os sons da língua, por meio da ligação sucessiva de diferentes \textit{tiers}. De acordo com esta abordagem, a criança iniciaria a construção do seu sistema com estruturas básicas, não marcadas, responsáveis pelas grandes classes de sons das línguas: obstruintes,\is{modo de articulação!obstruinte} nasais,\is{modo de articulação!nasal} líquidas\is{modo de articulação!líquida} e vogais \citep{matzenauer1996}, sendo \textipa{/p, t, m, n/} as primeiras consoantes a emergir na constituição do inventário fonológico. Ao considerar-se, por exemplo, o emprego da forma fonética \textipa{[s5mi\textprimstress nE]} para \textit{chaminé}, a interpretação, neste quadro teórico, é o reconhecimento da não ligação, no sistema da criança, do traço [$-$anterior] à estrutura interna do segmento \textipa{/S/}, e, consequentemente, da ausência, no inventário consonantal, do valor contrastivo do traço [$\pm$anterior]. A incorporação da fricativa\is{modo de articulação!fricativa} palatal no inventário de segmentos da criança ocorrerá com a aquisição do valor contrastivo do traço [$\pm$anterior].

\subsection{Teoria da Otimidade\is{Teoria da Otimidade/Otimalidade} \citep{princesmolensky1993,mccarthyprince93}}
\label{subsec:matzenauer_ot}

A Teoria da Otimidade (\textit{Optimality Theory} - OT)\is{Teoria da Otimidade/Otimalidade} propõe a existência de um \textit{input} (representação fonológica) e de um \textit{output} (representação fonética) e de uma relação entre os dois, sendo o mapeamento entre esses níveis, diferentemente dos outros modelos teóricos, mediado por restrições, num processamento linguístico em paralelo. As restrições, que são requisitos estruturais universais e violáveis, podem militar a favor da preservação, no \textit{output}, das unidades do \textit{input} (restrições de fidelidade) ou podem militar a favor de um \textit{output} não marcado (restrições de marcação). A OT\is{Teoria da Otimidade/Otimalidade} pressupõe que a Gramática Universal (GU) contém um conjunto de restrições universais \textsc{Con} (Constraint), bem como dois mecanismos formais: \textsc{Gen} (Generator) e \textsc{Eval} (Evaluator): o primeiro cria objetos linguísticos, isto é, candidatos potenciais a \textit{outputs}, e \textsc{Eval} usa a hierarquia de restrições para selecionar o candidato ótimo do conjunto de candidatos produzidos por \textsc{Gen}.

Nesta abordagem teórica, a gramática de um sistema linguístico é representada por uma hierarquia de restrições que lhe é específica e a aquisição da linguagem é vista como o processo de aquisição da hierarquia de restrições que caracteriza a língua alvo, sendo que os estádios desenvolvimentais, bem como as estratégias apresentadas pelas crianças no processo de aquisição, são entendidos como reflexo do encaminhamento para a hierarquia de restrições do sistema alvo \citep{bernhardtstemberger1998}. Os estádios são descritos por algoritmos de aprendizagem, que compreendem a demoção e a promoção de restrições até a aquisição da hierarquia da língua. No início da aquisição, são ordenadas em pontos mais altos, na hierarquia, as Restrições de Marcação, como, por exemplo, \textsc{Onset} (que proíbe sílabas sem ataque) e \textsc{NoCoda} (que proíbe sílabas com coda); esse ordenamento explica o licenciamento, nos estádios mais precoces da aquisição, de estruturas não marcadas.

A análise dos dados da aquisição assenta não só na seleção do modelo teórico, mas também na escolha da metodologia mais adequada, em função da natureza do estudo. Na próxima secção, será apresentada uma reflexão acerca dos aspetos metodológicos inerentes à área da aquisição fonológica.

\section{Aspetos metodológicos}
\label{sec:matzenauer_metodologia}

A investigação na área do desenvolvimento segmental tem sido caraterizada por alguma diversidade metodológica, particularmente no que diz respeito ao tipo de amostragem, à natureza da recolha dos dados e aos critérios de análise utilizados.

Globalmente, as amostras podem ser longitudinais ou transversais. As primeiras consistem na gravação de produções verbais de uma mesma criança em intervalos regulares -- quinzenal ou mensalmente -- durante um período de tempo. Já o segundo tipo de amostragem incide em grupos de crianças em faixas etárias específicas. Se, por um lado, os estudos longitudinais são valiosos pois permitem aceder aos padrões de desenvolvimento no percurso de cada criança, estes apresentam a desvantagem de abrangerem um número restrito de falantes e como tal não permitirem a generalização dos padrões observados. Por seu lado, os estudos transversais proporcionam essa generalização, pois fornecem dados sobre grupos mais alargados de sujeitos; no entanto, neste tipo de amostragem torna-se impossível aceder ao conhecimento das etapas do desenvolvimento individual.

No que diz respeito à metodologia de recolha dos dados, esta pode ser de natureza espontânea ou experimental. No primeiro caso, as produções são recolhidas normalmente em casa da criança, em situações do quotidiano. Já em contexto experimental, o foco da recolha é mais específico e as crianças são expostas a estímulos para a produção verbal, mediante a aplicação de um desenho experimental. As recolhas espontâneas são importantes pois permitem aceder o mais aproximadamente possível àquele que será o desempenho linguístico da criança em ambiente de descontração. Por outro lado, estas recolhas colocam alguns entraves nomeadamente à transcrição dos dados, pois a situação espontânea de fala compromete o controlo da qualidade acústica. Já nos estudos experimentais, o ambiente acústico pode ser controlado, garantindo maior fiabilidade nas transcrições fonéticas; no entanto o acesso às produções das crianças é canalizado para estruturas específicas, perdendo-se informação relativa ao efetivo grau de desenvolvimento segmental em que a criança se encontra. Há a salientar, contudo, o forte grau de complementaridade existente entre estes dois tipos de recolha.

No que diz respeito ao tratamento e à análise dos dados, há vários aspetos a ter em conta. De forma a ser possível a análise do desenvolvimento segmental, o material digital recolhido tem de ser convertido em transcrição fonética. Esta tarefa é extremamente exigente (em virtude, por exemplo, do grau de qualidade acústica das gravações e das idiossincrasias do trato vocal infantil), requerendo estratégias de aferição da fiabilidade das transcrições (consenso entre transcritores).

Outro aspeto que requer atenção no domínio dos estudos em aquisição segmental é o conceito de ``adquirido''. Quantas vezes tem um segmento de ser produzido conforme o alvo para que o possamos considerar adquirido? Na verdade, diferentes estudos têm utilizado diferentes critérios \citep{bernhardtstemberger1998}. Alguns optam por considerar \textit{adquirido} acima da barreira do 50\% e \textit{estabilizado} acima dos 90\%; outros consideram a aquisição ocorrida acima dos 75\%. No entanto, numa perspetiva geral, podemos afirmar que produções consentâneas com o alvo acima dos 75\% são tratadas na investigação da área como casos em que, no mínimo, a estrutura em causa já está em fase de aquisição.

Vários estudos sobre a aquisição fonológica observada em crianças portuguesas e brasileiras têm seguido a proposta de \citet{yavas_etal1991}, que estabelece os índices discriminados a seguir, tendo em conta o emprego de segmentos, pela criança, em consonância com o alvo da língua: 

\begin{enumerate}[label=\alph*)]
\item emprego (de acordo com o alvo) inferior a 50\%: a criança não possui o segmento contrastivo;
\item emprego (de acordo com o alvo) de 51\% a 75\%: a criança possui o segmento em concorrência com o que o substitui;
\item emprego (de acordo com o alvo) de 76\% a 85\%: a criança já adquiriu o segmento, mas são registados ainda casos de substituição; 
\item emprego (de acordo com o alvo) de 86\% a 100\%: o segmento foi efetivamente adquirido pela criança.
\end{enumerate}

Atualmente, o número de \textit{corpora} disponíveis para a investigação no domínio segmental é já considerável, particularmente devido ao desenvolvimento de ferramentas que facilitam a notação, o armazenamento, a análise e a partilha dos dados em formato digital  \citep{durand_etal2014}.

\section{Padrões de aquisição segmental nas línguas do mundo}
\label{sec:matzenauer_padroes_mundo}

A investigação no campo da aquisição da linguagem teve o marco inicial nos chamados \textit{diários}, em fins do séculos XIX e início do século XX, que constituíam estudos obtidos com o acompanhamento diário de crianças, com o registo e a descrição de sons por elas produzidos em determinado período do processo de aquisição. A observação da recorrência de sons em diferentes diários, no processo de aquisição de sistemas linguísticos diversos, ofereceu as primeiras bases para a identificação de tendências universais. A proposição de uma teoria universal de aquisição da fonologia é atribuída a Roman Jakobson, pela publicação, em \citeyear{jakobson1941}, de \citetitle{jakobson1941}.\footnote{A publicação original foi em alemão: \textit{Kindersprache, Aphasie und allgemeine Lautgesetze}.}

Defendia Jakobson que há um ordenamento na aquisição das oposições fonológicas, numa sequência consistente e previsível. Os contrastes presentes no inventário fonológico da língua alvo são adquiridos pela criança sob a influência de leis linguísticas, denominadas pelo autor de \textit{leis de solidariedade irreversível}, as quais, com base na distribuição de traços fonológicos\is{traço distintivo} nas línguas do mundo, representam \textit{leis implicacionais} que estabelecem que a presença de um traço, segmento ou classe de segmentos implica a presença de outro(s) nos inventários fonológicos. Assim, considerando, por exemplo, que o inventário de todas as línguas possui consoantes anteriores, mas não necessariamente consoantes posteriores, pela \textit{lei de solidariedade irreversível} a presença, num sistema, de consoantes posteriores pressupõe a presença de consoantes anteriores. O mesmo pressuposto é aplicado ao MA: a existência de segmentos contínuos (e.g. fricativas)\is{modo de articulação!fricativa} implica a presença no sistema de segmentos não contínuos (e.g. oclusivas);\is{modo de articulação!oclusiva} e ao vozeamento:\is{vozeamento!vozeado} a ocorrência de sons vozeados\is{vozeamento!vozeado} (e.g. \textipa{/b d g/}) implica a presença prévia de sons não vozeados\is{vozeamento!não vozeado} (e.g. \textipa{/p t k/}). Interpretadas no contexto da aquisição fonológica, leis dessa natureza estabelecem, por exemplo, que as consoantes anteriores são adquiridas mais precocemente do que as consoantes posteriores; que os sons oclusivos\is{modo de articulação!oclusiva} e os não vozeados\is{vozeamento!não vozeado} estabilizam nos sistemas fonológicos em desenvolvimento antes dos sons fricativos\is{modo de articulação!oclusiva} e dos vozeados,\is{vozeamento!vozeado} respetivamente. A pressuposição é a de que as crianças irão sempre adquirir traços,\is{traço distintivo} segmentos e conjuntos de contrastes considerados não marcados antes dos marcados, no que diz respeito a propriedades acústicas e articulatórias dos sons da fala.

Embora estudos subsequentes tenham vindo contradizer algumas das leis propostas por Jakobson e salientar a limitação de ter sido desconsiderada a \textit{variação} (diferenças entre as crianças, particularidades individuais), as pesquisas sobre a aquisição de variados sistemas linguísticos têm evidenciado, de facto, tendências universais no desenvolvimento fonológico. 

Assim, tendências gerais ou padrões na aquisição da fonologia de diversas línguas, como, por exemplo, o inglês,\il{inglês} o holandês\il{holandês} e o espanhol\il{espanhol} mostram a emergência precoce de unidades não marcadas, sejam sílabas, segmentos ou traços.\is{traço distintivo} No que concerne a segmentos, tem-se mostrado padrão, na aquisição fonológica, a tendência para as seguintes sequências na aquisição:

\begin{enumerate}[label=\alph*)]
\item \textit{vogais} -- aquisição da vogal baixa /a/ e de vogais altas antes de vogais médias;
\item \textit{consoantes} -- MA: aquisição de oclusivas\is{modo de articulação!oclusiva} e nasais\is{modo de articulação!nasal} antes de fricativas\is{modo de articulação!fricativa} e líquidas;\is{modo de articulação!líquida} PA: aquisição de consoantes labiais e coronais antes de dorsais; Vozeamento:\is{vozeamento} aquisição de obstruintes\is{modo de articulação!obstruinte} não vozeadas\is{vozeamento!não vozeado} antes de vozeadas.\is{vozeamento!vozeado}
\end{enumerate}

De um modo geral, os segmentos de emergência mais precoce substituem os segmentos mais tardios no percurso de aquisição.

Existem vários pontos de convergência entre os padrões gerais acima explicitados e as etapas de desenvolvimento evidenciadas pelos falantes em aquisição do Português, como veremos em seguida.

\section{Padrões de aquisição segmental no português}
\label{sec:matzenauer_padroes_portugues}

A aquisição do inventário segmental é um processo de desenvolvimento fonológico gradual, em que as crianças portuguesas e as brasileiras apresentam padrões comuns, embora também registem diferenças individuais e particularidades vinculadas a cada variante da língua (PE e PB). 

\subsection{Ordem geral de aquisição}
\label{subsec:matzenauer_ordem}

\subsubsection{Segmentos vocálicos}
\label{subsubsec:matzenauer_vogais}

\paragraph*{Dados do PB}
Os estudos sobre a aquisição fonológica em crianças brasileiras evidenciam que o sistema vocálico é integralizado mais precocemente do que o \isi{sistema consonântico}: antes de a criança completar a idade de 2 anos, já as vogais da língua fazem parte da sua gramática. Essa aquisição apresenta etapas, que se diferenciam sobretudo em função de dois condicionamentos: o acento da sílaba e a \isi{altura} da vogal.

Considerando o acento silábico, o sistema vocálico estabiliza-se primeiramente em posição tónica e postónica e, subsequentemente, em posição pretónica. Já no que diz respeito à \isi{altura} da vogal, os estudos registam três etapas, resumidas em (\ref{ex:matzenauer_ordem_tonicas_pb}) \citep{rangel2002,matzenauermiranda2009}:\footnote{Os estudos aqui referidos têm como objeto a variante do PB falada no sul do Brasil, em que as vogais médias baixas \textipa{/E, O/} se manifestam apenas na posição tónica; na variante do Nordeste do País, as médias baixas são empregues também na posição pretónica (veja-se \citealt{vogeley2011}). Salienta-se que, na sílaba pretónica, não há o estabelecimento de contraste entre as vogais médias.} na primeira etapa são adquiridas as vogais periféricas \textipa{/a, i, u/}, com a oposição de \isi{altura} apenas entre a vogal baixa \textipa{/a/} e as altas \textipa{/i, u/}; na segunda etapa, emergem as vogais médias altas \textipa{/e, o/} e, na terceira, as vogais médias baixas \textipa{/E, O/}.

\begin{exe}
\ex\label{ex:matzenauer_ordem_tonicas_pb} Ordem de aquisição do sistema vocálico tónico do PB - três estádios:\\\textipa{/a, i, u/} $>>$ \textipa{/e, o/} $>>$ \textipa{/E, O/}
\end{exe}

No decorrer do processo da aquisição vocálica, é frequente a substituição de vogais médias por periféricas (e.g. \textipa{[\textprimstress pa]} para o alvo \textit{pé} - B., 1;2) e de vogais médias baixas por médias altas ou altas, sendo preservado o seu ponto de articulação (e.g. r\textipa{[o]}da $\sim$ r\textipa{[u]}da para o alvo \textit{roda} - G., 1;5).

Na posição pretónica, o sistema vocálico do PB conta com apenas cinco vogais, sendo a ordem de aquisição consistente com a da posição tónica: mais precocemente emergem as vogais periféricas e, posteriormente, as vogais médias, conforme sistematizado em \ref{ex:matzenauer_ordem_pretonicas_pb}.

\begin{exe}
\ex\label{ex:matzenauer_ordem_pretonicas_pb} Ordem de aquisição das vogais do PB em posição pretónica - dois estádios:\\\textipa{/a, i, u/} $>>$ \textipa{/e, o/}
\end{exe}

Saliente-se, no entanto, que o sistema pretónico se completa numa etapa posterior ao sistema tónico. Até cerca dos 2:6, as vogais em posição pretónica podem apresentar-se como alvo para processos de assimilação, o que evidencia a sua vulnerabilidade nos estádios mais precoces da aquisição fonológica (e.g. \textipa{[ki\textprimstress iw]} para o alvo \textit{caiu} - J., 1;4.21; \textipa{[po\textprimstress kosu]} para o alvo \textit{pescoço} - L., 2;0).

Em suma, embora as crianças brasileiras possam apresentar diferenças individuais no processo de aquisição do sistema vocálico, o padrão geral mostra que em qualquer posição (tónica ou átona), num primeiro estádio, emergem as vogais \textipa{/a, i, u/}, de oposição máxima quanto à \isi{altura}.

\paragraph*{Dados do PE}

No que diz respeito à variante europeia do português, a área da aquisição vocálica constitui um campo de investigação ainda pouco explorado. Os escassos dados disponíveis neste âmbito provêm sobretudo de estudos que focam o desenvolvimento da estrutura silábica \citep{freitas1997}\footnote{Estudo baseado em sete crianças monolingues em fase de aquisição do português europeu como língua materna, numa faixa etária compreendida entre os 0;10 e os 3;7.} ou dos processos fonológicos \citep{freitas2004,fikkertfreitas2006} e não permitem ainda estabelecer uma ordem de aquisição vocálica nesta variedade da língua.

Os estudos atrás referidos têm permitido, no entanto, aceder a alguns padrões. Sabe-se, por exemplo, que a aquisição vocálica pelas crianças portuguesas é influenciada, à semelhança do que sucede no PB, por fatores como o acento, o ponto de articulação e o grau de \isi{altura}. Sabe-se também que entre as primeiras vogais a emergir no PE estão as recuadas \textipa{[a, 5]}, embora com variabilidade na produção das mesmas (e.g. \textipa{[\textprimstress da]/[\textprimstress d5]} para o alvo \textit{dá} (\textit{Inês}, 1;0.25 - \citealt{freitas1997}), mostrando uma tendência para o ponto de articulação estabilizar antes da especificação dos traços de \isi{altura}.\is{traço distintivo} A mesma oscilação entre graus de \isi{altura} foi observada aquando da aquisição das vogais não recuadas \textipa{/e, E, i/} (e.g. \textipa{[bibi]/[bebe]} para o alvo \textit{bebé}, Inês, 1;1.30) e labiais \textipa{/o, O, u/} (e.g. \textipa{[kO]/[ku]} para o alvo \textit{corda}, Inês, 1;5.11).

\subsubsection{Segmentos consonânticos}
\label{subsubsec:matzenauer_consoantes}

\paragraph*{Dados do PB}

A aquisição consonântica é um processo gradual que, em crianças brasileiras, tende a mostrar-se estabilizado até aos 4:6. Mais tardiamente podem emergir as sequências que constituem ataques silábicos ramificados, cuja aquisição pode estender-se até à idade de 5 anos.

Embora os estudos registem diferenças individuais no processo de construção do inventário fonológico consonântico, verificam-se padrões na ordem de emergência dos segmentos. Tais padrões, referidos a seguir, estão discriminados de acordo com a posição que a consoante ocupa na sílaba, já que o tipo de constituinte silábico constitui um condicionamento significativo no processo de desenvolvimento (e.g. \citealt{freitas2017}, neste volume). Ao ser definida uma ordem na aquisição das consoantes, faz-se a distinção entre quatro posições silábicas: ataque absoluto, ataque medial, coda medial e coda final. Nas tendências gerais aqui discriminadas, evidenciam-se alguns estádios na emergência de consoantes da língua.\footnote{Os dados exemplificados sobre o processo de aquisição de consoantes por crianças brasileiras são retirados de \citet{matzenauer1990} e de \citet{lamprecht_etal2004}; nestes trabalhos estão referidas as idades de emergência das consoantes na fonologia das crianças. A obra de Lamprecht et al. resume os resultados de diversas pesquisas realizadas sobre o desenvolvimento fonológico de crianças brasileiras.}

O padrão na emergência dos segmentos consonânticos licenciados para ocupar a posição de ataque de sílaba no início da palavra tende a apresentar quatro estádios, especificados em (\ref{ex:matzenauer_ordem_absoluto_pb}):

\begin{exe}
\ex\label{ex:matzenauer_ordem_absoluto_pb} Ordem de aquisição do \isi{sistema consonântico}, em ataque absoluto (PB):\\\textipa{/p, b, t, d, f, v, m, n/} $>>$ \textipa{/k, g, s, z/} $>>$ \textipa{/l, \;R/} $>>$ \textipa{/S, Z/}
\end{exe}

Uma ordem muito semelhante é aquela registada em ataque de sílaba medial de palavra, com o acréscimo de três consoantes que a fonologia da língua licencia nessa posição: \textipa{/\textltailn, L, R/}. Neste contexto, as crianças brasileiras tendem a apresentar seis estádios na
emergência das consoantes, conforme listado em (\ref{ex:matzenauer_ordem_medial_pb}).

\begin{exe}
\ex\label{ex:matzenauer_ordem_medial_pb} Ordem de aquisição do \isi{sistema consonântico}, em ataque medial (PB):\\\textipa{/p, b, t, d, f, v, m, n/} $>>$ \textipa{/k, g, \textltailn/} $>>$ \textipa{/S, Z/} $>>$ \textipa{/l, \;R/} $>>$ \textipa{/L/} $>>$ \textipa{/S, Z, R/}
\end{exe}

Em relação às consoantes que, na língua, podem ocupar a coda silábica, os estudos sobre a aquisição do PB referem quatro segmentos,\footnote{Os estudos sobre a fonologia do PB e sobre o processo de aquisição fonológica por crianças brasileiras, na sua maioria, consideram a possibilidade de o segmento nasal\is{modo de articulação!nasal} ocupar a posição de coda silábica.} cuja ordem de emergência tende a ser a mesma em coda medial e em coda final, com a especificidade de as consoantes em coda final serem adquiridas mais precocemente do que em coda medial. O padrão geral de aquisição em coda silábica surge em (\ref{ex:matzenauer_ordem_coda_pb}).

\begin{exe}
\ex\label{ex:matzenauer_ordem_coda_pb} Ordem de aquisição do \isi{sistema consonântico}, em coda (medial e final) (PB):\\/N/ $>>$ \textipa{/l/} $>>$ \textipa{/s/} $>>$ \textipa{/R/}
\end{exe}

Saliente-se que os dados aqui apresentados refletem as tendências gerais. Há no entanto variação, podendo diferentes crianças registar diferentes estádios de desenvolvimento.

\paragraph*{Dados do PE}

À semelhança do desenvolvimento observado no PB, as crianças em fase de aquisição do PE mostram que a integração dos segmentos consonânticos se processa de forma gradual e faseada. 

No que diz respeito ao ataque de sílaba (não ramificado, em posição inicial ou medial de palavra), os padrões gerais apontam para a emergência precoce dos fonemas listados em  (\ref{ex:matzenauer_ordem_inicial_medial_pe}).

\begin{exe}
\ex\label{ex:matzenauer_ordem_inicial_medial_pe} Primeiros segmentos consonânticos a estabilizar no PE - Ataque inicial e medial:\\\textipa{/p, b, t, d, m, n/}
\end{exe}

Numa fase posterior, emergem os segmentos fricativos\is{modo de articulação!oclusiva} e, por último, estabilizam as líquidas,\is{modo de articulação!líquida} particularmente a vibrante dorsal\is{ponto de articulação!dorsal} e a lateral palatal \citep{costa2010}.\footnote{Trabalho de investigação de natureza longitudinal, realizado com base em dados de cinco crianças em fase de aquisição do PE como língua materna, com idades compreendidas entre os 0;11 e os 4;10.}

Tal como no PB, também as crianças portuguesas apresentam padrões de aquisição consonântica determinados pela posição dos sons na sílaba. Os estudos na aquisição do PE mostram, por exemplo, que as consoantes são adquiridas primeiramente em posição de ataque não ramificado (\textit{ama\underline{r}elo} $\rightarrow$ \textipa{[m5\textprimstress REw]}, Luís:1;9.29) e só depois em segundo elemento de
ataque ramificado (\textit{p\underline{r}eto} -> [\textprimstress pReti], Pedro: 3;7.24 -- \citealt{freitassantos2001}). A pesquisa aponta ainda para o facto de alguns segmentos, como as fricativas\is{modo de articulação!fricativa} palatais \textipa{[S]} e \textipa{[Z]}, poderem surgir primeiro na posição de coda (\textit{meus} $\rightarrow$ \textipa{[\textprimstress mewS]}, Inês: 1;9) e só mais tarde em início de
sílaba (\textit{chorar} $\rightarrow$ \textipa{[Su\textprimstress RaR1]}, Inês: 2;9 – \citealt{almeida_etal2010}). Em suma, a produção de um
segmento numa determinada posição silábica não implica necessariamente a sua produção noutra posição da sílaba \citep{freitas1997,freitas2017}.

Segundo \citep{costa2010}, outros factores poderão também condicionar o desenvolvimento consonântico. Com base nos dados da aquisição de crianças portuguesas, a autora constatou que, à semelhança do que ocorre noutras línguas, o percurso de aquisição é influenciado (i) por restrições a determinadas coocorrências de traços distintivos;\is{traço distintivo} (ii) pela posição ocupada na palavra. 

No que diz respeito às combinações de traços, \citet{costa2010} mostra que as crianças processam como estruturas marcadas coocorrências do tipo [nasal,\is{modo de articulação!nasal} coronal $-$anterior]\is{ponto de articulação!coronal} ou [dorsal,\is{ponto de articulação!dorsal} $+$vozeado],\is{vozeamento!vozeado} resultando esse processamento numa dilatação temporal na aquisição das respetivas classes naturais. Por exemplo, a Inês adquire as nasais\is{modo de articulação!nasal} \textipa{/m/} e \textipa{/n/} aos 0;11 e 1;1, respetivamente, mas a produção da nasal\is{modo de articulação!nasal} palatal em conformidade com o alvo só estabiliza aos 3;4. O mesmo acontece em relação às oclusivas dorsais: as crianças portuguesas adquirem primeiro a dorsal\is{ponto de articulação!dorsal} não vozeada\is{vozeamento!não vozeado} \textipa{/k/}, sendo que a homorgânica vozeada\is{vozeamento!vozeado} \textipa{/g/} estabiliza apenas posteriormente, num intervalo que pode corresponder a mais de doze meses \citep{costa2010}. Ainda no campo das interações entre traços,\is{traço distintivo} há a referir que algumas das crianças portuguesas estudadas apresentaram um percurso de aquisição consonântica que é influenciado também pelo traço [$\pm$vozeado]:\is{vozeamento!vozeado} a ordem de aquisição é pautada por um sentido [$+$anterior] $>>$ [$-$anterior], mas em subcategorias determinadas pelo \isi{vozeamento} (primeiro as não vozeadas\is{vozeamento!não vozeado} e só depois as vozeadas\is{vozeamento!vozeado}). Repare-se, por exemplo, no percurso de aquisição dos sons oclusivos\is{modo de articulação!oclusiva} pela Inês, ilustrado em (\ref{ex:matzenauer_ordem_oclusivas_ines_pe}).

\begin{exe}
\ex\label{ex:matzenauer_ordem_oclusivas_ines_pe} Ordem de aquisição das oclusivas\is{modo de articulação!oclusiva} - interação entre PA e \isi{vozeamento} (PE - Inês):\\\textipa{[p]} (1;9) $>>$ \textipa{[t]} (2;2) $>>$ \textipa{[k]} (2;6) $>>$ \textipa{[b,d]} (2;10) $>>$ \textipa{[g]} (3;0)
\end{exe}

No que diz respeito à interação entre desenvolvimento segmental e unidade \textit{palavra}, \citet{costa2010} mostrou que, numa etapa inicial, a especificação de traços\is{traço distintivo} surge associada a toda a palavra e não às raízes segmentais individuais. Nesta fase emergem produções homorgânicas quer no modo quer no ponto de articulação consonântico, como ilustram produções como \textipa{[pa\textprimstress bEw]} para o alvo \textit{chapéu} (Inês, 1;8.2); \textipa{[\textprimstress dOd5]} para \textit{roda} (João, 1;10.26)
ou \textipa{[\textprimstress pop5]} para \textit{bola} (Luma, 2;2.22). Posteriormente, a uniformidade dá lugar à heterogeneidade consonântica no domínio da palavra. No entanto, numa primeira fase, a emergência das consoantes é condicionada, em função da posição. Por exemplo, estruturas com uma consoante labial\is{ponto de articulação!labial} em início de palavra e uma consoante coronal\is{ponto de articulação!coronal} em ataque medial (C\textsubscript{Lab}\ldots C\textsubscript{Cor}]) são já produzidas em conformidade, mas a combinação inversa ([C\textsubscript{Cor}\ldots C\textsubscript{Lab} é produzida de forma alternativa, conforme exemplificado na Tabela \ref{tab:matzenauer_assimetrias}.


\begin{table}
  \begin{tabular}{llllll}
    \lsptoprule
    Criança  & Idade & Ortogr. &  Alvo & PA-alvo & Produção \\
    \midrule
Inês & 1;8.2 & pente & \textipa{/p\~et1/} & [Lab\ldots Cor] & \textipa{[\textprimstress p\super hit5]}\\
 & 1;8.2 & tampa & \textipa{/t\~5p5/} & [Cor\ldots Lab] & \textipa{[\textprimstress pat5]}\\
\lspbottomrule
  \end{tabular}
  \caption{Assimetrias posicionais no percurso de aquisição}
  \label{tab:matzenauer_assimetrias}
\end{table}

Os dados do PE mostram, assim, que a produção segmental consonântica é fortemente condicionada, numa fase inicial (desde os 0;11 até aos 2;2, aproximadamente) pela unidade \textit{palavra}.

Em suma, o processo de aquisição segmental no português (PB e PE) é determinado por diferentes fatores, entre os quais as restrições à coocorrência de traços,\is{traço distintivo} a interação com o acento e a posição dos segmentos na sílaba e na palavra.


\subsection{Ordem de aquisição das consoantes por classes}
\label{subsec:matzenauer_ordem_classes}

Ao considerar-se o processo de aquisição do sistema de consoantes, especialmente em relação ao modo de articulação (MA), a tendência observada, nas crianças portuguesas e brasileiras, é muito semelhante à ordem proposta por \citet{jakobson1941}, com a emergência precoce de oclusivas\is{modo de articulação!oclusiva} e nasais.\is{modo de articulação!nasal} A ordem mais frequentemente observada na aquisição de segmentos consonânticos, quanto ao MA, aparece em (\ref{ex:matzenauer_ordem_consoantes_ma}).

\begin{exe}
\ex\label{ex:matzenauer_ordem_consoantes_ma} Ordem de aquisição de classes de segmentos consonânticos, por MA (PB e PE):\\oclusivas, nasais\is{modo de articulação!nasal} $>>$ fricativas\is{modo de articulação!fricativa} $>>$ líquidas\is{modo de articulação!líquida}\is{modo de articulação!líquida}
\end{exe}

No que diz respeito a faixas etárias de aquisição, estas são variáveis.\footnote{Essa variabilidade decorre do percurso de desenvolvimento percorrido por cada criança, assim como das opções metodológicas e dos critérios de aquisição que norteiam os diferentes estudos.} Todavia, podemos referir, a título exemplificativo, os dados do estudo TFF-ALPE \citep{mendes_etal2009,mendes_etal2013} -- que analisa dados de 768 crianças portuguesas --, e a partir do qual é possível estabelecer a seguinte cronologia de desenvolvimento:  oclusivas\is{modo de articulação!oclusiva} e nasais\is{modo de articulação!nasal} entre os 3;0 e os 3;6; fricativas\is{modo de articulação!fricativa} entre os 4;0 e os 4;6; líquidas\is{modo de articulação!líquida} em posição de Ataque silábico simples (e.g. <\textit{r}ato, ca\textit{r}o, \textit{l}ata ou mi\textit{lh}o>) até aos 4;6 e em final de sílaba (e.g. <ma\textit{r}>)  ou em grupo consonântico (e.g. <f\textit{l}or ou p\textit{r}ato>) até aos 5;6.

Quanto ao PA, os estudos em aquisição das consoantes evidenciam também padrões comuns às crianças falantes das duas variedades do português; esse padrão surge explicitado em  (\ref{ex:matzenauer_ordem_consoantes_pa}).

\begin{exe}
\ex\label{ex:matzenauer_ordem_consoantes_pa} Ordem de aquisição de classes de segmentos consonânticos, por PA (PB e PE):\\labial\is{ponto de articulação!labial} $>>$ coronal [$+$anterior], dorsal\is{ponto de articulação!dorsal} $>>$ coronal  [$-$anterior]\is{ponto de articulação!coronal}
\end{exe}

Repare-se que esta sequência de aquisição do PA tende a ocorrer no âmbito de cada classe de MA e pode ser determinada também pelo \isi{vozeamento}. Assim, no contexto de cada modo de articulação, verifica-se a tendência para os sons não vozeados\is{vozeamento!não vozeado} serem adquiridos antes dos vozeados\is{vozeamento!vozeado}, pela ordem anterior $>>$ recuado. No caso das oclusivas,\is{modo de articulação!oclusiva} a tendência de aquisição, embora variável, é \textipa{/p} $>>$ \textipa{t} $>>$ \textipa{k} $>>$ \textipa{b} $>>$ \textipa{d} $>>$ \textipa{g/} (3;0-3;6) e no caso das fricativas\is{modo de articulação!fricativa} \textipa{/f}  $>>$ \textipa{s} $>>$ \textipa{S} $>>$ \textipa{v}  (3;0 - 3;6) $>>$ \textipa{z} $>>$ \textipa{Z/} (4;0 -4;6) (dados do PE - \citealt{mendes_etal2009,mendes_etal2013}). 

Saliente-se, no entanto, que outros padrões de aquisição têm também sido verificados com frequência no PB, particularmente aquele em que as crianças adquirem em primeiro lugar as obstruintes\is{modo de articulação!obstruinte} não vozeadas\is{vozeamento!não vozeado} (\textipa{/p t k f s S/}) e depois as vozeadas\is{vozeamento!vozeado} (\textipa{/b d g v z Z/}), ou seja, um percurso de desenvolvimento segmental em que a especificação do traço de \isi{vozeamento} ([$\pm$vozeado]) parece sobrepor-se à especificação do traço [contínuo].

No que concerne a segmentos soantes, em que o \isi{vozeamento} não é distintivo, há a salientar que as nasais\is{modo de articulação!nasal} tendem também a seguir a mesma sequência de PA na ordem de desenvolvimento: a aquisição da labial\is{ponto de articulação!labial} \textipa{/m/} antecede a coronal\is{ponto de articulação!coronal} anterior \textipa{/n/}, seguida pela palatal \textipa{/\textltailn/}. No entanto, tem sido detetado um padrão divergente nas líquidas,\is{modo de articulação!líquida} especificamente nas vibrantes,\footnote{No PB, comummente designadas por róticas.} pois a recuada \textipa{/\;R/} tende a estabilizar no sistema das crianças antes da coronal\is{ponto de articulação!coronal} anterior \textipa{/R/} \citep{costa2010,miranda2007}; esse intervalo na aquisição das duas vibrantes pode compreender vários meses (um ano, de acordo com \citealt{mendes_etal2009,mendes_etal2013}). Este padrão de aquisição pode estar relacionado com diferentes graus de sonoridade inerentes aos dois segmentos, estando a dorsal\is{ponto de articulação!dorsal}\textipa{/\;R/} mais próxima das fricativas\is{modo de articulação!fricativa} do que das líquidas\is{modo de articulação!líquida} (para maior detalhe desta análise, consulte-se \citealt{miranda2007}).

\subsection{Padrões de substituição mais frequentes na aquisição das consoantes}
\label{subsec:matzenauer_subs_consoantes}

A aquisição gradual do inventário de consoantes da língua evidencia que os segmentos vão emergindo enquanto partes de classes naturais, ou seja, uma oclusiva\is{modo de articulação!oclusiva} emerge enquanto elemento integrante da classe das oclusivas,\is{modo de articulação!oclusiva} ou, num âmbito mais geral, como parte da classe das obstruintes;\is{modo de articulação!obstruinte} uma líquida\is{modo de articulação!líquida} surge como parte da classe das líquidas,\is{modo de articulação!líquida} ou, num âmbito mais geral, como integrante da classe das aproximantes, e assim por diante. Este aspeto essencial da constituição da gramática fonológica é revelado pelos padrões de substituição observados durante o desenvolvimento linguístico da criança.

No processo de formação do inventário fonológico por crianças portuguesas e brasileiras, as consoantes de aquisição mais tardia tendem a ser alvo de substituições. Esse fenómeno, que é recorrente na aquisição de diferentes sistemas linguísticos, não é aleatório: o segmento escolhido como substituto de um alvo ainda não adquirido denuncia a classe natural em que a criança localiza aquele segmento alvo. O modo de articulação (MA) é a classe que mais frequentemente gera a emergência de padrões de substituição. 

Na Tabela \ref{tab:matzenauer_padroes} apresentam-se os padrões de substituição mais frequentes\footnote{Os dados aqui apresentados são retirados de \citet{lamprecht_etal2004} e de \citet{costa2010}.} registados no processo de aquisição do português (PE e PB),\footnote{Na substituição de fricativas\is{modo de articulação!fricativa} por oclusivas,\is{modo de articulação!oclusiva} tende a ser preservado o PA do segmento alvo.} com a indicação dos contrastes não estabelecidos na fonologia da criança.\footnote{Por traço/contraste não adquirido ``[$\pm$contínuo]/[$-$soante]'' entenda-se: não foi adquirido o contraste estabelecido pelo traço [$\pm$contínuo] em coocorrência com o traço [$-$soante].}\is{traço distintivo}


\begin{table}
\resizebox{\textwidth}{!}{
  \begin{tabular}{lll}
    \lsptoprule
Substituições mais frequentes  & Exemplos & Traço/contraste não adquirido \\
    \midrule
\textbf{a.} fricativas\is{modo de articulação!fricativa} $\rightarrow$ oclusivas\is{modo de articulação!oclusiva} & \textit{faca} \textipa{[\textprimstress pak5]} & [$\pm$contínuo]/[$-$soante]               \\
\textbf{b.} obstruinte vozeada\is{vozeamento!vozeado}\is{modo de articulação!obstruinte} $\rightarrow$ obstruinte desvozeada & \textit{bola} \textipa{[\textprimstress pOl5]} & [$\pm$voz]/[$-$soante]            \\
\textbf{c.} oclusiva dorsal\is{ponto de articulação!dorsal} $\rightarrow$ oclusiva coronal\is{modo de articulação!oclusiva} & \textit{quero} \textipa{[\textprimstress tElu]} & [$+$voz]/[$-$soante]               \\
\textbf{d.} nasal\is{modo de articulação!nasal} cor. [$-$ant] $\rightarrow$ nasal cor. [$+$ant] & \textit{dinheiro} \textipa{[dZi\textprimstress nelu]} & [$\pm$anterior]/[$+$lateral]               \\
\textbf{e.} fricativas\is{modo de articulação!fricativa} cor. [$-$ant] $\rightarrow$ fricativas\is{modo de articulação!fricativa} cor. [$+$ant]& \textit{chapéu} \textipa{[sa\textprimstress pEw]} & [$\pm$anterior]/[cor +cont]               \\
\textbf{f.} fricativa\is{modo de articulação!fricativa}s cor. [$+$ant] $\rightarrow$ fricativas cor. [$-$ant] & \textit{sapo} \textipa{[\textprimstress Sapu]}& [$\pm$anterior]/[cor +cont]               \\
\textbf{g.} líquida\is{modo de articulação!líquida} $\rightarrow$ glide & \textit{barata} \textipa{[ba\textprimstress jat5]} & [$\pm$conson]/[$+$aproximante]               \\
\textbf{h.} líquida\is{modo de articulação!líquida} lat. [$-$ant] $\rightarrow$ líquida lat. [$+$ ant]& \textit{palhaço} \textipa{[pa\textprimstress lasu]} & [$\pm$anterior]/[+lateral]               \\
\textbf{i.} líquida [$-$lat]$ \rightarrow$ líquida [$+$lat]& \textit{nariz} \textipa{[na\textprimstress lis]} & [$\pm$lateral]/[$+$aproximante]               \\
\lspbottomrule
  \end{tabular}}
  \caption{Padrões de substituições mais frequentes}
  \label{tab:matzenauer_padroes}
\end{table}

As substituições referidas da alínea a. à d. tendem a verificar-se nos estádios mais precoces de aquisição da fonologia; as outras substituições podem estender-se por estádios mais avançados do desenvolvimento fonológico, especialmente aquelas identificadas nas alíneas e., f., i.

\section{Nota Final}
\label{sec:matzenauer_nota_final}

Encerramos este capítulo com o reforço da ideia de que os estudos em aquisição segmental e a consequente identificação de padrões gerais são fulcrais para o traçar de um perfil de desenvolvimento, o que viabilizará posteriormente a criação de bases para o diagnóstico e a terapia de desvios de fala. 

Destacamos também que os modelos teóricos que explicam o funcionamento da componente fonológica das línguas oferecem ferramentas importantes para o entendimento do processo de aquisição da linguagem pelas crianças. É, no entanto, preciso ter em conta o facto de que cada modelo, na decorrência dos seus pressupostos, implica uma interpretação linguística distinta para a natureza desse processo, incluindo a construção gradual dos segmentos do sistema linguístico como parte do próprio conhecimento fonológico. 

Neste contexto de especificidade dos diferentes modelos, podem ser elencadas duas amplas perspectivas de descrição e análise do processo de aquisição: o segmento pode ser visto (i) ou como uma unidade complexa, cuja estrutura interna é constituída por unidades menores (os traços distintivos), e nesta perspectiva os traços são capazes de caracterizar cada segmento e também estabelecer relações entre segmentos, explicitando classes naturais; (ii) ou como uma unidade que integra estruturas linguísticas maiores, como a sílaba, o pé métrico, a palavra e os constituintes prosódicos ainda mais altos, e nesta perspectiva os segmentos veem-se relacionados com unidades prosódicas, que condicionam o valor destes na língua. Por fim, a relevância dos estudos sobre a aquisição segmental está não apenas na possibilidade de desvendar o complexo processo que a caracteriza, descrevendo-o, analisando-o e explicando-o, mas também no descortinar do processo de desenvolvimento da competência fonológica das crianças.\is{traço distintivo}





{\sloppy
\printbibliography[heading=subbibliography,notkeyword=this]
}
\end{document}