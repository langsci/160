\documentclass[output=paper]{LSP/langsci} 
\author{Inês Sim-Sim\affiliation{Instituto Politécnico de Lisboa} 
}
\title{Aquisição da linguagem: Um olhar retrospetivo sobre o percurso do conhecimento
}  
\abstract{}
\maketitle
\begin{document}
\section{Da curiosidade à sistematização de dados}
\label{sec:simsim_curiosidade}

O processo de aquisição da linguagem pela criança é intrigante para qualquer adulto que, no convívio direto com uma criança, se apercebe da facilidade e da rapidez com que a mesma apreende e domina a língua da comunidade a que pertence. A curiosidade sobre esta realidade aparece-nos espelhada em mitos e religiões de muitos povos, tendo captado, ao longo do tempo, o interesse de domínios do conhecimento tão diversos quanto a filosofia, a psicologia, a linguística, a neurociência.

A linguagem é uma das grandes maravilhas do mundo natural. Possuir e conhecer uma língua é a quinta essência da nossa condição de humanos. Ao contrário dos outros animais, em poucos anos de vida, tornamo-nos falantes exímios da nossa língua materna. O processo é rápido, eficaz e universal e não carece de lições formais. Para que tal aconteça, apenas é necessário que a criança seja exposta a sons da fala e a situações de interação em que esses sons ocorram na convivência quotidiana com falantes. Dito de uma outra forma, que ouça falar e que falem com ela.\footnote{No caso da criança surda,\is{surdez} na modalidade visuo-manual (cf. Secção \ref{subsubsec:simsim_surdez}).} Tão cedo quanto no século IV, St Agostinho\ia{St. Agostinho} (354-430) apercebeu-se desta realidade ao escrever nas Confissões:\footnote{Livro I, Cap. VIII. 13.}

\begin{quote}
Não eram as pessoas mais velhas que me ensinavam, facultando-me as palavras pela ordem formal [\ldots] mas eu próprio, com a mente que me deste, meu Deus, com gemidos e vários sons e vários gestos, queria exprimir os sentimentos do meu coração. [\ldots] Fixava na memória quando eles nomeavam um objecto, e quando, consoante a palavra, moviam o corpo em direcção a alguma coisa, eu via e registava que designavam essa coisa com o som que proferiam quando queriam mostrá-la. [\ldots] Assim, ia eu deduzindo pouco a pouco de que coisas eram signos as palavras colocadas nas várias frases em posição apropriada \citep[23]{stagostinho2000}
\end{quote}

O desenvolvimento da linguagem na criança é materializado em modificações quantitativas e qualitativas na compreensão e produção verbal. A descrição e explicação dessas modificações é o objeto de estudo do ramo de conhecimento que se designa por aquisição da linguagem.

Compreender a evolução de um ramo do conhecimento, neste caso a aquisição da linguagem, implica debruçarmo-nos retrospectivamente sobre um processo em construção contínua e em que, através da procura de um fio condutor subjacente, se podem identificar as grandes questões formuladas na busca do conhecimento, as controvérsias geradoras de polémicas teóricas produtivas, as metodologias de investigação experimentadas, abandonadas ou melhoradas em pesquisas no domínio em causa.

A recolha, a análise e a organização sistematizada dos dados e a consequente interpretação à luz de uma teoria explicativa são a fronteira que marca a separação entre a curiosidade, científica ou não, e a construção do conhecimento científico. O primeiro registo conhecido de \isi{observações sistemáticas} da evolução de produções linguísticas da criança remonta ao século XIX e tem a assinatura de Charles Darwin.\ia{Darwin, Charles}\footnote{\ia{Darwin, Charles}Charles Darwin (1809-1882), autor da teoria da evolução, da origem das espécies e do processo de selecção natural.} As notas rigorosas e objetivas de Darwin num diário sobre o desenvolvimento do seu filho mais velho\footnote{William E. Darwin; os registos compreendem o período desde o dia do nascimento, 27/12/1839, até Setembro de 1844; posteriormente, Darwin registou também as produções de Anne Darwin, nascida 1841.} são um ponto de partida crucial na formulação de questões sobre o que pode ou não ser inato na expressão das emoções dos seres humanos e na relação com as outras espécies. Em 1877, 37 anos depois de recolhidas as notas do diário, Darwin publica de forma organizada o que observou e registou sobre o desabrochar da linguagem do filho William: 

\begin{quote}
At exactly the age of a year, he made the great step of inventing a word for food, namely \emph{mum}, but what led him to it I did not discover. And now instead of beginning to cry when he was hungry, he used this word in a demonstrative manner or as a verb, implying ``Give me food''. [\ldots] But he also used \emph{mum} as a substantive of wide signification; thus he called sugar \emph{shu-mum}, and a little later  after he had learned the word ``black,'' he called liquorice \emph{black-shu-mum}, - black-sugar-food. [\ldots] The interrogatory sound which my child gave to the word mum when asking for food is especially curious; for if anyone will use a single word or a short sentence in this manner, he will find that the musical pitch of his voice rises considerably at the close. [\ldots]. Before he was a year old, he understood intonations and gestures, as well as several words and short sentences. He understood one word, namely, his nurse's name, exactly five months before he invented his first word \emph{mum} and this is what might have been expected, as we know that the lower animals easily learn to understand spoken words. \citep[293--294]{darwin1877}\footnote{\url{http://psychclassics.yorku.ca/Darwin/infant.htm} (consultado em 13/6/14).}
\end{quote}

A recolha e sistematização das referidas notas e a sua leitura interpretativa podem ser consideradas o passo inaugural no caminho do conhecimento da aquisição da linguagem.

\section{Questões centrais e eixos teóricos no desenvolvimento do ramo do conhecimento}
\label{sec:simsim_questoes_centrais}

As reflexões filosóficas sobre a linguagem humana e a curiosidade sobre como as crianças se apropriam da língua da comunidade a que pertencem teve, no início do século XX, um ponto de viragem que vale a pena realçar. Referimo-nos à construção em 1905 do primeiro teste de inteligência de Binet\ia{Binet, Alfred}\footnote{Alfred Binet\ia{Binet, Alfred} (1857 -1911), psicólogo francês que desenvolveu medidas de avaliação que permitiam referenciar crianças para escolas de ensino especial.}-Simon (\emph{Échelle Métrique d'Intelligence}), reformulado e adaptado por Lewis Terman\ia{Terman, Lewis} em 1925 (\emph{Stanford-Binet Intelligence Scale}).

Pela primeira vez, foram criadas medidas que permitiam avaliar a evolução das capacidades verbais da criança. Para estes autores, o nível verbal era uma componente importante na definição da idade mental do sujeito. A versão de 1905 implicava identificar e nomear objetos, definir conceitos, repetir e completar frases e produzir rimas. Na versão adaptada de Terman,\ia{Terman, Lewis} que abrangia uma faixa etária dos dois aos 14 anos de idade, alargada a adultos, foram acrescentadas sub-provas verbais que avaliavam a compreensão de frases de complexidade crescente, a memorização de narrativas, a resolução de analogias simples, a identificação verbal de semelhanças e diferenças e a deteção de absurdos verbais. Surgiu assim, pela mão da psicologia, uma forma de recolher e tratar objetiva e sistematicamente dados sobre a produção e compreensão da linguagem oral das crianças.

A reflexão conceptual sobre esta realidade e o desenvolvimento de estudos na procura de evidência empírica que suportasse ou refutasse os quadros teóricos emergentes nas primeiras décadas do século XX beneficiaram do contributo oferecido, primeiro pela psicologia e depois, de forma consistente, pela linguística. Um breve olhar retrospetivo permite-nos perceber que os primeiros \isi{estudos empíricos} em aquisição da linguagem partilharam campos profissionais. Os linguistas concentraram os seus primeiros ``esforços descritivos'' principalmente na aquisição da fonologia e da morfologia e os psicólogos nos domínios pragmático e semântico, particularmente lexical.

No que respeita ao contributo teórico da psicologia, são de salientar os trabalhos realizados na Europa ocidental, designadamente pela escola de Genebra, Piaget\ia{Piaget, Jean}\footnote{\ia{Piaget, Jean}Jean Piaget (1896-1980), biólogo suíço de formação, que, a partir de 1921, na Escola de Genebra, desenvolve pesquisas e teorização sobre epistemologia e psicologia genética da inteligência, tornando-se um marco determinante no estudo do desenvolvimento cognitivo da criança.} e os seus seguidores cognitivistas, em que o foco é colocado na evolução do desenvolvimento humano numa perspetiva construtivista.\is{aquisição da linguagem!perspetiva construtivista}\footnote{A perspetiva construtivista\is{aquisição da linguagem!perspetiva construtivista} assume que as representações mentais (conhecimento) são subjetivas, construídas através da interação entre as ideias e as experiências do sujeito; para Piaget,\ia{Piaget, Jean} são as estruturas cognitivas do sujeito, através de processos de adaptação e organização, que permitem a elaboração das experiências pessoais, dando-lhes uma interpretação particular.} Por sua vez, nos Estados Unidos da América, o contributo da psicologia para o domínio da aquisição da linguagem surge pela mão da escola comportamentalista (behaviorista)\is{aquisição da linguagem!perspetiva behaviorista},\footnote{O behaviorismo (comportamentalismo) postula a objetividade e mensurabilidade do comportamento humano, o qual é o resultado de estímulos mais ou menos complexos, podendo ser condicionado (treinado e alterado) através da interação com o meio ambiente.} particularmente com Skinner\ia{Skinner, Burrhus Frederic}\footnote{\ia{Skinner, Burrhus Frederic}Burrhus. F. Skinner (1904-1990), psicólogo americano a quem se deve uma consistente teorização sobre o papel e contingências do reforço na aprendizagem.} e os seus continuadores, em que é destacado o papel da imitação e do reforço social nas aprendizagens verbais.  Finalmente, e não menos importante, é o contributo oriundo da \isi{psicologia soviética}, cujos princípios filosóficos subjacentes realçam o papel da “compreensão consciente” do universo e do real por parte do indivíduo. Para a então corrente soviética,\is{psicologia soviética} a linguagem da criança ancora em princípios \isi{reguladores do discurso}, como planificador e orientador da ação. Lev Vygotsky\ia{Vygotsky, Lev}\footnote{\ia{Vygotsky, Lev}Lev S. Vygotsky (1896-1934), psicólogo soviético da Bielorússia, tardiamente conhecido no ocidente (1962), desenvolveu pesquisas e conceptualizou sobre a \isi{interação social} como determinante no desenvolvimento de funções mentais superiores (\emph{Teoria do Desenvolvimento Social}).\is{Teoria do Desenvolvimento Social}} e, posteriormente, Alexander Luria\ia{Luria, Alexander}\footnote{\ia{Luria, Alexander}Alexander Luria (1902-1977), neuropsicólogo soviético, com trabalhos precursores na área da neuropsicologia do desenvolvimento infantil e em patologias da linguagem. Juntamente com Vygotsky foi um dos pilares fundadores da \emph{Psicologia Histórico-Cultural}.} são dois marcos\is{marcos de desenvolvimento} relevantes na compreensão do papel da linguagem na formação dos processos mentais da criança.

Para estes autores, o desenvolvimento da linguagem é teoricamente perspetivado como o resultado da \isi{interação social} e das necessidades práticas de comunicar.

Do ponto de vista da linguística, embora o grande ponto de ancoragem para o desenvolvimento de estudos sobre a aquisição da linguagem sejam os trabalhos de Noam Chomsky\ia{Chomsky, Noam}\footnote{\ia{Chomsky, Noam}Noam Chomsky (1928- ), linguista americano, professor emérito do MIT, teorizou sobre universais linguísticos\is{universais} e sobre princípios subjacentes à linguagem humana; concebeu e elaborou a teoria da \textit{Gramática Generativa},\is{gramática generativa} rompendo com as correntes do estruturalismo e do behaviorismo dominantes no estudo das línguas naturais e da sua aquisição.} (a partir de 1956 / 1965), será de toda a justiça mencionar Leonard Bloomfield\ia{Bloomfield, Leonard}\footnote{\ia{Bloomfield, Leonard}Leonard Bloomfield (1887-1949), linguista americano, cuja análise linguística foi designada posteriormente por estruturalismo linguístico, dada a metodologia usada para analisar, identificar e classificar as estruturas linguísticas e as línguas. A sua obra de referência \emph{Language} foi editada pela primeira vez em 1933, antecedida por \emph{Introduction to Language}, em 1914.} que, acerca do fenómeno de aquisição da linguagem pela criança, afirmava em 1933:

\begin{quote}
This is doubtless the greatest intellectual feat any one of us is ever required to perform. Exactly how children learn to speak is not known; the process seems to be something like this: [\ldots] Under various stimuli the child utters and repeats vocal sounds. This seems to be an inherited trait.

[\ldots] At the same time and by the same process, the child learns also to act the part of a hearer. [\ldots] This twofold character of the speech-habits becomes more and more unified, since the two phases always occur together. In each case where the child learns the connection S $\rightarrow$ r (for instance, to say doll when he sees his doll), he learns also the connection s $\rightarrow$ R (for instance, to reach for his doll or handle it when he hears the word doll). \citep[29--31]{bloomfield1933}
\end{quote}

É interessante salientar que a perspetiva inatista\is{aquisição da linguagem!perspetiva inatista} (a capacidade para a linguagem geneticamente herdada) defendida por Chomsky\ia{Chomsky, Noam} também fora assumida anteriormente por Bloomfield, embora o último tivesse optado por uma aproximação aos behavioristas\is{aquisição da linguagem!perspetiva behaviorista} no que respeita ao processo de aquisição da linguagem, o que Chomsky contrariou fortemente. No fundo estamos perante posicionamentos teóricos com pontos de contacto (o inatismo) e de divergência (processo de aquisição) acentuados. 

No modelo teórico defendido por Chomsky, ao adquirir a linguagem a criança descobre a gramática da língua a que é exposta. O processo de aquisição espelha a descoberta da criança das regras da \isi{gramática generativa}. O salto qualitativo decisivo no conhecimento sobre aquisição da linguagem teve o seu ponto de apoio na conceção da essência da linguagem e na consequente teoria linguística defendida por Chomsky, muito particularmente em 1959,\footnote{\citetitle{chomsky1959}, \citeyear{chomsky1959}.} na resposta à obra \citetitle{skinner57} \citeyearpar{skinner57} de Skinner. 

Os referenciais atrás mencionados estão na origem de grandes questões que alimentaram fecundas polémicas entre os defensores de perspetivas teóricas diferentes, nem sempre antagónicas, mas muitas vezes extremadas. As posições defendidas durante a primeira metade do século XX originaram os quadros conceptuais que orientaram, a partir dos anos setenta, os primeiros \isi{estudos empíricos} sobre aquisição de linguagem (cf. a Secção \ref{sec:simsim_evolucao} deste capítulo). 

Podemos condensar em três as grandes questões seminais que enquadram e ancoram as principais perspetivas teóricas sobre aquisição da linguagem.

\subsection{É a linguagem uma capacidade inata ou um comportamento aprendido?}
\label{subsec:simsim_inata_aprendido}
\is{capacidade inata}
Subjacente a esta questão está a velha controvérsia entre a dominância do que é geneticamente herdado e a preponderância do que é aprendido pela influência do meio social, traduzido sinteticamente na expressão inglesa \emph{nature} versus \emph{nurture}\is{nature@\emph{nature versus nurture}} (em português, hereditariedade versus meio social).

Deve-se a Skinner\ia{Skinner, Burrhus Frederic} a primeira grande teorização sobre a linguagem como um \isi{comportamento verbal} aprendido e sobre a especificação das variáveis que controlam esse comportamento. Durante mais de duas décadas, Skinner foi construindo a arquitetura teórica sobre o tema que tornou público nas Conferências William James\footnote{\textit{Verbal Behavior} by B. F. Skinner William James Lectures Harvard University 1948 \url{http://store.behavior.org/resources/595.pdf} (consultado em 6 de Julho de 2014).} em Harvard, em 1948 e, posteriormente, através da publicação da obra Verbal Behavior, em 1957. Como o autor afirma, 

\begin{quote}
Verbal behavior is shaped and sustained by a verbal environment -- by people who respond to behavior in certain ways because of the practices of the group of which they are members. These practices and the resulting interaction of speaker and listener yield the phenomena which are considered here under the rubric of verbal behavior. \citep[226]{skinner57}
\end{quote}

Em contraponto a uma análise estruturalista da linguagem, Skinner procura identificar, entre 1948/1957, uma relação funcional entre a resposta verbal e as variáveis operantes de controlo.\is{condicionamento operante} Para Skinner,\ia{Skinner, Burrhus Frederic} a linguagem é um \isi{comportamento verbal} que a criança aprende através da mediação social e de práticas de reforço da comunidade que convive com ela. Assim, o domínio da língua da comunidade de pertença é conseguido através da experiência verbal, de acordo com os princípios propostos pela Teoria da Aprendizagem.\is{Teoria da Aprendizagem} Para este autor, a aquisição da linguagem não é diferente de qualquer outra aprendizagem. Os pré-requisitos inatos não são linguísticos; estão confinados à capacidade de associação entre o estímulo e a resposta, à discriminação e à generalização de estímulos, comuns a qualquer animal. A imitação de sons e o reforço oferecido pelos adultos consolidam ou extinguem comportamentos verbais.

A aquisição da linguagem, numa perspetiva behaviorista\is{aquisição da linguagem!perspetiva behaviorista} (ou comportamental) traduz-se na aprendizagem de um conjunto de respostas verbais, consolidadas através da imitação e de processos de \isi{condicionamento operante}.\footnote{Processo de aprendizagem de um comportamento que ocorre como consequência de uma associação entre o estímulo, a resposta e os acontecimentos do ambiente que reforçam ou punem o comportamento.} Em síntese, a tese de Skinner é a de que a linguagem humana é um comportamento comunicativo aprendido pela criança e essa aprendizagem depende essencialmente de fatores externos ao próprio sujeito, i.e., os estímulos que a criança ouve e a recompensa à resposta verbal que emite. Como em qualquer comportamento animal, a recompensa funciona como um reforço que estabiliza o comportamento. 

Em 1959, dois anos após a publicação da obra \citetitle{skinner57} de B. Skinner, Noam Chomsky publica \citetitle{chomsky1959}. Para Chomsky, as analogias assumidas por Skinner com o comportamento animal, observado em laboratório, não se aplicam a funções humanas de caráter superior como a linguagem.

Nesse artigo de referência, Chomsky contesta a perspetiva behaviorista\is{aquisição da linguagem!perspetiva behaviorista} da linguagem e contraria a explicação de que a aquisição pela criança depende das contingências do reforço recebido da comunidade verbal que com ela interage. Para Chomsky, os princípios subjacentes à estrutura das línguas naturais são universais\is{universais} e geneticamente transmitidos. As línguas naturais são realizações particulares da linguagem humana, não divergem arbitrariamente entre si, partilham \isi{propriedades universais}, e as crianças nascem predispostas biologicamente para adquirir qualquer língua constituída de acordo com essas propriedades. Chomsky designou esta capacidade especificamente humana como \emph{Language Acquisition Device}\is{Language Acquisition Device}  (Dispositivo/Mecanismo de Aquisição da Linguagem) (LAD). Para Chomsky e os seus seguidores, a rapidez e a uniformidade do processo de aquisição da linguagem derivam da arquitetura da mente, da qual faz parte a faculdade da linguagem. É essa mesma capacidade que permite que o falante de qualquer língua, tendo adquirido um número finito de regras gramaticais da referida língua, compreenda e produza qualquer frase nunca anteriormente ouvida ou produzida nessa língua. Adquirir uma língua significa para Chomsky progredir de um estado zero de conhecimento até ao conhecimento adulto, ou estável.\footnote{\textit{Steady state}, na terminologia de Chomsky.} Um tema central na defesa de uma perspetiva inatista\is{aquisição da linguagem!perspetiva inatista} prende-se com o argumento que Chomsky designou pela ``pobreza do estímulo''.\is{pobreza do estímulo} Com base neste argumento, o processo de aquisição da linguagem não poderia ser tão rápido, perfeito e universal se estivesse dependente da riqueza dos estímulos linguísticos a que as crianças são expostas. Com efeito, os estímulos linguísticos do meio ficam muito aquém da grande complexidade de qualquer língua. A capacidade para descobrir princípios, condições e regras particulares da língua a que as crianças são expostas deve ser explicada, na perspectiva chomskyana, não pela riqueza dos estímulos, mas pela \isi{capacidade inata} para a linguagem.  Pela voz de Chomsky, em 1959,

\begin{quote}
As far as acquisition of language is concerned, it seems clear that reinforcement, casual observation, and natural inquisitiveness (coupled with a strong tendency to imitate) are important factors, as is the remarkable capacity of the child to generalize, hypothesize, and "process information" in a variety of very special and apparently highly complex ways which we cannot yet describe or begin to understand, and which may be largely innate, or may develop through some sort of learning or through maturation of the nervous system. [\ldots] it is possible that ability to select out of the complex auditory input those features that are phonologically relevant may develop largely independently of reinforcement, through genetically determined maturation. (p.15) [\ldots] The fact that all normal children acquire essentially comparable grammars of great complexity with remarkable rapidity suggests that human beings are somehow specially designed to do this, with data-handling or ``hypothesis-formulating'' ability of unknown character and complexity. \citep[50]{chomsky1959}
\end{quote}

Em suporte da teoria da \isi{capacidade inata} para a linguagem, surge em 1967 a obra \citetitle{lenneberg1967} do neurologista Eric Lenneberg.\footnote{\ia{Lenneberg, Eric Heinz}Eric Lenneberg (1921-1975) foi um neurologista alemão, refugiado nos EUA durante a II Guerra Mundial, que levantou hipóteses teóricas sobre o desenvolvimento da linguagem e a biologia, designadamente a de um período crucial (ou crítico)\is{período crítico} para a aquisição da linguagem.} Para Lenneberg, o desenvolvimento da linguagem na criança pode ser explicado através da biologia. Como o autor afirma, a sua perspectiva sobre o desenvolvimento da linguagem baseia-se na interpretação de factos observáveis, tais como a correlação entre o desenvolvimento motor e \isi{marcos de desenvolvimento} da linguagem, e.g., sentar-se, gatinhar e andar pela mão de um adulto e produzir \isi{lalação},\footnote{\is{lalação}Produção de sílabas, quase sempre CV (consoante/vogal).} compreender e produzir as primeiras palavras. A observação de crianças com um desenvolvimento motor e linguístico normal, de crianças surdas\is{surdez} ou ouvintes filhas de pais surdos,\is{surdez} assim como de crianças com atraso cognitivo e motor, bem como o estudo de adultos e de crianças com lesões corticais, responsáveis por comportamentos afásicos, levam-no a afirmar que

\begin{quote}
it is possible to correlate the variable language development with the variables chronological age or motor development, it is possible to relate it to the physical indications of brain maturation, such as the gross weight of the brain, neurodensity in the cerebral cortex, or the changing weight proportions of given substances in either gray or white matter. \citep[635]{lenneberg1969}

\end{quote}

\begin{quote}
[\ldots] Neurological material strongly suggests that something happens in the brain during the early teens that changes the propensity for language acquisition. We do not know the factors involved, but it is interesting that the critical period coincides with the time at which the human brain attains its final state of maturity in terms of structure, function, and biochemistry. \citep[639]{lenneberg1969}
\end{quote}

Baseado na observação clínica, Lenneberg assume que a capacidade para adquirir a linguagem é uma consequência da maturação neurológica, na medida em que os \isi{marcos de desenvolvimento} linguístico ocorrem em simultâneo com outros \isi{marcos de desenvolvimento} físico e de coordenação motora e que parece haver um período ideal ou preferencial (período crucial ou crítico),\is{período crítico}\footnote{Em inglês, \textit{critical period}.} entre o nascimento e a adolescência, para que a aquisição tenha lugar. A controvérsia gerada à volta da hipótese de um período preferencial ou crítico\is{período crítico} para aquisição da língua materna permanece ainda hoje atual. Em nome da verdade científica, talvez valha a pena lembrar que o próprio Lenneberg referia, em 1967, que a sua obra deveria ser entendida mais como um contributo para discussão do que como uma demonstração das bases biológicas da linguagem. 

As discussões teóricas entre os anos 50 e 60 do século passado, com posicionamentos a favor e contra a linguagem como um \isi{comportamento adquirido} ou uma \isi{capacidade inata}, foram a semente que gerou a grande explosão de \isi{estudos empíricos} nas décadas seguintes. Igualmente fértil na produção empírica é a questão que se segue. 

\subsection{Qual a relação entre a aquisição da linguagem e o desenvolvimento cognitivo?}
\label{subsec:simsim_relacao}

Como discutido na secção anterior, para os inatistas,\is{aquisição da linguagem!perspetiva inatista} a capacidade da linguagem é inata,\is{capacidade inata} codificada nos genes humanos e herdada biologicamente, justificando a universalidade do respectivo processo de aquisição. Para Chomsky, existe um ``núcleo fixo''\is{núcleo fixo},\footnote{Conferir a nota de rodapé nº \ref{ftn:simsim_rodape}.} inato e responsável pela universalidade da capacidade para a linguagem. 

A universalidade do processo é também assumida pelos cognitivistas construtivistas.\is{aquisição da linguagem!perspetiva construtivista} Contudo, para estes, a universalidade reside nas capacidades cognitivas, mais latas do que a linguagem e que determinam como é conhecido e interpretado o real. Um expoente deste ponto de vista foi Jean Piaget. Para Piaget, o primado está na cognição,\is{cognição e linguagem} inteligência ou pensamento, na terminologia piagetiana. A linguagem faz parte de uma organização cognitiva mais vasta que tem as suas raízes em ações e mecanismos sensório-motores que estão na base da função semiótica \citep{piagetinhelder1966}. Os esquemas sensório-motores são uma pré-condição para o aparecimento e desenvolvimento da linguagem e constituem a premissa lógica para as estruturas linguísticas. Para Piaget, só o funcionamento da inteligência é hereditário. Numa perspetiva tradicional piagetiana, a aquisição da linguagem depende do desenvolvimento cognitivo. 

O grande marco histórico da controvérsia entre inatistas\is{aquisição da linguagem!perspetiva inatista} e construtivistas\is{aquisição da linguagem!perspetiva construtivista} teve lugar em outubro de 1975, na abadia de Royalmont, perto de Paris, no célebre Debate entre Chomsky e Piaget.\footnote{\ia{Piaget, Jean}Editado em inglês em 1980 por Massimo Piattelli-Palmarini e traduzido para português, em \citeyear{piattelli-palmarini1987}, da versão francesa.} Tratou-se de um debate de argumentos e contra-argumentos de natureza lógica, carecendo, à data, de evidência empírica. Piaget procurava um ``compromisso'' com Chomsky, já que, na opinião de Piaget, as posições de ambos tinham pontos em comum que seria importante aprofundar e enfatizar. Assim, afirma Piaget,

\begin{quote}
\ldots não vejo aí um afastamento notável com aquilo que sempre defendi, porque, se não acredito na existência de estruturas cognitivas inatas no seio da 	inteligência, é evidente que considero que o funcionamento desta implica 	mecanismos nervosos hereditários [\ldots].Tudo o que defendo [\ldots] é que a partir deste funcionamento inato são necessárias novas regulações, desta vez construídas passo a passo pelo sujeito, para 	elaborar as estruturas pré-operatórias, aliás lógicas, das quais, em particular, as da inteligência sensório-motora conduzem ao ``núcleo fixo''\is{núcleo fixo} de Chomsky.\footnote{Para Chomsky, “núcleo fixo\is{núcleo fixo} [\ldots] é aquilo a que chamei gramática universal” \citep[102]{piattelli-palmarini1987}; para Piaget, ``núcleo fixo\is{núcleo fixo} [\ldots] não inato, constituía o resultado ``necessário'' das construções próprias à inteligência sensório-motora, anterior à linguagem'' \citep[58]{piattelli-palmarini1987}, portanto, um conjunto não específico de estruturas cognitivas. \label{ftn:simsim_rodape}} \citep[395]{piattelli-palmarini1987}\footnote{Tradução da versão francesa \textit{Theories du Langage, Theories de l’Apprendissage} (s/d). A versão inglesa \textit{Language and Learning--The debate between Jean Piaget and Noam Chomsky} foi publicada em 1980 pela Harvard University Press}
\end{quote}

Contudo, em vez de um compromisso conceptual, o debate foi, ao invés, um ponto de viragem no extremar de posições, com os inatistas\is{aquisição da linguagem!perspetiva inatista} defendendo a especificidade inata\is{capacidade inata} da linguagem e os construtivistas\is{aquisição da linguagem!perspetiva construtivista} assumindo a dependência desta do funcionamento de estruturas cognitivas. 

Em apoio da tese inatista,\is{aquisição da linguagem!perspetiva inatista} no debate de Royalmont, esteve, entre outros Jerry Fodor.\ia{Fodor, Jerry}\footnote{\ia{Fodor, Jerry}Jerry Fodor (1935- ), filósofo americano que, na linha do inatismo, desenvolveu a hipótese da modularidade\is{modularidade na aquisição da linguagem} da mente.} É a ele, a partir de 1983, que se deve a abordagem da linguagem na perspetiva de modularidade.\is{modularidade na aquisição da linguagem}\footnote{Trata-se de uma perspetiva que defende que a linguagem é processada no cérebro de forma modular e específica, i.e., encapsulada (o que determina a impossibilidade de interferir no processo interno do funcionamento do módulo, o qual é muito rápido e inconsciente); o módulo é inato e neurologicamente localizado.\label{ftn:simsim_rodape_27}} De acordo com esta teoria, existe uma organização modular no funcionamento da mente que permite um funcionamento e desenvolvimento específicos, neste caso para a linguagem. Os estudos com crianças com perturbações específicas de linguagem\is{Perturbação Específica da Linguagem}\footnote{\emph{Specific language impairment}, conferir, a propósito, a Secção \ref{subsubsec:simsim_perturbacoes}} e com a síndrome de Williams\footnote{Conferir, a propósito, a Secção \ref{subsubsec:simsim_williams} deste capítulo.} ofereceram, durante décadas, alguma evidência empírica sobre a dissociação entre a linguagem e a cognição (inteligência geral),\is{cognição e linguagem} o que reforçou a hipótese teórica da independência entre a cognição e a linguagem e,\is{cognição e linguagem} consequentemente, entre o desenvolvimento linguístico e o desenvolvimento cognitivo. Uma terceira abordagem teórica adveio com Lev Vygotsky.\ia{Vygotsky, Lev} Para os que seguem este autor, o pensamento, leia-se cognição, e a linguagem têm raízes diferentes, portanto sem qualquer dependência linear entre si. Segundo Vygotsky, 

\begin{quote}
a estrutura da linguagem não se limita a reflectir como num espelho a estrutura do pensamento; é por isso que não se pode vestir o pensamento com palavras, como se de um ornamento se tratasse. O pensamento sofre muitas alterações ao transformar-se em fala. Não se limita a encontrar expressão na fala;\footnote{Leia-se \emph{linguagem}.} encontra nela a sua realidade e forma \citep[166]{vygotsky1979}\footnote{Publicado em russo em 1934 e traduzido no ocidente em 1962; a edição portuguesa é de 1979.}
\end{quote}

Para este autor, a linguagem e o pensamento são duas realidades distintas, com existência autónoma, que partilham um espaço comum, o \emph{pensamento verbal}.\is{pensamento verbal} Numa perspetiva de desenvolvimento, existe um \isi{período pré-cognitivo da linguagem} e um \isi{período pré-verbal da cognição} (pensamento), confirmando ontogeneticamente que as raízes e o curso seguido pelo desenvolvimento cognitivo diferem dos da linguagem. Acrescenta o autor, 

\begin{quote}
a determinada altura\footnote{Por volta dos dois anos de idade.} estas duas trajectórias encontram-se e, em consequência disso, o pensamento torna-se verbal e a linguagem racional \citep[65]{vygotsky1979}
\end{quote}

Tal como na questão \ref{subsec:simsim_inata_aprendido}, as perspetivas teóricas sobre a relação entre a cognição e a linguagem\is{cognição e linguagem} estão na origem de muitos \isi{estudos empíricos} sobre aquisição da linguagem que decorreram a partir dos anos setenta do século XX. 

A próxima questão seminal foca a atenção nos fatores externos ou sociais que podem influenciar o processo de aquisição da linguagem.

\subsection{Qual a importância do contexto e da interação social na aquisição da linguagem?}
\label{subsec:simsim_contexto}
\is{interação social}
Na década de setenta do século XX, à medida que foi crescendo a base empírica de pesquisas sobre aquisição da linguagem, e que novos instrumentos de análise possibilitaram descrições mais precisas e detalhadas do fenómeno em estudo, começaram a surgir questões sobre a \isi{variabilidade individual} no ritmo e nas estratégias de crescimento linguístico da criança. A explicação da variabilidade devida à influência da interação com falantes da língua de aquisição passou a ser então objeto de teorização. É nesta perspetiva que se posicionam os \isi{interacionistas},\footnote{Os \isi{interacionistas}, no domínio do desenvolvimento humano, têm Vygotsky como mentor e Jerome Bruner\ia{Bruner, Jerome} (1915- )  como um dos teóricos de referência.} que defendem que a linguagem é biológica e social e que o processo de desenvolvimento da linguagem é influenciado pela interação da criança com os falantes que a rodeiam. Para Bruner,

\begin{quote}
the development of language [\ldots] involves two people negotiating. Language is not encountered willy-nilly by the child; it is shaped to make communicative interaction effective -- fine-tuned. If there is a \isi{Language Acquisition Device}, the input to it is not a shower of spoken language but a highly interactive affair shaped [\ldots] by some sort of an adult Language Acquisition Support System. \citep[39]{bruner1983}
\end{quote}

A importância da vertente social da linguagem no processo de aquisição de\-sen\-vol\-veu\x-se em redor de duas grandes linhas de pensamento, uma direcionada para a especificidade da formatação e do conteúdo do discurso do adulto quando se dirige diretamente à criança (CDS)\footnote{Em inglês, CDS, \emph{child directed speech},\is{child@\emph{child directed speech}} que substituiu a designação de \emph{baby talk} e \emph{motherese} (linguagem de bebé e maternalês, respetivamente).} e uma outra sobre as características do ambiente linguístico influenciadas por fatores sociais. 

Na teorização sobre a relação entre classe social e o discurso ouvido e produzido pela criança sobressai Basil Bernstein,\ia{Bernstein, Basil}\footnote{\ia{Bernstein, Basil}Linguista britânico (1924-2000) que desenvolveu uma teoria sociolinguística sobre os \emph{códigos da linguagem} (elaborado\is{código elaborado} e restrito).\is{código restrito}} que, em \citeyear{bernstein1971}, teorizou sobre como a estrutura social de pertença e os papéis sociais atribuídos e assumidos refletem e afetam a forma de transmitir verbalmente a informação. São os chamados código restrito\is{código restrito} e código elaborado.\is{código elaborado}\footnote{O \textit{código restrito}\is{código restrito} é fortemente dependente do contexto não verbal, caracterizado por uma estreita amplitude lexical, uma estrutura gramatical simplificada e uma grande utilização de chaves paralinguísticas; o \textit{código elaborado},\is{código elaborado} sendo funcionalmente mais flexível e independente do contexto, é veículo de qualquer tipo de informação, caracterizado por uma maior elaboração sintática e por um domínio lexical vasto e diversificado. } Para este autor, o uso preferencial de um dos códigos por parte dos adultos, assim como os temas discutidos no círculo familiar, afetam o desempenho linguístico das crianças. Bernstein defende que há uma relação profunda entre os papéis sociais atribuídos e a realização verbal de tais papéis e das respetivas interações subjacentes.

Uma outra autora de referência neste domínio é Courtney Cazden,\ia{Cazden, Courtney}\footnote{\ia{Cazden, Courtney}Courtney Cazden (1926- ), professora emérita da Universidade de Harvard, estabeleceu pontes entre a linguística aplicada, a sociologia e a educação. O texto clássico mais conhecido desta autora é \emph{Classroom Discourse}, publicado em 1988.} que assume que os fatores que determinam a qualidade do input\footnote{\emph{Linguistic input},\is{input@\emph{input}} expressão ou forma linguística que o sujeito ouve e processa.\label{ftn:simsim_input}} extravasam as características linguísticas do discurso ouvido pela criança, incluindo nesses fatores os padrões de interação adulto/criança e as especificidades do ambiente não linguístico em que a criança está inserida \citep{cazden1971}.

Os defensores do posicionamento teórico sobre a importância da sintonização entre o adulto e a criança, através da especificidade do discurso dirigido à criança (\emph{CDS}), não pretendem anular ou confirmar fatores inatos no processo de aquisição da linguagem. Para eles, a complexidade da aquisição da linguagem implica a intervenção de mecanismos com efeitos facilitadores, dos quais o discurso dirigido à criança é uma peça chave. Afirma Catherine Snow:\ia{Snow, Catherine}\footnote{\ia{Snow, Catherine}Catherine Snow (1945- ), professora na Universidade de Harvard, fundadora, com Brian McWhinney, da CHILDES (Child Language Data Exchange System), e com produção científica no domínio da interação criança-adulto.}

\begin{quote}
In our view the greatest potential value of research on CDS, and on facilitating features in CDS, is to constrain hypotheses concerning the nature and  variety of language learning mechanisms. The finding that any particular feature in CDS facilitates the speed or ease of language acquisition should be taken as a starting point for hypothesizing a language acquisition mechanism that operates better because of that feature. [\ldots] A variety of facilitative effects implies a variety of learning mechanisms, all operating whenever the enabling conditions hold. Such a picture is consonant with the most remarkable fact about language acquisition – its reliable occurrence in a wide variety of social settings. \citep[88--89]{snow1985}
\end{quote}

A influência da \isi{interação social} no processo de aquisição da linguagem gerou muitos estudos, procurando a evidência empírica que esclareça o papel e a importância de características específicas que afetam a aquisição da linguagem. 

\section{A evolução da evidência empírica: Metodologias e domínios de pesquisa}
\label{sec:simsim_evolucao}

\subsection{Dos estudos naturalistas à indução de respostas}
\label{subsec:simsim_estudos}

As grandes hipóteses conceptuais, espelhadas e sintetizadas nas questões formuladas na Secção  \ref{sec:simsim_questoes_centrais}, são a plataforma que tem alimentado os \isi{estudos empíricos} que, a partir da década de sessenta, no ocidente,\footnote{Foram conhecidos em 1966, pela mão de Dan Slobin,\ia{Slobin, Dan} \isi{estudos empíricos} realizados na então União Soviética, a partir de 1948.} ajudaram a confirmar, infirmar e reformular as referidas posições teóricas sobre aquisição da linguagem. 

Até à década de cinquenta, os estudos referenciados no ocidente foram recolhas naturalistas\is{estudos naturalistas} de discurso espontâneo, na maior parte dos casos, de filhos de psicólogos ou de linguistas. Encontramos referências a este tipo de estudos em inglês,\il{inglês} alemão,\il{alemão} russo,\il{russo} francês,\il{francês} turco.\footnote{Conferir Bar-Adon, A. \& Leopold, W. (eds.), \citeyear{baradonleopold1971}.}

São análises do discurso espontâneo das crianças, por vezes numa perspetiva longitudinal.\footnote{Estudos que descrevem o desenvolvimento ou as alterações num determinado período de tempo.}

A medida alvo era a produção adulta, e a linguagem da criança era vista como uma versão empobrecida da linguagem do adulto. Procedia-se à identificação e contagem de erros na produção infantil, em comparação com a linguagem do adulto, determinava-se a frequência de ocorrência de determinadas classes gramaticais e avaliava-se a maior ou menor dimensão lexical em amostras específicas.

As características das amostras e dos processos de recolha, análise e interpretação dos dados, embora tivessem permitido a obtenção de evidência empírica, não se configuravam como suficientemente poderosas para estabelecer a universalidade do fenómeno e enquadrar os resultados numa teoria explicativa de aquisição da linguagem. Aos \isi{estudos naturalistas} do discurso espontâneo das crianças foi acrescentado em 1958 um novo paradigma metodológico, o da \isi{indução de respostas} orais. Berko Gleason,\ia{Gleason, Berko} nessa data, criou o chamado teste \emph{wug},\is{teste Wug}\footnote{\is{teste Wug}Uma pseudo-palavra em inglês.} um método de \isi{interação controlada}, em que a criança deve completar uma frase, tendo para isso de usar morfemas específicos.\footnote{Através da apresentação de um desenho (Aqui está um wug.\is{teste Wug} Agora temos aqui dois. São dois \underline{\hspace{6em}}.), a criança é levada a usar as marcas para o plural (exemplo aqui apresentado), ou para o pretérito, diminuitivo, etc. (noutros casos).} Esta engenhosa metodologia de incitação à resposta resultou num salto metodológico na recolha experimental\is{estudos experimentais} de produções orais. 

Se o teste \emph{wug}\is{teste Wug} corresponde a uma substancial alteração metodológica na recolha de produções orais, o ponto de viragem no domínio do conhecimento da aquisição da linguagem ocorre com Roger Brown,\ia{Brown, Roger}\footnote{\ia{Brown, Roger}Roger Brown (1925-1997), professor americano de psicologia social, autor da obra \emph{The First Language}, que, sob o ponto de vista metodológico, pode ser considerado o pai da investigação moderna em aquisição da linguagem.} considerado um dos fundadores da moderna investigação neste campo. A melhoria qualitativa na investigação neste ramo do conhecimento alimenta-se de dois fatores: da qualidade tecnológica do registo sonoro, graças à fidelidade conseguida, que permite transcrever e analisar de forma sistemática as produções das crianças e, principalmente, da descrição e interpretação das produções dos informantes, não como aproximações ao discurso adulto, mas como produções autónomas reguladas por regras passíveis de serem explicadas através de ``instrumentos” teóricos da Gramática Generativa.\is{gramática generativa} Pela voz de R. Brown,

\begin{quote}
and it appears to be the case that the derivational complexity of English constructions within a generative grammar predicts fairly well the order in which the constructions will be acquired in childhood. \citep[115]{brown1973inbook}
\end{quote}

A obra \citetitle{brown1973} First Language \citeyearpar{brown1973} de Brown espelha o percurso de uma investigação longitudinal, realizada por uma equipa coordenada por ele, através da recolha do discurso espontâneo de três crianças,\footnote{Adam, Sarah e Eve.} durante um prolongado período de tempo, e que permitiu estabelecer os primeiros estádios do desenvolvimento gramatical, particularmente sintático e morfológico, posteriormente confirmadas noutros estudos e em línguas diferentes. A Roger Brown se deve a aplicação consistente do indicador/medida \emph{Mean Length of Utterance}\is{mean length@\textit{mean length of utterance}} (MLU), desenhada para avaliar o desenvolvimento sintático da criança e calculada através da divisão do número de morfemas pelo número de enunciados produzidos pela criança. Quanto mais elevado for este quociente, mais elaborado é o desenvolvimento gramatical da criança. 

Na procura de evidência empírica, continuaram a realizar-se \isi{estudos naturalistas}, com base na análise do discurso espontâneo das crianças, em simultaneidade com o uso de metodologias experimentais\is{estudos experimentais} de \isi{indução de respostas},\footnote{Técnica experimental que implica a criação de situações que provocam o aparecimento de determinados comportamentos ou a oportunidade de uso de determinadas estruturas linguísticas.} tanto na produção, como no julgamento da gramaticalidade de estruturas particulares (fonológicas, semânticas, sintáticas, pragmáticas) em línguas específicas e em \isi{estudos translinguísticos}.\footnote{Estudos desenvolvidos em diversas línguas com o objetivo de procurar universais\is{universais} e particularidades linguísticas; a grande figura de referência em \isi{estudos translinguísticos} em aquisição da linguagem é Dan Slobin\ia{Slobin, Dan} (1939 -), professor emérito da Universidade de Berkeley, Califórnia, que demonstrou a importância das comparações translinguísticas na compreensão da aquisição da linguagem.}

Para além dos \isi{estudos naturalistas} e experimentais,\is{estudos experimentais} nos quais incluímos os \isi{estudos correlacionais}, tiveram lugar diversas pesquisas com grandes amostras da população\footnote{Em inglês, \textit{surveys}.} escolar (estudos epidemiológicos),\is{estudos epidemiológicos} principalmente para avaliar o domínio lexical ou despistar patologias/disfunções fonológicas. O desenvolvimento de testes de avaliação da linguagem surge como uma resposta à necessidade da recolha massiva de informação e da disponibilização de medidas padronizadas de desenvolvimento. 

Durante as últimas seis décadas de investigação em aquisição da linguagem, a importância da fiabilidade dos dados obrigou a uma procura e aperfeiçoamento de metodologias de recolha e análise de informação e é hoje evidente que não há apenas um único e fiável método de obtenção de evidência empírica. É a adequação metodológica ao objetivo de investigação que permite confirmar ou infirmar a hipótese investigativa formulada.

A riqueza e a transparência dos dados e a consequente interpretação serão tanto mais promissoras quanto maior for a longevidade da informação obtida e maior o número de questões suscitadas. Os grandes problemas de adequação metodológica, e para os quais se têm procurado soluções, estão relacionados com: (i) a obtenção de informação fiável, tomando em linha de conta os instrumentos usados e as características da situação de interação e do observador;\footnote{É a questão do chamado \textit{artefacto experimental},\is{estudos experimentais} ou seja, os efeitos de interferência nos resultados, provocados pela metodologia usada ou pela situação de interação.} (ii) a representatividade da informação obtida no universo linguístico em causa;  (iii) o processo de quantificação e análise dos dados recolhidos dos informantes. É o ciclo contínuo de \emph{questões} -- \emph{dados} -- \emph{novas questões} que gera a evidência empírica e tem feito progredir o conhecimento. A metodologia de recolha, os dados obtidos e a teoria interpretativa\footnote{Conferir a Secção \ref{sec:simsim_busca}.} dos mesmos constituem-se, assim, como uma relação triangular de interdependência. 

\subsection{A diversidade metodológica na busca de evidência empírica}
\label{subsec:simsim_diversidade}

Num olhar retrospetivo sobre os estudos produzidos em aquisição da linguagem nos últimos sessenta anos, o domínio mais significativo diz respeito à descrição do desenvolvimento linguístico da criança, e a explicação dessa progressão à luz de diversas teorias. Com menor frequência, têm sido também desenvolvidas pesquisas que procuram as causas subjacentes à variação individual e ao efeito de \isi{variáveis genéticas} (sexo, inteligência/cognição),\is{cognição e linguagem} ou de \isi{variáveis sociais} (estrutura familiar, meio social e cultural) no processo de aquisição da linguagem. 

Centrando-nos no primeiro grupo, e no que respeita ao \isi{período pré-linguístico},\footnote{Desde o nascimento até à produção da primeira palavra.} a investigação tem incidido na capacidade de a criança discriminar voz humana, sons da fala e padrões prosódicos, assim como no estudo da sequência de vocalizações, desde os \isi{sons vegetativos} à \isi{reduplicação} silábica. Os paradigmas mais usados nesta fase do desenvolvimento\footnote{São de referir, dada a precocidade dos informantes, as pesquisas de Jacques Mehler \citeyearpar{mehler_etal1988} com bebés com 4h de vida.} para avaliar a discriminação de sons da voz humana utilizam a reação ao \isi{reflexo de sucção}, a alteração do ritmo cardíaco ou o aumento da sudação da pele das mãos por alteração dos estímulos sonoros apresentados ao bebé. Na procura da compreensão semântica e sintática no período anterior ao aparecimento das primeiras palavras, o paradigma com maior eficácia é o da chamada perceção intermodal,\is{perceção!perceção intermodal}\footnote{São apresentadas ao bebé duas imagens, acompanhadas por uma palavra ou expressão linguística e a criança tende a fixar o olhar na que corresponde ao estímulo auditivo \citep[cf.][]{mcdaniel_etal1998}.} concretizada na fixação preferencial do olhar do bebé no ecrã (ou no objeto) correspondente ao estímulo verbal.

Na linha de continuação do desenvolvimento fonológico, e já no período linguístico, têm sido desenvolvidas pesquisas em diversas línguas, na busca de universais,\is{universais} de \isi{marcos de desenvolvimento}, de especificidades de aquisição em cada língua e de regras que regem as estratégias de substituição, de assimilação ou de supressão de sons na produção das crianças. A aquisição de padrões prosódicos tem merecido também particular atenção. Ainda no que respeita ao desenvolvimento fonológico, um novo campo foi aberto por Isabelle Liberman \citeyearpar{liberman1973}, cunhado posteriormente como consciência fonológica (1984),\footnote{Virgínia Mann e Isabelle Liberman \citeyearpar{mannliberman1984}.} e que veio a revelar-se fortemente correlacionado com os processos de aprendizagem da linguagem escrita.

Quanto a metodologias usadas para avaliar a produção oral, para além da recolha do discurso espontâneo, foram criados procedimentos orientados pelo método de \isi{interação controlada} para indução de resposta,\is{indução de respostas} iniciado por \citet{berko1958}. Dentro da \isi{indução de respostas}, podemos distinguir como mais utilizados os paradigmas metodológicos de \isi{completamento de frases} e de imitação/\isi{repetição induzida}. Qualquer deles consistentemente usados em várias línguas para estudar a aquisição da produção de estruturas morfológicas e sintáticas específicas, designadamente marcadores de plural ou de género, de formas da conjugação verbal, da construção da passiva, da produção de frases relativas, de anáforas, de frases sujeitas a restrições semânticas, sintácticas ou pragmáticas.

No \isi{completamento de frases}, é pedido à criança que conclua a frase iniciada pelo investigador e que implica a utilização de uma determinada estrutura linguística. A formulação de questões específicas que não suportam uma resposta de sim/não é um procedimento paralelo e com objetivos idênticos ao do \isi{completamento de frases}. 

O paradigma da \isi{repetição induzida}, diferente da \isi{imitação espontânea},\footnote{Nesta, a criança repete espontaneamente uma expressão ou palavra previamente ouvida.} tem sido particularmente utilizado na avaliação de estruturas gramaticais complexas. Esta metodologia, iniciada por Brown \& Fraser em 1963, revelou-se ao longo de décadas uma interessante ``janela” para avaliação da competência linguística da criança, na medida em que o sujeito repete a frase, não por mero \isi{psitacismo},\footnote{\is{psitacismo}Como um papagaio ou um gravador.} mas de acordo com as estruturas que domina.\footnote{Por exemplo, perante a frase estímulo \emph{A rapariga \textbf{que eu vi} partiu a cabeça}, a criança pode repetir \emph{eu vi uma rapariga; a rapariga partiu a cabeça}.} Se a estrutura em análise não se encontra ainda consolidada, a repetição surge simplificada; a frase só é repetida corretamente se a estrutura alvo já fizer parte do conhecimento gramatical da criança.

O uso de metodologias de indução de resposta,\is{indução de respostas} simples ou combinadas, e consequentemente de \isi{estudos experimentais}, não anulou a importância da análise do discurso espontâneo e de \isi{estudos naturalistas} que continuaram a ser usados ao longo dos anos. Quando a recolha de dados é obtida e gravada de forma apropriada, a informação decorrente da produção espontânea pode ser um recurso muito produtivo a ser reutilizado repetidamente em estudos posteriores.

No que respeita à compreensão da linguagem oral, a recolha e tratamento de evidência empírica\footnote{Roger Brown \citeyearpar{brown1957} desenhou e realizou, no ocidente, o primeiro estudo experimental\is{estudos experimentais} sobre compreensão da linguagem, especificamente a compreensão de marcadores morfológicos em categorias sintáticas específicas.} na compreensão da linguagem pode implicar o uso de produção verbal ou ater-se a tarefas sem produção verbal. É disso exemplo o paradigma da perceção intermodal\is{perceção!perceção intermodal}, atrás mencionado. Dentro da mesma linha metodológica, embora mais elaborado, é o procedimento da \isi{seleção de imagens},\footnote{\is{seleção de imagens}Perante duas ou várias imagens, a criança deve escolher a que representa a frase ouvida; este procedimento é também usado em estudos sobre o conhecimento lexical.} o qual permite avaliar não só a interpretação semântica de contrastes morfossintáticos,\footnote{Perante pares ou séries de gravuras, pedir à criança que indique qual a gravura que mostra a frase alvo.} mas também detetar a sensibilidade da criança à gramaticalidade da estrutura linguística apresentada.

Um outro procedimento utilizado em estudos sobre compreensão da linguagem é o da \isi{manipulação figurativa}\footnote{Em inglês, \emph{acting out task}.} em que é pedido à criança que atue com os objetos presentes, de acordo com a frase alvo ouvida.\footnote{Por exemplo, pedir à criança que reproduza com um boneco e uma boneca a frase: ``O rapaz foi beijado pela rapariga''.} O racional subjacente é que o sujeito age de acordo com a interpretação da frase que ouviu.\footnote{Deve-se a Carol Chomsky \citeyearpar{chomsky1969} o primeiro uso deste procedimento.} Pelas características específicas, este procedimento tem-se mostrado muito eficaz ao longo do tempo, quer com crianças de diferentes idades, quer em \isi{estudos translinguísticos} de aquisição da linguagem.

Na recolha e análise de evidência empírica, os \isi{estudos translinguísticos} começaram por ser uma forma de comparação e de procura de universais\is{universais} de aquisição, evidenciando que diferentes línguas colocam problemas específicos ao aprendiz de falante. Iniciados nos anos setenta do século XX,\footnote{Melissa Bowerman, em 1965 (publicado em \citeyear{bowerman1973}), desenvolveu o que pode ser considerado o primeiro estudo translinguístico\is{estudos translinguísticos} em aquisição da linguagem, recolhendo e analisando dados de produções de crianças finlandesas, de acordo com critérios específicos já utilizados com falantes de inglês.\il{inglês}} este tipo de pesquisas ganha com Dan Slobin\footnote{A obra \citetitle{slobin1967} \citeyearpar{slobin1967} é uma referência importante, a que se seguiram os diversos volumes de \citetitle{slobin1985} \citeyearpar{slobin1985}.} uma metodologia decisiva e torna-se metodologicamente num paradigma para a recolha de evidência empírica no estudo da aquisição da linguagem.

\begin{quote}
By combining attention to universals and particulars, we are beginning to discern a more differentiated picture of child language – one in which we can see why patterns of acquisition of specific properties vary from language to language, while they are determined by common principles of a higher order.  
\citep[5]{slobin1985}
\end{quote}

Embora na pesquisa translinguística\is{estudos translinguísticos} a busca de universais\is{universais} seja um alvo de pesquisa, as propriedades específicas das línguas a adquirir são o fator crucial. Um contributo interpretativo para o desenvolvimento de \isi{estudos translinguísticos} adveio do modelo linguístico Princípios e Parâmetros\footnote{Os princípios representam as propriedades e operações invariantes em todas as línguas naturais, portanto universais;\is{universais} os parâmetros definem o espaço restrito de possível variação entre as línguas.} \citep{chomsky1986} nos anos oitenta. Nesta perspetiva,

\begin{quote}
language acquisition can be characterized as the process whereby the child, genetically endowed with the principles and parameters, fixes the values of the parameters on the basis of evidence and thus derives a specific instance of UG\footnote{\emph{Universal Grammar} ou, em português, Gramática Universal (GU).} namely, the grammar of the language to which she is exposed \citep[262]{jakubowicz1996}
\end{quote}

\subsection{O contributo de populações específicas no suporte à evidência empírica}
\label{subsec:simsim_contributo}

O desenvolvimento atípico é um ângulo produtivo na construção do conhecimento sobre o desenvolvimento humano. No que respeita à aquisição da linguagem, a demanda de universais\is{universais} de desenvolvimento é desafiada pelos sujeitos que, devido a características específicas, apresentam padrões de desenvolvimento particulares. É o caso de três grupos específicos, cujo manancial de dados, obtidos ao longo de dezenas de anos, desafia algumas teorias explicativas de aquisição da linguagem e corrobora outras: as crianças surdas,\is{surdez} as crianças com disfunções (perturbações) específicas de linguagem\is{Perturbação Específica da Linguagem}\footnote{\is{Perturbação Específica da Linguagem}Em inglês, \emph{Specific Language Impairment} (SLI).} e as crianças com a síndrome de Williams\is{Síndrome de Williams}.

\subsubsection{A surdez e a aquisição de línguas gestuais}
\label{subsubsec:simsim_surdez}
\is{língua gestual}
\is{surdez}
A aquisição da linguagem pelas crianças surdas\is{surdez} reveste-se de uma importância especial para a compreensão do fenómeno de aquisição da linguagem, na medida em que a comunicação oral, via audição, está comprometida nesta população. No Congresso de Milão (1880) foi aprovada uma resolução que estabelecia a educação oralista para os alunos surdos,\is{surdez} proibindo o uso de \isi{língua gestual} nas escolas. Até à década de setenta do século XX, os primeiros e raros estudos desenvolvidos visavam a comparação da aquisição da linguagem oral entre crianças surdas\is{surdez} e ouvintes.

A partir dos anos oitenta, quando a generalização da \isi{língua gestual} se tornou uma realidade, as pesquisas passaram a gravitar à volta de um dos seguintes domínios: (i) o paralelismo (similitudes e diferenças) entre a aquisição da \isi{língua gestual} pelas crianças surdas\is{surdez} e da língua oral pelas crianças ouvintes; (ii) o estudo da gramática de variadas línguas gestuais,\is{língua gestual} a partir dos resultados da aquisição dessas línguas; (iii) o estudo da aquisição da \isi{língua gestual} na testagem de teorias explicativas da aquisição da linguagem; (iv) a cultura e a identidade da sociedade surda\is{surdez} no processo de aquisição da \isi{língua gestual}.

A evidência empírica sobre a aquisição de uma modalidade visuo-manual de linguagem (uma \isi{língua gestual}), via exposição, vem em apoio de universais\is{universais} de desenvolvimento linguístico, presentes na aquisição da linguagem oral pelas crianças ouvintes, potenciando, por isso, a compreensão do fenómeno da aquisição da linguagem, em geral.

\subsubsection{Perturbações específicas de linguagem}
\label{subsubsec:simsim_perturbacoes}
\is{Perturbação Específica da Linguagem}
As perturbações específicas de linguagem\is{Perturbação Específica da Linguagem} configuram-se como uma forma atípica de desenvolvimento da linguagem, caracterizada pela dessintonia entre os indicadores linguísticos e outros aspetos do crescimento, designadamente o desenvolvimento cognitivo. Datam do século XIX \citep{gall1822} as primeiras referências clínicas a crianças com problemas de linguagem não associados a outras problemáticas.

Há registos desta realidade em diferentes línguas, ao longo dos dois últimos séculos, com designações variadas: \textit{linguagem desviante}, \textit{desordens de linguagem}, \textit{atrasos de linguagem}, \textit{perturbações no desenvolvimento da linguagem}, \textit{afasias de desenvolvimento}, \textit{disfasias}, etc. O termo \textit{Specific Language Impairment} (SLI) surgiu em \citeyear{leonard1981}, cunhado por Laurence Leonard, e tem sido usado consistentemente nas últimas décadas para denominar perturbações mais ou menos graves no desenvolvimento da linguagem, não associadas a défices cognitivos, sensoriais, ou neurológicos, nem a privação social.

O desenvolvimento linguístico destas crianças é caracterizado (i) pelo início retardado da produção das primeiras palavras (aos dois anos ou depois); (ii) por uma produção imatura ou desviante dos sons da fala; (iii) pelo uso simplificado de estruturas gramaticais, com omissão de palavras com função gramatical; (iv) por um léxico reduzido, em termos de compreensão e de produção; (v) por dificuldades na compreensão de um discurso complexo, particularmente quando produzido rapidamente pelo interlocutor.

Ao invés do que sucede com o desenvolvimento linguístico desta população, que espelha entraves em termos de velocidade e robustez do processo típico de aquisição da linguagem, o desenvolvimento cognitivo apresenta padrões e marcos\is{marcos de desenvolvimento} normais para a idade cronológica. A dissociação entre o desenvolvimento linguístico e o desenvolvimento cognitivo, assim como a atipicidade do respetivo crescimento linguístico levantam profícuas questões que podem ser um contributo importante para a compreensão teórica do processo de aquisição da linguagem pelo ser humano.

\subsubsection{A síndrome de Williams}
\label{subsubsec:simsim_williams}
\is{Síndrome de Williams}
Um outro grupo de sujeitos em que a dissociação entre o desenvolvimento linguístico e desenvolvimento cognitivo tem sido alvo de interesse científico são indivíduos com a síndrome de Williams\is{Síndrome de Williams}\footnote{\is{Síndrome de Williams}Definida pela ASHA (\textit{American Speech-Language-Hearing Association}) como uma disfunção genética muito rara, com acentuadas perturbações no processo de desenvolvimento e de saúde, particularmente, ao nível cardio-vascular.} Trata-se de uma desordem genética clinicamente identificada por John Williams em 1961, embora os primeiros estudos sobre aquisição da linguagem destas crianças só tenham surgido vinte anos depois e com resultados contraditórios. As contradições podem ser devidas à dimensão diminuta das amostras\footnote{Visto a prevalência na população ser de 1 em 7500 nascimentos.} e ou a lacunas metodológicas, quer na recolha de dados linguísticos e cognitivos, quer nos grupos de controlo usados.\footnote{Conferir, a propósito, \citet{brock2007}.}

As crianças com a síndrome de Williams\is{Síndrome de Williams} foram frequentemente descritas como tendo um desenvolvimento linguístico perto do normal para a idade cronológica, face a atrasos cognitivos graves ou moderados. A referência mais proeminente desta posição provém de \citet{bellugi_etal1988}, assumindo que esta síndrome é uma evidência da independência da linguagem da cognição.\is{cognição e linguagem} A posição de Bellugi constitui-se como um suporte em defesa da abordagem teórica da modularidade\is{modularidade na aquisição da linguagem}\footnote{Conferir nota de rodapé nº \ref{ftn:simsim_rodape_27}.} para a linguagem, defendida por \citet{fodor1983}.

Investigações posteriores confirmam o atraso cognitivo desta população e apontam para um desenvolvimento linguístico caracterizado por uma marcada desarmonia e inconsistência. Em síntese, ao invés do que inicialmente postulado, e tomando como comparação a idade cronológica ou o nível do desenvolvimento cognitivo, o desenvolvimento linguístico de crianças com a \isi{Síndrome de Williams} revela-se atípico, com realce positivo para a memória fonológica de curto prazo, para a compreensão lexical e para certos domínios da compreensão gramatical.

A continuação de pesquisas sobre aquisição da linguagem nesta população pode gerar um manancial de informação que ajudará à melhor compreensão do desenvolvimento linguístico, em geral, e, em particular, da relação entre cognição e linguagem. \is{cognição e linguagem}

\section{Na busca de uma teoria integradora da aquisição da linguagem}
\label{sec:simsim_busca}

Ao longo das páginas anteriores deste capítulo, subjaz a grande questão: Como é que as crianças adquirem a respetiva língua materna?

Na secção \ref{sec:simsim_questoes_centrais} deixámos espelhadas as principais abordagens e paradigmas que motivaram debates e posicionamentos subjacentes a perspetivas teóricas e consequente alinhamento ou interpretação de dados empíricos\is{estudos empíricos} na base dessas perspetivas. Numa breve síntese, podemos apontar para diversos enfoques interpretativos: (i) a perspetiva inatista\is{aquisição da linguagem!perspetiva inatista}/ generativista,\is{gramática generativa} na qual a teoria da modularidade\is{modularidade na aquisição da linguagem} se integra, que postula que a criança descobre a gramática da língua a que é exposta em virtude da capacidade geneticamente herdada para a linguagem, materializada na existência de mecanismos inatos da mente; (ii) a perspetiva comportamentalista (behaviorista),\is{aquisição da linguagem!perspetiva behaviorista} que defende que a linguagem é um \isi{comportamento verbal} e que a criança aprende respostas verbais por imitação e pelo reforço dos falantes adultos; (iii) a perspetiva cognitivista, que sustenta que as capacidades cognitivas determinam a aquisição da linguagem e essas capacidades têm as suas raízes em mecanismos sensório-motores, mais profundos do que os mecanismos linguísticos e só o funcionamento da inteligência, leia-se funcionamento cognitivo, é hereditário; (iv) a perspetiva interativa, para a qual a linguagem é biológica e social e o processo de desenvolvimento da linguagem é influenciado pela interação da criança com os falantes que a rodeiam.

Os quadros conceptuais referenciados, permitindo interpretações parcelares do fenómeno, deixam ainda em aberto a necessidade de uma rede articulada e sistematizada de constructos, de definições e de asserções com o propósito de justificar e predizer factos, ou seja, uma teoria científica \citep{kerlinger1973} que, em termos da aquisição da linguagem, seja capaz de explicar: (i) a rapidez de aquisição da linguagem, (ii) a compreensão e produção de sequências articulatórias\footnote{Visuo-gestuais nas crianças surdas.\is{surdez}} nunca ouvidas; (iii) a previsão e antecipação de etapas do desenvolvimento fonológico, semântico, sintático e pragmático na criança.

Na procura de uma articulação teórica surgiu a chamada Learnability Theory (teoria da aprendibilidade),\is{aprendibilidade e aquisição da linguagem} baseada no modelo matemático de E. Marc Gold (1967) e que procura identificar os procedimentos de aprendizagem na aquisição da gramática de uma língua alvo, perante \textit{inputs}\is{input@\emph{input}} linguísticos\footnote{Conferir nota de rodapé nº \ref{ftn:simsim_input}.} dessa língua. Em 1984 Steven Pinker\ia{Pinker, Steven}\footnote{\ia{Pinker, Steven}Steven Pinker (1954-  ), canadiano, professor no MIT e na Universidade de Harvard, autor de uma vasta obra sobre linguagem, cognição\is{cognição e linguagem} e aprendizagem.} busca uma leitura dessa teoria computacional aplicada à aquisição da linguagem. Segundo Pinker,\ia{Pinker, Steven} em 1995,

\begin{quote}
Learning theory has defined learning as a scenario involving four parts:

1. A class of languages. One of them is the “target” language, to be attained by the learner [\ldots]; the target language is the one spoken in their community.

 2. An environment. [\ldots]. In the case of children, it might include the sentences that parents utter, the context in which they utter them, feedback to the child (verbal or non verbal) in response to the child’s own speech [\ldots]. Parental utterances can be a random sample of the language, or they might have some special properties: [\ldots] ordered in certain ways, [\ldots] repeated or only uttered once [\ldots].
 
3. A learning strategy. [\ldots]. The learning strategy is the algorithm that creates the hypotheses and determines whether they are consistent with the input information [\ldots] For children, it is the “grammar-forming” mechanism in their brains, their “language acquisition device”.\is{Language Acquisition Device}

4. A success criterion.[\ldots]. Learners may arrive at a hypothesis identical to the target language [\ldots] they may arrive at an approximation to it; they may waver among a set of hypotheses, one of which is correct. \citep[147]{pinker1995}
\end{quote}

Para \citet{pinker1984}, o cerne desta teoria é que as crianças herdam geneticamente capacidades algorítmicas destinadas a adquirir as regras gramaticais e as entradas lexicais de qualquer língua. 

\begin{quote}
The algorithms are triggered at first by the meaning of the words in the input sentences and knowledge of what their referents are doing, gleaned from the context. Their first outputs […] are used to help analyze subsequent inputs and to trigger other learning algorithms, which come in sets tailored to the major components of language \citep[xv]{pinker1984}
\end{quote}

Esta, como qualquer outra teoria explicativa, será confirmada, negada ou reformulada com o vigor dos dados recolhidos na evidência empírica que a investigação já disponibilizou e venha a disponibilizar.

O curso da busca do conhecimento sobre como nos tornamos falantes exímios de uma língua continua em marcha. Essa busca, necessária e premente, não deve porém ofuscar o fascínio que a conversa com uma criança nos desperta e de que o relato de Kornei Chukovsky\footnote{Poeta russo (1882-1969), autor de poesia para crianças.} é apenas um exemplo:

\begin{quote}
Quando a avó disse que o inverno estava a chegar em breve, a criança de quarto anos riu e comentou “queres dizer que o inverno tem pernas?'' \citeyearpar[tradução livre da versão inglesa de][11]{chukovsky1963}
\end{quote}

{\sloppy
\printbibliography[heading=subbibliography,notkeyword=this]
}

\begin{sidewaystable}
\begin{tabular}{lll}
\lsptoprule
Data & Autor & Título \\
\midrule
\citeyear{darwin1877} & C. Darwin & \citetitle{darwin1877} \\
\citeyear{skinner57} & B. F. Skinner & \citetitle{skinner57} \\
\citeyear{berko1958} & J. Berko Gleason & \citetitle{berko1958} \\
\citeyear{chomsky1959} & N. Chomsky & \citetitle{chomsky1959} \\
\citeyear{brown1973} & R. Brown & \citetitle{brown1973} \\
\citeyear{piattelli-palmarini1980} & M. Piattelli-Palmarini & \citetitle{piattelli-palmarini1980} \\
\citeyear{fodor1983} & J. Fodor & \citetitle{fodor1983} \\
\citeyear{bruner1983} & J. Bruner & \citetitle{bruner1983} \\
\citeyear{mannliberman1984} & V. Mann \& I. Liberman & \citetitle{mannliberman1984} \\
\citeyear{pinker1984} & S. Pinker & \citetitle{pinker1984} \\
\citeyear{slobin1985} & D. Slobin & \citetitle{slobin1985} \\
\citeyear{chomsky1986} & N. Chomsky & \citetitle{chomsky1986} \\
\lspbottomrule
\end{tabular}
\caption{Síntese cronológica de publicações referenciais no conhecimento da aquisição da linguagem.}
\end{sidewaystable}
\end{document}