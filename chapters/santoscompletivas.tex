\documentclass[output=paper]{LSP/langsci} 
\author{Ana Lúcia Santos\affiliation{Universidade de Lisboa, Faculdade de Letras, Centro de Linguística}
}
\title{Alguns aspetos da aquisição de orações subordinadas completivas}  
\ChapterDOI{10.5281/zenodo.889437}
\abstract{}
\maketitle
\begin{document}
\section{Alguns factos gerais sobre \isi{subordinação} completiva} 
\label{sec:santoscompletivas_alguns_factos}

As orações subordinadas completivas, ou, numa terminologia tradicional, orações subordinadas substantivas completivas, são orações que desempenham na frase funções tipicamente associadas a grupos nominais: são argumento externo (sujeito) - veja-se (\ref{ex:santoscompletivas_1}) - ou argumentos internos (complementos) - veja-se (\ref{ex:santoscompletivas_2}).

\ea\label{ex:santoscompletivas_1}
\ea\label{ex:santoscompletivas_1a} [Subir a árvores] assusta-me.
\ex\label{ex:santoscompletivas_1b} Surpreende-me [que a Teresa falte à aula].
\zl

\ea\label{ex:santoscompletivas_2}
\ea\label{ex:santoscompletivas_2a} A criança quer [comer a sopa depressa].
\ex\label{ex:santoscompletivas_2b} A mãe disse [que a criança comeu depressa].
\ex\label{ex:santoscompletivas_2c} A mãe obrigou o menino [a tomar o remédio].
\ex\label{ex:santoscompletivas_2d} A ideia [de que os meninos podem fazer tudo] é em geral desastrosa.
\ex\label{ex:santoscompletivas_2e} Os alunos estão desejosos [de terem boas notas no exame].
\zl

As subordinadas completivas podem ser finitas (\ref{ex:santoscompletivas_1b}, \ref{ex:santoscompletivas_2b}, \ref{ex:santoscompletivas_2d}) ou infinitivas (\ref{ex:santoscompletivas_1a}, \ref{ex:santoscompletivas_2a}, \ref{ex:santoscompletivas_2c}, \ref{ex:santoscompletivas_2e}). Ocupam uma posição argumental, podendo desempenhar a função sintática de sujeito ou complemento de verbos (veja-se (\ref{ex:santoscompletivas_1}) e (\ref{ex:santoscompletivas_2a}--\ref{ex:santoscompletivas_2c})) ou ainda a função de complemento de nomes ou adjetivos - (\ref{ex:santoscompletivas_2d}) e (\ref{ex:santoscompletivas_2e}) respetivamente. As orações completivas que são argumento interno de verbos podem ser ou não introduzidas por preposição (veja-se (\ref{ex:santoscompletivas_2a}, \ref{ex:santoscompletivas_2b} vs. \ref{ex:santoscompletivas_2c}). Já as completivas de adjetivo ou nome são sempre introduzidas por preposições (\ref{ex:santoscompletivas_2d}, \ref{ex:santoscompletivas_2e}).

Tanto as subordinadas substantivas \isi{completivas finitas} como as infinitivas exibem propriedades específicas. As finitas são sempre introduzidas por um \isi{complementador} (tipicamente \textit{que}, mas também \textit{se} no caso de interrogativas indiretas – veja-se (\ref{ex:santoscompletivas_3})).

\ea\label{ex:santoscompletivas_3}
Ela perguntou [se sabias dançar].
\z

A maioria das \isi{completivas infinitivas} tem um \isi{complementador} nulo (i.e., a posição de \isi{complementador} não é preenchida, veja-se (\ref{ex:santoscompletivas_1a}) e (\ref{ex:santoscompletivas_2a},\ref{ex:santoscompletivas_2c})). No entanto, algumas \isi{completivas infinitivas}, selecionadas por alguns verbos declarativos de ordem, são introduzidas pelo \isi{complementador} \textit{para}, que resulta da reanálise da preposição homófona – veja-se (\ref{ex:santoscompletivas_4}).

\ea\label{ex:santoscompletivas_4}
O João disse à Maria [para comer a sopa depressa].
\z

As \isi{completivas finitas} e as \isi{completivas infinitivas} exibem ainda outras propriedades distintivas. As \isi{completivas finitas} exibem contrastes de modo, em português como em outras línguas românicas. Há, assim, \isi{completivas finitas} de \isi{indicativo} (veja-se (\ref{ex:santoscompletivas_2b}, \ref{ex:santoscompletivas_2d}, \ref{ex:santoscompletivas_5a}) ou de \isi{conjuntivo} (veja-se (\ref{ex:santoscompletivas_1b}) e (\ref{ex:santoscompletivas_5b})), dependendo do verbo que as seleciona como argumento (veja-se (\ref{ex:santoscompletivas_5}), estando o verbo relevante, o da frase matriz, em itálico).

\ea\label{ex:santoscompletivas_5}
\ea\label{ex:santoscompletivas_5a} A mãe \textit{disse} [que a criança comeu depressa].
\ex\label{ex:santoscompletivas_5b} A mãe \textit{quer} [que a criança coma depressa].
\zl

As \isi{completivas finitas} que são complemento estão ainda associadas a um contraste interessante no que diz respeito à interpretação de sujeitos: como mostram os índices em (\ref{ex:santoscompletivas_6a}) e (\ref{ex:santoscompletivas_6b}), enquanto o \isi{sujeito nulo} de um complemento finito de \isi{indicativo} pode ser correferente com o sujeito da matriz (leitura preferencial) ou ter a sua referência estabelecida por uma entidade saliente no discurso ou contexto pragmático (\ref{ex:santoscompletivas_6a}), na maioria dos casos o sujeito dos complementos finitos de \isi{conjuntivo} não pode ser correferente com o sujeito da matriz (\ref{ex:santoscompletivas_6b}). A este bloqueio da correferência com o sujeito matriz associado ao \isi{conjuntivo} chama-se \isi{obviação} referencial.

\ea\label{ex:santoscompletivas_6}
\ea\label{ex:santoscompletivas_6a} [A mãe]$_i$ disse [que [-]$_{k/i}$ comia depressa].
\ex\label{ex:santoscompletivas_6b} [A mãe]$_i$ quer [que [-]$_{k/^{*}i}$ coma depressa].
\zl

Já as \isi{completivas infinitivas} exibem, em português, outro tipo de contraste: podem exibir um verbo no \isi{infinitivo} não flexionado (\ref{ex:santoscompletivas_7a}) ou no \isi{infinitivo flexionado} (\ref{ex:santoscompletivas_8}).

\ea\label{ex:santoscompletivas_7}
\ea[]{\label{ex:santoscompletivas_7a} Os arquitetos querem [reconstruir a muralha].}
\ex[*]{\label{ex:santoscompletivas_7b} Os arquitetos querem [reconstruírem a muralha].}
\zl

\ea\label{ex:santoscompletivas_8}
\ea\label{ex:santoscompletivas_8a} [(Eles) Reconstruírem a muralha] desagradou à oposição.
\ex\label{ex:santoscompletivas_8b} A oposição declarou/lamentou [reconstruírem (eles) a muralha].
\zl

A possibilidade de \isi{infinitivo flexionado} é uma propriedade do português que o distingue de outras línguas, inclusivamente línguas românicas, mas que sobrevive também, em contextos limitados, no galego.\footnote{Se considerarmos diferentes variedades do português, podemos dizer que o \isi{infinitivo flexionado} existe em português europeu na norma e em variedades coloquiais. Também se mantém no português moçambicano. Já em português brasileiro, mantém-se como parte da norma, mas não faz parte de algumas variedades coloquiais (veja--se \citealt{piresrothman2009} e \citealt{pires_etal2011} para discussão).} Os infinitivos flexionados podem ocorrer com sujeitos nulos ou com sujeitos plenos, que, quando são pronomes, exibem Caso nominativo (veja-se a opcionalidade do sujeito pleno em \ref{ex:santoscompletivas_8}). O \isi{infinitivo flexionado} tem, contudo, uma distribuição mais limitada do que o infinitivo não flexionado em completivas (veja-se \citealt{duarte_etal2016} para uma descrição pormenorizada da distribuição dos infinitivos flexionados). O \isi{infinitivo flexionado} pode ocorrer em geral em completivas que desempenham a função sintática de sujeito (\ref{ex:santoscompletivas_8a}); no caso de completivas objeto de verbos com um único argumento interno, assume-se geralmente que o \isi{infinitivo flexionado} pode ocorrer em complementos de verbos declarativos, epistémicos ou factivos (\ref{ex:santoscompletivas_8b}), mas não como complemento de verbos volitivos (\ref{ex:santoscompletivas_7b}) \citep{raposo1987}. 

As \isi{completivas infinitivas} de \isi{infinitivo flexionado} são ainda possíveis como complemento de verbos causativos e perceptivos (\ref{ex:santoscompletivas_9a}). Neste caso, o \isi{infinitivo flexionado} é uma alternativa a um complemento de infinitivo não flexionado (\ref{ex:santoscompletivas_9b}): neste último caso, como o infinitivo não flexionado não pode legitimar um sujeito com caso nominativo, esse sujeito exibe excecionalmente caso acusativo, como se verifica observando a forma do pronome em (\ref{ex:santoscompletivas_9b}). Este exemplo pode ser tratado como uma construção de \isi{elevação} para objeto, assumindo-se que o sujeito da oração encaixada se elevou para uma posição de objeto na oração superior, razão pela qual exibe caso acusativo.\largerpage

\ea\label{ex:santoscompletivas_9}
\ea\label{ex:santoscompletivas_9a} \{Mandei/deixei/vi\} [\{as crianças/elas\} fazerem o puzzle].
\ex\label{ex:santoscompletivas_9b} \{Mandei/deixei/vi\}  \{-as/as crianças\} fazer o puzzle.
\zl

No caso de completivas que são complemento de verbos com mais de um argumento interno, o \isi{infinitivo flexionado} também ocorre (\ref{ex:santoscompletivas_10}). No entanto, neste último caso, ao contrário do que acontece com os outros casos de \isi{infinitivo flexionado}, se o verbo é um verbo de controlo de objeto\is{controlo!controlo de objeto} (a referência do sujeito do complemento oracional é obrigatoriamente estabelecida pelo objeto direto ou indireto do verbo), então a leitura de controlo de objeto\is{controlo!controlo de objeto} (ou seja, a leitura em que se verifica identidade referencial entre o sujeito encaixado e o objeto matriz) mantém-se mesmo com o \isi{infinitivo flexionado} (vejam-se os índices em \ref{ex:santoscompletivas_10}). 

\ea\label{ex:santoscompletivas_10}
O presidente obrigou [os arquitetos]$_i$ [a [-]$_{i/^{*}k}$ reconstruírem a muralha].
\z

Esta muito breve descrição dos diferentes tipos de orações subordinadas completivas torna evidente a complexidade da sua aquisição.\footnote{Para uma descrição pormenorizada destas estruturas, veja-se \citealt{duarte2003}.} Como veremos, a existência de \isi{infinitivo flexionado} em português aumenta consideravelmente essa complexidade.

\section{Aquisição de orações completivas: algumas questões gerais}
\label{sec:santoscompletivas_questoes_gerais}

A aquisição de orações completivas levanta uma série de questões gerais: por um lado, as completivas são uma instância de \isi{subordinação} e discutir a sua aquisição leva a colocar toda uma série de questões gerais sobre a emergência de capacidades de produção e compreensão de processos de \isi{subordinação}. Por outro lado, as orações completivas são argumento (externo ou interno) de predicadores (verbos, nomes, adjetivos) e as suas propriedades são condicionadas pelas propriedades (sintáticas e semânticas) desses predicadores. Finalmente, e de um ponto de vista semântico, a produção de orações completivas permite expressar conceitos complexos, como os que manifestam aquilo que se tem designado \textit{Theory of Mind} (ToM).\is{Theory of Mind} Nesta secção, abordamos de forma breve estas questões.

De um ponto de vista geral, as estruturas com completivas, como acontece com as estruturas que envolvem \isi{subordinação}, exemplificam a propriedade de \textit{recursividade} da linguagem humana: um dos elementos constituintes de uma frase é ele próprio uma outra frase, sendo possível, usando esta propriedade, e de um ponto de vista puramente gramatical, construir uma frase ilimitadamente longa (veja-se (\ref{ex:santoscompletivas_11})).

\ea\label{ex:santoscompletivas_11}
O João disse [que a Ana sugeriu [que o Pedro quer [que a Rita diga [que o Francisco lamenta [que a Eva jure [que \ldots]]]]]]
\z

Da mesma forma, estruturas como as do excerto do conhecido poema de Carlos Drummond de Andrade em (\ref{ex:santoscompletivas_12}), que envolvem o encaixe sucessivo de orações relativas, exemplificam a propriedade de recursividade da linguagem humana (para questões que dizem respeito à aquisição de orações relativas, veja-se o Capítulo 10):

\ea\label{ex:santoscompletivas_12}
João amava Teresa que amava Raimundo\\
que amava Maria que amava Joaquim que amava Lili\\
que não amava ninguém \jambox{(C. Drummond de Andrade, ``Quadrilha'')}
\z

Evidentemente, as nossas capacidades, limitadas, de processamento tornam difícil a compreensão de frases como (\ref{ex:santoscompletivas_11}) e ainda mais difícil a compreensão de frases ainda mais longas. Contudo, a propriedade da recursividade é uma propriedade distintiva da linguagem humana \citep{hauser_etal2010}.\footnote{A ideia de que a recursividade é uma característica da linguagem verbal humana foi posta em causa por Daniel Everett, que defende que o Pirahã, uma língua falada na Amazónia, não exibe esta propriedade. No entanto, esta posição foi questionada por vários linguistas (veja-se a crítica em \citealt{nevins_etal2009}).} Decorrendo a possibilidade de \isi{subordinação} de uma propriedade geral e distintiva da linguagem humana, interessa observar as suas manifestações no processo de aquisição.

As estruturas envolvendo orações subordinadas não se encontram entre as primeiras combinações de palavras produzidas pelas crianças, que, na verdade, nem apresentam uma extensão (em número médio de palavras) que corresponda à extensão de frases contendo subordinadas. As orações subordinadas completivas são em geral observadas pela primeira vez no discurso das crianças cerca ou pouco depois dos 2 anos de idade, sendo o período entre os 2 e os 3 anos um período de grande expansão da variedade de estruturas subordinadas produzidas. Como daremos conta neste capítulo, as primeiras completivas produzidas são infinitivas, assumindo as estruturas com o verbo \textit{querer} um lugar de destaque, e emergindo as \isi{completivas finitas} mais tarde, tipicamente ainda no período entre os 2 e os 3 anos. Este tipo de observação já se encontra em \citet{limber1973}, um trabalho clássico centrado na aquisição do inglês.\il{inglês} Os dados de produção espontânea de que dispomos para o estudo da aquisição do português europeu confirmam estas generalizações.

A observação da ausência de orações subordinadas entre as primeiras estruturas produzidas pelas crianças poderá levar à discussão de uma série de hipóteses teóricas. Por exemplo, se se assumir que a projeção de uma subordinada (completiva ou outra) implica a projeção do domínio do \isi{complementador} (CP, de \textit{Complementizer Phrase}, em inglês),\il{inglês} como é visível em enunciados como (\ref{ex:santoscompletivas_1b}) ou (\ref{ex:santoscompletivas_2b}), que apresentam a posição C ocupada por um \isi{complementador} (\textit{que}) lexicalmente realizado, poder-se-á pensar que a ausência de \isi{subordinação} nos primeiros enunciados resulta da ausência do domínio CP (veja-se a discussão em \citet{meiselmuller1992,radford1996}). A hipótese da possível ausência de projeção do domínio CP em estádios iniciais de aquisição poderia ainda encontrar apoio em factos como a ocorrência de frases em que o \isi{complementador} \textit{que} parece ter sido omitido (veja-se \ref{ex:santoscompletivas_13}) – a omissão ocasional de \isi{complementador} foi também observada para outras línguas (veja-se \citealt{meiselmuller1992} para o alemão).\il{alemão}

\ea\label{ex:santoscompletivas_13}
SAN: Passou Ø a Ma(r)ta (es)tava a chorar.\\\jambox{(SAN 2;6.3) \citep[365]{soares2006}}
\z

Contudo, \citet{soares2006} discute esta hipótese, rejeitando-a: como mostra a autora, com base num corpus longitudinal-transversal de aquisição do português europeu,\footnote{Constituído por dados de três crianças (MAR – 1;2-2;2; SAN – 2;6-3;5; CAR – 3;6-4;6).} a ausência de \isi{complementador} realizado em (\ref{ex:santoscompletivas_13}) não se pode explicar pela ausência de C, visto que, no mesmo dia, a criança produz complementadores\is{complementador} noutros enunciados (veja-se \ref{ex:santoscompletivas_14}). Assim, os dados de produção espontânea de que dispomos não nos permitem dizer que exista um estádio em que os complementadores\is{complementador} são sistematicamente omitidos nas subordinadas; ao contrário, o que parece é que, numa fase inicial de produção de orações completivas, eles podem ser opcionalmente omitidos.

\ea\label{ex:santoscompletivas_14}
ADU: O que é que disseste?\\
SAN: Que ab(r)i isto.\jambox{(SAN 2;6.3) \citep[366]{soares2006}}
\z

Por outro lado, assumir a ausência do domínio CP na gramática das crianças de dois anos, como é o caso de SAN, cujos dados foram considerados em (\ref{ex:santoscompletivas_13}) e em (\ref{ex:santoscompletivas_14}), teria ainda o problema de deixar por explicar que, antes de produzirem completivas com \isi{complementador} realizado, as crianças produzam interrogativas wh-, que também implicam a presença do domínio CP (sobre interrogativas wh- e a questão da projeção do domínio CP nestas estruturas, veja-se o Capítulo 10). Deixaria ainda por explicar o facto de as crianças que adquirem línguas V2, que exibem movimento do verbo para C, o núcleo do domínio CP, produzirem ordens de palavras de acordo com o esperado na gramática adulta, na mesma idade (veja-se o Capítulo 7).\footnote{Nas línguas V2, o verbo flexionado na frase matriz encontra-se obrigatoriamente em segunda posição, i.e. precedido obrigatoriamente por um (apenas um) outro constituinte, que pode ser o sujeito ou outro, como um complemento ou um modificador. Nas línguas V2, o verbo sobe para ocupar C, núcleo do CP, e o constituinte em primeira posição sobe para ocupar uma posição linearmente à esquerda do verbo, no domínio CP.} Assim, será de assumir que as crianças nesta faixa etária podem projetar o domínio CP, embora eventualmente nem todas as propriedades desse domínio na língua-alvo tenham sido adquiridas. Finalmente, é também possível que o encaixe de orações, nomeadamente em estruturas de \isi{subordinação}, seja mais complexo, em termos de processamento, do que a projeção de frases simples. Essa hipótese é explorada por \citet{soares2006} e está de acordo com o facto de a emergência de estruturas com encaixe de várias subordinadas finitas ser particularmente tardia. Em (\ref{ex:santoscompletivas_15}) apresentamos o único caso de encaixe sucessivo de duas subordinadas observado por Soares (2006), que ocorre já depois dos 4 anos de idade.

\ea\label{ex:santoscompletivas_15}
Eu acho [que sei [onde é que está o relógio]].\\\jambox{(CAR 4;15.19) \citep[369]{soares2006}}
\z

Do mesmo modo, a extração longa de constituintes wh- a partir de subordinadas completivas é rara e tardia (veja-se \ref{ex:santoscompletivas_16}, o primeiro dos únicos dois exemplos identificados por Soares no seu \textit{corpus}), contrastando com o movimento de wh- em frases raiz, que em geral já se observa por volta dos dois anos (veja-se o Capítulo 10).

\ea\label{ex:santoscompletivas_16}
O que$_i$ é que queres [que eu vista [-]$_i$]?\jambox{(CAR 4;5.19) \citep[368]{soares2006}]}
\z

Uma outra questão a considerar quando discutimos a aquisição de estruturas de \isi{subordinação} completiva diz respeito à interação entre o léxico e a sintaxe. Como foi já apontado na Secção \ref{sec:santoscompletivas_alguns_factos}, as orações completivas são argumentos selecionados por predicadores (verbos, nomes, adjetivos) e quer a possibilidade de ter como complemento uma oração quer determinadas propriedades dessa oração dependem do predicador. Deste último facto é exemplo a alternância entre \isi{infinitivo flexionado} e não flexionado nas \isi{completivas infinitivas} (por exemplo, uma completiva selecionada por \textit{querer} não admite \isi{infinitivo flexionado}) ou a alternância de modo (\isi{indicativo} / \isi{conjuntivo}) nas \isi{completivas finitas} (a completiva finita selecionada por \textit{querer}, verbo volitivo, exibe modo \isi{conjuntivo}; já a selecionada por \textit{dizer}, verbo declarativo, exibe modo \isi{indicativo}). A aquisição de completivas relaciona-se, assim, intimamente com a aquisição de propriedades dos predicadores. Em alguns casos, pelo menos, essas propriedades são semânticas, dizem respeito ao significado do verbo: por exemplo, de acordo com \citet{marques1995}, os verbos matriz que expressam uma atitude epistémica (de crença ou conhecimento) positiva selecionam o \isi{indicativo}, selecionando os restantes verbos o \isi{conjuntivo} para a sua completiva finita. Nesse caso, como se dá a aquisição? Serão pistas semânticas a conduzir a aquisição de alguns aspetos da sintaxe (\textit{semantic bootstrapping})? Ou serão pistas sintáticas a fornecer evidência para a aquisição da semântica (\textit{syntactic bootstrapping})?

A questão da relação entre léxico, semântica e sintaxe tem sido central em todos os trabalhos centrados na aquisição de verbos. \citet{pinker1984} sugere que as crianças usam noções semânticas básicas para a identificação de classes de palavras no \textit{input}: por exemplo, palavras que designam coisas ou pessoas corresponderão a nomes; palavras que designam ações ou mudanças de estado corresponderão a verbos. Assim, Pinker defende o que chama \textit{semantic bootstrapping} na aquisição da sintaxe. Por outro lado, outros autores, nomeadamente Leila Gleitman (\citealt{gleitman_etal2005} e referências aí contidas), sugerem que os contextos sintáticos em que os diferentes verbos ocorrem, que correspondem à sua estrutura argumental, funcionam como restrição ao significado que as crianças lhes podem atribuir. Os verbos que selecionam completivas não designam ações concretas, pelo que as ações físicas observáveis no contexto situacional em que ocorrem não são suficientemente informativas para guiar as crianças na aquisição do seu significado. No caso desses verbos, as pistas sintáticas podem ser particularmente relevantes, podendo a ocorrência do verbo com uma oração completiva ser uma pista relevante para o seu significado. Na verdade, confrontados com um verbo desconhecido (como o pseudo-verbo \textit{protar}, em (\ref{ex:santoscompletivas_17}), inferimos, com base no contexto sintático em que ocorre, que esse verbo deverá referir-se a um estado mental, como \textit{achar}, \textit{pensar}, ou poderá corresponder a um verbo declarativo, como \textit{dizer}, mas não será um verbo com um significado causativo, como \textit{destruir}, por exemplo (veja-se a revisão em \citealt[90--98]{guasti2002}).

\ea\label{ex:santoscompletivas_17}
O menino \textit{protou} que a mãe saiu de casa.
\z

No entanto, nem todos os verbos que se referem a atitudes mentais têm significados semelhantes, como podemos perceber se pensarmos em verbos como \textit{querer} e \textit{pensar}. Muito recentemente, \citet{hacquard2014} explorou a possibilidade de que a estrutura sintática do complemento de verbos como \textit{want} `querer' e think `pensar' funcione como pista para o seu significado - nomeadamente, o facto de \textit{want} selecionar obrigatoriamente um complemento infinitivo, enquanto \textit{think} seleciona um complemento finito (numa língua como o português, o contraste relevante pode envolver o modo \isi{indicativo} / modo \isi{conjuntivo}). Neste caso, estaríamos perante uma situação de recurso a pistas sintáticas (\textit{syntactic bootstrapping}) para aquisição de propriedades finas que dizem respeito à semântica lexical de verbos e, consequentemente, à interpretação dos seus complementos.

Em síntese, é possível pensar que as pistas sintáticas e semânticas são complementares na aquisição lexical. A aquisição de verbos que selecionam orações completivas como argumento é certamente uma área em que a exploração das interfaces entre léxico, sintaxe e semântica na aquisição é particularmente relevante e em que muito está ainda por descobrir. Voltaremos a esta questão na Secção \ref{subsec:santoscompletivas_questoes_controlo}.

A aquisição de verbos que selecionam completivas interage ainda muito diretamente com o desenvolvimento cognitivo das crianças. Por exemplo, sabe-se que o verbo \textit{want} (‘querer’), volitivo, emerge no discurso das crianças mais precocemente (antes do terceiro ano de vida) do que verbos como \textit{think} (‘pensar’) ou \textit{know} (‘saber’), que expressam crença ou conhecimento (e que emergem, geralmente, durante o terceiro ano de vida) (veja-se \citealt{devilliers2007}). Particularmente relevantes são os casos em que estes verbos permitem falar das crenças de outros, como em frases como a que se apresenta em (\ref{ex:santoscompletivas_18}). 

\ea\label{ex:santoscompletivas_18}
A Maria pensa que o chocolate está na caixa.
\z

Na verdade, alguns autores mostraram que estes verbos emergem na produção das crianças antes de estas mostrarem ser capazes de considerar as crenças dos outros, incluindo casos em que essas crenças são falsas (por exemplo, o caso em que a Maria pensa que o chocolate está na caixa, mas ele está afinal no frigorífico).

A capacidade de considerar crenças falsas (de outros) é uma das manifestações de desenvolvimento de \textit{Theory of Mind},\is{Theory of Mind} capacidade essa que se começa a observar por volta dos 4 anos de idade. Neste sentido, vários autores salientam a relevância da interação entre desenvolvimento linguístico e desenvolvimento cognitivo nesta área, sugerindo que o desenvolvimento linguístico, neste caso, desempenha um papel importante no desenvolvimento do conceito de crença (veja-se \citealt{devilliers2007} para uma revisão).

Esta breve síntese indica já a complexidade da questão da aquisição de estruturas com orações completivas. Parte dessa complexidade resulta do facto de a aquisição destas estruturas interagir diretamente quer com várias áreas do desenvolvimento linguístico quer com o desenvolvimento cognitivo. Nas secções seguintes, apresentam-se, de forma sucinta, alguns factos hoje conhecidos sobre a aquisição de diferentes tipos de orações completivas, separando-se as \isi{completivas infinitivas} e as \isi{completivas finitas}.

\section{Completivas infinitivas\is{completivas infinitivas}}
\label{sec:santoscompletivas_completivas_infinitivas}

Nesta secção, consideraremos apenas dois tipos de oração completiva infinitiva: estruturas que envolvem controlo e estruturas de \isi{infinitivo flexionado}. Estas últimas são, como já referimos, uma característica particular do português. 

\subsection{Questões específicas da aquisição de controlo}
\label{subsec:santoscompletivas_questoes_controlo}

As orações completivas de \isi{infinitivo} não flexionado, como as que se encontram em (\ref{ex:santoscompletivas_19a}--\ref{ex:santoscompletivas_19c}), têm sido objeto de investigação em aquisição de língua materna. O interesse da investigação tem-se sobretudo centrado na interpretação que as crianças atribuem a estas frases, particularmente no que diz respeito à forma como as crianças interpretam a posição vazia de sujeito da infinitiva (representada como [-]).

\ea\label{ex:santoscompletivas_19}
\ea\label{ex:santoscompletivas_19a} Os gatos querem [[-] saltar o muro].
\ex\label{ex:santoscompletivas_19b} Os gatos ensinaram os porcos [a [-] saltar o muro].
\ex\label{ex:santoscompletivas_19c} Os gatos prometeram aos porcos [[-] saltar o muro].
\zl

Como se mostra em (\ref{ex:santoscompletivas_20}), através da utilização de índices, a referência do sujeito da infinitiva é obrigatoriamente estabelecida por um grupo nominal na frase matriz (a sua referência não pode ser estabelecida por uma entidade externa à frase), dizendo-se que esse sujeito é controlado por um dos argumentos na frase matriz. Estas estruturas são, assim, chamadas estruturas de \textit{controlo} e o elemento na matriz que estabelece a referência do sujeito da encaixada é chamado \textit{controlador}.

\ea\label{ex:santoscompletivas_20}
\ea\label{ex:santoscompletivas_20a} [Os gatos]$_i$ querem [[-]$_{i/^{*}k}$ saltar o muro].
\ex\label{ex:santoscompletivas_20b} [Os gatos]$_i$ ensinaram [os porcos]$_m$ [a [-]$_{m/^{*}i/^{*}k}$ saltar o muro].
\ex\label{ex:santoscompletivas_20c} [Os gatos]$_i$ prometeram [aos porcos]$_m$ [[-]$_{i/^{*}m/^{*}k}$ saltar o muro].
\zl

No caso de (\ref{ex:santoscompletivas_20a}), o verbo \textit{matriz},  \textit{querer}, seleciona apenas um argumento interno, a própria oração completiva infinitiva. Assim, o único controlador possível do sujeito da infinitiva é o sujeito da matriz. Esta estrutura é, portanto, uma estrutura de controlo de sujeito\is{controlo!controlo de sujeito}. O verbo \textit{querer} é um dos verbos que mais precocemente ocorre com completivas no discurso espontâneo das crianças (veja-se a Secção \ref{sec:santoscompletivas_questoes_gerais}) e essas primeiras ocorrências integram precisamente complementos infinitivos: em (\ref{ex:santoscompletivas_21}) apresentam-se frases em que \textit{querer} ocorre com uma completiva infinitiva, produzidas antes ou cerca dos dois anos. Importa, claro, perceber se a interpretação que as crianças atribuem a estas frases é semelhante à que os adultos lhes atribuem.

\ea\label{ex:santoscompletivas_21}
\ea\label{ex:santoscompletivas_21a} a  (I)nês que(r) i(r) pa(ra) a casa.\jambox{(INI, 1;10.29) [\textit{corpus} Santos]}
\ex\label{ex:santoscompletivas_21b} a foca qu(er)ia sai(r).\jambox{(TOM, 2;1.7) [\textit{corpus} Santos]}
\zl

Os dados recolhidos para o português por \citet{agostinho2014} mostram que as crianças em idade pré-escolar (3--5 anos) têm um comportamento adulto no que diz respeito à interpretação da categoria vazia na posição de sujeito da infinitiva quando o verbo é \textit{querer} ou \textit{conseguir}, ambos verbos de controlo de sujeito\is{controlo!controlo de sujeito} com um único argumento interno, a oração infinitiva. No entanto, não é impossível que crianças mais novas atribuam uma interpretação não adulta a enunciados como (\ref{ex:santoscompletivas_20a}). \citet{landauthornton2011} mostram que uma criança monolingue que adquire o inglês\il{inglês} produz muito precocemente (antes dos dois anos) estruturas com \textit{want} equivalentes a (\ref{ex:santoscompletivas_20a}), i.e. estruturas de controlo de sujeito\is{controlo!controlo de sujeito} (o caso em \ref{ex:santoscompletivas_22a}). A mesma criança só mais tardiamente produz uma estrutura com \textit{want} em que não há controlo de sujeito\is{controlo!controlo de sujeito} (\ref{ex:santoscompletivas_22b}) – neste caso, o sujeito lógico da oração encaixada é \textit{somebody} (em português, a interpretação equivalente corresponderia a uma oração finita com \isi{conjuntivo}, como se verifica pela tradução apresentada).

\ea\label{ex:santoscompletivas_22}
\ea{\label{ex:santoscompletivas_22a}
\gll I want {to see} paper.\\
eu quero ver.\textsc{inf} papel\\\jambox{(1;08.10)}
\glt `Eu quero ver o paper.'
}
\ex{\label{ex:santoscompletivas_22b}
\gll I just want somebody {to play} {with me}. ~~ (2;05.02)\\
	eu só quero alguém brincar.\textsc{inf} comigo\\
\glt	`Eu só quero que alguém brinque comigo.'\\\jambox{\citep[926,928]{landauthornton2011}}
}
\zl

O que é interessante é que, no período em que ainda não produz a estrutura em (\ref{ex:santoscompletivas_22b}) e numa situação em que quer dar conta da sua vontade de que outra pessoa faça determinada coisa, a criança produz estruturas como (\ref{ex:santoscompletivas_22a}) mas que não têm uma interpretação correspondente a controlo de sujeito\is{controlo!controlo de sujeito} (é este o caso em \ref{ex:santoscompletivas_23}). Neste caso, o sujeito de ``push'' (``empurrar'') parece poder ser interpretado como arbitrário. 

\ea\label{ex:santoscompletivas_23}
Contexto: A Laura quer que a mãe a empurre.\\
\gll Laura: I want \_ push Laura.\\
{~} eu quero ~ empurrar.\textsc{inf} Laura.\\\jambox{(1;7.19) \citep[927]{landauthornton2011}}
\z

Também \citet{mcdaniel_etal1990} verificaram que algumas das crianças que testaram (apenas duas crianças, no entanto) podem ter uma interpretação não adulta de enunciados que um adulto interpretará como contextos de controlo obrigatório. Este assunto carece ainda de uma investigação mais aprofundada.

Voltemos agora às estruturas em (\ref{ex:santoscompletivas_20}), centrando-nos nos enunciados em (\ref{ex:santoscompletivas_20b}) e em (\ref{ex:santoscompletivas_20c}). Nestes casos, o verbo matriz (\textit{ensinar}, \textit{prometer}) é um predicado de três lugares, selecionando um argumento externo e dois argumentos internos. Assim, o sujeito não realizado lexicalmente na infinitiva, a ser obrigatoriamente controlado por um argumento na oração matriz, poderá hipoteticamente ser controlado ou pelo sujeito da matriz ou por um argumento interno do verbo (o complemento direto, no caso de \textit{ensinar}; o complemento indireto, no caso de \textit{prometer}). No entanto, sabemos que apenas uma das opções é gramatical para o adulto, dependendo a escolha do controlador do verbo matriz: \textit{ensinar} é um verbo de controlo de objeto\is{controlo!controlo de objeto} (veja-se \ref{ex:santoscompletivas_20b}); \textit{prometer} é um verbo de controlo de sujeito\is{controlo!controlo de sujeito} (veja-se \ref{ex:santoscompletivas_20c}). Serão as crianças capazes de estabelecer esta distinção?

Carol Chomsky, num trabalho clássico publicado em \citeyear{chomsky1969}, mostra que há assimetrias na forma como as crianças interpretam estruturas de controlo com \textit{promise} (‘prometer’) vs. \textit{tell} (‘dizer’), sendo este último um verbo de controlo de objeto\is{controlo!controlo de objeto}. A autora usou uma tarefa de representação, como se exemplifica em (\ref{ex:santoscompletivas_24}).

\ea\label{ex:santoscompletivas_24}
\ea{\label{ex:santoscompletivas_24a}
\gll Bozo tells Donald {to hop} up and down. Make him hop.\\
Bozo diz Donald saltar.\textsc{inf} {para cima} e {para baixo}. Faz o saltar.\textsc{inf}\\
\glt `O Bozo diz ao Donald para saltar para cima e para baixo. Fá-lo saltar.'
}
\ex{\label{ex:santoscompletivas_24b}
\gll Bozo promises Donald {to do} {a somersault}. Make him do it\\
Bozo prometeu Donald fazer.\textsc{inf} cambalhota faz o fazer.\textsc{inf} isso\\
\glt	`O bozo prometeu ao Donald fazer uma cambalhota. Fá-lo fazer isso.'\jambox{\citep[33]{chomsky1969}}
}
\zl

Os resultados obtidos com a aplicação desta tarefa mostraram que as crianças preferem em geral leituras de controlo de objeto,\is{controlo!controlo de objeto} o que resulta em interpretações não adultas de frases com \textit{promise} (\ref{ex:santoscompletivas_24b}) e em interpretações semelhantes à adulta de frases com \textit{tell} (\ref{ex:santoscompletivas_24a}). Verifica-se, portanto, uma assimetria entre a interpretação de estruturas de controlo de objeto\is{controlo!controlo de objeto} e a interpretação de estruturas de controlo de sujeito\is{controlo!controlo de sujeito} com predicados de três lugares como \textit{promise}. Uma assimetria semelhante foi observada para o português em \citet{agostinho2014}: crianças entre os 3 e os 5 anos, falantes monolingues do português europeu, têm menos resultados convergentes com os adultos na interpretação de frases com \textit{prometer}, em que se espera controlo pelo sujeito da matriz (como em \ref{ex:santoscompletivas_25a}), do que na interpretação de frases com \textit{ensinar} ou \textit{proibir}, em que se espera controlo de objeto\is{controlo!controlo de objeto} (como em \ref{ex:santoscompletivas_25b}). Na verdade, no que diz respeito à interpretação de frases como em (\ref{ex:santoscompletivas_25a}), com \textit{prometer}, mesmo o grupo de 5 anos testado neste estudo apresenta apenas 55\% de respostas adultas (i.e. controlo de sujeito\is{controlo!controlo de sujeito}), o que indica que a interpretação deste tipo de estrutura é de desenvolvimento mais tardio, prolongando-se já em idade escolar. 

\ea\label{ex:santoscompletivas_25}
\ea\label{ex:santoscompletivas_25a} Contexto: Numa casa vivem três animais: o pato, o galo e o coelho. São vizinhos dos animais da quinta. O pato diz um dia: ``E se convidássemos alguns amigos para virem cá jantar?'' Os outros dizem: ``Sim! Sim! É uma óptima ideia!'' Então, dividem as tarefas, e\ldots\\
Item de teste: \textit{O galo promete ao coelho cozinhar o jantar}.\\
Quem é que vai cozinhar o jantar?\\
Resposta esperada: O galo. (controlo de sujeito\is{controlo!controlo de sujeito})
\ex\label{ex:santoscompletivas_25b} Contexto: Um texugo, um ganso e um esquilo são vizinhos num bosque. É janeiro, e o Natal foi há pouco tempo. Chega o esquilo e diz para os outros dois: ``Olhem o que me deram no Natal: um skate!'' Então\ldots\\
Item de teste: \textit{O texugo ensina o ganso a andar de skate}.\\
Quem é que vai andar de skate?\\
Resposta esperada: O ganso. (controlo de objeto\is{controlo!controlo de objeto})
\zl

No entanto, os dados de \citet{agostinho2014} mostram também que, aos 3--4 anos, as crianças não têm um comportamento completamente adulto mesmo na interpretação de estruturas de controlo de objeto\is{controlo!controlo de objeto} (como \ref{ex:santoscompletivas_25b}): embora a maioria das respostas correspondam a controlo de objeto\is{controlo!controlo de objeto} (a leitura adulta), os grupos de 3 e 4 anos apresentaram cerca de 30\% de respostas não esperadas, que correspondem a controlo de sujeito\is{controlo!controlo de sujeito}. Esta é, também, uma área que merece investigação aprofundada.

\subsection{O caso particular do \isi{infinitivo flexionado} em português}
\label{subsec:santoscompletivas_caso_particular}

Como vimos na Secção \ref{sec:santoscompletivas_alguns_factos}, o português exibe, a par de orações completivas de \isi{infinitivo} não flexionado (de que são exemplo as estruturas discutidas na Secção \ref{subsec:questoes_controlo}), também orações completivas de \isi{infinitivo flexionado}. Como vimos, em orações completivas, o \isi{infinitivo flexionado} tem uma distribuição limitada: pode ocorrer em geral em completivas que desempenham a função de sujeito na frase matriz; ao contrário, no caso de completivas que desempenham a função de objeto na frase matriz, o \isi{infinitivo flexionado} pode ocorrer em complementos de verbos declarativos, epistémicos e factivos, bem como em complementos de verbos causativos e percetivos, mas não pode ocorrer em complementos de verbos volitivos. Além disso, o \isi{infinitivo flexionado} pode ainda ocorrer geralmente em orações subordinadas adverbiais, nomeadamente as finais\is{orações finais} introduzidas por \textit{para}, de que falaremos particularmente.

Neste momento, dispomos já de dados que nos indicam que o \isi{infinitivo flexionado}, apesar de raro nas línguas do mundo, emerge cedo no discurso espontâneo das crianças. A análise de um \textit{corpus} de produção espontânea que reúne dados de três crianças entre 1;5 e 3;11 mostrou que o \isi{infinitivo flexionado} emerge no discurso espontâneo por volta dos dois anos, em geral no mesmo período em que emergem complementadores\is{complementador} (\textit{que}) lexicalmente realizados. Estes dados mostram-nos ainda que as crianças produzem infinitivos flexionados restringindo-os aos contextos esperados (nomeadamente, não produzem infinitivos flexionados como complementos do verbo volitivo \textit{querer}, um verbo frequente no discurso espontâneo); por outro lado, mostram-nos que os infinitivos flexionados emergem em contextos diferentes em idades diferentes.

Os primeiros contextos em que o \isi{infinitivo flexionado} ocorre não são, no entanto, completivas, tratando-se antes de \isi{orações finais} introduzidas por \textit{para} (vejam-se os exemplos em \ref{ex:santoscompletivas_26}). Como se observa também em (\ref{ex:santoscompletivas_26}), muitas das primeiras ocorrências de infinitivos flexionados são casos de 1.ª ou 3.ª pessoas do singular, casos em que o \isi{infinitivo flexionado} não tem um morfema de pessoa/número foneticamente realizado: nesses casos, é a presença de um sujeito lexicalmente realizado (\textit{o u(r)so} em \ref{ex:santoscompletivas_26a}) que assinala a presença do \isi{infinitivo flexionado}.\footnote{O infinitivo não flexionado não legitima um sujeito lexicalmente realizado, como se observa em (\ref{ex:santoscompletivas_i}).

\ea\label{ex:santoscompletivas_i}
\ea[*]{\label{ex:santoscompletivas_ia} Os meninos querem os pais fazer bife para o jantar.}
\ex[*]{\label{ex:santoscompletivas_ib}Ponho aí a peça para tu montar.}
\zl
}
No entanto, a flexão ocorre quando é esperada, noutras combinações de pessoa-número, como em (\ref{ex:santoscompletivas_26b}) (para mais informação, veja-se \citealt{santos_rothman_etal2013}).

\ea\label{ex:santoscompletivas_26}
\ea\label{ex:santoscompletivas_26a} MAE: vão buscar papa?\\MAE: para quem?\\TOM:  pó [: para o] \textit{u(r)so} \textbf{come(r)}.\jambox{(1;11.12) [\textit{corpus} Santos]}
\ex\label{ex:santoscompletivas_26b} TOM: ponh(o) aí pa(ra) \textbf{faze(re)s} [?] \# (es)tá?\\\jambox{(2;8.9) [\textit{corpus} Santos]}
\zl

A observação dos mesmos dados mostrou que, mais tarde, mas ainda antes ou por volta dos três anos, o \isi{infinitivo flexionado} emerge em completivas, embora num tipo de completivas em particular: complementos de verbos percetivos e causativos, nomeadamente \textit{ver} e \textit{deixar} (\ref{ex:santoscompletivas_27}).\footnote{No caso dos complementos de verbos percetivos, o \isi{infinitivo flexionado} é possível em complementos que são construções de \isi{infinitivo} preposicionado \citep{raposo1989}, como é o caso de (\ref{ex:santoscompletivas_27a}), em que ocorre a preposição \textit{a}; é possível ainda em complementos não preposicionados, como (\ref{ex:santoscompletivas_27c}). Em (\ref{ex:santoscompletivas_27c}), bem como em (\ref{ex:santoscompletivas_27b}), a presença de um \isi{infinitivo flexionado} é assinalada pela presença de um sujeito pronominal com Caso nominativo.}

\ea\label{ex:santoscompletivas_27}
\ea\label{ex:santoscompletivas_27a} TOM. ainda vi \# os [/] os [/] os senhores a [/] a jogarem \# ténis.\jambox{(3;0.22) [\textit{corpus} Santos]}
\ex\label{ex:santoscompletivas_27b} TOM: deixa eu \# vi(rar) [/] virar.\jambox{(3;0.22) [\textit{corpus} Santos]}
\ex\label{ex:santoscompletivas_27c} TOM: viste \# ela \# sa(ir) [/] sair?\jambox{(3;1.25) [\textit{corpus} Santos]}
\zl

A ocorrência de infinitivos flexionados em orações complementos destes verbos foi confirmada num estudo experimental, apresentado em \citet{santos_etal2016}. As autoras aplicaram um teste de produção provocada a crianças em idade pré-escolar (3 a 5 anos) e a um grupo de controlo adulto. Quando chamadas a completar frases com verbos como \textit{deixar} e \textit{mandar}, as crianças produziram frequentemente (sobretudo a partir dos 4 anos) complementos com \isi{infinitivo flexionado} (\ref{ex:santoscompletivas_28}).

\ea\label{ex:santoscompletivas_28} O pai mandou os outros tigres saírem da casota.\jambox{(3;8.23)}
\z

Quando chamadas a completar frases com o percetivo \textit{ver}, as crianças produziram sobretudo, para além de \isi{completivas finitas} com \isi{indicativo}, como em (\ref{ex:santoscompletivas_29}), construções de \isi{infinitivo} preposicionado, frequentemente com \isi{infinitivo flexionado}, como é o caso em (\ref{ex:santoscompletivas_30}).

\ea\label{ex:santoscompletivas_29} (O leão) viu que eles tinham comido o bife.\jambox{(3;5.16)}
\z
\ea\label{ex:santoscompletivas_30} O leão viu os tigres a comerem o bife todo.\jambox{(3;11.04)}
\z

As estruturas em (\ref{ex:santoscompletivas_28}) a (\ref{ex:santoscompletivas_30}) correspondem às estruturas preferidas pelas crianças em idade pré-escolar quando se trata de produzir complementos de verbos causativos e percetivos, respetivamente. Estes verbos permitem, no entanto, um outro tipo de complemento (\ref{ex:santoscompletivas_31}), um complemento com \isi{infinitivo} não flexionado que coocorre com um grupo nominal marcado com Caso acusativo (visível quando se trata de um pronome, como em \ref{ex:santoscompletivas_31b}). Esta estrutura só raramente ocorreu nas respostas das crianças à tarefa de completamento (os casos como \ref{ex:santoscompletivas_31} são raros), o que significa que os complementos de percetivos e de causativos são contextos em que as crianças parecem efetivamente preferir infinitivos flexionados às alternativas com \isi{infinitivo} não flexionado, disponíveis na língua.\footnote{
As estruturas em (\ref{ex:santoscompletivas_31}) podem ser analisadas como estruturas de \isi{elevação} para objeto, i.e. estruturas em que o grupo nominal que coocorre com o \isi{infinitivo}, não podendo ser legitimado pelo \isi{infinitivo} não flexionado, se elevaria para a posição de objeto do verbo superior e, por isso, manifestaria Caso acusativo (veja-se \citet{santos_etal2016}). Há um longo debate na literatura acerca da disponibilidade de \isi{elevação} nas gramáticas de crianças em idade pré-escolar, centrado sobretudo em estruturas de \isi{elevação} para sujeito, como a que se observa com o verbo \textit{seem} ‘parecer’. Alguns autores, como \citet{hirschwexler2007}, sugerem que a compreensão de estruturas de \isi{elevação} para sujeito, como a que se observa em (\ref{ex:santoscompletivas_i2}) só começam a ser compreendidas de forma adulta por volta dos sete anos de idade (veja-se ainda\citet{orfitelli2012} para uma discussão mais alargada destas estruturas). Já outros autores, como \citet{becker2005}, sugerem que as crianças compreendem mais precocemente estruturas de \isi{elevação}. Não desenvolveremos, neste capítulo, discussão sobre aquisição de estruturas de \isi{elevação}, cujo estudo, para o português, é ainda incipiente.

\ea\label{ex:santoscompletivas_i2}
\gll Bart seems to Lisa {to be} playing an instrument.\\ 
Bart parece à Lisa  estar tocando um instrument\\
\glt ‘À Lisa, parece que o Bart está a tocar um instrumento.’
\z
}

\ea\label{ex:santoscompletivas_31}
\ea\label{ex:santoscompletivas_31a} \ldots mandou os filhos empurrar o carro do polícia.\jambox{(3;10.21)}
\ex\label{ex:santoscompletivas_31b} O pai deixou-\textit{os} sair da casa um bocadinho.\jambox{(3;7.08)}
\zl

O \isi{infinitivo flexionado} pode ainda ocorrer em complementos de verbos de controlo de objeto\is{controlo!controlo de objeto}, um caso especial, em que tal \isi{infinitivo} não pode coocorrer com um sujeito lexicalmente realizado e em que se mantém a interpretação de uma estrutura de controlo de objeto\is{controlo!controlo de objeto} (veja-se a Secção \ref{sec:santoscompletivas_alguns_factos}). Este contexto foi escassamente explorado para o português, mas a mesma tarefa experimental aqui mencionada, descrita em \citet{santos_etal2016}, mostra que este é também um contexto frequente de produção de \isi{infinitivo flexionado}, pelo menos por crianças de 5 anos: mais de 30\% das respostas ao pedido de completamento de frases em que se esperava um verbo de controlo de objeto\is{controlo!controlo de objeto} corresponderam, nesta faixa etária, a respostas deste tipo (veja-se \ref{ex:santoscompletivas_32}).

\ea\label{ex:santoscompletivas_32} (O macaco) ensinou os patinhos a saltaram.\jambox{(5;0.13)}
\z

\section{Completivas finitas}
\label{sec:santoscompletivas_completivas_finitas}

Nesta secção, sintetizaremos alguns resultados de trabalhos recentes que dão conta do percurso de aquisição de \isi{completivas finitas}. No caso das orações \isi{completivas finitas}, é de destacar, em português, o contraste entre \isi{indicativo} e \isi{conjuntivo}, bem como a questão associada da interpretação de sujeitos em \isi{completivas finitas} de \isi{indicativo} e de \isi{conjuntivo} – são esses os objetos das Secções \ref{subsec:santoscompletivas_contraste}. e \ref{subsec:santoscompletivas_interpretacao_nulos}, respetivamente. Antes de avançarmos para essas questões, na Secção \ref{subsec:santoscompletivas_realizacao_complementador} faremos algumas observações sobre emergência de diferentes tipos de \isi{completivas finitas} associadas a diferentes tipos de \isi{complementador}.


\subsection{Realização do \isi{complementador}}
\label{subsec:santoscompletivas_realizacao_complementador}

Do conjunto das orações completivas produzidas espontaneamente pelas crianças, e que emergem geralmente por volta dos dois anos ou mesmo um pouco antes, parece-nos seguro dizer que as mais precocemente produzidas são \isi{completivas infinitivas}, particularmente as selecionadas por \textit{querer} (veja-se o exemplo \ref{ex:santoscompletivas_21}, na Secção \ref{subsec:santoscompletivas_questoes_controlo}, aqui repetido em \ref{ex:santoscompletivas_33}). 

\ea\label{ex:santoscompletivas_33}
\ea\label{ex:santoscompletivas_33a} a  (I)nês que(r) i(r) pa(ra) a casa.\jambox{(INI, 1;10.29) [\textit{corpus} Santos]}
\ex\label{ex:santoscompletivas_33b} a foca qu(er)ia sai(r).\jambox{(TOM, 2;1.7) [\textit{corpus} Santos]}
\zl

Nesses casos, não se espera um \isi{complementador} lexicalmente realizado e, visto que a estrutura é superficialmente (apenas superficialmente) semelhante a estruturas com verbos auxiliares, poderá sempre considerar-se a possibilidade de que as infinitivas complemento de \textit{querer} em estádios iniciais não sejam forçosamente tratadas pela criança como orações com o tipo de estrutura que encontramos na gramática adulta (nomeadamente, estruturas em que se projeta um domínio CP na encaixada – mas veja-se a discussão na Secção \ref{sec:santoscompletivas_questoes_gerais}). Importou sempre, por isso, aos investigadores, determinar em que momento emergem \isi{completivas finitas}, com \isi{complementador} realizado. Estas ocorrem no discurso espontâneo tipicamente entre os dois e os três anos (sabemos, além disso, que, neste período inicial de produção de completivas, pode ocasionalmente ocorrer omissão de \isi{complementador}, como se observou na Secção \ref{sec:santoscompletivas_questoes_gerais}). Por exemplo, no caso das crianças que produziram as frases em (\ref{ex:santoscompletivas_33}),\largerpage[2] encontram-se no \textit{corpus} \isi{completivas finitas} um pouco mais tarde, aos 2;5 e 2;8,\footnote{Antes, ocorrem, aos 2;1, na produção destas duas crianças, estruturas elípticas como:
\ea\label{ex:santoscompletivas_i3}INI: a(cho) que não.\jambox{(2;1.10)}
\z\vspace*{-\baselineskip}} como se observa em (\ref{ex:santoscompletivas_34}).\footnote{Estes dados foram tratados no âmbito do projeto Completivas na Aquisição do Português (PTDC/CLE-LIN/120897/2010), projeto com financiamento da Fundação para a Ciência e a Tecnologia e desenvolvido no Centro de Linguística da Universidade de Lisboa.}

\ea\label{ex:santoscompletivas_34}
\ea\label{ex:santoscompletivas_34a} eu acho que é meu.\jambox{(INI, 2;8.23) [\textit{corpus} Santos]}
\ex\label{ex:santoscompletivas_34b} acho que é difí(cil).\jambox{(TOM, 2;5.3) [\textit{corpus} Santos]}
\zl

\citet{soares2006} discute exatamente esta questão, com base num outro \textit{corpus} de produção espontânea. A autora encontra produção de uma completiva finita com \isi{complementador} realizado aos 2;2, no caso de uma criança, ou aos 2;9, no caso de outra (exemplos em \ref{ex:santoscompletivas_35}). 

\ea\label{ex:santoscompletivas_35}
\ea\label{ex:santoscompletivas_35a} Queres ver que eu ando? \jambox{(Marta;2.17) \citep[363]{soares2006}}
\ex\label{ex:santoscompletivas_35b} Agora acho que eu vou \# arranjar.\jambox{(Sandra 2;9.22) \citep[365]{soares2006}}
\zl

Marginalmente, podemos observar alguma coincidência nos verbos com que ocorrem estas primeiras \isi{completivas finitas}: \textit{achar} parece, à luz destes dados, estar associado a muitas destas primeiras ocorrências; o verbo percetivo \textit{ver} também parece poder ocorrer precocemente com \isi{completivas finitas} (veja-se \ref{ex:santoscompletivas_35a}), o que está de acordo com os dados de produção provocada de crianças de 3 anos obtidos por \citet{santos_etal2016} (\ref{subsec:santoscompletivas_caso_particular}). Outros verbos que selecionam estas primeiras \isi{completivas finitas} são o declarativo \textit{dizer} ou o epistémico \textit{saber}, como se observa em (\ref{ex:santoscompletivas_36}). Já o verbo \textit{pensar} com completiva finita, no mesmo \textit{corpus}, ocorre muito raramente e um pouco mais tarde (\ref{ex:santoscompletivas_37}).

\ea\label{ex:santoscompletivas_36}
\ea\label{ex:santoscompletivas_36a} Adulto: olha \# o [/] o princípio agarrou no sapato \# e foi fazer o quê?\\
Criança: dizer [?] à Branca+de+neve \# que a Branca+de+neve (es)tava doente.\jambox{(INM 2;6.19) [\textit{corpus} Santos]}
\ex\label{ex:santoscompletivas_36b} sabes que eu tenho um cavalo \# que mordeu uma vez.\\(TOM 3;1.23) [\textit{corpus} Santos]
\zl

\ea\label{ex:santoscompletivas_37} pensava que ele não tinha a boca aberta.\jambox{(TOM 3;4.25) [\textit{corpus} Santos]}
\z

Para além de \isi{completivas finitas} introduzidas pelo \isi{complementador} \textit{que}, existem em português completivas introduzidas pelo \isi{complementador} \textit{se}: é este o caso das completivas interrogativas indiretas. Estas completivas são menos frequentes no discurso espontâneo e emergem em geral um pouco mais tarde na produção espontânea das crianças, como observou já \citet{soares2006}. Esta autora regista no seu \textit{corpus} uma completiva finita introduzida por \textit{se} apenas aos 3;2, neste caso dependente do verbo \textit{saber}. A observação do \textit{corpus} Santos permite confirmar a ocorrência de completivas de \textit{se} selecionadas por \textit{saber} no discurso de crianças entre os 3 e os 4 anos:\largerpage

\ea\label{ex:santoscompletivas_38}
\ea\label{ex:santoscompletivas_38a} É pa(ra) \# saber se está montado.\jambox{(Sandra, 3;2.11) \citep[370]{soares2006}}
\ex\label{ex:santoscompletivas_38b} não sei se ele deitou-se.\jambox{(TOM, 3;6.17) [\textit{corpus} Santos]}
\zl

No entanto, a observação do mesmo \textit{corpus} permite afirmar que o verbo com que o \isi{complementador} \textit{se} ocorre mais frequentemente e mais cedo é o verbo \textit{ver} (não percetivo), em estruturas como em (\ref{ex:santoscompletivas_39}):


\ea\label{ex:santoscompletivas_39}
\ea\label{ex:santoscompletivas_39a} ve(r) s(e) encont(ro) um gato!\jambox{(TOM, 2;7.13) [\textit{corpus} Santos]}
\ex\label{ex:santoscompletivas_39b} vê lá se consegues fechar \# assim.\jambox{(TOM, 3;2.29) [\textit{corpus} Santos]}
\ex\label{ex:santoscompletivas_39c} <vê lá> [/] \# vê lá s(e) é pa(ra) c(omer)!\jambox{(INI, 2;11.21) [\textit{coprus} Santos]}
\zl

Ocorre ainda neste \textit{corpus} uma frase em que o \isi{complementador} \textit{se} é (possivelmente) produzido numa completiva selecionada por \textit{perguntar} (neste caso, há um \textit{se} omitido, que não sabemos se é o \isi{complementador} ou o clítico):

\ea\label{ex:santoscompletivas_40}
eu já uma vez \# perguntei ao Martim \# se \# lembrava \# do Duarte. \jambox{(TOM, 3;2.29) [\textit{corpus} Santos]}
\z

É ainda relevante observar que também com completivas introduzidas por \textit{se} se observam, num estádio inicial do desenvolvimento típico, omissões de \isi{complementador}, como observou também já \citet{soares2006} (veja-se \ref{ex:santoscompletivas_41}):

\ea\label{ex:santoscompletivas_41} e(u) [//] \# <e(u) vo(u)> [/] e(u) vo(u) <ver e(n)co(n)t(r)o> [?].\\\jambox{(INI, 2:4.19) [\textit{corpus} Santos]}
\z

Finalmente, alguns dados apresentados em trabalhos sobre \isi{desenvolvimento atípico}, nomeadamente desenvolvimento linguístico em crianças com \textit{Specific Language Impairment} (SLI) (veja-se \citealt{kay1997}), sugerem que a omissão de \isi{complementador} (\textit{que} ou \textit{se}) pode acontecer até muito tarde no discurso destas crianças. Nas produções em (\ref{ex:santoscompletivas_42}), os complementadores\is{complementador} entre parênteses não foram produzidos.

\ea\label{ex:santoscompletivas_42}
\ea\label{ex:santoscompletivas_42a} BRU: e vê [//] e viu (que) o cão estava aqui preso (6;9)
\ex\label{ex:santoscompletivas_42b} LUI: e o cão estava a ver (se) a rã estava lá dentro (7;3)\\\jambox{\citep[79]{kay1997}}
\zl

\subsection{O contraste entre \isi{indicativo} e \isi{conjuntivo}}
\label{subsec:santoscompletivas_contraste}

A observação dos exemplos apresentados na secção anterior, que correspondem a primeiras \isi{completivas finitas} na produção espontânea de crianças que adquirem o português europeu, mostra que estas primeiras completivas exibem modo \isi{indicativo}. No entanto, a observação do mesmo \textit{corpus} de produção espontânea referido na secção anterior mostra que, embora raras (4 casos), mais tardias (ocorrendo perto dos 3 anos ou entre os 3 e os 4 anos) e restritas a complementos do verbo \textit{querer}, as \isi{completivas finitas} de \isi{conjuntivo} ocorrem no discurso espontâneo das crianças:

\ea\label{ex:santoscompletivas_43}
\ea\label{ex:santoscompletivas_43a} num [: não] quero \# que fiqu(e) a girafa \# feita \# mas quero \# oh@i.\\\jambox{(INI 3;10.1)}
\ex\label{ex:santoscompletivas_43b} queres que eu jogue?\jambox{(TOM, 2;11.12) [\textit{corpus} Santos]}
\zl

O trabalho recente de \citet{jesus2014}, centrado na escolha de modo \isi{indicativo} / \isi{conjuntivo} em completivas produzidas por crianças entre os 4 e os 9 anos, confirma que os complementos finitos de \textit{querer} são um dos contextos mais precocemente estabilizados como contexto de uso de \isi{conjuntivo}. Alguns trabalhos anteriores para o espanhol\il{espanhol} tinham já sugerido que as crianças começavam a produzir \isi{conjuntivo} por volta dos 3 anos \citep{giligaya1972}, mas não teriam estabilizado a sua distribuição até aos 10 anos \citep{blake1983}, sendo, no entanto, os contextos diretivos ou volitivos (como o complemento de \textit{querer}) os que são mais precocemente associados ao \isi{conjuntivo} \citep{giligaya1972,blake1983}. \citet{jesus2014} testou a escolha de modo \isi{conjuntivo} por crianças entre os 4 e os 9 anos, em completivas selecionadas por predicadores não epistémicos, i.e que não expressam crença (\textit{querer}, \textit{mandar}, \textit{deixar}, \textit{achar bem}) e epistémicos fracos (i.e. epistémicos que expressam crença, mas não positiva) (foram testados \textit{duvidar} e \textit{não acreditar}) (em \ref{ex:santoscompletivas_44} apresenta-se um exemplo de uma completiva de \isi{conjuntivo} selecionada por um verbo epistémico fraco).

\ea\label{ex:santoscompletivas_44} O cão duvidava que o gato apanhasse a bola.
\z

Os dados obtidos por \citet{jesus2014} mostram que os melhores resultados (i.e. percentagens mais elevadas de uso do \isi{conjuntivo}) se observam com \textit{querer} e \textit{mandar}, sendo neste caso a escolha de \isi{indicativo} residual, mesmo aos 4 anos. Mesmo em completivas selecionadas por \textit{deixar} e por \textit{achar bem} a preferência pelo \isi{conjuntivo} é precoce: aos 5 anos, a maioria das crianças testadas produz \isi{conjuntivo} nestes contextos. Isto significa que ainda em idade pré-escolar as crianças já adquiriram como contexto de \isi{conjuntivo} as completivas selecionadas por predicadores não epistémicos. Ao contrário, e como mostra \citet{jesus2014}, a estabilização da escolha de \isi{conjuntivo} em completivas selecionadas por epistémicos fracos, como \textit{duvidar} e \textit{não acreditar}, não está conseguida sequer aos 9 anos (quase final da idade escolar). De acordo com \citet{jesus2014}, as crianças primeiro associam o \isi{conjuntivo} a contextos não epistémicos (i.e. que não expressam crença) e só depois conseguem determinar quais os contextos epistémicos que se associam ao uso do \isi{conjuntivo} (na verdade, os que não expressam crença positiva).

\subsection{Interpretação de sujeitos nulos em \isi{completivas finitas}}
\label{subsec:santoscompletivas_interpretacao_nulos}

Como se viu na Secção \ref{sec:santoscompletivas_alguns_factos}, nas línguas românicas, nomeadamente em português, o contraste de modo (\isi{indicativo}/\isi{conjuntivo}) na completiva tem consequências relativamente à interpretação do seu sujeito. Como se observa em (\ref{ex:santoscompletivas_45a}), o \isi{sujeito nulo} de uma completiva finita de \isi{indicativo} pode ser correferente do sujeito da oração matriz (no caso [a rapariga]) – essa é, aliás, a interpretação preferencial do \isi{sujeito nulo} neste caso. Ao contrário, a maioria das completivas de \isi{conjuntivo} exibe o chamado efeito de \isi{obviação}, i.e. o sujeito da completiva não pode ser correferente do sujeito da matriz (\ref{ex:santoscompletivas_45b}).\footnote{
Há exceções ao efeito de \isi{obviação}, dependendo do verbo matriz. Por exemplo, com o verbo ``duvidar'', é possível ter correferência entre o sujeito da matriz e o sujeito da encaixada (veja-se \ref{ex:santoscompletivas_i4}). A natureza e a distribuição da \isi{obviação} é um facto ainda não completamente compreendido pelos linguistas.
\ea\label{ex:santoscompletivas_i4} Eu$_i$ duvido [que [-]$_{i/k}$ faça isso a tempo].
\z
}

\ea\label{ex:santoscompletivas_45}
\ea\label{ex:santoscompletivas_45a} A rapariga$_i$ disse [que [-]$_{i/k}$ vai ao cinema amanhã].
\ex\label{ex:santoscompletivas_45b} A rapariga$_i$ quer [que [-]$_{^{*}i/k}$ vá ao cinema amanhã].
\zl

Num trabalho muito recente, \citet{silva2015} apresenta vários testes que pretendem determinar a interpretação preferencial, por parte de crianças entre os 3 e os 6 anos, de sujeitos nulos e realizados (pronominais) em \isi{completivas finitas}. Aqui, destacamos os resultados relativos à interpretação de sujeitos nulos de completivas de \isi{indicativo} selecionadas por \textit{dizer} (veja-se o exemplo em \ref{ex:santoscompletivas_45a}) e de sujeitos nulos de completivas de \isi{conjuntivo} selecionadas por \textit{querer} (veja-se o exemplo em \ref{ex:santoscompletivas_45b}). Os resultados apresentados por \citet{silva2015} mostram que, no caso dos sujeitos nulos de completivas de \isi{indicativo} selecionadas por \textit{dizer}, as crianças aceitam correferência entre o sujeito da matriz e o sujeito da encaixada em níveis iguais ou próximos dos níveis de aceitação apresentados pelos adultos (i.e. em mais de 90\% dos casos). No entanto, no caso da interpretação de sujeitos nulos em completivas de \isi{conjuntivo} selecionadas por \textit{querer}, o comportamento das crianças difere do dos adultos: os adultos rejeitam todos os casos de correferência entre o sujeito da encaixada e o sujeito da matriz, mas as crianças aceitam-nos frequentemente (aos 3 anos, aceitam essa leitura em 62\% dos casos; aos 6 anos, aceitam a mesma leitura em 49\% dos casos). Estes dados confirmam alguns resultados obtidos no âmbito do trabalho apresentado por \citet{ambulate2008}, que encontrou também dificuldades na interpretação de sujeitos nulos de completivas de \isi{conjuntivo} (embora a questão da \isi{obviação} não fosse foco do trabalho). Por outro lado, a dificuldade na interpretação do sujeito da finita selecionada por \textit{querer} confirma resultados anteriores de \citet{padilla1990} para o espanhol,\il{espanhol} língua que apresenta efeitos semelhantes de \isi{obviação} (mas veja-se também a discussão em \citealt{avrutinwexler2000}).

Os dados apresentados pelos estudos sobre \isi{obviação} em português, centrando\x-se no comportamento de completivas de \isi{conjuntivo} selecionadas por verbos volitivos, nomeadamente \textit{querer}, permitem ainda uma observação interessante: embora no trabalho de \citet{jesus2014} se mostre que a seleção de \isi{conjuntivo} nas finitas complemento de \textit{querer} já está estabilizada pelos 4--5 anos (veja-se a Secção \ref{subsec:santoscompletivas_contraste}), os dados sobre \isi{obviação} mostram que, nas mesmas idades, as crianças não restringem de forma adulta a interpretação do \isi{sujeito} dessas completivas de \isi{conjuntivo}. A distribuição de \isi{conjuntivo} e a leitura obviativa são, por isso, adquiridas de forma relativamente independente.

\section{Em síntese}
\label{sec:santoscompletivas_sintese}

Neste capítulo, mostrámos que as completivas surgem, no discurso espontâneo, a par de outras estruturas que ativam a projeção do domínio CP (como interrogativas wh-, clivadas e relativas). Embora as primeiras completivas surjam no discurso espontâneo por volta dos dois anos ou entre os dois e os três anos, vimos que (i) nem todas as estruturas, quer no caso de \isi{completivas infinitivas} quer no caso de \isi{completivas finitas}, emergem ao mesmo tempo e que (ii) nem todas as estruturas são igualmente compreendidas por crianças em idade pré-escolar. No que diz respeito às \isi{completivas infinitivas}, o \isi{infinitivo flexionado} é produzido precocemente, mas não em todos os contextos possíveis; para além disso, algumas estruturas de controlo oferecem dificuldades mesmo a crianças em idade pré-escolar (e, presumivelmente, em idade escolar). No que diz respeito às \isi{completivas finitas}, a estabilização da distribuição do \isi{conjuntivo} prolonga-se até pelo menos aos 9 anos. Finalmente, saliente-se que as subordinadas completivas são ainda um domínio em que se podem observar diferenças entre o desenvolvimento típico e o atípico, podendo observar-se dificuldades na sua produção (por exemplo, omissão de \isi{complementador} nas finitas) de forma prolongada no \isi{desenvolvimento atípico}. 

{\sloppy
\printbibliography[heading=subbibliography,notkeyword=this]
}
\end{document}