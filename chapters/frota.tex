\documentclass[output=paper]{LSP/langsci} 
\author{Sónia Frota\affiliation{Universidade de Lisboa, Faculdade de Letras, Centro de Linguística}\lastand 
Cristina Name\affiliation{Universidade Federal de Juiz de Fora, CNPq}
}
\title{Questões de perceção em língua materna}
\abstract{}
\ChapterDOI{10.5281/zenodo.889419}
\maketitle
\begin{document}

\section{Perceção e aquisição da língua materna}
\label{sec:frota_percecao}

Durante muito tempo\is{tempo}, a pesquisa em aquisição da linguagem concentrou-se no estudo da compreensão e da produção de enunciados pela criança. Porém, quando a criança começa a responder ao adulto ou a outra criança, seja através de gestos ou da fala, demonstrando que entendeu o que foi dito, ela já sabe muito sobre a língua da sua comunidade. Por volta dos dois anos, já conhece várias palavras e suas posições na frase; já percebe relações de concordância, por exemplo, de género, entre artigos e nomes; já é capaz de combinar duas palavras produzindo frases simples. 

Mas como, em tão pouco tempo, a criança é capaz de conhecer tanto sobre a língua que está a ser adquirida? Os estudos sugerem uma sensibilidade precoce do bebé a uma série de propriedades da fala (ver Secção \ref{sec:frota_sensibilidade}). No final da sua gestação, o feto já perceciona o \isi{contorno melódico} -- uma parte da \isi{prosódia} --  da língua materna e reconhece a voz da mãe. Bebés recém-nascidos preferem ouvir sons linguísticos em detrimento de sons não linguísticos, e também preferem a fala normal, comparada com a fala apresentada de trás para frente; discriminam línguas não ouvidas anteriormente, com base no ritmo (por exemplo, inglês\il{inglês} versus japonês);\il{japonês} percebem características acústicas que podem sinalizar a fronteira entre uma palavra e outra; e são ainda capazes de distinguir itens funcionais\is{palavras funcionais} de itens lexicais,\is{palavras lexicais} também a partir de suas características acústicas \citep{gervainmehler2010}.

Com o contacto com a língua materna, a perceção do bebé vai se especializando naquilo que é específico da língua materna \citep{kuhl2004}. Inicialmente, o bebé é sensível a contrastes fonéticos mesmo quando não são produtivos na sua língua, sendo capaz de discriminar vogais\is{discriminação fonética} e consoantes de diferentes línguas, às quais nunca foi exposto. Em torno dos seis meses, há uma perda de sensibilidade às vogais que não fazem parte de sua língua, e o mesmo ocorre a partir dos dez meses em relação às consoantes. Por volta dos nove meses, o bebé torna-se sensível à forma típica das palavras, à fonotática (as sequências de fonemas permitidas) e às regularidades distribucionais de sequências de fonemas da língua em aquisição. Todo este desenvolvimento na perceção da fala ocorre ainda antes de o bebé produzir as primeiras palavras.

Portanto, desde os primeiros dias de vida, os bebés são sensíveis a uma série de propriedades sonoras da fala e são capazes de \textit{processar} os estímulos linguísticos de dada maneira, de modo que a aquisição da língua começa, de facto, muito cedo, possivelmente antes mesmo do seu nascimento, no caso de bebés ouvintes. Tal \textit{processamento} dos estímulos linguísticos decorreria de habilidades percetivas -- mecanismos ou \isi{primitivos percetivos} -- que permitiriam ao bebé analisar, tratar a fala, atentando (inconscientemente) para algumas das suas características. Por sua vez, essas características sinalizariam propriedades abstratas da língua -- de qualquer língua natural -- , que não se apresentam explicitamente nos enunciados, como a identificação das palavras, a ordem das palavras (Sujeito-Verbo-Objeto), a categoria e função das palavras, as relações morfossintáticas (concordância verbal, concordância nominal\ldots), ou propriedades discursivas e pragmáticas (tipos de frase, informação nova). Dessa maneira, a \isi{perceção} de tais características sonoras parece ser crucial para o desencadeamento da aquisição da língua.

Antes de falarmos mais sobre as habilidades percetivas do bebé, é importante destacar que, para estudarmos a \isi{perceção} no processo de aquisição de uma língua pela criança, são necessários métodos e técnicas experimentais específicas, que permitam que se verifique, indiretamente, o que ela já sabe. Afinal, se pedimos a crianças de dois anos para pegar, por exemplo, numa bola, dentre vários objetos, e elas pegam na bola, podemos afirmar que entenderam o que foi pedido. Mas não temos como perguntar a bebés de seis ou onze meses se já conhecem as vogais e consoantes da língua a que estão expostos. Para isso, precisamos de encontrar \textit{evidência} de que eles reconhecem determinados sons como pertencentes -- ou não -- ao inventário de vogais e consoantes da língua, colocando as nossas questões de investigação de forma indireta. Utilizamos, então, um método experimental, que usa uma situação \textit{controlada} -- através de uma experiência ou uma atividade experimental -- para se observar o comportamento dos participantes (no caso, bebés e crianças) em reação a estímulos linguísticos que lhes são apresentados. Esses estímulos vão apresentar variações do objeto de investigação, de forma a gerar dados para análise (as respostas dos bebés e crianças). 

Precisaremos, ainda, de escolher uma técnica experimental que seja adequada à idade do bebé ou da criança, de modo a captar as variações relevantes no seu comportamento. A ideia é que o bebé terá uma reação comportamental (no caso de se recorrer a métodos comportamentais) ou uma reação involuntária não comportamental (no caso de se utilizar métodos não comportamentais) se perceber diferenças ou reconhecer padrões entre os tipos de estímulos apresentados. Por exemplo, se queremos saber se os bebés, com poucos dias de vida, já reconhecem a \isi{prosódia} da língua a que estão expostos, iremos comparar o modo como reagem diante de estímulos com a \isi{prosódia} da língua materna e estímulos com a \isi{prosódia} de outra língua, que apresenta características diferentes da primeira \citep{nazzi_etal1998}.  Para isso, podemos usar, por exemplo, o método de \textit{Sucção Não Nutritiva},\is{métodos experimentais em perceção da fala!sucção não nutritiva} apresentado na secção seguinte.

\section{Métodos experimentais usados nos estudos de perceção da fala}
\label{sec:frota_metodos}

São vários os métodos usados na investigação em \isi{perceção} da fala. Alguns tomam como medida uma reação comportamental\is{medidas comportamentais} da criança (a sucção,\is{métodos experimentais em perceção da fala!sucção não nutritiva} o movimento da cabeça, o olhar),\is{métodos experimentais em perceção da fala!fixação visual e olhar preferencial} outros captam uma reação não comportamental involuntária (como mudanças no batimento cardíaco, na oxigenação sanguínea, ou na atividade cerebral). De uma forma ou de outra, as medidas obtidas são medidas indiretas que refletem aspectos do conhecimento e processamento linguísticos. Selecionamos quatro métodos, três de tipo comportamental e um não comportamental, por serem os mais comummente usados até o momento na investigação em \isi{perceção} infantil.

\paragraph*{Sucção Não Nutritiva (\textit{High-Amplitude Sucking} -- HAS)}
\is{métodos experimentais em perceção da fala!sucção não nutritiva}
Este método foi utilizado pela primeira vez em estudos de \isi{perceção} de fala nos anos 1970 \citep{eimas_etal1971} e pode ser usado com bebés recém-nascidos e até os quatro meses de vida. Toma como medida uma reação comportamental básica do bebé: a sucção.\is{métodos experimentais em perceção da fala!sucção não nutritiva} O bebé chupa uma chupeta ligada a um computador que regista a taxa de sucção.\is{métodos experimentais em perceção da fala!sucção não nutritiva} Antes da apresentação dos estímulos auditivos, é determinada a linha de base da taxa de sucção do bebé. Depois, o bebé ouve um dado estímulo e a taxa de sucção\is{métodos experimentais em perceção da fala!sucção não nutritiva} a ele associada é registada. Quando a frequência de sucção está estabilizada, tendo decrescido de acordo com um critério de habituação previamente estabelecido (por exemplo, uma diminuição de 20 a 30\%), apresenta-se ao bebé uma nova sequência de estímulos, que pode ser igual (na condição de controlo) ou diferente (na condição experimental) daquela que ele estava a ouvir. Se há um aumento significativo e constante na taxa de sucção\is{métodos experimentais em perceção da fala!sucção não nutritiva} dos bebés ao ouvirem estímulos diferentes, em comparação com a situação de controlo, temos evidência de que os bebés foram sensíveis à diferença entre os estímulos. A medida para se avaliar a \isi{perceção} do bebé é, portanto, a taxa de sucção.\is{métodos experimentais em perceção da fala!sucção não nutritiva} Os primeiros estudos sobre as capacidades de discriminação de contrastes segmentais\is{discriminação fonética} por bebés (por exemplo, \textipa{[pa]/[ba]}) foram realizados recorrendo a esta metodologia \citep{jusczyk1997,gerkenaslin2005}.

\paragraph*{Movimento Preferencial da Cabeça ou Escuta Preferencial (\textit{Head-Turn Preference Procedure} -- HPP)
}

O Movimento Preferencial da Cabeça (ou Escuta Preferencial)\is{métodos experimentais em perceção da fala!movimento preferencial da cabeça/ escuta preferencial} é um método adequado e especialmente produtivo para bebés e crianças entre quatro e dezoito meses. Este método explora a tendência natural que os bebés têm de se orientarem visualmente para uma fonte sonora, utilizando como medida o tempo total do movimento da cabeça (\textit{head-turn}) que sinaliza, assim, a atenção prestada aos estímulos auditivos \citep{kemlernelson_etal1995}. O bebé, sentado ao colo do cuidador, encontra-se no centro de uma cabine com três lados. À sua frente, no painel central da cabine, está uma luz verde e em cada um dos painéis laterais existe um altifalante próximo de uma luz vermelha. Os estímulos sonoros saem, aleatoriamente, de um dos dois altifalantes colocados à direita e à esquerda da criança. Esta técnica explora uma relação de contingência entre estímulo e comportamento (movimento da cabeça). Por outras palavras, a apresentação do estímulo é dependente do comportamento da criança: se ela se interessa pelo que ouve, mantendo a cabeça virada na direção da fonte sonora, o estímulo continua sendo emitido; se a criança desvia a cabeça por mais de dois segundos, o som é interrompido e um novo estímulo é apresentado. A criança controla a emissão dos estímulos, ouvindo mais aqueles da sua preferência. Para além de testar preferências (como, por exemplo, a preferência em relação a passagens de fala com pausas\is{pausa} coincidentes com constituintes sintáticos versus pausas\is{pausa} no interior de constituintes sintáticos), esta metodologia é também adequada para testar o reconhecimento e \isi{segmentação} de palavras, bem como a discriminação\is{discriminação fonética} (ver Secção \ref{sec:frota_sensibilidade}). Inicialmente, na fase de treino ou familiarização, a criança é exposta aos dois tipos de estímulos auditivos que irá ouvir (em paradigmas de preferência), ou a um tipo de estímulo (em paradigmas de discriminação).\is{discriminação fonética} Em seguida, na fase de teste, são apresentados dois tipos de estímulos, aleatoriamente distribuídos por cada um dos dois lados da cabine. Mede-se o tempo médio de orientação da cabeça (logo de escuta) para cada tipo de estímulo. Se há uma clara preferência por um tipo de estímulo (i.e., tempo maior de escuta), podemos defender que as crianças preferem as características desse tipo de estímulo (em paradigmas de preferência, ou de reconhecimento), ou que as crianças perceberam a diferença entre os estímulos (em paradigmas de discriminação).\is{discriminação fonética} A medida comportamental\is{medidas comportamentais} usada, portanto, é o tempo de orientação da cabeça/tempo de escuta, existindo variantes diferentes de implementação deste procedimento experimental \citep{jusczyk1997}. O Movimento Preferencial da Cabeça tem sido uma metodologia amplamente utilizada para estudar as capacidades iniciais de \isi{segmentação} lexical \citep{gerkenaslin2005}.

\paragraph*{Fixação Visual e Olhar Preferencial (\textit{Visual Fixation and Preferential Looking Procedures})}
\is{métodos experimentais em perceção da fala!fixação visual e olhar preferencial}
Estes métodos têm em comum a utilização do olhar e partilham algumas semelhanças com o Movimento Preferencial da Cabeça,\is{métodos experimentais em perceção da fala!movimento preferencial da cabeça/ escuta preferencial} diferindo na apresentação dos estímulos através de uma única fonte emissora colocada à frente da criança, que está associada à apresentação de estímulos visuais num écran. Dado que está demonstrado que os tempos de fixação visual\is{métodos experimentais em perceção da fala!fixação visual e olhar preferencial} dos bebés são afetados pela estimulação auditiva concomitante, este comportamento é explorado na investigação da perceção infantil. Os bebés apresentam um aumento sistemático do tempo de fixação visual\is{métodos experimentais em perceção da fala!fixação visual e olhar preferencial} quando há mudança no estímulo auditivo \citep{jusczyk1997}. Em algumas aplicações deste procedimento experimental, existe também uma relação de contingência entre estímulo auditivo e fixação visual:\is{métodos experimentais em perceção da fala!fixação visual e olhar preferencial} o bebé/a criança escuta o estímulo auditivo enquanto estiver a olhar para o écran.  A medida usada é o tempo de fixação do olhar\is{métodos experimentais em perceção da fala!fixação visual e olhar preferencial} nos estímulos visuais apresentados. Como estes são formados pela mesma imagem ou animação, independentemente do tipo de estímulo sonoro apresentado, podemos defender que o tempo de fixação do olhar\is{métodos experimentais em perceção da fala!fixação visual e olhar preferencial} é decorrente do maior ou menor interesse da criança pelo estímulo sonoro, sendo que a presença de discriminação é assinalada por maior tempo de fixação durante a audição do estímulo novo (não apresentado na fase anterior de familiarização ou habituação). Em variantes destes métodos, designadamente no caso do Olhar Preferencial (Preferential Looking),\is{métodos experimentais em perceção da fala!fixação visual e olhar preferencial} as imagens apresentadas podem variar consoante os estímulos auditivos, explorando-se a tendência para olhar espontaneamente para a imagem que se relaciona com a sequência que se está a ouvir (como, por exemplo, em estudos de aprendizagem inicial de palavras em que se testa a formação do elo associativo entre som e imagem -- ver Secção \ref{sec:frota_sensibilidade}). Estes métodos são particularmente adequados a crianças entre quatro e dezoito meses, sendo todavia utilizados até mais tarde, especialmente em estudos que implicam escolha visual durante a audição de estímulos de fala \citep{gervainmehler2010}. Foi através deste tipo de procedimento experimental que se determinou, por exemplo, que os bebés aos 6 meses já associam a palavra \textit{mamã} à imagem da mãe \citep{tincoffjusczyk1999}.

\paragraph*{Estudos Electrofisiológicos e Potenciais Evocados}
\is{métodos experimentais em perceção da fala!estudos electrofisiológicos}
\is{métodos experimentais em perceção da fala!potenciais evocados}
Ao contrário dos procedimentos experimentais descritos anteriormente, a eletroencefalografia e os potenciais evocados\is{métodos experimentais em perceção da fala!potenciais evocados} constituem um método não comportamental de estudo da perceção infantil, que permite medir o processamento da linguagem sem ser necessária a mediação de uma resposta comportamental (como a sucção,\is{métodos experimentais em perceção da fala!sucção não nutritiva} o movimento da cabeça ou o olhar).\is{métodos experimentais em perceção da fala!fixação visual e olhar preferencial} Ao ser medida a atividade elétrica cerebral em resposta direta a um estímulo sensorial particular, obtém-se uma curva média que corresponde a um potencial evocado.\is{métodos experimentais em perceção da fala!potenciais evocados} Por exemplo, de entre os potenciais evocados auditivos, o MMN (\textit{mismatch negativity}) constitui um correlato neuronal da presença de discriminação fonética.\is{discriminação fonética} Este método pode ser utilizado em bebés e crianças de qualquer idade, sendo particularmente ajustado ao estudo da perceção de fenómenos que se sucedem rapidamente no tempo, como a fala, mostrando-nos como os padrões de atividade neuronal mudam em tempo real durante o processamento de estímulos linguísticos \citep{kuhlrivera2008}.

\section{Primitivos percetivos}
\label{sec:frota_primitivos}
\is{primitivos percetivos}
Vimos que, desde os primeiros dias de vida, os bebés são sensíveis a várias propriedades acústicas da fala e que, ao longo do primeiro ano de vida, a partir da exposição à língua materna, sofrem uma especialização ou um \textit{estreitamento percetivos} (\textit{perceptual narrowing}). Tais habilidades iniciais podem ser entendidas como \isi{primitivos percetivos}, mecanismos gerais não específicos da linguagem e, em alguns casos, não específicos da espécie humana, que são usados na obtenção de informações depois utilizadas na aquisição da língua \citep{gervainmehler2010}. Um exemplo seria a sensibilidade às fronteiras de uma sequência de elementos, fronteiras essas que constituem posições salientes com impacto na perceção, na memória e na aprendizagem dos elementos que aí ocorrem. Este mecanismo geral percetivo facilitaria a computação -- e a aprendizagem -- de regularidades gramaticais que ocorrem nas margens de elementos ou estruturas linguísticas, como no caso da prefixação e sufixação que ocorrem nos limites da palavra. 

Para além da saliência das posições de fronteira, são também \isi{primitivos percetivos} a sensibilidade a princípios de agrupamento, tipicamente baseada em propriedades prosódicas como a melodia e o ritmo, e a sensibilidade a repetições e relações de identidade, que podem estar subjacentes à deteção de padrões envolvida na computação estatística (ver Secções \ref{sec:frota_sensibilidade} e \ref{sec:frota_descoberta}). Esses mecanismos auxiliariam o desencadeamento da aquisição do léxico e da estrutura morfossintática. 

Os \isi{primitivos percetivos} constituem, assim, capacidades percetivas gerais que seriam recrutadas para o processo de aquisição da língua, e não seriam necessariamente específicas nem da linguagem, nem dos humanos. Estudos experimentais realizados com outras espécies demonstraram, por exemplo, que ratos e macacos tamarindos são também capazes de discriminar línguas (com base em informação rítmica), que macacos tamarindos possuem capacidades de aprendizagem estatística e que chinchilas e macacos rhesus mostram habilidades de perceção fonémica categorial. 

\section{Sensibilidade à prosódia}
\label{sec:frota_sensibilidade}
\is{prosódia}
A perceção infantil é caracterizada por uma sensibilidade inicial a propriedades prosódicas da linguagem, o que sugere que os bebés estejam equipados com um mecanismo de processamento do sinal de fala (o input a que estão expostos) inicialmente sintonizado para informação prosódica \citep{morgandemuth1996,jusczyk1997}. Esta informação consiste nas pistas fonéticas, nomeadamente a frequência fundamental, a duração e a energia, que fazem o ritmo e a melodia das sequências de fala. A sensibilidade precoce à \isi{prosódia} poderá estar relacionada com \isi{primitivos percetivos} ou capacidades percetivas iniciais (ver Secção \ref{sec:frota_primitivos}), algumas das quais partilhadas com outras espécies, como por exemplo a sensibilidade a padrões rítmicos diferentes ou o agrupamento de sequências com base em informação melódica.

Vários estudos, utilizando o método da sucção não nutritiva,\is{métodos experimentais em perceção da fala!sucção não nutritiva} demonstraram que bebés recém-nascidos são sensíveis ao ritmo das línguas \citep{nazzi_etal1998,gervainmehler2010}, pois conseguem discriminar línguas\is{discriminação fonética} a que nunca foram expostos e que têm ritmos diferentes (como por exemplo o inglês\il{inglês} e o italiano),\il{italiano} mas não línguas com propriedades rítmicas semelhantes (como o inglês\il{inglês} e o holandês).\il{holandês} Com poucos dias de vida, os bebés são também sensíveis às melodias das palavras, distinguindo entre palavras com melodias ascendentes e descendentes. Esta sensibilidade inicial parece não depender da língua materna, mas rapidamente evolui para uma sensibilidade direcionada para os padrões prosódicos específicos da língua materna,\largerpage de acordo com o estreitamento percetivo que caracteriza a evolução da perceção no primeiro ano de vida (ver Secção \ref{sec:frota_percecao}).

Aos 4--5 meses de idade os bebés são já sensíveis aos padrões melódicos\is{contorno melódico}\is{padrão melódico|see {contorno melódico}} particulares da língua materna, como demonstrado em estudos recorrendo ao método da fixação visual,\is{métodos experimentais em perceção da fala!fixação visual e olhar preferencial} em línguas tão diferentes quanto o japonês,\il{japonês} uma língua que usa contornos melódicos\is{contorno melódico} para distinguir entre palavras com significados diferentes (por exemplo, \textit{hana} com uma melodia descendente significa `flor', enquanto \textit{hana} sem esta propriedade melódica lexical significa `nariz'), e o português, uma língua em que a melodia contribui para o significado ao nível da frase (por exemplo, contrastando frases declarativas e interrogativas), como é característico das línguas entoacionais \citep{frota_etal2014}. Estudos com o procedimento experimental do movimento preferencial da cabeça\is{métodos experimentais em perceção da fala!movimento preferencial da cabeça/ escuta preferencial} mostraram que aos seis meses de idade os bebés são sensíveis à presença e localização de fronteiras prosódicas\is{fronteira prosódica} assinaladas por pausas,\is{pausa} inflexões melódicas e alongamentos\is{alongamento} (por exemplo, distinguem entre passagens como \textit{\ldots os coelhos comem. Legumes com muitas folhas\ldots} e \textit{os coelhos comem legumes com muitas folhas}). Interessantemente, as pistas prosódicas cruciais para esta distinção variam consoante a língua materna, refletindo já aspetos da gramática adulta (no alemão\il{alemão} a pausa\is{pausa} é a pista determinante, enquanto no inglês\il{inglês} a entoação é crucial). Utilizando o mesmo paradigma experimental, foi também demonstrado que pelos 9 meses de idade os bebés são sensíveis a diferenças entre padrões acentuais presentes na língua materna (por exemplo, o contraste entre dissílabos em que o elemento tónico é a primeira sílaba e dissílabos em que a segunda sílaba é a tónica). Assim, os bebés aprendentes de inglês\il{inglês} e de castelhano\il{espanhol} mostram esta sensibilidade, contrariamente aos bebés aprendentes de francês.\il{francês} Todavia, a sensibilidade a padrões de proeminência (isto é, a relações entre elementos foneticamente salientes ou fortes e elementos fracos) parece estar presente ainda mais cedo no desenvolvimento. Estudos electrofisiológicos\is{métodos experimentais em perceção da fala!estudos electrofisiológicos} mostraram a influência da língua materna já aos 4--5 meses de idade, revelada pela perceção assimétrica dos padrões forte-fraco (trocaico)\is{padrão!trocaico|see {padrão!trocaico}} e fraco-forte (jâmbico):\is{padrão!jâmbico|see {padrão!iâmbico}} bebés aprendentes do alemão\il{alemão} favorecem o primeiro padrão (como no caso de \textit{búbu}, em que a primeira sílaba é a mais forte); já bebés aprendentes do francês\il{francês} favorecem o segundo (como em \textit{bubú}, em que a última sílaba é a mais forte), de acordo com os padrões de proeminência dominantes da língua \citep{gerkenaslin2005,seidlcristia2008,skoruppa2013development}.

Dado que as propriedades prosódicas tendem a estabelecer correlações frequentes com outras propriedades linguísticas, ao nível da sílaba, da palavra e da frase, esta sensibilidade inicial à \isi{prosódia} poderá ser usada na aquisição da língua materna, fornecendo pistas relevantes para a descoberta das palavras (a \isi{segmentação} lexical do input) e de aspectos da estrutura sintáctica\is{segmentação} \citep[a segmentação de unidades linguísticas maiores, de tipo sintagmático e frásico -- ver, entre outros,][]{morgandemuth1996,hohle2009}. Por exemplo, fronteiras entre frases ou orações, como a fronteira que segue \textit{chegou} em \textit{O Paulo chegou, mas a Ana saiu}, tipicamente correspondem a fronteiras prosódicas\is{fronteira prosódica} fortes (ou fronteiras de sintagmas entoacionais).  Sendo os bebés sensíveis às pistas que assinalam estas fronteiras prosódicas,\is{fronteira prosódica} estas pistas podem ser utilizadas na \isi{segmentação} de unidades estruturais de tipo sintático. A ordem das palavras na língua também se correlaciona com propriedades prosódicas, nomeadamente com o padrão de proeminência prosódica dentro do \isi{sintagma entoacional}. Assim, línguas com a ordem cabeça-complemento tendem a apresentar um padrão rítmico fraco-forte que se repete no \isi{sintagma entoacional} (como em italiano\il{italiano} ou português, em que a ordem é verbo+nome), enquanto línguas com a ordem complemento-cabeça tendem a apresentar um padrão forte-fraco (como em turco\il{turco} ou japonês,\il{japonês} em que a ordem é nome+verbo). Estas diferenças prosódicas podem constituir uma pista relevante para a aprendizagem da ordem de palavras, pois os bebés são desde cedo sensíveis a padrões de proeminência contrastantes. De forma semelhante, pistas prosódicas podem também assinalar unidades linguísticas menores, como as palavras, como veremos na secção seguinte.

\section{À descoberta das palavras}
\label{sec:frota_descoberta}

Para adquirirem o léxico da língua materna, as crianças necessitam de segmentar palavras ou potenciais candidatos a palavras a partir das sequências de fala a que estão expostas e que são, por natureza, contínuas, pois ao contrário da escrita não existem na fala fronteiras óbvias a separar as palavras umas das outras. A aquisição lexical começa por volta dos 6 meses de vida, com a \isi{segmentação} do input com base em pistas prosódicas e informação estatística \citep{gervainmehler2010}.

Entre as pistas prosódicas, encontra-se o ritmo global dos enunciados característico da língua materna. Numa língua como o inglês,\il{inglês} o padrão rítmico assenta numa unidade acentual\is{ritmo!acentual} básica formada por uma sílaba forte seguida de uma sílaba fraca (o chamado pé trocaico).\is{padrão!trocaico} Já no francês,\il{francês} a unidade básica é a sílaba. Assim, inglês\il{inglês} e francês\il{francês} apresentam tipos rítmicos diferentes, respetivamente conhecidos como ritmo acentual\is{ritmo!acentual} e ritmo silábico.\is{ritmo!silábico} Dada a capacidade dos bebés em diferenciar tipos rítmicos e identificar o ritmo da língua materna, este elemento prosódico poderá estar na base da emergência das capacidades de \isi{segmentação} lexical ao fornecer um candidato inicial para a estratégia de segmentação apropriada a cada língua \citep{hohle2009,mersad_etal2010}. De facto, bebés a adquirir o inglês (e o holandês ou o alemão, também línguas de ritmo acentual) começam por segmentar sequências formadas por uma sílaba forte seguida de uma sílaba fraca (palavras trocaicas) e falham a segmentação de sequências com o padrão inverso ou de monossílabos (por exemplo, a palavra \textit{candle} `vela' é facilmente segmentada; na sequência \textit{guitar is} `guitarra é', a sequência \textit{taris}, e não \textit{guitar}, é percecionada como um candidato a palavra porque tem o formato forte-fraco; um monossílabo como \textit{can} `lata' não é reconhecido). Pelo contrário, bebés a adquirir o francês\il{francês} começam por segmentar monossílabos e falham a \isi{segmentação} de dissílabos.

Para além do ritmo, a presença de uma fronteira prosódica\is{fronteira prosódica} constitui igualmente uma pista forte para o reconhecimento de palavras \citep{shukla_etal2011}. Os bebés distinguem entre sequências de sílabas separadas por uma fronteira prosódica\is{discriminação fonética} e sequências de sílabas agrupadas no mesmo constituinte prosódico. Aos 6 meses de idade, associam as sequências de sílabas antes da fronteira prosódica,\is{discriminação fonética} mas não as separadas por fronteira, a um referente visual, sugerindo que sequências alinhadas com fronteiras prosódicas\is{discriminação fonética} são bons candidatos a palavras. Bebés de 13 meses a adquirir o português do Brasil familiarizados com uma sequência de sílabas também a reconhecem quando seguida, mas não quando separada por uma fronteira prosódica\is{discriminação fonética} (por exemplo, a sequência \textit{bar-co} em \textit{[A sócia do nosso \textbf{BARCO}] [fechou contrato com turistas]} versus \textit{[A sócia do nosso \textbf{BAR}]  [\textbf{CO}chila durante o trabalho]}; \citealt{silvaname2014}).

Uma outra pista prosódica para a \isi{segmentação} lexical, mas que depende crucialmente das propriedades da língua materna, é o padrão acentual das palavras. A posição do acento na palavra, que apresenta regularidades fortes em várias línguas, é um marcador potencialmente útil para a \isi{segmentação} lexical. Por exemplo, no inglês,\il{inglês} apesar de a posição do acento na palavra ser variável, a grande maioria das palavras multissilábicas (cerca de 90\%) começa com sílaba tónica. Logo, segmentar o sinal de fala antes da sílaba tónica é uma boa estratégia para descobrir palavras nesta língua e aos 7.5 meses de idade os bebés a adquirir o inglês\il{inglês} mostram privilegiar este padrão de \isi{segmentação} \citep{jusczyk_etal1999}.

Para além de pistas prosódicas, a distribuição e frequência de ocorrência de sons e sílabas em sequências adjacentes constituem informação presente em qualquer língua, que pode ser estatisticamente relevante para determinar as suas unidades linguísticas (como morfemas e palavras). No português europeu, a probabilidade de o segmento \textipa{[S]} ocorrer em final de palavra é bastante alta ($0.644$), enquanto a possibilidade de \textipa{[S]} iniciar palavra ou ocorrer em posição interna é muito mais baixa (respectivamente, $0.006$ e $0.281$ -- cf. \citealt{vigario_etal2012}). No inglês,\il{inglês} considerando a sequência \textit{pretty baby} \textipa{[\textprimstress prItI\textprimstress beIbI]} `bebé lindo', a probabilidade de a sílaba \textipa{[tI]} seguir a sílaba \textipa{[\textprimstress prI]} é bastante mais alta do que a probabilidade de a sílaba \textipa{[\textprimstress beI]} seguir a sílaba \textipa{tI}, dada a existência da palavra \textit{pretty} mas não da palavra \textipa{[tI\textprimstress beI]}. Sabe-se que os bebés são sensíveis a informação deste tipo pelo menos desde os 6 meses de idade, usando as probabilidades sequenciais para detetar palavras. Vários estudos demonstraram que os bebés, no primeiro ano de vida, combinam o tratamento estatístico do input e pistas prosódicas nas suas estratégias de \isi{segmentação} lexical, dando maior peso a umas ou a outras consoante a idade e o tipo de pistas em estudo \citep{kuhl2004,gerkenaslin2005,shukla_etal2011}. Interessantemente, tal como a sensibilidade a algumas pistas prosódicas também a aprendizagem estatística não constitui uma capacidade exclusivamente humana (ver Secção \ref{sec:frota_primitivos}).

Pistas prosódicas e informação estatística conjugam-se igualmente para facilitar a identificação de grandes categorias de palavras, como a separação entre \isi{palavras funcionais} (como artigos e preposições) e \isi{palavras lexicais} (como nomes, verbos e adjetivos). As primeiras são habitualmente itens muito frequentes na língua, com forma monossilábica, sem acento e localizadas nas margens de unidades prosódicas; as segundas têm frequências de ocorrência muito inferiores, são tipicamente constituídas por várias sílabas e são acusticamente mais salientes. Pelo menos em algumas línguas, como é o caso do português, o inventário segmental utilizado nas \isi{palavras funcionais} é bem mais reduzido que o instanciado nas \isi{palavras lexicais} \citep{vigario_etal2012}. Regularidades deste tipo incluem-se no conjunto de propriedades a que os bebés demonstram ter uma sensibilidade precoce e podem, portanto, conduzir a uma classificação inicial rudimentar das duas grandes classes de palavras. Bebés recém-nascidos são capazes de discriminar entre \isi{palavras funcionais} e \isi{palavras lexicais}, mesmo que os estímulos não pertençam àquela que virá a ser a sua língua materna, aos 6 meses têm preferência por \isi{palavras lexicais} e a partir dos 7 meses mostram reconhecer a presença de \isi{palavras funcionais} em sequências \citep{gervainmehler2010}. Crianças adquirindo o português do Brasil reconhecem itens funcionais e fazem uso deles para identificar a palavra seguinte como um nome ou um verbo a partir dos 13 meses de idade \citep{namecorrea2003,name_etal2015}.

À \isi{segmentação} e ao reconhecimento das palavras, segue-se a aprendizagem inicial de palavras, ou seja, o desenvolvimento das primeiras associações entre forma e significado. Vários estudos têm mostrado que, se o contraste fonético for suficientemente saliente, os bebés são bem sucedidos na formação do elo associativo mais cedo no desenvolvimento, isto é, entre os 12 e os 17 meses \citep{gervainmehler2010}. Por exemplo, palavras potenciais que diferem entre si em vários sons são adquiridas mais cedo do que palavras que diferem numa única consoante. Um estudo para o português europeu mostrou ainda que candidatos a palavras que contrastam apenas nas suas propriedades prosódicas, como o acento e a entoação, são considerados inicialmente como potenciais palavras diferentes mesmo que tal não esteja de acordo com a fonologia da língua materna, pois diferenças melódicas (e.g., \textit{milo} pronunciado com entoação descendente ou ascendente) não estabelecem contrastes lexicais no português \citep{frota_etal2012}.  Estes resultados mostram que contrastes prosódicos podem ser tão salientes quanto múltiplas diferenças segmentais. Também a frequência de combinação dos segmentos sonoros na língua materna (a frequência do padrão fonotático) tem impacto na aprendizagem inicial de palavras, com sequências com maior probabilidade fonotática a serem adquiridas mais cedo do que sequências com menor probabilidade (por exemplo, \textit{bide} versus \textit{dibe} em francês\il{francês} -- ver \citealt{gonzalezgomez_etal2013}).

\section{Interação social e aquisição da linguagem}
\label{sec:frota_interacao}

O desenvolvimento inicial da linguagem depende de uma teia complexa de fatores, tais como a sensibilidade à \isi{prosódia} e a aprendizagem estatística, mas também a interação social. Em contextos naturais, a linguagem é adquirida em interação social e o papel deste fator no processamento da linguagem é relevante tanto no domínio da produção como no da perceção \citep{kuhl2004}. A fala dirigida a bebés e crianças apresenta características particulares que a distinguem da fala entre adultos, como um uso mais expandido da melodia e propriedades rítmicas mais salientes, e os bebés demonstram preferência em ouvir este tipo de discurso. O interesse acrescido pelos sinais de fala típicos da interação linguística com o bebé é já por si revelador da importância de fatores sociais no processo de aquisição.

O impacto da interação social na perceção e aquisição da linguagem foi demonstrado em estudos de \isi{discriminação fonética} e aprendizagem de palavras, que compararam situações de interação social naturalística com situações de mera exposição a input sem a intervenção direta humana (como através de estímulos auditivos gravados, ou através de estímulos áudio-visuais em televisão). Verificou-se que bebés de 9 meses de idade expostos a padrões de uma língua não materna os aprendiam com sucesso a partir da interação direta com tutores (interação social naturalista), mas não a partir da mera exposição áudio ou mesmo áudio-visual (com a imagem dos tutores em écran televisivo). A interação social parece, assim, ser essencial à aquisição da linguagem, que tende a privilegiar contextos naturais de socialização, à semelhança da aquisição de formas de comunicação em outras espécies.

\section{Preditores precoces do desenvolvimento da linguagem}
\label{sec:frota_preditores}

O processo de aquisição da língua materna, designadamente através do estreitamento percetivo que caracteriza o primeiro ano de vida, introduz mudanças no cérebro do bebé, cuja arquitetura e ligações vão progressivamente estando mais moldadas pelos padrões linguísticos nativos \citep{kuhl2004}. 

Este compromisso com a língua materna, que caracteriza o desenvolvimento típico, tem dois efeitos importantes: por um lado, facilita a aquisição de unidades e padrões mais complexos da língua, dependentes do conhecimento linguístico inicial; por outro lado, reduz as capacidades de atenção e aquisição de padrões alternativos que são diferentes dos da língua materna, como os que ocorrem em outras línguas. Neste contexto, a sensibilidade e \textit{performance} precoces do bebé em vários domínios da perceção da língua materna podem funcionar como preditores do desenvolvimento da linguagem em fases posteriores, mais avançadas do processo de aquisição.

O exemplo mais estudado de marcadores precoces no desenvolvimento da linguagem é a perceção de contrastes fonéticos presentes e ausentes da língua materna, no primeiro ano de vida. Bebés com boas capacidades percetivas dos contrastes fonéticos da língua materna mostraram ter um desenvolvimento linguístico posterior mais bem sucedido. Pelo contrário, bebés com boas capacidades percetivas dos contrastes fonéticos ausentes da língua materna apresentam um desenvolvimento linguístico subsequente inferior. Tanto medidas comportamentais\is{medidas comportamentais} como electrofisiológicas\is{medidas electrofisiológicas} demonstraram, por exemplo, que as capacidades de \isi{discriminação fonética} aos 6 meses predizem o vocabulário recetivo e expressivo aos 13, 16 e 24 meses, e que uma melhor discriminação\is{discriminação fonética} dos contrastes fonéticos da língua materna aos 7.5 meses se correlaciona positivamente com o vocabulário expressivo e a complexidade frásica aos 24 meses, o tamanho das frases produzidas aos 30 meses e o crescimento do léxico entre os 14 e os 30 meses \citep{kuhlrivera2008}.

O estudo de preditores precoces do desenvolvimento da linguagem tem sido alargado a outros domínios da perceção, como no caso da aprendizagem inicial de palavras, e à investigação comparativa entre desenvolvimento típico e desenvolvimento atípico, como no caso de bebés de risco para o desenvolvimento de perturbações da linguagem.

\section{Conclusão}
\label{sec:frota_conclusao}

Neste capítulo, apresentamos o essencial do estado da arte no domínio dos estudos de perceção e do seu papel na aquisição da língua materna. No primeiro ano de vida, o bebé atravessa etapas cruciais para o desenvolvimento da linguagem, em que \isi{primitivos percetivos}, isto é, mecanismos gerais não específicos para a linguagem, são inicialmente recrutados para possibilitar o processo de aquisição da língua. Partindo de uma sensibilidade especialmente orientada para estímulos linguísticos e em que as propriedades prosódicas assumem papel de destaque, ao longo do primeiro ano o bebé, por exposição à língua materna, especializa-se percetivamente para as propriedades específicas desta língua. Este \textit{estreitamento} percetivo constitui um passo determinante no processo de aquisição, tornando o bebé particularmente sensível aos padrões prosódicos específicos da língua materna e a outras pistas que lhe permitirão a \isi{segmentação} do input em unidades fonológicas e morfossintáticas, a descoberta de palavras e o reconhecimento de aspetos da estrutura sintática. A evolução percetiva do bebé -- de capacidades gerais para habilidades especializadas -- decorre, portanto, da sua inserção num ambiente linguístico, e a interação social tem um importante papel neste processo. É precisamente em contextos naturais de socialização, e à semelhança da aquisição de formas de comunicação em outras espécies, que a aquisição inicial da linguagem é promovida. O desempenho percetivo do bebé no decorrer destas fases iniciais pode predizer o desenvolvimento da linguagem em etapas posteriores, de modo que a observação de um padrão atípico de desenvolvimento poderá auxiliar na identificação precoce de perturbações da linguagem. 

Os estudos em perceção da fala constituem uma área de investigação muito produtiva, que emergiu na década de 1970 e tem beneficiado grandemente do aparecimento de métodos experimentais cada vez mais sensíveis à captação das capacidades percetivas dos bebés \citep{gerkenaslin2005,kuhlrivera2008}. As bases biológicas da aquisição da linguagem, a importância do input, o papel da interação social, e em particular da fala dirigida ao bebé, permanecem tópicos de pesquisa fundamentais para a compreensão do processo da aquisição da linguagem. A investigação multidisciplinar, particularmente nas áreas da ciência cognitiva e das neurociências, juntando contributos da genética, da comunicação animal e da neurolinguística, surge como um campo de investigação promissor que aprofundará o nosso entendimento sobre os correlatos comportamentais e neuronais das capacidades percetivas do bebé, e da forma como o seu cérebro é moldado pela exposição à língua materna.

\section*{Agradecimentos}
A investigação que conduziu a este capítulo foi parcialmente financiada pelos projetos PTDC\slash CLE-LIN\slash 108722\slash 2008, EXCL\slash MHC-LIN\slash 0688\slash 2012 e PTDC\slash MHCLIN\slash 3901\slash 2014 da Fundação para a Ciência e Tecnologia (Portugal), e 312833\slash 2013-0, 485171\slash 2012-0 e 307823\slash 2010-5 do Conselho Nacional de Desenvolvimento Científico e Tecnológico\slash CNPq (Brasil). As autoras agradecem a Marina Vigário e Susana Correia, por comentários e apoio à escrita do texto.


{\sloppy
\printbibliography[heading=subbibliography,notkeyword=this]
}
\end{document}