 \documentclass[output=paper]{LSP/langsci} 
\author{Ana Lúcia Santos\affiliation{Universidade de Lisboa, Faculdade de Letras, Centro de Linguística}\lastand 
Ruth E. V. Lopes\affiliation{Universidade Estadual de Campinas}
}
\title{Primeiros passos na aquisição da sintaxe: direcionalidade, movimento do verbo e flexão}  
\abstract{}
\rohead{\thechapter\hspace{0.5em}Primeiros passos na aquisição da sintaxe}
\ChapterDOI{10.5281/zenodo.889429}
\maketitle
\begin{document}
\section{Introdução}
\label{sec:santoslopes_intro}

Os primeiros enunciados com mais de uma palavra produzidos pelas crianças suscitaram sempre o interesse dos investigadores. Esses enunciados são frequentemente caracterizados como “telegráficos”, visto que são, em alguns deles, observáveis características como as seguintes: ausência de verbos copulativos (\ref{ex:santoslopes_1}), ausência de determinantes e verbos auxiliares (\ref{ex:santoslopes_2}), ausência de flexão\is{flexão} (\ref{ex:santoslopes_3}) (veja-se também o Capítulo 6, sobre a aquisição do sintagma nominal).

\ea\label{ex:santoslopes_1}
\gll Paula good girl. (por Paula is a good girl)\\
Paula boa menina\\\jambox{(18 meses)}
\glt `A Paula é uma boa menina.'\jambox{(inglês; \citealt{radford1988})}
\z

\ea\label{ex:santoslopes_2}
\ea\label{ex:santoslopes_2a}
\gll Ball go? (por Did the ball go?/ Where did the ball go?)\\
bola foi\\
\glt `(Onde é que) a bola foi?'\jambox{(inglês; \citealt{klimabellugi1966} \emph{apud} \citealt{radford1988})}
\ex\label{ex:santoslopes_2b}
\gll Car go? (por Did the car go?) \\
carro ir\\\jambox{(25--25 meses)}
\glt `O carro foi?'\jambox{(inglês; \citealt{hill1983} \emph{apud} \citealt{radford1995})}
\zl
\ea\label{ex:santoslopes_3}
\gll That go there. (por That goes there.) \\
essa vai aí\\\jambox{(23 meses)}
\glt `Essa vai para aí.'\jambox{(inglês; \citep{radford1988})}
\z

Estes dados incluem ainda vários casos de infinitivos em frases raiz,\is{infinitivos raiz} como (\ref{ex:santoslopes_4}), que correspondem a estruturas que não são possíveis na gramática adulta. Na verdade, os infinitivos raiz\is{infinitivos raiz} foram objeto de uma atenção especial, como veremos neste capítulo.

\ea\label{ex:santoslopes_4}
\ea\label{ex:santoslopes_4a}
\gll Dormir là Michel.\\
dormir lá Miguel\\\jambox{(francês;\il{francês} \citealt{deprezpierce1993})}
\ex\label{ex:santoslopes_4b}
\gll Thorsten Caesar haben.\\
{Thorsten (o boneco)} Caeser ter\\\jambox{(alemão; \citealt{poeppelwexler1993})}
\zl

A observação destes enunciados, diferentes do que são os enunciados produzidos pelos adultos, justificou um aceso debate entre investigadores no campo da aquisição das línguas maternas. Em geral, esse debate acabou por mostrar que, embora diferentes das produções adultas, estas produções respeitam as propriedades específicas das línguas-alvo, nomeadamente as que determinam a ordem de palavras nas línguas. Mostrou, contudo, que em alguns aspetos a gramática das crianças nos primeiros estádios poderá não ser totalmente convergente com a gramática-alvo.

\section{Direcionalidade}
\label{sec:santoslopes_direcionalidade}
\is{direcionalidade}
Desde o trabalho de \citet{greenberg1963}, que formulou generalizações descritivas sobre propriedades tomadas como universais nas línguas humanas, observou-se que a ordem de palavras nas línguas obedece a alguma regularidade: por exemplo, em línguas em que o verbo precede o objeto (designadas por Greenberg línguas VSO) existem preposições e em línguas SOV tendem a existir posposições (elementos comparáveis a preposições mas que seguem o seu complemento). Estas observações foram muito importantes, na medida em que sugerem que há padrões de variação nas línguas no que diz respeito à ordem relativa de núcleo e complementos.

De facto, trabalhos posteriores exploraram a ideia de que as línguas diferem quanto à ordem relativa de núcleo e complementos, o que se reflete na ordem relativa do verbo (o núcleo do sintagma verbal) e seus complementos. Isso justifica que em português, uma língua VO (i.e. com a ordem básica Verbo-Objeto), a ordem de palavras seja a que se observa em (\ref{ex:santoslopes_4wrong}), mas em japonês,\il{japonês} uma língua OV (Objeto-Verbo), a ordem de palavras seja a que se observa em (\ref{ex:santoslopes_5}).

\ea\label{ex:santoslopes_4wrong}
[comprei [o doce]]
\z
\ea\label{ex:santoslopes_5}
\gll [[okashi-o] kau]\\
doce.\textsc{acc} comprar\\
\z

Este tipo de variação entre as línguas foi concebido como uma diferença paramétrica, no âmbito da gramática generativa, sendo o parâmetro\is{parâmetro} relevante aquele que regula a direcionalidade\is{direcionalidade} na língua, i.e. a posição do núcleo, que pode ser final ou inicial (em inglês,\il{inglês} \textit{Head Parameter} – veja-se \citealt{travis1984}; em português, Parâmetro do Núcleo).\is{parâmetro!do núcleo} Em português, o Parâmetro do Núcleo\is{parâmetro!do núcleo} é fixado como núcleo-inicial (o verbo precede os complementos); em japonês,\il{japonês} o Parâmetro do Núcleo\is{parâmetro!do núcleo} é fixado como núcleo-final (o verbo segue os complementos).

Na verdade, a observação das primeiras combinações de palavras espontaneamente produzidas pelas crianças mostra que os erros de ordem de palavras, nomeadamente no que diz respeito a verbo e complementos, são residuais \citep{bloom1970,brown1973}. As crianças mostram, assim, uma convergência precoce com a gramática adulta, pelo menos no que diz respeito à ordem de palavras, o que aponta para uma fixação muito precoce do valor para este parâmetro.\is{parâmetro}

Para além da observação da produção espontânea das crianças, forçosamente restringida a idades em que as crianças já produzem combinações de duas ou mais palavras, foram desenvolvidos estudos experimentais que visavam avaliar a sensibilidade à ordem de palavras-alvo na língua por crianças em estádios mais precoces, que não produziam ainda enunciados de duas palavras. Um desses trabalhos, clássico na literatura, é o de \citet{hirshpasekgolinkoff1996}. Nesse estudo, mostrou-se que crianças de 17 meses compreendem frases ativas reversíveis,\footnote{Frases ativas reversíveis são aquelas que contêm um sujeito e um objeto animados, como em “\textit{João lavou Pedro}”. São assim chamadas porque esses elementos são intercambiáveis. No exemplo, pode-se ter \textit{Pedro} lavou \textit{João}, implicando uma troca possível de papéis.} como \textit{Big Bird is washing Cookie Monster} (O Garibaldo está lavando o Come-come (PB)/ O Poupas está a lavar o Monstro das Bolachas (PE)), tal como seria esperado de acordo com a gramática adulta. Usando o \textit{preferential looking paradigm} (Paradigma da Preferência de Olhar),\footnote{Nesse paradigma experimental, a criança vê dois quadros distintos projetados. Um coincide com a frase sendo testada e o outro não. Mede-se o tempo de fixação do olhar da criança para cada quadro. No caso da experiência relatada, um quadro conteria O Garibaldo lavando o Come-come/O Poupas a lavar o Monstro das Bolachas e o outro o Come-come/Monstro das Bolachas lavando o Garibaldo/ Poupas.} mostraram que as crianças, quando ouvem essa frase, preferem olhar para uma tela que mostra o Big Bird a lavar o Cookie Monster do que para uma tela que mostra o Cookie Monster a lavar o Big Bird. 

Estes resultados, contudo, não provam necessariamente que a criança trata o inglês\il{inglês} como língua VO e que a distingue de uma língua OV, visto que o que é testado, afinal, é a ordem relativa do sujeito e do objeto. Mais recentemente, \citet{franck_etal2013} testaram especificamente a sensibilidade ao contraste VO / OV por crianças de 19 meses que adquirem o francês,\il{francês} uma língua VO, como o português. Nesta experiência usaram pseudo-verbos (verbos que não existem na língua, mas que têm um formato fonológico possível na língua, como seria o caso de \textit{mipar} para o português, por exemplo), em frases que têm a ordem SVO
(\ref{ex:santoslopes_6a}) ou a ordem SOV (\ref{ex:santoslopes_6b}) (SOV é a ordem esperada numa língua como o japonês,\il{japonês} como vimos acima).

\ea\label{ex:santoslopes_6}
\ea\label{ex:santoslopes_6a}
\gll Le lion poune le cheval.\\
o leão pseudo-verbo o cavalo\\
\ex\label{ex:santoslopes_6b}
\gll La vache le lion dase.\\
a vaca o leão pseudo-verbo\\\jambox{\citep{franck_etal2013}}
\zl

\citet{franck_etal2013} mostraram que, quando ouvem a frase em (\ref{ex:santoslopes_6a}), as crianças preferem olhar para uma tela em que um animal pratica uma ação sobre outro animal do que olhar para uma tela em que cada animal pratica a ação de forma reflexiva, i.e. sobre si próprio. Ao contrário, quando as crianças ouvem uma frase com a ordem SOV, como (\ref{ex:santoslopes_6b}), agramatical na língua que adquirem, não mostram preferência por nenhuma das telas em particular. Muito recentemente, \citet{gavarro_etal2015} replicaram a experiência de \citet{franck_etal2013} com crianças também de 19 meses, expostas agora a uma língua OV, o Hindi-Urdu.\il{hindi-urdu} Os resultados confirmam os obtidos na experiência anterior: as crianças que adquirem uma língua OV mostram já nesta idade reconhecer a ordem de palavras esperada na língua. Estes resultados sugerem, pois, que, antes de produzirem as combinações de palavras relevantes, as crianças já determinaram algumas propriedades centrais da língua, nomeadamente no que diz respeito à ordem de palavras. Se assumirmos que a escolha de OV ou VO é determinada por um parâmetro,\is{parâmetro} o Parâmetro do Núcleo,\is{parâmetro!do núcleo} então esse parâmetro é fixado muito precocemente.
\section{Posição do verbo}
\label{sec:santoslopes_posicao_verbo}

Para além do que foi observado na secção anterior, mesmo olhando apenas para línguas em que o núcleo ocupa uma posição inicial, i.e. em que o verbo precede os complementos no sintagma verbal (SV), doravante VP (de \textit{verb phrase}, no inglês),\il{inglês} encontramos diferenças relativamente à ordem de palavras que é observada, sobretudo quando consideramos frases que contêm também advérbios\is{advérbio} (para uma discussão já clássica deste assunto, consulte-se \citealt{pollock1989}). Veja-se o contraste entre os enunciados em francês,\il{francês} em (\ref{ex:santoslopes_7}) e os enunciados em inglês,\il{inglês} em (\ref{ex:santoslopes_8}).

\ea\label{ex:santoslopes_7}
\ea[]{\label{ex:santoslopes_7a}
\gll Marie regarde souvent la télé.\\
Marie vê frequentemente a televisão\\
}
\ex[*]{\label{ex:santoslopes_7b}
Mary souvent regarde la télé.}
\zl

\ea\label{ex:santoslopes_8}
\ea[*]{\label{ex:santoslopes_8a}
\gll Mary watches often television.\\
Mary vê frequentemente televisão\\
}
\ex[]{\label{ex:santoslopes_8b}
Mary often watches television.}
\zl

Esta diferença entre línguas é uma diferença que pode ser concebida como resultado da fixação de um valor diferente para um parâmetro,\is{parâmetro} o parâmetro que determina se o verbo se move para o domínio-I (\textit{Inflection}, “flexão”),\is{flexão} domínio relacionado com traços de Tempo e de Concordância, e que designamos, como é habitual nos estudos em gramática generativa, IP\is{IP} (de \textit{inflection phrase}, em inglês).\il{inglês} Essa diferença de tipo paramétrico pode, aliás, ser vista como o reflexo de diferentes especificações de traços. Assim, e de forma simplificada, nos casos em que um advérbio\is{advérbio} ocorre entre o verbo principal e um complemento, consideramos que o verbo não se encontra dentro do sintagma verbal (VP), tendo subido para uma posição mais alta, o núcleo do domínio IP\is{IP} (veja-se a representação simplificada em \ref{ex:santoslopes_9}).\footnote{Assumimos aqui uma versão bastante simplificada da estrutura da frase.}

\ea\label{ex:santoslopes_9}
[\textsubscript{IP} Marie regarde [souvent[\textsubscript{VP} \sout{regarde} la télé]] 
\z

A agramaticalidade de (\ref{ex:santoslopes_8a}) indica que os verbos principais em inglês\il{inglês} não sobem para a mesma posição que é ocupada pelo verbo em francês.\il{francês} No entanto, dados como (\ref{ex:santoslopes_10}) mostram que o comportamento do verbo copulativo \textit{be} (\textit{ser}/\textit{estar}) e dos auxiliares, como \textit{do}, é distinto, ocupando estes uma posição mais alta na estrutura.

\ea\label{ex:santoslopes_10}
\ea\label{ex:santoslopes_10a}
\gll Mary is often sick.\\
Mary está frequentemente doente.\\
\ex\label{ex:santoslopes_10b}
\gll Mary does often watch television.\\
Mary \textsc{aux} frequentemente vê televisão\\
\glt `A Mary vê frequentemente televisão.'\\
\zl

Na verdade, nem todas as formas verbais na mesma língua dão origem a frases com ordem de palavras semelhante: é o que acontece quando observamos a ordem relativa do verbo e da negação\is{negação} em línguas como o francês, \il{francês}em (\ref{ex:santoslopes_11}).

\ea\label{ex:santoslopes_11}
\ea\label{ex:santoslopes_11a}
\gll Elle ne mange pas.\\
ela \textsc{neg} come \textsc{neg}\\
\glt `Ela não come.'\\
\ex\label{ex:santoslopes_11b}
\gll \ldots pour ne pas manger.\\
para \textsc{neg} \textsc{neg} comer\\
\glt `Para não comer.'
\zl

Se considerarmos, como é habitual fazer-se, que \textit{pas} é o elemento que assinala a posição da negação\is{negação} em francês\il{francês} (\textit{ne} é um clítico ao verbo e é frequentemente omitido na oralidade), percebemos que as formas finitas do verbo aparecem à esquerda da negação\is{negação} (\textit{pas}), enquanto o infinitivo aparece à direita da negação.\is{negação} Esta diferença na ordem linear tem sido interpretada do seguinte modo: a forma finita terá subido para uma posição no domínio IP,\is{IP} mas a forma verbal no infinitivo ocupará uma posição mais baixa do que aquela que é definida pela negação\is{negação} frásica, como se vê em (\ref{ex:santoslopes_12}).

\ea\label{ex:santoslopes_12}
\ea\label{ex:santoslopes_12a}
[\textsubscript{IP} Elle ne mange [pas[\textsubscript{VP} \sout{mange}]]]
\ex\label{ex:santoslopes_12b} pour [\textsubscript{IP} [ne pas manger]]
\zl

Com efeito, vários trabalhos exploraram a ordem de palavras dos primeiros enunciados com mais de uma palavra produzidos pelas crianças, procurando determinar se a ordem de palavras observada mostra sensibilidade às propriedades específicas da língua-alvo. Um desses trabalhos, \citet{pierce1992}, mostrou que, mesmo antes dos dois anos de idade, quando começam a produzir enunciados em que coocorrem um verbo e negação\is{negação} frásica, as crianças produzem formas finitas à esquerda da negação\is{negação} (\textit{pas}) e formas de infinitivo à direita da negação\is{negação} (para uma síntese destes dados, veja-se \citealt[109--111]{guasti2002}), como se observa em (\ref{ex:santoslopes_13}):

\ea\label{ex:santoslopes_13}
\ea\label{ex:santoslopes_13a}
\gll Pas manger la poupée\\
\textsc{neg} comer a boneca\\\jambox{(francês; Nathalie, 1;9)}
\glt `A boneca não come.'\\
\ex\label{ex:santoslopes_13b}
\gll Elle roule pas.\\
ela rola \textsc{neg}\\\jambox{(francês; Grégoire, 1;11; \citealt[110]{guasti2002})}
\glt `Ela não rola.'\\
\zl

Este tipo de dados tem servido para defender que já neste estádio bastante precoce de desenvolvimento linguístico as crianças não só projetam um IP\is{IP} como já determinaram se a língua que adquirem tem movimento do verbo\is{movimento!movimento do verbo} para I e em que casos isso acontece. Nesse sentido, estes dados permitem argumentar contra posições como a defendida por \citet{radford1988}, que vê as primeiras combinações de palavras produzidas pelas crianças como meras projeções de categorias lexicais, despojadas de domínios funcionais, como é o caso de I.

Passemos agora ao caso específico do português. O português é uma língua que exibe movimento do verbo\is{movimento!movimento do verbo} para o domínio que aqui designamos de IP.\is{IP}\footnote{Na verdade, \citet{cyrino_matos2002} sugerem que o verbo, no português brasileiro, ocupa uma categoria Aspetual. Não vamos mais explorar essas distinções aqui, assumindo que, mesmo que o verbo não chegue ao núcleo de I em português brasileiro, está, de alguma forma, num domínio acima do VP.} Esse facto é observável na frase (e respetiva representação) que apresentamos em (\ref{ex:santoslopes_14}):

\ea\label{ex:santoslopes_14}
\ea\label{ex:santoslopes_14a}
A criança come bem a sopa.
\ex\label{ex:santoslopes_14b}
[\textsubscript{IP} A criança come [bem [\textsubscript{VP} \sout{come} a sopa]]]
\zl

A evidência de que o verbo ocupará uma posição acima daquela em que é basicamente gerado no sintagma verbal, advém normalmente da observação da presença de um advérbio\is{advérbio} ou da negação\is{negação} frásica intervindo entre o verbo e um complemento. No entanto, em português, a negação\is{negação} frásica não é evidência relevante, visto que precede sistematicamente o verbo.

Assim, frases como (\ref{ex:santoslopes_14}) têm sido usadas pelos linguistas para mostrar que o verbo em português sobe para o IP.\is{IP} Se as encontrarmos entre as primeiras produções das crianças, teremos um argumento a favor da subida do verbo mesmo nas gramáticas que correspondem aos primeiros estádios de produção de unidades multi-palavra. \citet{goncalves2004,goncalves2006}, baseando-se num corpus de produção espontânea, explora esta possibilidade, encontrando de facto enunciados em idades razoavelmente precoces em que se observa a ordem de palavras V ADV O. Foi possível confirmar esse facto com base na observação de um \textit{corpus} independente, tratado em \citet{santos2006}, e com base em produções como as que se apresentam em (\ref{ex:santoslopes_15}).

\ea\label{ex:santoslopes_15}
\ea\label{ex:santoslopes_15a}
e(u) vo(u) pa [: pôr] tamãe [: também] no p(r)a(to) da Jul(i)eta.\jambox{(INI 2;5.24)}
\ex\label{ex:santoslopes_15b}
eu go(s)to muito do Paulo.\jambox{(TOM 2;6.6 [\textit{corpus} Santos])}
\zl

No entanto, como se pode observar, se compararmos os enunciados apresentados em (\ref{ex:santoslopes_15}) para o português com os que foram apresentados em (\ref{ex:santoslopes_13}) para o francês,\il{francês} verificamos que os enunciados portugueses são mais longos e produzidos em idades mais avançadas. Na verdade, os enunciados em (\ref{ex:santoslopes_15}) já não exemplificam as primeiras combinações de palavras produzidas pelas crianças, que são normalmente curtas, marcadas por limitações ao número de palavras por enunciado (MLUw, do inglês\il{inglês} \textit{Mean Length of Utterance in words},\is{mean length@\textit{mean length of utterance}} ou “tamanho médio do enunciado em palavras”). Interessar-nos-ia, pois, encontrar evidência para a subida do verbo para um domínio funcional acima do VP em enunciados mais curtos. Nesse sentido, a ordem relativa de verbo e advérbios\is{advérbio} não é a evidência ideal: por natureza, é necessário termos pelo menos um enunciado de três palavras para observarmos a ordem V ADV O, enquanto o contraste que envolve negação\is{negação} frásica (NEG V / V NEG) é em teoria observável em enunciados de duas palavras.

Nesta medida, a evidência mais precoce para a subida do verbo em português pode ser encontrada noutro tipo de estruturas. A subida do verbo na língua não tem só consequências para a posição do verbo face a advérbios\is{advérbio} ou à negação.\is{negação} Na verdade, a existência na língua de elipse do VP\is{elipse do VP} tem sido sistematicamente relacionada com a existência de subida do verbo para um domínio acima do VP. A elipse do VP\is{elipse do VP} é observável em orações coordenadas, mas também em pares pergunta-resposta (veja-se \ref{ex:santoslopes_16} e \ref{ex:santoslopes_17}). A resposta em (\ref{ex:santoslopes_16a}) ou em (\ref{ex:santoslopes_17a}) corresponde a uma estrutura em que apenas o verbo, que se encontra numa posição acima do VP, é pronunciado, sendo apagado todo o material numa posição mais baixa (veja-se a representação simplificada em \ref{ex:santoslopes_16b} e \ref{ex:santoslopes_17b}).

\ea\label{ex:santoslopes_16}
\ea\label{ex:santoslopes_16a}
P: Tu foste ao cinema com a Maria?\\
R: Fui. (= Fui ao cinema com a Maria)
\ex\label{ex:santoslopes_16b}
[\textsubscript{IP} fui [\textsubscript{VP} \sout{fui ao cinema com a Maria}]]
\zl
\ea\label{ex:santoslopes_17}
\ea\label{ex:santoslopes_17a}
P: Você entregou o livro à Sónia?\\
R: Entreguei. (= Entreguei o livro à Sónia)
\ex\label{ex:santoslopes_17b}
[\textsubscript{IP} entreguei [\textsubscript{VP} \sout{entreguei o livro à Sónia}]]
\zl

Embora nem todas as línguas que apresentam movimento do verbo\is{movimento!movimento do verbo} para I permitam elipse do VP\is{elipse do VP}\is{elipse do VP} (o francês,\il{francês} por exemplo, não tem elipse do VP,\is{elipse do VP} embora apresente movimento do verbo\is{movimento!movimento do verbo} finito para I), este tipo de estrutura só é possível nas línguas e nas estruturas em que o verbo se encontra em I. Isso é visível em inglês,\il{inglês} língua em que há verbos que ocupam uma posição no domínio IP\is{IP} e legitimam elipse de VP (auxiliares ou o verbo copulativo) e verbos que permanecem numa posição baixa e não legitimam elipse do VP (verbos principais) (veja-se o contraste entre \ref{ex:santoslopes_18a} e \ref{ex:santoslopes_18b}).

\ea\label{ex:santoslopes_18}
\gll P: Do you like soup?\\
~ \textsc{aux} tu gostar sopa\\
\glt ~~ `Gostas de sopa?'\\
\ea[]{\label{ex:santoslopes_18a}
\gll R:  I do.\\
~ eu \textsc{aux}\\
\glt ~~ `Gosto.'}
\ex[]{\label{ex:santoslopes_18b}
\gll R: *I like.\\
~~ eu gosto\\
\glt ~~ `Gosto.'}
\zl

Apoiada neste tipo de raciocínio e na análise de um \textit{corpus} com 18492 enunciados produzidos por três crianças entre 1;5 e 3;11, \citet{santos2006} mostra que as crianças que adquirem o português como língua materna produzem precocemente o tipo de respostas afirmativas em (\ref{ex:santoslopes_16}) e (\ref{ex:santoslopes_17}), analisáveis como casos de elipse de VP, e argumenta que este é, na verdade, o tipo de evidência mais precoce que podemos encontrar para a subida do verbo para I na aquisição do português como L1. O mesmo pode ser afirmado para o português brasileiro, segundo \citet{lopes2009}, que examinou dados de produção espontânea em três crianças brasileiras entre 1;8 e 3;7. Alguns exemplos destas produções são apresentados em (\ref{ex:santoslopes_19a} -- \ref{ex:santoslopes_19d}, \citealt{santos2006}; \ref{ex:santoslopes_19e}, \citealt{lopes2009}):

\ea\label{ex:santoslopes_19}
\ea\label{ex:santoslopes_19a}
MAE: o cavalo vai papar?\\
TOM: vai.\jambox{(TOM 1;9.14)}
\ex\label{ex:santoslopes_19b}
ALS: <está a tirar> [//] estás a tirar os olhos \# da rã?\\
INM: +< (es)tá.\jambox{(INM 1;7.6)}
\ex\label{ex:santoslopes_19c}
MAE: é o quê?\\
TOM: ah@i.\\
MAE: olha \# são legos?\\
TOM: são.\jambox{(TOM 1;8.16)}
\ex\label{ex:santoslopes_19d}
MAE: fez ai+ai ao Tomás?\\
TOM: fez.\jambox{(TOM 2;2.9)}
\ex\label{ex:santoslopes_19e}
ADULTO: Tomou remédio também?\\
AC: Tomou.\jambox{(AC 2;1)}
\zl

No entanto, serão estes dados realmente casos de elipse do VP\is{elipse do VP} em idades tão precoces? Poder-se-á pensar que se trata de casos em que a criança se limita a repetir uma palavra que encontra no discurso imediatamente anterior. \citet{santos2006} argumenta extensamente contra a ideia de que se trate de mera repetição. Neste capítulo, destacamos dois dos argumentos apresentados. Em primeiro lugar, uma análise de todo o \textit{corpus} mostra que as crianças usam o verbo na resposta e não outra palavra, sendo que uma mera repetição devia ser insensível à categoria da palavra repetida, i.e. não deveria ser guiada sintaticamente. Em segundo lugar, se este tipo de enunciados corresponder a mera repetição, esperar-se-á que sejam observados na aquisição de qualquer língua, independentemente de a língua permitir ou não elipse do VP.\is{elipse do VP} Santos (2006) comparou os dados do português com dados do \textit{corpus} York, do francês\il{francês} \citep{decatplunkett2002,plunkett2002} e com dados do corpus Brown \citep{brown1973}, para o inglês,\il{inglês} ambos disponíveis na base de dados CHILDES \citep{macwhinney2000}. Tal como se espera, se os enunciados em (\ref{ex:santoslopes_19}) forem efetivamente casos de elipse do VP,\is{elipse do VP} não se observam enunciados semelhantes na aquisição do francês, língua que não permite elipse do VP.\is{elipse do VP} Ao contrário, observam-se casos inequívocos de elipse do VP\is{elipse do VP} em inglês,\il{inglês} também em contexto de par pergunta-resposta, em estádios em que a criança já produz o auxiliar \textit{do}, que legitima este tipo de elipse (veja-se o exemplo em \ref{ex:santoslopes_20}).

\ea\label{ex:santoslopes_20}
\gll MOT: would you please ask him?\\
~ \textsc{aux} tu {por favor} perguntar lhe\\
\glt ~~~~~~~~~ `Podes perguntar-lhe, por favor?'\\
\gll CHI: I \textbf{did}.\\
~ eu \textsc{aux}\\
\glt ~~~~~~~~ `Eu perguntei.'\\
\gll MOT: no you didn't.\\
~ não tu \textsc{aux.neg}\\
\glt ~~~~~~~~~ `Não, não perguntaste.'\\
\gll MOT: he didn't hear you.\\
~ ele \textsc{aux.neg} ouviu te\\
\glt ~~~~~~~~~ `Ele não te ouviu.'\\
\gll COL: were you going {to [?] ask} me something?\\
~ \textsc{aux} tu ir perguntar me {alguma coisa}\\
\glt ~~~~~~~~ `Ias perguntar-me alguma coisa?'\\
\gll MOT: go ahead.\\
~ ir {em frente}\\
\glt ~~~~~~~~~ `Vá.'\\
\gll CHI: I \textbf{did already}.\\
~ eu \textsc{aux} já\\\jambox{(Eve 2;2 - eve18.cha)}
\glt ~~~~~~~~ `Eu já perguntei.'
\z

\citet{lopes2009} discute ainda um outro tipo de evidência de que a criança tem movimento de verbo\is{movimento!movimento do verbo} precocemente. A evidência, independente, viria através do uso de advérbios\is{advérbio} aspectuais. Se estes advérbios\is{advérbio} estão no domínio IP\is{IP} e o verbo deve ser adjacente a eles, então mostram que o verbo saiu do domínio verbal, como vemos em (\ref{ex:santoslopes_21}):

\ea\label{ex:santoslopes_21}
\ea\label{ex:santoslopes_21a}
Aqui já comeu (= Aqui [o boneco] já comeu)\jambox{(AC 2;3)}
\ex\label{ex:santoslopes_21b}
Já tem out(r)o bicho.\jambox{(AC 2;3)}
\zl

\section{Posição do verbo em línguas V2}
\label{sec:santoslopes_posicao_verbo_v2}

A discussão sobre ordem de palavras e presença de domínios funcionais nos primeiros enunciados produzidos pelas crianças teve ainda desenvolvimentos interessantes a partir do estudo de línguas como o alemão\il{alemão} ou o holandês.\il{holandês} Estas línguas são chamadas línguas V2\is{V2} porque, nelas, nas frases raiz, o verbo finito ocupa forçosamente a segunda posição, sendo que a primeira posição é ocupada quer pelo sujeito quer por outro constituinte, como o complemento (objeto) direto ou um advérbio.\is{advérbio} Este tipo de ordem de palavras é esquematizado em (\ref{ex:santoslopes_22}).

\ea\label{ex:santoslopes_22}
XP V YP
\z
Na verdade, a ordem V2\is{V2} (verbo em segunda posição obrigatória, rígida) tem sido analisada como resultando do movimento obrigatório do verbo,\is{movimento!movimento do verbo} que tinha já subido para I, para um outro domínio mais alto, o domínio do complementador\is{complementador} (CP,\is{CP} de \textit{complementizer phrase}, em inglês),\il{inglês} mais propriamente para a posição de núcleo desse domínio. Esse movimento seria acompanhado do movimento de um outro constituinte (sujeito, objeto, modificador) para uma posição no domínio CP que linearmente precede o verbo. Veja-se a representação esquematizada em (\ref{ex:santoslopes_23}).

\ea\label{ex:santoslopes_23}
\gll {[\textsubscript{CP}} jetzt {gehe [\textsubscript{IP} \sout{jetzt}} ich nach Hause \sout{gehe}]]\\
~ agora vou eu para casa\\
\z

O domínio CP\is{CP} é um domínio periférico cujo núcleo acolhe o complementador\is{complementador} em orações subordinadas (veja-se o Capítulo 11). É ainda um domínio que acolhe relativos e interrogativos e que não se associa a uma função sintática específica (veja-se o Capítulo 10), o que explica que, no caso da ordem V2,\is{V2} o verbo apareça rigidamente em segunda posição mas possa ser precedido quer por um sujeito, quer por outro tipo de constituinte. A ordem V2\is{V2} não se confunde assim com a ordem SVO (sujeito – verbo - objeto) do português, que em frases simples se explica com a subida do verbo para o núcleo do IP\is{IP} e a subida do sujeito para uma posição que linearmente precede o verbo.

A ordem de palavras na gramática adulta do alemão\il{alemão} permite assim fazer predições específicas, que trabalhos como o de \citet{poeppelwexler1993} procuraram avaliar. A ser verdade que a gramática das crianças é desprovida de categorias funcionais (I, C) até cerca dos 2 anos ou 2;6 (como sugerem Radford e outros autores no final dos anos 80 e início dos anos 90 do século XX, mas veja-se discussão em \citealt{poeppelwexler1993}), não se esperará que as primeiras combinações de palavras em alemão\il{alemão} revelem efeitos de V2,\is{V2} já que V2\is{V2} implica a presença de CP.\is{CP} No entanto, tendo observado a produção espontânea de frases declarativas nos dados de uma criança de 2;1 falante de alemão,\il{alemão} \citet[7]{poeppelwexler1993} mostram que, de 208 frases com formas finitas do verbo (as formas verbais que se espera que sofram movimento para C), 197 são efetivamente identificáveis como casos em que o verbo se encontra em segunda posição (sendo esta segunda posição uma posição não final, i.e. trata--se de enunciados com mais de duas palavras). Apresentamos exemplos de produções precoces de frases com ordem V2\is{V2} em (\ref{ex:santoslopes_24a}) e (\ref{ex:santoslopes_24b}) – estes são casos em que o que ocorre em primeira posição não é um sujeito, o que é característico deste tipo de ordem de palavras em línguas V2.\is{V2}

\ea\label{ex:santoslopes_24}
\ea\label{ex:santoslopes_24a}
\gll Da bin ich.\\
aqui estou eu\\
\ex\label{ex:santoslopes_24b}
\gll Eine Fase hab ich.\\
uma jarra tenho eu\\\jambox{(alemão; \citealt[14]{poeppelwexler1993})}
\zl

Estes dados foram interpretados por \citeauthor{poeppelwexler1993} como sugerindo convergência precoce entre a gramática das crianças neste estádio e a gramática adulta. A hipótese de que as crianças têm, desde estes estádios precoces, uma gramática que é convergente com a gramática adulta, incluindo o elenco de categorias funcionais na gramática-alvo, é a Hipótese de Competência Plena (\textit{Full Competence Hypothesis}).

No entanto, é preciso dizer que esta interpretação de dados como (\ref{ex:santoslopes_24a}) ou (\ref{ex:santoslopes_24b}) não foi unanimemente aceite pelos investigadores. Por exemplo, \citet{meiselmuller1992} tratam os enunciados com verbo em segunda posição nas produções precoces do alemão\il{alemão} como casos de aparente V2,\is{V2} sugerindo que as crianças obtêm essa ordem de palavras movendo o verbo e outro constituinte para o domínio IP\is{IP}\footnote{\is{IP}Estamos aqui a simplificar significativamente a posição de \citet{meiselmuller1992}.} e não para o domínio CP,\is{CP} porque ainda não projetariam CP.\is{CP} Fundamentam a sua posição no facto de crianças que produzem a aparente ordem V2\is{V2} produzirem também enunciados que podem ser tomados como casos de subordinação, mas com omissão de complementador\is{complementador} (como é o caso de \ref{ex:santoslopes_25}).

\ea\label{ex:santoslopes_25}
\gll pa'auf teddy tombe pas\\
cuida urso cai \textsc{neg}\\\jambox{Ivar (bilingue alemão\il{alemão}/inglês)\il{inglês} 2;4.23}
\glt `Cuide para que o ursinho não caia'\jambox{\citep[120]{meiselmuller1992}}
\z

\citet{poeppelwexler1993} discutem esta posição de \citeauthor{meiselmuller1992}, dizendo que a omissão do complementador\is{complementador} (nomeadamente, \textit{dass} ‘que’) não significa forçosamente que CP\is{CP} não seja projetado. Por outro lado, se as crianças puderem derivar a ordem de palavras na gramática adulta sem terem adquirido a gramática adulta (i.e. por exemplo, interpretando a posição à esquerda do verbo no IP\is{IP} como uma posição que não é forçosamente reservada para sujeitos), é difícil explicar de que forma poderão adquirir a gramática-alvo. 


\section{O caso particular dos infinitivos raiz}
\label{sec:infinitivos_raiz}
\is{infinitivos raiz}
Nas secções anteriores, mostrámos que as primeiras combinações de palavras produzidas pelas crianças, frequentemente descritas como “telegráficas”, mostram, na verdade, um grau elevado de convergência com a gramática da língua a que a criança está exposta, nomeadamente no que diz respeito à ordem de palavras. No entanto, nem tudo nestes dados revela já uma gramática adulta: as crianças produzem estruturas impossíveis na gramática-alvo, por exemplo, no que diz respeito à ausência inicial de determinantes ou de verbos auxiliares ou flexão\is{flexão} numa língua como o inglês\il{inglês} (vejam-se os dados em \ref{ex:santoslopes_1} a \ref{ex:santoslopes_3}) ou ainda no que diz respeito à produção de infinitivos em frases raiz em certas línguas (vejam-se os exemplos em \ref{ex:santoslopes_4}, reproduzidos de seguida e acrescidos de um novo caso).

\ea\label{ex:santoslopes_26}
\ea\label{ex:santoslopes_26a}
\gll Dormir là Michel.\\
dormir lá Miguel\\\jambox{(francês; \citealt{deprezpierce1993})}
\ex\label{ex:santoslopes_26b}
\gll Thorsten Caesar haben.\\
{Thorsten (o boneco)} Caeser ter\\\jambox{(alemão;\il{alemão} \citealt{poeppelwexler1993})}
\ex\label{ex:santoslopes_26c}
\gll Tourner dans l´autre sens.\\
virar no outro sentido\\\jambox{(francês; \citealt{guasti2002})}
\zl

Na verdade, dados como os apresentados em (\ref{ex:santoslopes_26}) despertaram a curiosidade dos investigadores e têm sido objeto de intensa investigação (destaquem-se os trabalhos de \citealt{wexler1994,wexler1998,rizzi1993,hoekstrahyams1998}). Essa investigação acabou por reunir uma série de factos relevantes sobre este tipo de estrutura, entre os quais se destacam os que referimos de seguida.

\begin{enumerate}[label=(\roman*)]
\item Os infinitivos raiz\is{infinitivos raiz} não são universais: tendo em conta as línguas até agora estudadas, é possível identificar um estádio de produção de infinitivos raiz\is{infinitivos raiz} em línguas que não permitem sujeitos nulos, como o francês,\il{francês} o alemão,\il{alemão} o holandês\il{holandês} ou o sueco,\il{sueco} mas não em línguas de sujeito nulo,\is{sujeito nulo} como o português, o espanhol\il{espanhol} ou o italiano\il{italiano} (para o português, veja-se \citealt{lopes2003,goncalves2004,santosduarte2011}).
\item Os infinitivos raiz\is{infinitivos raiz} não resultam de uma incapacidade para produzir flexão\is{flexão} de pessoa e número ou realizar concordância sujeito-verbo: nos mesmos estádios em que produzem infinitivos raiz,\is{infinitivos raiz} as crianças produzem formas finitas com flexão\is{flexão} relevante e exibindo concordância sujeito-verbo.
\end{enumerate}
Para além dos factos acima enunciados, destaca-se outra generalização, que correlaciona o fenómeno dos infinitivos raiz\is{infinitivos raiz} com outro fenómeno observado nas mesmas idades e nas mesmas línguas (até cerca dos 3 anos): a produção espontânea de sujeitos nulos em línguas que não o permitem. Observa-se, então, que:

\begin{enumerate}[label=(\roman*),resume]
\item os infinitivos raiz ocorrem muito frequentemente com sujeitos nulos (o caso em \ref{ex:santoslopes_26c}).
\end{enumerate}

Diversas explicações foram avançadas na literatura para o fenómeno dos infinitivos raiz,\is{infinitivos raiz} algumas delas argumentando que as crianças poderiam opcionalmente não projetar domínios funcionais que seriam obrigatórios na gramática adulta (IP,\is{IP} CP)\is{CP} (veja-se \citealt{rizzi1993}) ou que poderiam deixar subespecificados traços, nomeadamente traços de Tempo (veja-se \citealt{wexler1994}). Estas explicações são complexas e não poderão ser aqui expostas pormenorizadamente (veja-se \citealt[128 e ss.]{guasti2002} para uma síntese).

Importa, contudo, sublinhar que em português europeu, no desenvolvimento típico, não se observam em geral estruturas de infinitivo raiz\is{infinitivos raiz} – como é esperado, já que o português europeu é uma língua de sujeito nulo.\is{sujeito nulo} O mesmo se pode dizer do português brasileiro, que tem sido caracterizado como uma língua de sujeito nulo\is{sujeito nulo} parcial. Todavia, não é impossível encontrar estruturas semelhantes a infinitivos raiz\is{infinitivos raiz} na produção de crianças diagnosticadas com \textit{Specific Language Impairment} (SLI) \textit{(Perturbação Específica da Linguagem}\is{Perturbação Específica da Linguagem} – PEL, em português europeu ou \textit{Déficit Específico de Linguagem} – DEL, em português brasileiro), neste caso com idades muito superiores às normalmente associadas aos infinitivos raiz\is{infinitivos raiz} no desenvolvimento típico (sobre este tipo de patologias, veja-se o Capítulo 16). Vejam-se os dados seguintes, extraídos de \citet{kay1997}. Está por determinar quão frequente é este tipo de produção no desenvolvimento atípico e até que ponto exibe as mesmas características que os infinitivos raiz\is{infinitivos raiz} no desenvolvimento típico.

\ea\label{ex:santoslopes_27}
\ea\label{ex:santoslopes_27a}
depois ficar na água\jambox{(LUI 7;3)}
\ex\label{ex:santoslopes_27b}
o menino ver o cão saltar.\jambox{(SAM 9;9)}
\zl

\section{Sujeitos nulos nos primeiros estádios de aquisição}
\label{sec:santoslopes_sujeitos_nulos}

Como se observou na secção anterior, as primeiras frases produzidas pelas crianças exibem ainda uma outra propriedade: presença de sujeitos nulos. Sabemos que nem todas as línguas permitem que um sujeito não seja lexicalmente realizado, sendo possível dizer, de forma simplificada, que as línguas se dividem em (i) línguas de sujeito nulo,\is{sujeito nulo} como o português, o espanhol\il{espanhol} ou o italiano\il{italiano} e (ii) línguas que não permitem sujeito nulo,\is{sujeito nulo} como o inglês\il{inglês} ou o francês,\il{francês} e, ainda, (iii) línguas que parecem estar em processo de mudança, como é o caso do português brasileiro, que permite sujeitos nulos em ambientes sintáticos específicos. Essa diferença reflete-se em dados como os que apresentamos em (\ref{ex:santoslopes_28}).

\ea\label{ex:santoslopes_28}
\ea[]{\label{ex:santoslopes_28a}
\underline{\hspace{1em}} fui à praia.\\vs. Eu fui à praia.\\}
\ex[*]{\label{ex:santoslopes_28b}
\gll \underline{\hspace{1em}} went {to the} beach.\\
~ fui à praia\\
\glt ~\\
\gll vs. I went {to the} beach.\\
~ eu fui à praia\\}
\ex[*]{\label{ex:santoslopes_28c}
\gll \underline{\hspace{1em}} suis allé {à la} plage.\\
~ \textsc{aux} ido à praia\\
\glt ~~~~`Fui à praia'\\
\gll vs. Je suis allé {à la} plage.\\
~ eu \textsc{aux} ido à praia\\
\glt ~~~~`Fui à praia'\\}
\zl

No entanto, mesmo as crianças que adquirem línguas como o inglês\il{inglês} e o francês\il{francês} produzem inicialmente frases com sujeitos nulos (a par de outras com sujeitos realizados), sendo este também um aspeto em que se observa divergência entre as produções iniciais das crianças e a gramática adulta. Vejam-se os exemplos em (\ref{ex:santoslopes_29}):

\ea\label{ex:santoslopes_29}
\ea\label{ex:santoslopes_29a}
\gll \underline{\hspace{1em}} tout tout tout mangé\\
~ \textsc{aux} tudo tudo tudo comido.\\\jambox{(francês; Augustin 2;0)}
\glt ~~~~`Comeu tudo.'\\
\ex\label{ex:santoslopes_29b}
\gll \underline{\hspace{1em}} was a green one.\\
~ era uma verde uma\\\jambox{(Eve 1;10, \citealt{brown1973})}
\glt ~~~~`Era uma verde.'\jambox{(inglês; \citealt[270]{rizzi2000})}
\zl 

Este fenómeno recebeu várias explicações, sendo a mais conhecida a que explora a ideia de parâmetro\is{parâmetro} na aquisição. \citet{hyams1986} explora a ideia generativista de que a aquisição de uma língua em particular resulta da fixação de parâmetros\is{parâmetro} pré-definidos na Gramática Universal, sendo um desses parâmetros\is{parâmetro} o Parâmetro do Sujeito Nulo.\is{parâmetro! do Sujeito Nulo} \citeauthor{hyams1986} sugere que os parâmetros\is{parâmetro} poderão ter um valor que é assumido como o valor por defeito (default)- no caso do Parâmetro do Sujeito Nulo,\is{parâmetro!do sujeito nulo} esse valor seria o positivo, razão pela qual as crianças começariam por assumir que a língua a que estão expostas e adquirem é uma língua de sujeito nulo\is{sujeito nulo} (produzindo então frases como em \ref{ex:santoslopes_29}). Mais tarde, a observação dos dados da língua a que estão expostas levá-las-ia a fixar o valor alvo do parâmetro.\is{parâmetro}

Essa proposta, entretanto, veio a mostrar-se problemática, pois, se as crianças   começassem com o valor de sujeito nulo,\is{sujeito nulo} não teriam como “voltar” para o valor do parâmetro\is{parâmetro} de sujeito preenchido, já que uma língua de sujeito nulo\is{sujeito nulo} também apresenta dados de sujeitos preenchidos. Isso quer dizer que uma gramática com tal valor, o do sujeito nulo,\is{sujeito nulo} não poderia ser aquela com o valor por defeito e sim o marcado.

\citet{hyams1991} refez a sua análise posteriormente assumindo que a criança teria ao seu dispor dados que a levariam a diferentes opções: (i) uma língua de sujeito nulo,\is{sujeito nulo} como o italiano,\il{italiano} que dependeria de uma flexão\is{flexão} verbal mais robusta recuperando as pessoas do discurso, (ii) uma língua como o chinês, que teria, na realidade, não um sujeito nulo\is{sujeito nulo} do tipo do italiano,\il{italiano} mas um tópico discursivo nulo, (iii) línguas como o inglês,\il{inglês} em que apenas a opção com o sujeito preenchido é gramatical, excluindo, assim, as opções em (i) e (ii); e (iv) línguas como a \textit{American Sign Language} (ASL), em que as opções (i) e (ii) são possíveis. 

A explicação avançada por \citeauthor{hyams1986} é uma explicação gramatical, já que assume que os sujeitos nulos nas produções iniciais das crianças refletem uma diferença entre a gramática das crianças e a gramática adulta. No entanto, foram avançadas outras explicações para o mesmo fenómeno na literatura, sugerindo que os sujeitos nulos nas primeiras produções resultariam de limitações de processamento, que teriam como resultado uma limitação ao número de palavras por enunciado que as crianças seriam capazes de produzir (veja-se \citealt{bloom1990,valianeisenberg} para dados do português). Encontramos ainda em \citet{rizzi2000,rizzi2005} uma explicação gramatical, mas que assume peso do processamento na produção de sujeitos nulos nos primeiros estádios.

De facto, persiste, na literatura, uma tensão entre abordagens gramaticais e abordagens de processamento ao fenómeno da produção de sujeitos nulos nos primeiros estádios de aquisição de línguas que não os permitem. Esse facto levou a que, mais recentemente, \citet{orfitellihyams2012} tenham explorado a compreensão de enunciados com sujeito nulo\is{sujeito nulo} por crianças entre os 2;6 e os 3;11 adquirindo o inglês,\il{inglês} que ou se encontram num estádio em que ainda produzem sujeitos nulos (até pouco depois dos 3 anos, em geral) ou estão progressivamente a deixar de produzir esse tipo de enunciados. Os resultados mostram que as crianças mais novas, que ainda produzem sujeitos nulos, interpretam enunciados com sujeito nulo\is{sujeito nulo} como se se tratasse de enunciados equivalentes numa língua como o português ou o castelhano.\il{espanhol} Tais resultados favorecem uma abordagem gramatical do fenómeno do sujeito nulo\is{sujeito nulo} nos primeiros estádios de aquisição.\footnote{Para uma abordagem alternativa, mas igualmente gramatical, cf. \citealt{lopes2003}.}

\section{Considerações finais}
\label{sec:santoslopes_conclusao}

Em geral, trabalhos como os que se debruçaram sobre a distribuição de verbo e negação\is{negação} em francês\il{francês} ou a posição do verbo em alemão\il{alemão} mostram que uma caracterização das primeiras combinações de palavras das crianças como “telegráficas” é demasiado superficial. Estes dados permitem argumentar contra posições como a defendida por \citet{radford1988}, que vê as primeiras combinações de palavras produzidas pelas crianças como meras projeções de categorias lexicais, despojadas de domínios funcionais, como é o caso de IP.\is{IP} Mostramos, ao longo do capítulo, que há evidências contundentes contra esse tipo de análise, sustentando a presença, no caso deste capítulo, de material no IP\is{IP} e mesmo no domínio CP.\is{CP} As evidências foram trazidas pelo movimento do verbo,\is{movimento!movimento do verbo} através da distinção entre verbos finitos e infinitivos em línguas como o francês,\il{francês} V2\is{V2} no alemão\il{alemão} e dados de elipse de VP\is{elipse do VP} no português e inglês.\il{inglês}

Vimos, ainda, que crianças adquirindo o português, ou outras línguas de sujeito nulo,\is{sujeito nulo} não passam por um estádio de infinitivos raiz.\is{infinitivos raiz} Esse tipo de evidência mostra que as crianças reconhecem muito precocemente a estrutura da sua língua a partir justamente das categorias funcionais relevantes.

Finalmente, mostrou-se que o “estádio de sujeito nulo” \is{sujeito nulo} por que passam as crianças adquirindo línguas de sujeito preenchido ou parcialmente preenchido encontra uma melhor explicação em abordagens gramaticais.


{\sloppy
\printbibliography[heading=subbibliography,notkeyword=this]
}
\end{document}