\documentclass[output=paper]{LSP/langsci} 
\author{João Costa\affiliation{Centro de Línguistica da Universidade Nova de Lisboa \& Faculdade de Ciências Sociais e Humanas da Universidade Nova de Lisboa}\lastand 
Elaine Grolla\affiliation{Universidade de São Paulo}
}
\title{Pronomes, clíticos e objetos nulos: dados de produção e compreensão}  
\abstract{}
\maketitle
\begin{document}
\section{Introdução}
\label{sec:costa_intro}

Os pronomes têm características morfo-fonológicas, sintáticas e semântico-pragmáticas específicas que os tornam particularmente interessantes para o estudo da aquisição. Neste capítulo, enunciamos algumas destas características, explicitando o seu interesse para a aquisição e apresentamos alguns resultados da investigação sobre a produção e a compreensão de pronomes em português europeu e em português brasileiro. 

O capítulo tem a seguinte organização. Na secção \ref{sec:costa_aspetos}, apresentamos uma breve descrição do sistema pronominal do português. Na secção seguinte, enunciam-se alguns dados relevantes sobre a aquisição dos pronomes em diversas línguas, tanto na produção como na compreensão. A secção \ref{sec:costa_aquisicao_pe} reporta alguns dados principais sobre a produção de pronomes clíticos\is{pronome!clítico} por crianças portuguesas e, finalmente, na secção \ref{sec:costa_aquisicao_pb} damos conta de resultados relevantes sobre a interpretação de pronomes no português europeu e brasileiro.

Ao longo do capítulo, mostrar-se-á que as crianças dominam desde cedo as propriedades fundamentais do comportamento sintático das formas pronominais, sendo os aspetos mais tardios na aquisição explicáveis por um domínio mais lento de algumas das suas propriedades interpretativas. Esta observação de que as propriedades básicas do comportamento dos pronomes se encontra estabilizada desde muito cedo reforça a ideia, defendida por vários autores, de que parte do conhecimento sintático é inato e estabiliza durante o primeiro ano de vida, quando as crianças ainda não combinam palavras (ver \citealt{guasti2002} e \citealt{wexler1998} para uma revisão desta proposta).

\section{Aspetos que caracterizam as formas pronominais}
\label{sec:costa_aspetos}

Os pronomes são a classe de palavras utilizada quando o falante não quer ou não pode (por razões linguísticas) repetir um nome. Vejamos os seguintes exemplos:

\ea\label{ex:costa_1}
A: Onde é que a Maria está?\\B: Vi-a à  porta da faculdade.
\z
\ea\label{ex:costa_2}
Tirei o prato da mesa e pu-lo na cozinha.
\z
\ea\label{ex:costa_3}
Pedi ao Pedro para sair com ele.
\z
\ea\label{ex:costa_4}
A Ana falou com o patrão dela.
\z

Nos exemplos entre (\ref{ex:costa_1}) e (\ref{ex:costa_4}), o pronome é utilizado para referir um nome que é facilmente recuperável no discurso anterior ou até na própria frase. Em (\ref{ex:costa_1}), o pronome \textit{-a} retoma o grupo nominal \textit{a Maria}, em (\ref{ex:costa_2}) o pronome \textit{–lo} retoma o grupo nominal \textit{o prato}, em (\ref{ex:costa_3}) o pronome \textit{ele} retoma o grupo nominal \textit{o Pedro} e, em (\ref{ex:costa_4}), o pronome \textit{ela} retoma o grupo nominal \textit{a Ana}.\footnote{Nos exemplos (\ref{ex:costa_3}) e (\ref{ex:costa_4}) existe ainda a possibilidade de a referência do pronome ser estabelecida por outra entidade saliente no contexto/discurso anterior.} Em todos estes contextos, o pronome permite recuperar a referência destes grupos nominais sem os repetir. Aliás, a sua repetição geraria resultados estranhos ou agramaticais, como podemos verificar nos seguintes exemplos:\footnote{Utilizamos o símbolo \# para assinalar inadequação contextual e o símbolo * para marcar agramaticalidade.}\textsuperscript{,}\footnote{As frases (\ref{ex:costa_7}) e (\ref{ex:costa_8}) são agramaticais na interpretação pretendida, em que os dois grupos nominais referem à mesma entidade.}

\ea\label{ex:costa_5}
A: Onde é que a Maria está?\\B: \#Vi a Maria à porta da faculdade.
\z

\ea[*]{\label{ex:costa_6}
Tirei o prato da mesa e pus o prato na cozinha.}
\z
\ea[*]{\label{ex:costa_7}
Pedi ao Pedro para sair com o Pedro.}
\z
\ea[*]{\label{ex:costa_8}
A Ana falou com o patrão da Ana.}
\z

Os pronomes distinguem-se dos grupos nominais por um conjunto de propriedades específicas, que os tornam particularmente interessantes no estudo da aquisição das línguas.

\subsection{Tipos de pronomes: fortes, clíticos e nulos}
\label{subsec:costa_tipos_pronomes}

Os pronomes não têm todos o mesmo comportamento, conforme notado por \citet{cardinalettistarke1994}, sendo possível distinguir pronomes fortes,\is{pronome!forte} pronomes clíticos\is{pronome!clítico} e pronomes nulos.\is{pronome!nulo} Em (\ref{ex:costa_9}), temos o exemplo de cada um destes tipos de pronomes:

\ea\label{ex:costa_9}
\ea\label{ex:costa_9a} A Ana falou com \textbf{ele}.\jambox{(Pronome forte)}
\ex\label{ex:costa_9b} A Ana viu-\textbf{o}.\jambox{(Pronome clítico)}
\ex\label{ex:costa_9c} \textbf{Ø} vi a Ana.\jambox{(Pronome nulo)}
\zl

Estes pronomes são diferenciados por várias razões. Podemos começar por apreciar a diferença entre pronomes fortes\is{pronome!forte} e clíticos. A primeira razão para os distinguir é de natureza interlinguística: nem todas as línguas têm pronomes clíticos,\is{pronome!clítico} embora todas pareçam ter pronomes fortes.\is{pronome!forte} Por exemplo, quando comparamos as variedades portuguesa e brasileira do português, podemos constatar que o português brasileiro tem vindo a perder alguns clíticos.\is{pronome!clítico} Assim, uma frase como (\ref{ex:costa_9b}) não é atualmente produtiva em português brasileiro, ocorrendo no mesmo contexto um pronome forte,\is{pronome!forte} como em (\ref{ex:costa_10}):

\ea\label{ex:costa_10}
A Ana viu \textbf{ele}.
\z

Para além da motivação interlinguística, há construções em que os dois tipos de pronome participam de forma distinta, conforme relatado em \citet{cardinalettistarke1994}. Ilustramos aqui algumas características que os distinguem:

\paragraph*{Propriedades prosódicas}
Os pronomes fortes\is{pronome!forte} são acentuados, podendo ser acentuados em contextos enfáticos,\footnote{A acentuação é aqui assinalada graficamente através do uso de letras maiúsculas.} algo que é impossível com os pronomes clíticos,\is{pronome!clítico} que são necessariamente átonos:

\ea\label{ex:costa_11}
\ea\label{ex:costa_11a} A Ana falou com ele/ELE.
\ex\label{ex:costa_11b} A Ana viu-o/*-O.
\zl
\paragraph*{Coordenação}
Os pronomes fortes\is{pronome!forte} podem ser coordenados com outro pronome ou com um grupo nominal, o que não acontece com os clíticos:\is{pronome!clítico}

\ea\label{ex:costa_12}
\ea[]{\label{ex:costa_12a} Ele e o Pedro ficaram em casa.}
\ex[*]{\label{ex:costa_12b} A Ana viu-o e o Pedro.}
\zl

\paragraph*{Modificação}
Ao contrário dos clíticos,\is{pronome!clítico} os pronomes fortes\is{pronome!forte} podem ser modificados por advérbios. Como se evidencia em (\ref{ex:costa_13}), o clítico\is{pronome!clítico} não pode ser modificado pelo advérbio. (13c) mostra que, mesmo que o advérbio não seja colocado em posição de modificação do clítico,\is{pronome!clítico} a frase não tem a leitura em que o advérbio tem escopo sobre ele:

\ea\label{ex:costa_13}
\ea[]{\label{ex:costa_13a} Só ele ficou em casa.}
\ex[*]{\label{ex:costa_13b} Eu vi só -o.}
\ex[]{\label{ex:costa_13c} Eu vi-o somente ($\neq$ eu vi só a ele.)}
\zl

\paragraph*{Posição}
Os clíticos\is{pronome!clítico} ocupam posições específicas na frase \citep{kayne1975,duartematos2000}, ao contrário dos pronomes fortes,\is{pronome!forte} que ocupam a mesma posição que qualquer outro grupo nominal. Conforme ilustrado em (\ref{ex:costa_14}), com verbos auxiliares, os clíticos\is{pronome!clítico} ocorrem em adjacência aos verbos flexionados, ao contrário dos pronomes fortes:\is{pronome!forte}\footnote{Pretende-se exemplificar aqui a posição dos clíticos\is{pronome!clítico} quando ocorrem com verbos auxiliares e não o que acontece em contextos de subida do clítico,\is{pronome!clítico} caso em que a subida do clítico\is{pronome!clítico} pode ser opcional.}

\ea\label{ex:costa_14}
\ea[]{\label{ex:costa_14a} O Pedro não o tinha visto.}
\ex[*]{\label{ex:costa_14b} O Pedro não tinha visto-o.}
\zl

As mesmas frases em português brasileiro revelam que o pronome forte\is{pronome!forte} ocupa a mesma posição que qualquer outro grupo nominal:

\ea\label{ex:costa_15}
\ea[*]{\label{ex:costa_15a} O Pedro não ele tinha visto.}
\ex[]{\label{ex:costa_15b} O Pedro não tinha visto ele.}
\zl

Em línguas com clíticos\is{pronome!clítico} sintáticos, como é o caso do português europeu, a posição dos pronomes é variável e sintaticamente condicionada. Ilustram-se, em (\ref{ex:costa_16}), alguns contextos de próclise,\is{próclise} em que o pronome ocorre em posição pré-verbal:\footnote{Ver \citet{duartematos2000}, entre outros, para uma descrição dos contextos de colocação dos pronomes.}

\ea\label{ex:costa_16}
\ea\label{ex:costa_16a} O Pedro não o tinha visto.\\(Negação)
\ex\label{ex:costa_16b} Disseram que o Pedro o tinha visto.\\(Subordinação finita com conjunção)
\ex\label{ex:costa_16c} Quando é que o Pedro o tinha visto?\\(Interrogativas parciais com constituinte interrogativo anteposto)
\ex\label{ex:costa_16d} Eu já o tinha visto.\\(Advérbios como ``já, também, só, ainda, \ldots'')
\zl

Estas são algumas das propriedades que distinguem os pronomes fortes\is{pronome!forte} dos pronomes clíticos.\is{pronome!clítico} Se pensarmos na distinção entre pronomes clíticos\is{pronome!clítico} e pronomes nulos,\is{pronome!nulo} torna-se evidente que esta diferença é real. Em primeiro lugar, deve notar-se que há línguas que têm pronomes clíticos\is{pronome!clítico} e não têm pronomes nulos.\is{pronome!nulo} O francês\il{francês} é um exemplo de língua assim. Como vemos em (\ref{ex:costa_17}), nesta língua há clíticos\is{pronome!clítico} (quer em posição de sujeito, quer em posição de objeto), mas as contrapartidas nulas são agramaticais:

\ea\label{ex:costa_17}
\ea[]{
\gll Je l'aime.\\
Eu {o amo}\\
\glt `Eu o amo.'
}
\ex[*]{
\gll	\textbf{Ø} l'aime.\\
		Ø {o amo}\\
\glt	`Amo-o.'
}
\ex[*]{
\gll	J'aime \textbf{Ø}.\\
		{Eu amo} \textbf{Ø}\\
\glt	`Eu amo.'
}
\zl

Ao contrário do que acontece em francês,\il{francês} o português (tanto brasileiro quanto europeu) admite formas nulas tanto em contexto de sujeito, como em contexto de objeto. Estes pronomes nulos\is{pronome!nulo} permitem caracterizar o português como língua de sujeito nulo\is{sujeito nulo} e como língua de objeto nulo \citep{raposo1986}:

\ea\label{ex:costa_18}
\ea\label{ex:costa_18a} Eu vejo o João.
\ex\label{ex:costa_18b} \textbf{Ø} vejo o João.\jambox{(sujeito nulo)}
\ex\label{ex:costa_18c} Eu vejo \textbf{Ø}.\jambox{(objeto nulo)}
\zl

As duas propriedades são independentes, já que há línguas que admitem apenas sujeitos nulos e não objetos nulos (como é o caso do italiano\il{italiano} ou do espanhol),\il{espanhol}  e outras há que admitem objeto nulo,\is{objeto nulo} mas em que o sujeito nulo\is{sujeito nulo} é bastante mais restrito (o português brasileiro é um caso destes, conforme discutido em \citealt{duarte1995}).

\subsection{Propriedades sintático-semânticas dos pronomes}
\label{subsec:costa_propriedades_sint_sem_pronomes}

Como já vimos, diferentes pronomes exibem comportamentos sintáticos diferentes. Já se mostrou, por exemplo, que os pronomes clíticos\is{pronome!clítico} têm uma distribuição diferente da dos pronomes fortes.\is{pronome!forte} Com efeito, como foi mostrado, os pronomes clíticos\is{pronome!clítico} ocorrem junto do verbo flexionado e a sua posição pré ou pós-verbal é condicionada pelo contexto sintático, designadamente pela presença de alguns desencadeadores de próclise.\is{próclise}

Para além destas questões sintáticas que condicionam o posicionamento dos pronomes clíticos,\is{pronome!clítico} é interessante observar que a interpretação dos pronomes é parcialmente condicionada pelo contexto sintático de ocorrência.

Ao contrário dos grupos nominais, os pronomes não têm referência própria. Podemos comparar as frases em (\ref{ex:costa_19}), para o entender:

\ea\label{ex:costa_19}
\ea\label{ex:costa_19a} O diretor da Faculdade caiu.
\ex\label{ex:costa_19b} Ele caiu.
\zl

Em (\ref{ex:costa_19a}), a expressão \textit{o diretor da Faculdade} tem referência própria, sendo que o pronome \textit{ele}, em (\ref{ex:costa_19b}), precisa de contexto linguístico ou situacional para se lhe poder atribuir referência. A referência de um pronome pode ser fixada de formas diferentes. Uma frase como (19b) pode ser utilizada ao mesmo tempo que o falante aponta para alguém. Temos, nesse caso, o contexto extralinguístico a determinar a referência do pronome. A referência também pode ser fixada interfrasicamente, como em (\ref{ex:costa_20}):

\ea\label{ex:costa_20}
\ea\label{ex:costa_20a} O que é que aconteceu ao Pedro?
\ex\label{ex:costa_20b} Nem imaginas, quando o carro ia a chegar, ele caiu.
\zl

Em (\ref{ex:costa_20}), o contexto linguístico permite-nos interpretar \textit{ele} como \textit{o Pedro}. Dizemos que duas expressões que são interpretadas da mesma forma são co-referentes. Nestes casos, temos exemplos de fixação semântico-pragmática da referência dos pronomes, dado que estamos em contextos interfrásicos.

Sabe-se, contudo, desde o final dos anos 70 e graças a muita investigação conduzida durante os anos 80, sobretudo desde a publicação dos trabalhos de \citet{reinhart1976} e \citet{chomsky1981}, que a interpretação dos pronomes pode ser sintaticamente condicionada. Vejamos os exemplos seguintes:

\ea\label{ex:costa_21}
\ea\label{ex:costa_21a} O Pedro lavou-se.
\ex\label{ex:costa_21b} O Pedro lavou-o.
\zl

Em (\ref{ex:costa_21a}), \textit{o Pedro} e \textit{–se} são obrigatoriamente co-referentes, o que contrasta com (\ref{ex:costa_21b}), em que \textit{o Pedro} e \textit{–o} são obrigatoriamente não co-referentes, isto é, o pronome pode referir qualquer entidade (desde que compatível com masculino singular), exceto \textit{o Pedro}. A formas como \textit{–se} chamamos \textit{anáforas},\is{anáfora} reservando o termo \textit{pronome} para os que se comportam como \textit{–o}.

Como se pode ver nos exemplos seguintes, manipulando o contexto sintático, podem alterar-se as condições de interpretação dos pronomes. Se se aumentar a distância sintática entre o antecedente e o pronome, é possível verificar que as possibilidades de interpretação são afetadas. Atente-se a (\ref{ex:costa_22}): a introdução de um nível de subordinação condiciona a interpretação da anáfora\is{anáfora} em (\ref{ex:costa_22a}), sendo que esta tem de ter como antecedente o grupo nominal que se encontra na mesma oração. Em (\ref{ex:costa_22b}), vemos que, ao contrário da anáfora,\is{anáfora} o pronome não pode ter o seu antecedente na mesma oração, mas pode ter como antecedente o grupo nominal que é sujeito da oração matriz:

\ea\label{ex:costa_22}
\ea\label{ex:costa_22a} O Pedro disse que o João se lavou.\\(se $=$ o João; se $\neq$ o Pedro)
\ex\label{ex:costa_22b} O Pedro disse que o João o lavou.\\(-o $\neq$ o João; -o $=$ o Pedro)
\zl

A manipulação do contexto sintático mostra-nos ainda que a relação entre o antecedente e o pronome é estabelecida hierarquicamente e não de forma linear. Em (\ref{ex:costa_23}), vemos que o antecedente do pronome pode estar na mesma oração que o pronome, desde que se encontre hierarquicamente inacessível:

\ea\label{ex:costa_23}
\ea\label{ex:costa_23a} O Pedro lavou-o.\\(-o $\neq$ o Pedro)
\ex\label{ex:costa_23b} O filho do Pedro lavou-o.\\(-o $=$ o Pedro)
\zl

Desde os anos 80, a Teoria da Ligação\is{Teoria da Ligação} é o módulo da gramática responsável por descrever e explicar a forma como as anáforas\is{anáfora} e os pronomes adquirem referência, explicitando os contextos sintáticos que o legitimam. Perante dados como os de (\ref{ex:costa_22}) e (\ref{ex:costa_23}), podemos formular generalizações como as seguintes, que correspondem a versões muito informais dos princípios da Teoria da Ligação:\is{Teoria da Ligação}

\begin{enumerate}[label=\Alph*.]
\item  Uma anáfora\is{anáfora} tem obrigatoriamente o seu antecedente acessível na oração em que se encontra.
\item  Um pronome não pode ter um antecedente acessível na oração em que se encontra.
\end{enumerate}

Também os pronomes nulos\is{pronome!nulo} são regulados por condições sintáticas específicas. Para ilustrar as condições sintáticas a que os pronomes nulos\is{pronome!nulo} obedecem, podemos recorrer aos objetos nulos. Conforme demonstrado em \citet{raposo1986}, em português europeu (mas não em português brasileiro), os objetos nulos são legítimos em frases simples (\ref{ex:costa_24}B), mas não nos contextos sintáticos conhecidos como contextos-ilha\is{ilha} \citep{ross1969}, de que as orações adverbiais são um exemplo (\ref{ex:costa_25}B):

\ea\label{ex:costa_24}
Português Europeu:\\
A: E o teu carro?\\
B: Levei \textbf{Ø}/-o agora mesmo para a oficina.
\z

\ea\label{ex:costa_25}
A: E o teu carro?\\
B: Estou chateado porque não vi \textbf{*Ø}/-o vi na oficina.
\z

Já em português brasileiro, objetos nulos são possíveis tanto em frases simples (\ref{ex:costa_26}B) como em contextos-ilha\is{ilha} (\ref{ex:costa_27}B):

\ea\label{ex:costa_26}
Português Brasileiro:\\
A: E o seu carro?\\
B: Levei \textbf{Ø}/ele agora mesmo para a oficina.
\z

\ea\label{ex:costa_27}
A: E o teu carro?\\
B: Estou chateado porque não \textbf{*Ø}/ele na oficina.
\z

Vemos, assim, que o conhecimento das propriedades dos pronomes passa também pelo conhecimento dos contextos sintáticos em que são possíveis e da influência desses contextos sintáticos na sua interpretação.

\subsection{Propriedades semântico-pragmáticas dos pronomes}
\label{subsec:costa_propriedades_sem_prag_pronomes}

Para além das propriedades sintático-semânticas dos pronomes, é importante referir, ainda que brevemente, que os pronomes (ao contrário das anáforas)\is{anáfora} podem ter a sua referência fixada pragmaticamente, o que também é sujeito a restrições específicas. A legitimação textual e pragmática pode ser atestada em contextos como o que é ilustrado em (\ref{ex:costa_28}):

\ea\label{ex:costa_28}
O Pedro chegou a casa cansado. \textbf{Ele} tinha trabalhado dez horas seguidas.
\z

Em (\ref{ex:costa_28}), o pronome \textit{ele} é co-referente com o grupo nominal \textit{o Pedro}. Este é um processo de retoma textual, que não é sintaticamente condicionado, já que estamos perante expressões que se encontram em frases distintas, para além portanto do nível máximo de análise da sintaxe.\footnote{No mesmo contexto, seria possível um sujeito nulo,\is{sujeito nulo} o que é irrelevante para o que aqui se ilustra:
\ea\label{ex:costa_i}
O Pedro chegou a casa cansado. Tinha trabalhado dez horas seguidas.
\z} Curiosamente, a legitimação interfrásica também conhece limites. Por exemplo, se o antecedente for uma expressão quantificada e não um grupo nominal, torna-se impossível o estabelecimento de co-referência por esta via:

\ea\label{ex:costa_29}
Cada funcionário chegou a casa cansado. \textbf{*Ele} tinha trabalhado dez horas.
\z

Neste mesmo contexto, o pronome nulo\is{pronome!nulo} é preferível:

\ea\label{ex:costa_30}
Cada funcionário chegou a casa cansado. \textbf{Ø} tinha trabalhado dez horas.
\z

Vê-se, assim, que é necessário conhecer que propriedades um antecedente pode ter e saber se pode alternar livremente ou não com uma forma nula do pronome.

\section{Adquirir pronomes}
\label{sec:costa_adquirir_pronomes}

Com base na breve descrição das propriedades dos pronomes conduzida na secção anterior, podemos já adivinhar por que motivo o estudo da aquisição dos pronomes é tão importante. Em primeiro lugar, ao estudar-se como as crianças chegam a um conhecimento estável sobre pronomes, avaliamos como dominam uma área do seu conhecimento linguístico para a qual convergem questões fonológicas, morfológicas, sintáticas, semânticas e pragmáticas. O estudo da aquisição dos pronomes é, assim, uma janela sobre a aquisição de diferentes componentes da gramática em interação. Com base nas propriedades que descrevemos na secção anterior, podemos formular um conjunto de questões com relevância para os estudos em aquisição:

\begin{enumerate}[label=Q\arabic*]
\item\label{q:costa_1}  \textit{As crianças sabem que há diferentes tipos de pronomes?}

As crianças distinguem pronomes fortes,\is{pronome!forte} clíticos\is{pronome!clítico} e nulos\is{pronome!nulo} nas suas produções iniciais ou na forma como os compreendem? A resposta a esta pergunta permite-nos saber se as crianças lidam com conhecimento que se relaciona com diferentes propriedades morfológicas e sintáticas, em particular com a relevância da lexicalidade e dos níveis de projeção das palavras (como núcleos sintáticos ou como projeções máximas).

\item\label{q:costa_2} \textit{As crianças conhecem o tipo de língua que estão a adquirir no que concerne à disponibilidade de categorias nulas?}

A existência de sujeitos nulos e objetos nulos numa língua é sujeita a variação paramétrica, já que nem todas as línguas têm sujeito e objeto nulo.\is{objeto nulo} O estudo da aquisição dos pronomes permitirá saber quão precocemente as crianças fixam parâmetros\is{parâmetro} deste tipo.

\item\label{q:costa_3} \textit{As crianças conhecem as restrições de colocação dos pronomes?}

Vimos que os pronomes clíticos\is{pronome!clítico} têm uma colocação sintaticamente condicionada. Importará saber se as crianças dominam os contextos de colocação dos pronomes, para sabermos quão específico é o seu conhecimento sobre formas pronominais.

\item\label{q:costa_4} \textit{As crianças conhecem as condições sintáticas que restringem a interpretação dos pronomes?}

Por outras palavras, as crianças conhecem os princípios da Teoria da Ligação,\is{Teoria da Ligação} que determinam os contextos sintáticos para a distribuição e interpretação de pronomes e anáforas?\is{anáfora}

\item\label{q:costa_5} \textit{As crianças conhecem todas as restrições semântico-pragmáticas que condicionam a interpretação dos pronomes?}

Ainda que a resposta a \ref{q:costa_4} seja positiva e se perceba que as crianças têm um bom conhecimento sintático sobre os pronomes, tal não significa que as crianças dominem todas as restrições semânticas e pragmáticas que regulam a forma como os pronomes são interpretados.
\end{enumerate}

Nas últimas duas décadas, tem sido muita a literatura que se tem dedicado ao estudo da aquisição de formas pronominais, quer no que concerne à sua produção, quer no que diz respeito à sua compreensão. Antes de descrevermos resultados nos estudos sobre o português, resumimos alguns dos principais resultados disponíveis na literatura para diversas línguas.


\subsection{Produção de pronomes em diversas línguas}
\label{subsec:costa_producao_pronomes}

São vários os estudos que mostram que as crianças omitem pronomes nas suas produções iniciais. \citet{jakubowicz_etal1998} mostram, contudo, que, em francês,\il{francês} esta omissão de pronomes é seletiva: afeta apenas pronomes acusativos, como em (\ref{ex:costa_31a}) e não os dativos (como em \ref{ex:costa_31b}):

\ea\label{ex:costa_31}
\ea{\label{ex:costa_31a}
\gll Il \textbf{la} lave.\\
ele a lava\\
\glt `Ele a lava.'
}
\ex{\label{ex:costa_31b}
\gll	Il \textbf{lui} téléphone.\\
		Ele lhe telefona\\
\glt	`Ele lhe telefona.'
}
\zl

\citet{jakubowicz_etal1998} mostram ainda que a omissão de pronomes clíticos\is{pronome!clítico} não se deve à sua fraca proeminência fonológica. Na verdade, a comparação entre a produção de pronomes clíticos\is{pronome!clítico} e determinantes em francês\il{francês} (que são homófonos) mostra que as crianças apenas omitem os pronomes clíticos\is{pronome!clítico} e não os determinantes, o que permite argumentar que a omissão de clíticos\is{pronome!clítico} não se relaciona com a sua atonicidade. Em \citet{varlokosta_etal2015}, refere-se que, em línguas sem clíticos,\is{pronome!clítico} como o inglês,\il{inglês} não há omissão de pronomes.

Estes dados permitem já dar uma resposta a \ref{q:costa_1} enunciada acima: as crianças tratam os pronomes clíticos\is{pronome!clítico} de forma diferenciada, o que permite supor que têm algum conhecimento sobre as suas propriedades específicas.

Vários estudos para várias línguas se dedicaram a aferir se as crianças omitem pronomes clíticos\is{pronome!clítico} nas suas produções iniciais. Observou-se que a omissão não é atestada em todas as línguas. Encontra-se em francês,\il{francês} italiano\il{italiano} \citep{schaeffer1997}, catalão\il{catalão} \citep{wexler_etal2004}, mas não em espanhol\il{espanhol} \citep{wexler_etal2004} ou grego\il{grego} \citep{tskaliwexler2003}. Parte dos estudos na literatura tem tentado associar a omissão de clítico\is{pronome!clítico} a outras propriedades dos sistemas linguísticos, como a existência de concordância nos particípios passados, o que não é de todo consensual. Percebe-se, contudo, a relevância de avaliar a robustez da generalização de que os clíticos\is{pronome!clítico} são adquiridos tardiamente.

Os estudos sobre a produção dos pronomes pelas crianças têm mostrado que as crianças não exibem grandes problemas na colocação dos pronomes clíticos\is{pronome!clítico} em línguas como o italiano\il{italiano} ou o francês\il{francês} (conforme demonstrado em \citealt{guasti2002}), o que contribui para que se possa dizer que há um bom conhecimento das propriedades destas formas pronominais. Veremos, contudo, mais adiante que esta ideia de que as crianças não cometem erros na distribuição dos pronomes requer alguma qualificação, já que não se verifica em todas as línguas.

Importa ainda referir, no âmbito dos estudos de produção, os trabalhos de \citeauthor{hyams1992genesis}, sobretudo a partir de \citeyear{hyams1992genesis}, sobre sujeitos nulos na aquisição (veja-se sobre este tópico o Capítulo 7, particularmente a Secção 6).


\subsection{Compreensão de pronomes em diversas línguas}
\label{subsec:costa_compreensao_linguas}

No que concerne à compreensão de pronomes, o estudo seminal de \citet{chienwexler1990} veio mostrar que pode haver dificuldades na interpretação de alguns pronomes em inglês.\il{inglês} Estes autores testaram a compreensão de frases como as de (\ref{ex:costa_32}), mostrando que as crianças a adquirir inglês\il{inglês} não têm dificuldades na compreensão de anáforas,\is{anáfora} mas atribuem erradamente leituras co-referenciais a pronomes na compreensão de enunciados como (\ref{ex:costa_32b}):

\ea\label{ex:costa_32}
\ea{\label{ex:costa_32a}
\gll Mama Bear washed herself.\\
Mamãe Ursa lavou se\\
\glt `Mamãe Ursa lavou-se.'
}
\ex{\label{ex:costa_32b}
\gll Mama Bear washed her.\\
Mamãe Ursa lavou ela\\
\glt `Mamãe Ursa lavou-a.'
}
\zl

À primeira vista, poder-se-ia pensar que as crianças exibem um atraso no domínio do princípio B da Teoria da Ligação,\is{Teoria da Ligação} que regula a distribuição e interpretação de pronomes. Contudo, essa é uma hipótese bastante questionável, pelas seguintes razões:

\begin{enumerate}[label=\alph*)]
\item Os princípios da Teoria da Ligação\is{Teoria da Ligação} funcionam de forma complementar, isto é, as anáforas\is{anáfora} ocorrem em distribuição complementar com os pronomes. Assim, seria muito difícil de entender que haja um bom domínio do princípio A, mas não do princípio B.
\item A compreensão de frases como (\ref{ex:costa_33}) é perfeita. (\ref{ex:costa_33}) distingue-se de (\ref{ex:costa_32b}) por conter um antecedente quantificado. Nos dois casos, (\ref{ex:costa_32b}) e (\ref{ex:costa_33}), o pronome não pode ter o sujeito da frase como antecedente. Isto é, (\ref{ex:costa_32b}) não pode significar que `Mamãe ursa lavou-se’ e (\ref{ex:costa_33}) não pode significar que `toda ursa lavou-se’. As crianças rejeitam (\ref{ex:costa_33}) em contextos que mostram cada ursa se lavando, mas aceitam (\ref{ex:costa_32b}) em contextos em que mamãe ursa se lavou.  Como vimos acima, os quantificadores podem ser antecedentes de pronomes, mas apenas quando a relação de co-referência é estabelecida sintaticamente:

\ea\label{ex:costa_33}
\gll Every Bear washes her.\\
Toda ursa lava ela\\
\glt `Toda ursa lava-a.'
\z

O sucesso na compreensão de frases como (\ref{ex:costa_33}) mostra que não é o princípio B que está afetado, mas sim os modos extrassintáticos de atribuição de referência. Por outras palavras, as crianças terão apenas dificuldades no domínio de propriedades pragmáticas que regulam a interpretação das formas pronominais e não nos princípios sintáticos. Isto explica que a ligação por um antecedente quantificado não seja problemática e que não se verifiquem problemas com anáforas,\is{anáfora} cuja referência é sempre fixada sintaticamente.

\item Os estudos de \citet{mckee1992} para o italiano\il{italiano} e de \citet{padilha1990} para o espanhol\il{espanhol}  mostram que não há dificuldades na compreensão de pronomes quando estes são clíticos.\is{pronome!clítico} Para além de, novamente, ser reforçada a ideia de que as crianças distinguem clíticos\is{pronome!clítico} de pronomes fortes,\is{pronome!forte} estas observações permitem afirmar que os princípios sintáticos da Teoria da Ligação\is{Teoria da Ligação} não são sujeitos a maturação no desenvolvimento linguístico.

\end{enumerate}

Estes dados já nos permitem chegar a algumas respostas às questões formuladas acima. Parece ser possível supor que as crianças conhecem as restrições sintáticas que condicionam a interpretação dos pronomes (\ref{q:costa_4}), mas não conhecem todas as restrições semântico-pragmáticas que condicionam a interpretação dos pronomes.

Nas secções seguintes, veremos que os dados disponíveis para o português corroboram estas conclusões.


\section{Aquisição de pronomes em português europeu: produção}
\label{sec:costa_aquisicao_pe}

Apresentamos, nesta secção, os principais resultados dos estudos que têm vindo a ser feitos sobre a produção de pronomes clíticos\is{pronome!clítico} em português. Dada a ausência de tais pronomes em português brasileiro, a discussão ficará confinada ao português europeu. São particularmente relevantes os estudos sobre as taxas de produção desses pronomes e sobre a sua colocação.

\subsection{Produção vs. Omissão de pronomes clíticos}
\label{subsec:costa_producao_omissao}

Conforme vimos na secção anterior, há várias línguas em que os pronomes são omitidos nas produções iniciais das crianças. Os trabalhos de Ken Wexler e colegas têm colocado a hipótese de que os pronomes são omitidos apenas naquelas línguas em que existe concordância de particípio passado, como é o caso do francês,\il{francês} ilustrado em (\ref{ex:costa_34}):

\ea\label{ex:costa_34}
\gll J'ai repeint les fenêtres. Je les ai repeintes.\\
{eu tenho} repintado.\textsc{m.sg} as janelas.\textsc{f.pl} eu as.\textsc{f.pl} tenho repintado.\textsc{f.pl}\\
\glt `Eu repintei as janelas. Eu repintei-as.'
\z

De acordo com esta hipótese, prediz-se que não haja omissão de clíticos\is{pronome!clítico} em português europeu, já que esta língua não tem concordância de particípio passado. Numa sequência de estudos, Costa, Lobo e Silva \citep{costalobo2007,costalobo_etal2012,silva2008} avaliaram a produção induzida de clíticos\is{pronome!clítico} por crianças entre os 3 e os 6 anos, tendo chegado a duas conclusões principais:

\begin{enumerate}[label=\Alph*.]
\item  As crianças portuguesas omitem clíticos em taxas superiores às identificadas para outras línguas.
\item  As crianças portuguesas omitem clíticos\is{pronome!clítico} até mais tarde do que foi encontrado para outras línguas.
\end{enumerate}

Perante estes dados, duas hipóteses se apresentam: ou a omissão de clíticos\is{pronome!clítico} é diferente em línguas diferentes ou deverá haver uma explicação alternativa para o que se passa na aquisição do português europeu. 

Recorde-se que o português europeu tem objetos nulos, pelo que a produção de um verbo sem complemento pode não corresponder a uma omissão de clítico,\is{pronome!clítico} mas sim a uma produção de objeto nulo\is{objeto nulo} semelhante à que se encontra na gramática do adulto. Por este motivo, \citet{costalobo_etal2012} e \citet{silva2008} elicitaram a produção de clíticos\is{pronome!clítico} que não alternam livremente com objetos nulos ou em contextos em que essa alternância não é legítima. Foram, assim, testados os seguintes contextos:

\begin{enumerate}[label=\alph*)]
\item Dativos (não existe consenso sobre a disponibilidade de objeto nulo\is{objeto nulo} em contexto dativo, cf. \citealt{costaduarte2003})
\item Clíticos de 1.ª e 2.ª pessoa (que não alternam livremente com objeto nulo)
\item Clíticos reflexos (que não alternam livremente com objeto nulo)
\item Contextos ilha\is{ilha} (em que o objeto nulo não é legitimado)
\end{enumerate}

Os resultados dos testes aplicados são bastante robustos e podem ser resumidos da seguinte forma:

\begin{enumerate}[label=\alph*)]
\item As crianças produzem frases em que o pronome é omitido até tarde.
\item A omissão é igualmente alta em contextos acusativos não-reflexos e nos outros contextos estudados, não se verificando diferenças entre dativos, pessoa, reflexos e contextos de ilha.\is{ilha}
\end{enumerate}

Com base nestes resultados, \citet{costalobo2009} colocaram a hipótese de que a omissão encontrada em português europeu não é um caso de omissão de clítico,\is{pronome!clítico} mas sim de sobregeneralização da construção de objeto nulo.\is{objeto nulo}

Estes resultados contribuem para uma resposta clara a algumas das questões enunciadas na secção anterior. Se a hipótese se verificar, podemos dizer que as crianças portuguesas sabem que o seu sistema tem pronomes clíticos\is{pronome!clítico} (\ref{q:costa_1}), que a sua língua tem objetos nulos (\ref{q:costa_2}), mas ainda não dominam os contextos específicos em que o objeto nulo\is{objeto nulo} é legitimado (\ref{q:costa_5}).

\subsection{Posicionamento dos pronomes clíticos}
\label{subsec:costa_posicionamento}

\citet{duarte_etal1995}  e \citet{duartematos2000} relatam que as crianças portuguesas generalizam a posição enclítica\is{ênclise} (pós verbal) do pronome, conforme se ilustra nos seguintes exemplos:

\ea\label{ex:costa_35}
\ea\label{ex:costa_35a} Foste tu que daste-me  \jambox{(J. 4;8)}
\ex\label{ex:costa_35b} Foi a Mariana que deu-me este \jambox{(Sandra 3;0.21; in \citealt[375]{soares2006})}
\ex\label{ex:costa_35c} foi alguém que meteu-me nesta fotografia. \jambox{(J. G. 3;3; in \citealt{duarte_etal1995})}
\ex\label{ex:costa_35d} O mano não deixa-me dormir.\jambox{(J. 3;8)}
\ex\label{ex:costa_35e} não chama-se nada\jambox{(M. 20 meses; \citealt{duarte_etal1995})}
\ex\label{ex:costa_35f} Porque partiu-se, mãe?\jambox{(J. 3;4)}
\ex\label{ex:costa_35g} Porque é que foste-me interromper?\jambox{(R., 2;5; in \citealt{duarte_etal1995})}
\zl

Em \citet{costa_etal2014}, foi testada a produção de clíticos\is{pronome!clítico} em diferentes contextos de ênclise\is{enclise@ênclise} e próclise.\is{próclise} Testou-se, em particular, a produção de clíticos\is{pronome!clítico} em contextos com um desencadeador de próclise:\is{próclise} a negação, o advérbio \textit{já}, a subordinação completiva, a subordinação adverbial, interrogativas, com sujeitos negativos e com sujeitos quantificados, como se ilustra nos seguintes exemplos:

\ea\label{ex:costa_36}
\ea\label{ex:costa_36a} Eu não o vi.
\ex\label{ex:costa_36b} Eu já o vi.
\ex\label{ex:costa_36c} Eu disse que o vi.
\ex\label{ex:costa_36d} Eu tossi quando o vi.
\ex\label{ex:costa_36e} Quando é que eu o vi?
\ex\label{ex:costa_36f} Nenhum aluno o viu.
\ex\label{ex:costa_36g} Todos os alunos o viram.
\zl

\citet{costa_etal2014} observaram que as crianças, de facto, sobregeneralizam a ênclise,\is{enclise@ênclise} mas fazem-no em taxas diferenciadas nos vários contextos, sendo que adquirem a próclise\is{próclise} de forma gradual nos diferentes contextos. Estes autores identificaram a seguinte escala de desenvolvimento da próclise:\is{próclise}

\ea\label{ex:costa_37}
Negação $>$ Sujeitos negativos / subordinação completiva $>$ Advérbio \textit{já} $>$ subordinação adverbial $>$ Sujeitos quantificados
\z

De acordo com \citet{costa_etal2014}, esta sequência explica-se quando se tem em conta a complexidade inerente de cada um destes contextos: por exemplo, para adquirir próclise\is{próclise} com negação, basta saber que a negação é um desencadeador de próclise.\is{próclise} Já para adquirir a próclise\is{próclise} com sujeitos quantificados, é necessário saber qual o subconjunto de quantificadores que é, de facto, desencadeador de próclise.\is{próclise} Vemos, assim, que o conhecimento sobre próclise\is{próclise} e ênclise\is{enclise@ênclise} é relativamente precoce, mas que o domínio completo dos contextos para o posicionamento do clítico\is{pronome!clítico} vai depender do conhecimento de propriedades dos itens lexicais envolvidos e da sua complexidade inerente.

Em jeito de sumário, podemos concluir que os estudos sobre a produção dos pronomes por crianças que estão a adquirir o português europeu nos permitem afirmar que:

\begin{enumerate}[label=\alph*)]
\item As crianças distinguem clíticos\is{pronome!clítico} de outros pronomes desde cedo.
\item As crianças usam o objeto nulo\is{objeto nulo} produtivamente desde cedo, embora o sobregeneralizem.
\item As crianças usam próclise\is{próclise} e ênclise\is{enclise@ênclise} desde cedo, mas a estabilização dos contextos de próclise\is{próclise} depende da aquisição de aspetos lexicais e sintáticos que trazem complexidade para o input.
\end{enumerate}

\section{Aquisição dos pronomes em português europeu e brasileiro: compreensão}
\label{sec:costa_aquisicao_pb}

No que concerne à compreensão de pronomes, reportaremos estudos em duas áreas principais: a compreensão de pronomes nas duas variedades do português (europeia e brasileira) e a compreensão e aceitabilidade da construção de objeto nulo.\is{objeto nulo}

\subsection{Compreensão de pronomes clíticos e fortes em português}
\label{subsec:costa_compreensao_pt}

Tal como para outras línguas, o estudo de \citet{chienwexler1990} foi reproduzido em português. Curiosamente, os resultados obtidos para o português europeu e para o português brasileiro foram bastante diferentes.

Em português europeu, \citet{cristovao2006} mostrou que as crianças portuguesas interpretam corretamente tanto pronomes como anáforas\is{anáfora} em frases como as de (\ref{ex:costa_38}):

\ea\label{ex:costa_38}
\ea\label{ex:costa_38a} O menino lava-se.
\ex\label{ex:costa_38b} O menino lava-o.
\zl

Ao contrário do que aconteceu em inglês,\il{inglês} as crianças portuguesas não apresentam evidência de dificuldade na compreensão de pronomes. Este resultado é compatível com o de \citet{mckee1992} para o italiano,\il{italiano} reforçando a ideia de que o estatuto categorial do pronome é fundamental para se predizer se há ou não dificuldades na sua compreensão. Com efeito, \citet{grolla2006,grolla2010} replicou o mesmo teste em português brasileiro e encontrou dificuldades na compreensão de pronomes, à semelhança do que aconteceu em inglês.\il{inglês} É crucial termos em conta que, no teste em português brasileiro, as frases utilizadas foram como a de (\ref{ex:costa_39}):

\ea\label{ex:costa_39}
O menino lava ele.
\z

Aqui o pronome é forte\is{pronome!forte} e, como já foi referido, apenas os pronomes fortes\is{pronome!forte} induzem problemas de compreensão (ver \citealt{cristovao2006} e \citealt{costalobo_etal2012} para uma tentativa de análise desta assimetria entre pronomes fortes\is{pronome!forte} e clíticos).\is{pronome!clítico}

A hipótese de que o estatuto do pronome, enquanto clítico\is{pronome!clítico} ou forte,\is{pronome!forte} é relevante levou \citet{costaambulate2010} e \citet{silva2015} a testar se, numa mesma variedade do português, os pronomes fortes\is{pronome!forte} são igualmente mais difíceis de compreender para as crianças. Assim, estes autores testaram a compreensão de pronomes fortes\is{pronome!forte} em português europeu, em contextos de pronome sujeito subordinado (como em \ref{ex:costa_40}) e em contextos de complemento de preposição (como em \ref{ex:costa_41}):

\ea\label{ex:costa_40}
O Pedro disse ao Paulo que \textbf{ele} tem fome.
\z
\ea\label{ex:costa_41}
O Pedro está orgulhoso \textbf{dele}.
\z

\citet{silva2015} mostra que, em todos os contextos, a compreensão do pronome forte\is{pronome!forte} é menos bem sucedida do que a dos pronomes clíticos\is{pronome!clítico} ou dos pronomes nulos, o que permite sustentar a hipótese de que os mecanismos de legitimação dos pronomes fortes\is{pronome!forte} são diferentes e dependentes de aspetos semântico-pragmáticos e não apenas de restrições sintáticas.

Estes resultados parecem ir ao encontro dos estudos existentes que mostram que as crianças conhecem desde cedo os princípios da Teoria da Ligação,\is{Teoria da Ligação} podendo, contudo, desconhecer os princípios concretos de legitimação semântico-pragmática de algumas formas pronominais. Os estudos sobre a compreensão de objeto nulo\is{objeto nulo} que relatamos na subsecção seguinte confirmam esta conclusão.

\subsection{Compreensão de objeto nulo em português}
\label{subsec:costa_compreensao_obj_nulo}

Nos estudos de produção, levantou-se a hipótese de que as crianças sobregeneralizam objeto nulo\is{objeto nulo} e que isso explica as altas taxas de omissão de pronomes clíticos.\is{pronome!clítico} Na secção anterior, colocou-se a hipótese de que nem todas as propriedades semântico-pragmáticas dos pronomes estão adquiridas desde cedo. Os estudos de \citet{costalobo2009,costalobo2010} e de \citet{costagrolla_etal2015} sobre a compreensão e aceitabilidade de objeto nulo\is{objeto nulo} parecem corroborar esta hipótese.

\citet{costalobo2009} testaram se as crianças são capazes de atribuir interpretações transitivas a verbos que ocorrem sem complemento, em frases como as de (\ref{ex:costa_42}):

\ea\label{ex:costa_42}
\ea\label{ex:costa_42a} Acordou(-o).
\ex\label{ex:costa_42b} Balançou(-o).
\ex\label{ex:costa_42c} Mergulhou(-o).
\zl

Se a gramática das crianças não contiver a possibilidade de existência de objetos nulos, na ausência do pronome, as crianças apenas conseguiriam atribuir interpretações intransitivas aos verbos, à semelhança do que foi encontrado para o francês\il{francês} e para o inglês\il{inglês} por \citet{gruter2006}. No entanto, as crianças portuguesas conseguiram interpretar estas frases transitivamente, o que mostra que aceitam construções de objeto nulo\is{objeto nulo} e que as interpretam adequadamente. O mesmo foi encontrado por \citet{costagrolla_etal2015} para crianças a adquirir o português brasileiro.

Curiosamente, as crianças portuguesas, que, como se viu, sobregeneralizam a construção de objeto nulo\is{objeto nulo} na produção, também sobregeneralizam o objeto nulo\is{objeto nulo} na compreensão, aceitando objetos nulos em contextos em que os adultos os rejeitam (contextos ilha\is{ilha} e contextos reflexos). Estes resultados permitem levantar, de novo, a hipótese de que as crianças conhecem a gramática que estão a adquirir, sabendo que se trata de uma gramática de objeto nulo,\is{objeto nulo} mas não conhecem ainda todas as propriedades semântico-pragmáticas das categorias nulas envolvidas.

Em \citet{costalobo2010}, testámos o conhecimento das crianças sobre propriedades finas do sujeito nulo\is{sujeito nulo} e do objeto nulo,\is{objeto nulo} a partir de um estudo de \citet{miyagawa2010}. Em frases como as de (\ref{ex:costa_43}) e (\ref{ex:costa_44}), podemos verificar que o sujeito nulo\is{sujeito nulo} apenas permite uma interpretação estrita, retomando apenas o sujeito da frase matriz, enquanto o objeto nulo\is{objeto nulo} é ambíguo:

\ea\label{ex:costa_43} O Pedro disse que os pais estão doentes e o Paulo disse que \textbf{Ø} estão bons.\\\textbf{Ø} $=$ pais do Pedro\\\textbf{Ø} $\neq$ pais do Paulo
\z

\ea\label{ex:costa_44} O Pedro abraçou os pais e o Paulo beijou \textbf{Ø}.\\\textbf{Ø} $=$ pais do Pedro\\\textbf{Ø} $=$ pais do Paulo
\z

Conhecer as propriedades dos pronomes nulos\is{pronome!nulo} implicará conhecer este tipo de restrição imposta sobre a sua interpretação. Em \citet{costalobo2010} testou-se o conhecimento destas propriedades por crianças portuguesas e o mesmo foi feito em comparação com crianças brasileiras em \citet{costagrolla_etal2015}, tendo-se concluído que as crianças aos 5 anos ainda não dominam estes pormenores que são relevantes para uma interpretação adulta das formas pronominais. 

Estes resultados permitem-nos dar resposta a algumas das questões enunciadas. Os dados da compreensão sugerem que as crianças dominam as restrições estritamente sintáticas que regulam a interpretação dos pronomes, tais como os princípios da Teoria da Ligação\is{Teoria da Ligação} ou os parâmetros\is{parâmetro} que preveem a utilização de categorias nulas, mas não conhecem todas as propriedades semântico-pragmáticas associadas aos pronomes fortes\is{pronome!forte} e às categorias nulas.

\section{Conclusão}
\label{sec:costa_conclusao}

Os resultados dos estudos que resumimos neste capítulo permitem-nos corroborar as observações recorrentes na literatura segundo as quais grande parte do conhecimento linguístico é adquirido muito precocemente. Com efeito, ao longo do capítulo, pudemos constatar que os seguintes conhecimentos estão adquiridos nos primeiros anos de vida:

\begin{enumerate}[label=\alph*)]
\item A distinção entre pronomes fortes,\is{pronome!forte} clíticos\is{pronome!clítico} e nulos.\is{pronome!nulo}
\item A variação interlinguística que regula se as línguas têm ou não sujeitos nulos e objetos nulos.
\item Os princípios sintáticos que regulam a distribuição e interpretação de anáforas\is{anáfora} e pronomes.
\end{enumerate}

A evidência de que este conhecimento linguístico fino é dominado tão cedo constitui argumento para a assunção de que parte do conhecimento linguístico é inato e independente de aprendizagem ou parasita noutras propriedades do desenvolvimento cognitivo.



{\sloppy
\printbibliography[heading=subbibliography,notkeyword=this]
}
\end{document}