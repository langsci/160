%%%%%%%%%%%%%%%%%%%%%%%%%%%%%%%%%%%%%%%%%%%%%%%%%%%%
%%%                                              %%%
%%%                 Metadata                     %%%
%%%          fill in as appropriate              %%%
%%%                                              %%%
%%%%%%%%%%%%%%%%%%%%%%%%%%%%%%%%%%%%%%%%%%%%%%%%%%%%

\title{Aquisição de língua materna e não materna}  
\subtitle{Questões gerais e dados do português}
% \BackTitle{} % Change if BackTitle != Title
\BackBody{O presente volume é uma introdução ao estudo da aquisição e desenvolvimento linguísticos. Embora especialmente dedicado à aquisição do português como língua materna e não materna, assume uma perspetiva comparada, confrontando dados da aquisição desta língua com os disponíveis para outras línguas do mundo. Surge na sequência da necessidade de um livro de carácter introdutório de apoio à atividade pedagógica nesta área do saber, suprindo uma lacuna há muito sentida por docentes e discentes. Integra capítulos da autoria de vários especialistas portugueses e brasileiros, o que permitiu compilar uma parte substancial do trabalho de investigação sobre aquisição do português desenvolvido nas últimas décadas, em formato de texto de divulgação de fácil acesso. Começando com uma panorâmica histórica das questões centrais colocadas, no último século, sobre a aquisição das línguas naturais, o volume explora vários domínios da aquisição (particularmente, a fonologia e a sintaxe) e considera o desenvolvimento típico e atípico, bem como o problema da avaliação linguística. O bilinguismo e a aquisição de uma L2 são o tema de dois capítulos independentes. Finalmente, relaciona-se o desenvolvimento do conhecimento implícito com o do conhecimento metalinguístico e com a aprendizagem da escrita.\\

% \noindent The present volume is an introduction to the study of Language Acquisition, especially centered on Portuguese. Even though the different chapters always take Portuguese as a point of departure, a comparative perspective is assumed and Portuguese data is compared to data from other languages, when relevant. This book aims at filling a gap in the literature: an introductory textbook to be used by students of Language Acquisition in Portuguese-speaking countries. The book is composed by chapters authored by several Portuguese and Brazilian researchers and it presents in  textbook format a relevant part of the research results obtained during the last decades. The book starts with a general historical presentation of the field. The following chapters explore the acquisition of phonology and syntax and consider the problem of typical and atypical development, as well as linguistic assessment. Bilingualism and L2 acquisition are the topics of two independent chapters. Two final chapters discuss the development of linguistic awareness, in relation to the acquisition of writing.
}
%\dedication{Change dedication in localmetadata.tex}
\typesetter{Pedro Tiago Martins, Felix Kopecky}
%\proofreader{Change proofreaders in localmetadata.tex}
\author{Maria João Freitas\lastand Ana Lúcia Santos} %use this field for the volume editors
\renewcommand{\lsSpineAuthor}{Freitas\lastand Santos}

\renewcommand{\lsISBNdigital}{978-3-96110-016-3}     
\renewcommand{\lsISBNsoftcover}{978-1-976340-14-7}
\renewcommand{\lsBookDOI}{10.5281/zenodo.889261}
\renewcommand{\lsSeries}{tbls} % use lowercase acronym, e.g. sidl, eotms, tgdi
\renewcommand{\lsSeriesNumber}{3} %will be assigned when the book enters the proofreading stage
\renewcommand{\lsID}{160} % contact the coordinator for the right number

\renewcommand{\lsBackTitleFont}{\sffamily\addfontfeatures{Scale=MatchUppercase}\fontsize{20pt}{10mm}\selectfont}